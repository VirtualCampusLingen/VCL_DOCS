\subsection{Terminplanung und Zeitmanagement}
\label{sec:TerminplanungUndZeitManagement}

Jedes Projekt ist an feste terminliche Zeiten gebunden, welche eingehalten werden müssen. Bei der Durchführung dieses 
Projektes werden die relevanten Termine anhand der Projektterminplanung im Voraus geplant. Diese Vorgehensweise 
dient dazu, das Risiko zu minimieren das Projekt aufgrund mangelnder Terminplanung nicht fristgerecht fertigstellen zu 
können.

Das Projekt begann am 03.04.2013 mit einem Kick-Off-Meeting mit Herrn Feldker. Das Projekt 
soll am 14.07.2014 mit der Übergabe an den Auftraggeber abgeschlossenen werden. Darüber hinaus wird 
der weitere Verlauf des Projektes schrittweise anhand von 
Arbeitspaketen mit festgelegten Durchführungszeiten durchgeplant. Für detaillierte Informationen kann das Gantt-Diagramm 
im Anhang betrachtet werden\footnote{siehe Anhang XX}.
%ToDo: Gantt Diagramm in Anhang einfügen
%ToDo: Anhang XX Verweis auf Gantt Diagramm richtig einfügen
Durch diese schrittweise Planung kann bestmöglich gewährleistet werden, dass das  Projektrisiko im Bezug auf zeitliche 
Planung minimiert wird und das Projekt zeitgerecht abgeschlossen werden kann.

Im dem Gantt-Diagramm ist neben der Terminplanung ersichtlich, dass für das Projekt insgesamt ein Zeitaufwand von
\textbf{760} Stunden angesetzt sind. Diese 760 Stunden ergeben sich dabei aus:

\tabelleEinfg{Geplanter zeitlicher Aufwand in Stunden}{tab:geplanterAufwand}{GeplanterAufwand}
