\subsection{Terminplanung}
\label{sec:Terminplanung}

Jedes Projekt ist an feste terminliche Zeiten gebunden, welche eingehalten werden müssen. Bei der Durchführung dieses 
Projektes sollen die relevanten Termine anhand der Projektterminplanung im Voraus geplant werden. Diese Vorgehensweise 
dient dazu, das Risiko zu minimieren das Projekt aufgrund mangelnder Terminplanung nicht fristgerecht fertigstellen zu 
können. 
Der Beginn dieses Projektes findet am 03.04.2013 mit einem Kick-Off-Meeting mit Herrn Feldker statt. Das Projekt soll am 14.07.2014 abgeschlossenen werden. Darüber hinaus wird der weitere Verlauf des Projektes schrittweise anhand von 
Arbeitspaketen mit festgelegten Durchführungszeiten durchgeplant. Für detaillierte Informationen kann das Gantt-Diagramm 
im Anhang betrachtet werden.
Durch diese schrittweise Planung kann bestmöglich gewährleistet werden, dass das  Projektrisiko im Bezug auf zeitliche 
Planung minmiert wird und das Projekt zeitgerecht abgeschlossen werden kann.

// weitere Daten rein?? Implementierung bis Ende 2013 und Fotos bis Anfang 2014???

Bei der Projektterminplanung ist es notwendig den zeitlichen Ablauf des Projektes sowie der einzelnen Meilensteine oder 
Arbeitspakete darzustellen. Hierbei müssen selbstverständlich Abweichungen oder anderweitige zeitliche Verschiebungen 
beachtet werden. Dies kann durch eine falsche zeitliche Einschätzung oder ein unerwartetes Problem eintreten. Dadurch 
können sich die Anfangs- und Endzeitpunkte von nachfolgenden Arbeitspaketen verzögern. Die visuelle Darstellung solch 
einer terminlichen und zeitlichen Planung sowie die Reihenfolge und Priorität der Arbeitspakete ist mithilfe eines Gantt-Diagramms möglich.

Der Beginn des Projektes findet am 03.04.2013 mit einem Kick-Off-Meeting mit Herrn Feldker statt. In diesem soll die 
Projektidee vorgestellt und diskutiert werden.
Zum 04.04.2014 soll das Projekt vollständig abgeschlossen und zur Zufriedenheit des Kunden implementiert worden sein. Dies 
bedeutet, dass die Anwendung fehlerfrei und vollständig nutzbar ist und in den Internetauftritt der Hochschule Osnabrück 
integriert worden ist. Durch die Implementierung zu dem genannten Datum kann gewährleistet werden, dass sich 
Studieninteressierte für das darauffolgende Semester bereits über diese Plattform informieren und einen eigenen Eindruck 
über die Räumlichkeiten des Campus erhalten können. 

Um das gewünschte Ziel zu erreichen, wird ein Mindestmaß an Arbeitsstunden benötigt. So sind die Projektmitglieder 
tagtäglich an dem Projekt tätig, ebenfalls an Wochenenden, also Samstagen und Sonntagen. Die tägliche Arbeitszeit an dem 
Projekt variiert in diesem Fall je nach Auslastung durch anderweitige Projekte oder Vorlesungen in den unterschiedlichen 
Modulen. Es wird somit aufgrund anderweitiger Auslastungen angesetzt, dass ein Arbeitstag an dem Projekt variabel ist, 
aber im Schnitt vier Stunden beträgt. Während der Praxisphasen wird ebenfalls an dem Projekt gearbeitet und es werden Scrum-Meetings per Skype abgehalten. Durch die Distanz der Wohnorte aller Projektmitglieder zueinander, ist die 
Arbeitsintensität nicht so stark ausgeprägt wie während der Theoriephase. Hierbei werden aber weiterhin die Arbeitspakete 
per PHProjekt den Gruppenmitgliedern zugewiesen und zeitliche Fristen zu den einzelnen Paketen hinterlegt.