\subsection{Kostenplanung}
\label{sec:Kostenplanung}

Im nachfolgenden Abschnitt werden die Kosten geplant, die zur Weiterführung und
zum Erhalt des Projektes anfallen. Zu diesen Kosten zählen sowohl alle
Anschaffungskosten für Geräte, als auch alle laufenden Kosten für
Software und Lizenzen, die zur Erstellung der Panoramafotos benötigt werden.
Darüber hinaus werden an dieser Stelle Personalkosten auf Basis der geplanten
Aufwandszeit berechnet, um den Wert des erstellten Projektes zu
ermitteln.
Die Kosten sind in nachfolgenden Tabellen dargestellt. 

\textbf{Anschaffungskosten}

\tabelleEinfg{Übersicht über Anschaffungskosten}{tab:KostenaufstellungAnschaffung}{Anschaffungskosten}

% \footnotetext[1]{\url{http://www.wexcameras.de/kaufen-nikon-d90-dslr-gehaeuse/p1028011?source=webgains&siteid=42077&affiliate=10800}}
% \footnotetext[2]{\url{http://www.amazon.de/Manfrotto-055XPROB-Ausz\%C3\%BCge-Belastbarkeit-schwarz/dp/B000TSHPCO/ref=sr_1_1?s=ce-de&ie=UTF8&qid=1393199001&sr=1-1&keywords=stativ+manfrotto}}
% \footnotetext[3]{\url{http://www.amazon.de/Walimex-Pro-abnehmbarer-Gegenlichtbl-Objektivbajonett/dp/B008VGCW7O/ref=sr_1_20?s=ce-de&ie=UTF8&qid=1393198301&sr=1-20&keywords=nikon+d90+fisheye}}
% \footnotetext[4]{\url{http://www.pano-store.de/nodal-ninja-rd3l-6-8-30-mit-ez-leveler-ii}}
% \footnotetext[5]{\url{http://www.pano-store.de/nodal-ninja-rd8}}
% \footnotetext[6]{\url{http://www.pano-store.de/nodal-ninja-3-mkii-starter-paket-mit-standard-rotator}}

\textbf{Lizenzkosten}

\tabelleEinfg{Übersicht über Lizenzkosten}{tab:KostenaufstellungLizenzen}{Lizenzkosten}

% \footnotetext[7]{\url{https://www.adobe.com/de/products/catalog/cs6._sl_id-contentfilter_sl_catalog_sl_software_sl_creativesuite6.html}}
% \footnotetext[8]{\url{http://www.kolor.com/buy/software/autopano-giga-4.html}}

\clearpage
\textbf{Laufende Kosten}

\tabelleEinfg{Übersicht über laufende Kosten}{tab:KostenaufstellungLaufendeKosten}{LaufendeKosten}

% \footnotetext[9]{\url{https://www.hosteurope.de/de/Server/Virtual-Server/}}

\textbf{Personalkosten}

\tabelleEinfg{Übersicht über Personalkosten}{tab:KostenaufstellungPersonal}{Personalkosten}

\footnotetext[14]{berechnet aus 760 Stunden geplanter Aufwandszeit und einem Kostensatz von 60€ pro Stunde}

\textbf{Gesamt}

\tabelleEinfg{Übersicht über Gesamtkosten}{tab:KostenaufstellungGesamt}{Gesamtkosten}

Die vorherigen Tabellen zeigen eine detaillierte Aufstellung der Projekkosten.
Die aufgelisteten Kosten sind dabei aber nicht tatsächlich angefallen, da die
Kostenpositionen innerhalb des Projektteams reduziert werden konnten.

Die angesetzten Kosten für Anschaffungen konnten vermieden werden, 
da sich das gesamte Equipment im Besitz des Projektteams befand oder 
geliehen werden konnte.

Für die softwareseitige Erstellung der Panoramafotos wurden
zwei Programme verwendet. Zum einen Adobe Photoshop und zum anderen Autopano Giga. 
Die Software musste ebenfalls nicht für das Projekt angeschafft werden, da kostenlose
Testlizenzen genutzt werden konnten. Es 
werden Volllizenzen angeschafft, um die weitere Administration mit den Programmen zu gewährleisten.
Die Kosten für eine Domain und Webspace können zudem vernachlässigt werden, da die geplante Anwendung auf einem bereits 
vorhandenen Server der Hochschule laufen soll.
Die Personalkosten für dieses Projekt sind ebenfalls nicht angefallen, da die Durchführung im Zuge des Projektstudiums 
stattfand.

Im Zeitraum der Projektdurchführung sind daher \textbf{keine} Kosten enstanden, es enstehen aber
Kosten für die Weiterführung des Projektes.
