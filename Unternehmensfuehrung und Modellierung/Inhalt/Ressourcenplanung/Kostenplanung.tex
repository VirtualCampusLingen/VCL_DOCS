\subsection{Kostenplanung}
\label{sec:Kostenplanung}

Im nachfolgenden Abschnitt werden die Kosten geplant, die zur Weiterführung und zum Erhalt des Projektes anfallen. Zu 
diesen Kosten zählen sowohl alle Anschaffungskosten für die Geräte, die das Projektteam aus privatem Besitzt zur 
Anfertigung des Projektes gestellt hat, als auch alle laufenden Kosten für Software und Lizenzen, die zur Erstellung der 
Panoramafotos benötigt werden. 
Darüber hinaus werden an dieser Stelle Personalkosten auf Basis der geplanten Aufwandszeit berechnet, um den Wert des 
erstellten Projektes zu bewerten.
Die Kosten sind in nachfolgenden Tabellen dargestellt.

\textbf{Anschaffungskosten}

\begin{table}[h]
\centering
\begin{tabular}{ccccl}
\hline
\multicolumn{1}{l}{}      & angesetzte Kosten {[}€{]} & tatsächliche Kosten {[}€{]} \\ \hline
Kamera                    & 563\footnotemark[1]           & 0                    \\ \hline
Stativ                    & 139\footnotemark[2]           & 0                    \\ \hline
Fisheye-Objektiv          & 360\footnotemark[3]           & 0                    \\ \hline
Nivelliervorrichtung      & 195\footnotemark[4]           & 0                    \\ \hline
Rotationsvorrichtung      & 142\footnotemark[5]           & 0                    \\ \hline
Panoramakopf              & 215\footnotemark[6]           & 0                    \\ \hline

Gesamt                    & 1614           & 0                    \\ \hline
\end{tabular}
\caption{Soll ist Vergleich der Kostenplanung}%
\label{tab:KostenaufstellungAnschaffung}%
\end{table}

\footnotetext[1]{\url{http://www.wexcameras.de/kaufen-nikon-d90-dslr-gehaeuse/p1028011?source=webgains&siteid=42077&affiliate=10800}}
\footnotetext[2]{\url{http://www.amazon.de/Manfrotto-055XPROB-Ausz\%C3\%BCge-Belastbarkeit-schwarz/dp/B000TSHPCO/ref=sr_1_1?s=ce-de&ie=UTF8&qid=1393199001&sr=1-1&keywords=stativ+manfrotto}}
\footnotetext[3]{\url{http://www.amazon.de/Walimex-Pro-abnehmbarer-Gegenlichtbl-Objektivbajonett/dp/B008VGCW7O/ref=sr_1_20?s=ce-de&ie=UTF8&qid=1393198301&sr=1-20&keywords=nikon+d90+fisheye}}
\footnotetext[4]{\url{http://www.pano-store.de/nodal-ninja-rd3l-6-8-30-mit-ez-leveler-ii}}
\footnotetext[5]{\url{http://www.pano-store.de/nodal-ninja-rd8}}
\footnotetext[6]{\url{http://www.pano-store.de/nodal-ninja-3-mkii-starter-paket-mit-standard-rotator}}


\textbf{Lizenzkosten}

\begin{table}[h]
\centering
\begin{tabular}{ccccl}
\hline
\multicolumn{1}{l}{}      & angesetzte Kosten {[}€{]} & tatsächliche Kosten {[}€{]} &  Laufzeit \\ \hline
Photoshop CS6             & 950\footnotemark[7]       & 0                           &  unbegrenzt  \\ \hline
Autopano Giga             & 199\footnotemark[8]       & 0                           &  unbegrenzt  \\ \hline

Gesamt                    & 1149           & 0                    \\ \hline
\end{tabular}
\caption{Soll ist Vergleich der Kostenplanung}%
\label{tab:KostenaufstellungLizenzen}%
\end{table}

\footnotetext[7]{\url{https://www.adobe.com/de/products/catalog/cs6._sl_id-contentfilter_sl_catalog_sl_software_sl_creativesuite6.html}}
\footnotetext[8]{\url{http://www.kolor.com/buy/software/autopano-giga-4.html}}


\textbf{Laufende Kosten}

\begin{table}[h]
\centering
\begin{tabular}{ccccl}
\hline
\multicolumn{1}{l}{}      & angesetzte Kosten {[}€{]} & tatsächliche Kosten {[}€{]} \\ \hline
Hostingkosten             & 12,99\footnotemark[9]       & 0                         \\ \hline

Gesamt                    & 12,99           & 0                    \\ \hline
\end{tabular}
\caption{Soll ist Vergleich der Kostenplanung}%
\label{tab:KostenaufstellungLizenzen}%
\end{table}

\footnotetext[9]{\url{https://www.hosteurope.de/de/Server/Virtual-Server/}}


\textbf{Personalkosten}

\begin{table}[h]
\centering
\begin{tabular}{ccccl}
\hline
\multicolumn{1}{l}{}      & angesetzte Kosten {[}€{]} & tatsächliche Kosten {[}€{]} \\ \hline
Personal                  & 45.600\footnotemark[10]    & 0                         \\ \hline

Gesamt                    & 45.600           & 0                    \\ \hline
\end{tabular}
\caption{Soll ist Vergleich der Kostenplanung}%
\label{tab:KostenaufstellungLizenzen}%
\end{table}

\footnotetext[10]{berechnet 760 Stunden geplanter Aufwandszeit und einem Kostensatz von 60€ pro Stunde}


Anhand vorheriger Tabelle ist erkennbar, dass keine geplanten Kosten anfallen.
Die angesetzten Kosten konnten vermieden werden, da sich das gesamte Equipment im Besitz des Projektteams befand oder 
geliehen werden konnte.

Das Equipment umfasst die Spiegeflexkamera, das Stativ inkl. 
Panoramakopf und das Fisheye-Objektiv. Für die softwareseitige 
Erstellung der Panoramafotos wurden zwei Programme verwendet. Zum einen Adobe Photoshop und zum anderen Autopano Giga. 
Die Software musste ebenfalls nicht für das Projekt angeschafft werden, da kostenlose
Testlizenzen genutzt werden konnten. Es 
werden Volllizenzen angeschafft, um die weitere Administration mit den Programmen zu gewährleisten.
Die Kosten für eine Domain und Webspace konnten vernachlässigt werden, da die geplante Anwendung auf einem bereits 
vorhandenen Server der Hochschule laufen soll.
Die Personalkosten für dieses Projekt sind ebenfalls nicht angefallen, da die Durchführung im Zuge des Projektstudiums 
stattfand.