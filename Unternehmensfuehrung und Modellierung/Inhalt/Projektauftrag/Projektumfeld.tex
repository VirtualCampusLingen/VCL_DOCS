\subsection{Projektumfeld}
\label{sec:Projektumfeld}

Nachdem nun die Ziele des Projektes "`Virtueller Campus Lingen"' dargestellt
wurden sollen nachfolgend alle Stakeholder des Projektes herausgestellt werden.
Stakeholder sind hierbei alle Personen, die ein Berechtigtes Interesse sowohl an
dem Verlauf als auch an dem Ergebnis des Projektes haben. Das Projektteam,
welches aus Andreas Makeev, Raphael Otten, Carsten Sandker und Jannik Fangmann
besteht, stellt hierbei nur einen Teil der Stakeholder dar. Dieser Teil der
Stakeholder agiert als Entwickler und ist für die Umsetzung der Projektidee
zuständig.

Auftraggeber des vorleigenden Projektes ist die Leitung der Fakultät Management,
Kultur und Technink der Hochschule Osanbrück, da diese Fakultät an dem Standort
Lingen ansässig ist. Die Fakultätsleitung besteht hierbei aus dem Dekan der
Fakultät, Prof. Dr. Frank Blümel, und den Leitern der vier Institute der
Fakultät. Diese sind Prof. Dr. Michael Ryba für das Institut Management und
Technik, Prof. Dr.-Ing. Wolfgang Arens-Fischer für das Institut für Duale
Studiengänge, Prof. Dr. Dagmar Schütte für das Institut für
Kommunikationsmanagement und Prof. Dr. Bernd Ruping für das Institut für
Theaterpädagogik. Als Auftraggeber entscheidet dieses Gremium über die
Realisierung und spätere Einführung des Projektes.

% Weiterhin: Nutzen des Projektes muss den Auftraggebern verdeutlicht werden

Die Rolle der Betreuer wird von den Projektpaten Stefan Feldker und Prof. Dr.
Ralf Westerbusch übernommen. Diese sollen der Projektgruppe bei Fragen zur
Seite stehen und diese durch den Prozess der Projektumsetzung begleiten. Hierfür
müssen die Betreuer in das Projekt mit eingebunden und über den aktuellen Stand
des Projektes informiert werden. Die Betreuer stellen weiterhin die
Schnittstelle der Kommunikation zwischen den Entwicklern und der
Fakultätsleitung dar.

% Ist das so?

Auch der Datenschutzbeauftragte der Hochschule Osnabrück, Prof. Dr. Alfred
Scheerhorn ist am Projekt beteiligt. 
% Das Projekt soll, wie bereits
% herausgestellt, im Stile von Google Street View realisiert werden. Hierbei
% sollen die Benutzer durch 360-Grad Fotos des Campus navigieren. Die Umsetzung
% des Projektes beinhaltet also auch die Veröffentlichung dieser Fotos im
% Internet. Bei den datenschutzrechtlichen Fragen, die sich hieraus ergeben, soll
% Herr Scheerhorn dem Projektteam als Berater zur Seite stehen.

Neben den bereits erwähnten Personen und Personengruppen gehören auch die
Benutzer der zu realisierenden Anwendung zu den Stakeholdern des Projektes.
Alle Stakeholder, die an dem Projekt beteiligt sind, sind in Abbildung X
% TODO: Verweis auf Anhang
dargestellt.
