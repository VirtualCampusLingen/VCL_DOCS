\subsection{Projektzielsetzung}
\label{sec:Projektzielsetzung}

Die Projektzielsetzung stellt einen wichtigen Punkt der Projektplanung dar.
Durch die Projektzielsetzung soll eine gemeinsame Zielvorstellung entwickelt
werden, welche verbindlich festgehalten wird.\footnote{\citet[S.~34]{litke2007}}
Diese Ziele müssen mit dem Auftraggeber abgestimmt und von den Entwicklern im
Laufe des Projektes umgesetzt werden. Nur bei Erreichung der Projektziele kann
das Projekt als erfolgreich abgeschlossen bezeichnet werden. Die Zielerreichung
wird hierbei nach Abschluss des Projektes untersucht.

Ein Zielsystem, das für das vorliegende Projekt erstellt wurde, stellt alle 
Ziele des Projektes übersichtlich dar und ist in \verweis{Zielsystem} abgebildet. 
Der bereits in\verweis{Projektidee} definierte Projektauftrag stellt das Hauptziel, 
das Projektziel in der obersten Ebene, dar. Das Hauptziel des Projektes ist:

\begin{quote}
Konzeptionierung und Entwicklung einer Anwendung zur Darstellung des 
Campus Lingen als attraktiven Studienstandort.
\end{quote}

Um dieses sehr allgemein formulierte Hauptziel zu operationalisieren, wird
es in folgende sechs Oberziele zerlegt:

\begin{itemize}
  \item Außendarstellung des Campus Lingen verbessern
  \item Kosten gering halten / Finanzierung sichern
  \item Campus vollständig und realistisch abbilden
  \item Nachhaltigkeit sichern
  \item Zielgruppe junger Studieninteressierter direkt ansprechen
  \item Qualität des Internetauftritts erhalten
\end{itemize}

In diesen Oberzielen spiegelt sich das Hauptziel des Projektes wider. Die
sechs Oberziele bieten jedoch noch nicht den gewünschten Grad der
Operationalisierung. Aus diesem Grund werden die einzelnen Oberziele ebenfalls
in mehre Unterziele zerlegt. Die Unterziele sind in \verweis{Zielkatalog} dargestellt.

Diese Ziele sind nicht mehr durch andere Ziele beschreibbar und 
ihnen können konkrete Maßnahmen zur Erreichung zugeordnet werden. Zusätzlich sind die
Ergebnisse dieser Ziele messbar und machen die Erreichung des Oberziels und
somit schließlich die Erreichung des Hauptziels bewertbar. Um die Relevanz der
einzelnen Unterziele zu verdeutlichen und Missverständnisse bezüglich dieser
Relevanz zu vermeiden, werden diese in die Kategorien \textbf{Muss}-, \textbf{Soll}-, und
\textbf{Kann}-Ziele eingeordnet. 
Muss-Ziele sind hierbei elementar für die
Fertigstellung und den Erfolg des Projektes. Sollten diese Ziele nicht erfüllt
sein, kann das Projekt nicht erfolgreich abgeschlossen werden. Das Nichterfüllen von Soll-Zielen führt
hingegen nicht sofort zum Scheitern des Projektes. Das Endprodukt wird jedoch
stark in seinen Funktionen eingeschränkt sein. Die Erfüllung von Kann-Zielen
steigert nur den Wert des Projektes. Eine Erfüllung dieser Ziele ist nicht
zwingend erforderlich.

Die Muss- Soll- und Kann-Ziele werden im Zielsystem durch eine Ampeldarstellung
abgebildet. Muss-Ziele wurden im Zielsystem rot , Soll-Ziele gelb und 
Kann-Ziele grün hervorgehoben.

Neben der Unterteilung in Muss-, Kann- und Soll-Ziele wurden die Ziele auf der
untersten Ebene weiterhin in die Kategorien "`Sachziele"' und
"`Betriebswirtschaftliche Ziele"' eingeteilt. Die betriebswirtschaftlichen
Ziele repräsentieren hierbei besonders die Interessen der Auftraggeber. Durch
die Sachziele wird hingegen der angestrebte Funktionsumfang der Anwendung
verdeutlicht.

Eine abschließende Differenzierung der Ziele wird durch die Unterteilung 
in qualitative und quantitative Ziele gegeben. Diese Unterteilung ist wichtig
für die spätere Verifikation der Zielerreichung, denn nur die Erreichung von quantitativen
Zielen lässt sich durch Zielindikatoren bestimmen und verifizieren. 
Für qualititative Ziele lassen sich dagegen keine Zielindikatoren
finden, da diese einen Zielzustand beschreiben, der nur subjektiv wahrgenommen werden kann.
Zur Verifikation der Projektziele werden aus diesem Grund nur die quantitativen Ziele betrachtet.

Alle operationaliserten Ziele werden mit einem entsprechenden Zielindikator, der die Erreichung dieses
Ziel verifiziert, in \verweis{Zielkatalog} vorgestellt.