\subsection{Projektzielsetzung}
\label{sec:Projektzielsetzung}

Die Projektzielsetzung stellt einen wichtigen Punkt der Projektplanung dar.
Durch die Projektzielsetzung soll eine gemeinsame Zielvorstellung entwickelt
werden, welche verbindlich festgehalten wird.\footnote{\citet[S.~34]{litke2007}}
Diese Ziele müssen mit dem Auftraggeber abgestimmt und von den Entwicklern im
Laufe des Projektes umgesetzt werden. Nur bei Erreichung der Projektziele kann
das Projekt als erfolgreich abgeschlossen bezeichnet werden. Die Zielerreichung
wird hierbei nach Abschluss des Projektes untersucht. Die Ziele orientieren
sich dabei immer an einem Soll-Zustand, der folgende Merkmale aufweisen muss:

\begin{itemize}
  \item Er muss in der Zukunft liegen
  \item Er muss real sein
  \item Das Erreichen muss erwünscht sein
  \item Er muss bewusst gewählt sein
  \item Er muss nur durch Handlung erreicht werden
  können\footnote{\citet[S.~33]{litke2007}}
\end{itemize}

Als Methode für die Zielfestlegung wurde die Zielhierarchie gewählt. Hierbei
werden die Ziele hierarchisch vom Allgemeinen zum Speziellen definiert. Die
systematische Anordnung der Projektziele, welche hierbei entsteht, wird als
Zielsystem bezeichnet. Der Vorteil dieser Methode ist, dass Ziele auf der
hierarchisch letzten Stufe besser analysiert werden können. Hierdurch wird eine
Messbarkeit der Zielerreichung möglich. 

\subsubsection{Zielsystem des Projektes}
\label{sec:Zielsystem}

Für das Projekt wurde ein Zielsystem erstellt, welches sich in drei Ebenen
gliedert. Das Zielsystem ist in \anhang{Zielsystem} dargestellt. Der bereits in
\verweis{Projektidee} definierte Projektauftrag "`Konzeptionierung und
Entwicklung einer Anwendung zur Darstellung des Campus Lingen als attraktiven
Studienstandort"' stellt das Hauptziel, das Projektziel in der obersten Ebene,
dar. Um dieses sehr allgemein formulierte Hauptziel zu operationalisieren, wurde
es in folgende sechs Oberziele zerlegt:

\begin{itemize}
  \item Außendarstellung des Campus Lingen verbessern
  \item Kosten gering halten / Finanzierung sichern
  \item Campus vollständig und realistisch abbilden
  \item Nachhaltigkeit sichern
  \item Zielgruppe junger Studieninteressierter direkt ansprechen
  \item Qualität des Internetauftritts erhalten
\end{itemize}

In diesen Oberzielen spiegelt sich das Hauptziel des Projektes wider. Die
sechs Oberziele bieten jedoch noch nicht den gewünschten Grad der
Operationalisierung. Aus diesem Grund wurden die einzelnen Oberziele ebenfalls
in mehre Unterziele zerlegt.

Diese Ziele sind nicht mehr durch andere Ziele beschreibbar. Zur Erreichung
können diesen Zielen konkrete Maßnahmen zugeordnet werden. Zusätzlich sind die
Ergebnisse dieser Ziele messbar und machen die Erreichung des Oberziels und
somit schließlich die Erreichung des Hauptziels bewertbar. Um die Relevanz der
einzelnen Unterziele zu verdeutlichen und Missverständnisse bezüglich dieser
Relevanz zu vermeiden, wurden diese in die Kategorien Muss-, Soll-, und
Kann-Ziele eingeordnet. Muss-Ziele sind hierbei elementar für die
Fertigstellung und den Erfolg des Projektes. Sollten diese Ziele nicht erfüllt
sein, wird das Projekt scheitern. Das Nichterfüllen von Soll-Zielen führt
hingegen nicht sofort zum Scheitern des Projektes. Das Endprodukt wird jedoch
stark in seinen Funktionen eingeschränkt sein. Die Erfüllung von Kann-Zielen
steigert nur den Wert des Projektes. Eine Erfüllung dieser Ziele ist nicht
zwingend erforderlich.

Die Muss- Soll- und Kann-Ziele werden im Zielsystem durch eine Ampeldarstellung
abgebildet.  Muss-Ziele wurden im Zielsystem rot , Soll-Ziele
wurden gelb und Kann-Ziele wurden grün hervorgehoben.
%ToDo: Kann man das so sagen? Satz hatte vorher 3x hervorgehoben

Neben der Unterteilung in Muss-, Kann- und Soll-Ziele wurden die Ziele auf der
untersten Ebene weiterhin in die Kategorien "`Sachziele"' und
"`Betriebswirtschaftliche Ziele"' eingeteilt. Die betriebswirtschaftlichen
Ziele repräsentieren hierbei besonders die Interessen der Auftraggeber. Durch
die Sachziele wird hingegen der angestrebte Funktionsumfang der Anwendung
verdeutlicht.

\subsubsection{Zielkatalog}
\label{sec:Zielkatalog}

Die zuvor herausgestellten Unterziele sollen im Folgenden genauer beschrieben
werden. Hierbei soll eine Unterteilung der Ziele in qualitative und quantitative
Ziele erfolgen. Weiterhin sollen für die Unterziele Zielindikatoren festgelegt
werden. Mit diesen Zielindikatoren wird eine Möglichkeit zur Verifikation der
Ziele geschaffen. Das Ziel gilt als erfüllt, wenn der Zielindikator im zuvor von
der Projektgruppe festgelegten Maße erreicht wurde. Durch die Zuordnung der
Zielindikatoren zu den Unterzielen entsteht so ein Zielkatalog, mit dem sich
die Erreichung des Projektziels messen lässt.

Für die Muss-Ziele des Projektes wurde hierbei von der Projektgruppe folgender
Zielkatalog ausgearbeitet:

\begin{description}
	\item[Steigerung der Attraktivität:] Die Zielerreichung wird innerhalb eines
	Beta-Tests festgestellt. Die Tester werden hier nach der von ihnen empfundenen
	Attraktivität der Anwendung gefragt. Das Ziel gilt als erreicht, wenn innerhalb
	des Fragebogens beim Betatest die Frage "`Wie bewerten Sie die durch die
	Anwendung vermittelte Attraktivität des Campus Lingen?"' im durchschnitt
	mindestens mit dem Wert 3,0 oder besser bewertet wird.
	\item[Abbildung aller Institute:] Die Zielerreichung wird von der Projektgruppe
	festgestellt. Diese bewertet, ob alle Institute in der Anwendung in
	angemessenem Umfang präsentiert werden. Kriterium bei der Bewertung ist hierbei
	die Anzahl der erstellten Fotos je Institut. Das Ziel gilt als erreicht, wenn
	mindestens 10 oder mehr Bilder pro Institut in der Anwendung vorhanden sind.
	% TODO: Infotexte auch relevant?	
	\item[Verbesserung der Präsentation von Studienprojekten:] Die Zielerreichung
	wird innerhalb eines Beta-Tests festgestellt. Die Tester werden hier gefragt,
	ob sie durch die Anwendung auf Studienprojekte aufmerksam gemacht wurden. Das
	Ziel gilt als erreicht, wenn im durchschnitt alle Betatester die Frage "`Wie
	bewerten Sie die Art der Präsentation der Studienprojekte im Vergleich zur
	bereits bestehenden Hochschulseite? "' mit mindestens 3,0 bewerten.
	\item[Verbesserung der Präsentation aller Studiengänge:] Das Ziel
	gilt als erreicht, wenn im Durchschnitt alle Betatester die Frage "`Wie
	bewerten Sie die Art der Präsentation der einzelnen Studiengänge im Vergleich
	zur bereits bestehenden Hochschulseite?"' mit mindestens 3,0 bewerten. Ein
	weiteres Kriterium für dieses Ziel ist die Anzahl der Infotexte pro
	Studiengang. Es soll mindestens ein Infotext zu jedem Studiengang angelegt
	sein.
	Dies wird durch die Projektgruppe selbst verifiziert.
	\item[Aufbau einer Administrationsoberfläche:] Die Zielerreichung wird von der
	Projektgruppe festgestellt. Das Ziel gilt als erreicht, wenn es der
	Projektgruppe möglich ist ihre erstellten 360 Grad Fotos innerhalb der
	Administrationsoberfläche abzuspeichern, zu positionieren und Infotexte zu
	vergeben.%ToDo: Schreibweise von "`360 Grad Foto"' im Dokument einhaltlich ?
	\item[Entwicklung einer Anwendungsdokumentation:] Die Zielerreichung wird
	von der Projektgruppe festgestellt. Das Ziel gilt hierbei als erreicht, wenn
	für die zu realisierende Anwendung eine Dokumentation erstellt wurde. In dieser
	müssen alle relevanten Anwendungsfälle dokumentiert sein.
	\item[Verbesserung der Informationsbeschaffung:] Innerhalb eines Betatest wird
	der Tester nach der Einfachheit der Informationsbeschaffung im
	Gegensatz zur textbasierten Homepage gefragt. Das Ziel gilt als erreicht, wenn
	im durchschnitt alle Betatester auf die Frage "`Wie bewerten Sie die Art der
	Informationsdarstellung im Vergleich zur bereits bestehenden Hochschulseite?"'
	mit mindestens 3,0 oder besser beantworten.
	\item[Erstellung unter Einhaltung der Datenschutzrichtlinien:] Die in Absprache
	mit der Hochschule Osnabrück entstandenen Datenschutzrichtlinien müssen während
	der Erstellung beachtet werden. Die Zielerreichung wird hierbei von der
	Projektgruppe selbst verifiziert. Es sind keine Abweichungen zu den
	Datenschutzrichtlinien erlaubt.
	\item[Erstellung unter Beachtung der IT-Sicherheit:] Die erstellte Anwendung
	wird von Kommilitonen aus dem Studiengang "`Wirtschaftsinformatik"' einem
	Penetrationstest unterzogen. Das Ziel gilt als erreicht wenn kein
	Administrationszugriff erlangt werden kann. 
	%TODO: Beschreibung Penetrationstest
\end{description}

Im Nachfolgenden werden examplarisch ein Soll- und ein Kann-Ziel mit dem
dazugehörigen Zielindikator dargestellt. Es wurden alle Muss-Ziele mit den
dazugehörigen Indikatoren gezeigt, da diese per Definition bei einem
Nichterreichen zum Misserfolg des Projektes führen.

Ein Soll-Ziel des Oberziels "`Nachhaltigkeit sichern"' ist "`Verbesserung der
Usability"'. Die Zielerreichung wird innerhalb des Betatests verifiziert. Das
Ziel gilt als erreicht, wenn im Durchschnitt alle Betatester auf die Frage
"`Konnten Sie die Anwendung intuitiv verwenden?"' mit mindestens 3,0 oder besser
antworten. %ToDo: Betatests schreibweise im gesamten Dokument beachten

Ein Kann-Ziel des Oberziels "`Außendartellung des Campus Lingen
verbessern"' ist "`Förderung eines positiven Images"'. Hierbei ist das Festlegen
eines klaren Zielindikators schwer, da eine positives Image nicht direkt auf die
Anwendung zurückgeführt werden kann. Es handelt sich bei diesem Unterziel um ein
qualitatives Ziel. %ToDo: Begriff erklären in Fussnote? 
Dies bedeutet, dass sich das Erreichen des Ziels nicht direkt messen lässt. Es
müssen Beobachtungskriterien gefunden werden, die unter einer Annahme dieses
Ziel abbilden und bewertbar sind. Die Projektgruppe entscheidet sich hierbei für
eine Frage innerhalb des Betatestfragebogens. Der Tester wird nach dem
Potenzial der Anwendung zur Verbesserung des Images vom Campus Lingen gefragt.
Somit kann von der Projektgruppe die Zielerreichung verifiziert werden.

Ein detaillierter Zielkatalog mit der Auflistung aller Ziele mit den
zugeordneten Zielindikatoren zur Bestimmung der Zielerreichung ist im Anhang
%ToDo Anhang Zielindikatoren verlinken 
zu finden.











