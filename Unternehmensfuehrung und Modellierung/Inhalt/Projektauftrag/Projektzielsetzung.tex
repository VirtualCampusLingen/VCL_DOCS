\subsection{Projektzielsetzung}
\label{sec:Projektzielsetzung}

Die Projektzielsetzung stellt einen wichtigen Punkt der Projektplanung dar.
Durch die Projektzielsetzung soll eine gemeinsame Zielvorstellung entwickelt
werden welche verbindlich festgehalten wird.\footnote{\citet[S.~34]{litke2007}}
Diese Ziele müssen mit dem Auftraggeber abgestimmt und von den Entwicklern im
Laufe des Projektes umgesetzt werden. Nur bei erreichung der Projektziele kann
das Projekt als erfolgreich abgeschlossen bezeichnet werden. Die Zielerreichung
wird hierbei nach Abschluss des Projektes untersucht. Die Ziele orientieren
sich dabei immer an einen Soll-Zustand, der folgende Merkmale aufweisen muss:

\begin{itemize}
  \item Er muss in der in der Zukunft liegt.
  \item Er muss real sein.
  \item Das Erreichen muss erwünscht sein.
  \item Er muss bewusst gewählt sein.
  \item Er muss nur durch Handlung erreicht werden
  können.\footnote{\citet[S.~33]{litke2007}}
\end{itemize}

Als Methode für die Zielfestlegung wurde die Zielhierarchie gewählt. Hierbei
werden die Ziele hierarchisch vom Allgemeinen zum spezielleren definiert. Die
systematische Anordnung der Projektziele, welche hierbei entsteht, wird als
Zielsystem bezeichnet. Der Vorteil dieser Methode ist das Ziele auf der
hierarchisch letzten Stufe besser analysiert werden können. Hierdurch wird eine
Messbarkeit der Zielerreichung möglich. 

\subsubsection{Zielsystem des Projektes}
\label{sec:Zielsystem}

Für das Projekt wurde ein Zielsystem erstellt, welches sich in drei Ebenen
gliedert. Das Zielsystem ist in \anhang{Zielsystem} dargestellt. Der bereits in
\verweis{Projektidee} definierte Projektauftrag "`Konzeptionierung und
Entwicklung einer Anwendung zur Darstellung des Campus Lingen als attraktiven
Studienstandort"' stellt das Hauptziel, das Projektziel in der obersten Ebene,
dar. Um dieses sehr allgemein formulierte Hauptziel zu operationalisieren wurde
es in folgende sechs Oberziele zerlegt:

\begin{itemize}
  \item Außendarstellung des Campus Lingen verbessern
  \item Kosten gering / Finanzierung sichern
  \item Campus vollständig und realistisch abbilden
  \item Nachhaltigkeit sichern
  \item Zielgruppe junger Studieninteressierter direkt ansprechen
  \item Qualität des Internetauftritts Erhalten
\end{itemize}

In diesen Oberzielen spiegelt sich das Hauptziel des Projektes wieder. Die
sechs Oberziele bieten jedoch noch nicht den gewünschten Grad der
Operationalisierung. Aus diesem Grund wurden die einzelnen Oberziele ebenfalls
in mehre Unterziele zerlegt.

Diese Unterziele stellen Ziele auf der untersten Ebene dar, welche nicht
mehr durch andere Ziele beschreibbar sind. Zur Erreichung können diesen Zielen
konkrete Maßnahmen zugeordnet werden. Zusätzlich sind die Ergebnisse dieser
Ziele messbar und machen die Erreichung des Oberziels und somit schließlich
die Erreichung des Hauptziels bewertbar. Um die Relevanz der einzelnen
Unterziele zu verdeutlichen und Missverständnisse bezüglich dieser Relevanz zu
vermeiden wurden diese in die Kategorien Muss-, Soll-, und Kann-Ziele
eingeordnet. Muss-Ziele sind hierbei elementar für die Fertigstellung und den
Erfolg des Projektes. Sollten diese Ziele nicht erfüllt sein wird das Projekt
scheitern. Das Nichterfüllen von Soll-Zielen führt hingegen nicht sofort zum
Scheitern des Projektes. Das Endprodukt wird jedoch stark in seinen Funktionen
eingeschränkt sein. Die Erfüllung von Kann-Zielen steigert nur den Wert des
Projektes. Eine Erfüllung dieser Ziele ist nicht zwingend erforderlich.

Die Muss- Soll- und Kann-Ziele werden im Zielsystem durch eine Ampeldarstellung
abgebildet.  Muss-Ziele wurden im Zielsystem rot hervorgehoben, Soll-Ziele
wurden gelb hervorgehoben und Kann-Ziele wurden grün hervorgehoben.

Neben der Unterteilung in Muss-, Kann- und Soll-Zielen wurden die Ziele auf der
untersten Ebene weiterhin in die Kategorien "`Sachziele"' und
"`Betriebswirtschaftliche Ziele"' eingeteilt. Die betriebswirtschaftlichen
Ziele repräsentieren hierbei besonders die Interessen der Auftraggeber. Durch
die Sachziele wird hingegen der angestrebte Funktionsumfang der Anwendung
verdeutlicht.

\subsubsection{Zielkatalog}
\label{sec:Zielkatalog}

Die zuvor herausgestellten Unterziele sollen im Folgenden genauer beschrieben
werden. Hierbei soll eine Unterteilung der Ziele in qualitative und quantitative
Ziele erfolgen. Weiterhin sollen für die Unterziele Zielindikatoren festgelegt
werden. Mit diesen Zielindikatoren wird eine Möglichkeit zur Messbarmachung der
Ziele geschaffen. Das Ziel gilt als erfüllt, wenn der Zielindikator im zuvor von
der Projektgruppe festgelegten Maße erreicht wurde. Durch die Zuordnung der
Zielindikatoren zu den Unterzielen entsteht so ein Zielkatalog, mit dem sich
die Erreichung des Projektziels messen lässt.

Für die Muss-Ziele des Projektes wurde hierbei von der Projektgruppe folgender
Zielkatalog ausgearbeitet:



\begin{description}
	\item[Steigerung der Attraktivität:] Die Zielerreichung wird innerhalb eines
	Beta-Tests festgestellt. Die Tester werden hier nach der von ihnen empfundenen
	Attraktivität der Anwendung gefragt. Das Ziel gilt als erreicht, wenn\ldots
	\item[Abbildung aller Institute:] Die Zielerreichung wird von der Projektgruppe
	festgestellt. Diese bewertet, ob alle Institute in der Anwendung in
	angemessenem Umfang präsentiert werden. Kriterium bei der Bewertung ist hierbei
	die Anzahl der erstellten Fotos je Institut. Das Ziel gilt als erreicht,
	wenn\ldots
	% TODO: Infotexte auch relevant?
	\item[Verbesserung der Präsentation von Studienprojekten:] Die Zielerreichung
	wird innerhalb eines Beta-Tests festgestellt. Die Tester werden hier gefragt,
	ob sie durch die Anwendung auf Studienprojekte aufmerksam gemacht wurden. Das
	Ziel gilt als erreicht, wenn\ldots
	\item[Verbesserung der Präsentation aller Studiengänge:] Die Zielerreichung
	wird von der Projektgruppe festgestellt. Diese bewertet, ob alle Studiengänge
	in der Anwendung in angemessenem Umfang präsentiert werden. Das Ziel gilt als
	erreicht, wenn\ldots
	\item[Aufbau einer Administrationsoberfläche:] Die Zielerreichung wird von der
	Projektgruppe festgestellt. Diese Bewertet, ob eine Administrationsoberfläche
	erstellt wurde, mit der die Pflege der Anwendung möglich ist. Das Ziel gilt als
	erreicht, wenn\ldots
	\item[Entwicklung einer Anwendungsdokumentation:] Die Zielerreichung wird
	innerhalb eines Beta-Tests festgestellt. Testpersonen werden hierbei gebeten
	neue Fotos und zugehörige Informationen mithilfe der Dokumentation in die
	Anwendung einzupflegen. Das Ziel gilt als erreicht, wenn\ldots
	\end{description}
	
	\ldots
	
	\clearpage
	
	\begin{description}
	\item[Verbesserung der Informationsbeschaffung:] Innerhalb eines Betatest, wird
	der Tester nach der Einfachheit der Informationsbeschaffung innerhalb der
	Anwendung gefragt.
	\item[Erstellung unter Einhaltung der Datenschutzrichtlinien:] Wurden alle
	Datenschutzrichtlinien  der Hochschule Osnabrück beachtet? Hierzu werden die
	Vorgaben die in Rücksprache  mit dem Datenschutzbeauftragten der Hochschule
	Osnabrück  Herrn Prof. Dr. Alfred Scheerhorn entstanden sind beachtet.
	\item[Erstellung unter Beachtung der IT-Sicherheit:] Innerhalb des Betatest
	wird das Projekt von Kommilitonen aus dem Studiengang Wirtschaftsinformatik
	einem Penetrationstest unterzogen.
\end{description}