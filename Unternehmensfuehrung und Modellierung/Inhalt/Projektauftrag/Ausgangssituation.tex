\subsection{Ausgangssituation}
\label{sec:Ausgangssituation}

Der Studienstandort Lingen, der der Hochschule Osnabrück zugeordnet ist, wurde
in den Jahren 2007 bis 2013 komplett umgestaltet. Im Zuge dieser Umgestaltung
wurde das stillgelegte Eisenbahnausbesserungswerk der Stadt zu einem modernen
Campusgebäude umgebaut. Die vier Institute der Fakultät Management, Kultur und
Technik, welche zuvor in Lingen an verschiedenen Standorten verteilt
waren, sind nun alle in diesem Campusgebäude untergebracht.

Der neue Campus wirkt sich in vielerlei Hinsicht positiv auf den
Studienstandort Lingen und alle Studierenden aus. Seit der Zentralisierung der
Institute am gemeinsamen Campus, wächst das Zusammengehörigkeitsgefühl zwischen
den Instituten stetig. Die Studierenden der einzelnen Institute sehen sich nun
nicht mehr nur als Mitglied ihres Instituts, sondern als Teil der Hochschule
Osnabrück. Weiterhin bietet der Campus den Studierenden attraktive
Räumlichkeiten, die mit moderner Technik ausgestattet sind. Auch
Aufenthaltsräume für die Studierenden, eine Bibliothek, Labore und eine Mensa
wurden am Standort eingerichtet. Es wurde somit ein attraktiver Studienstandort
mit viel Potenzial geschaffen.