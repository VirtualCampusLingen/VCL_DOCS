\subsection{Projektphasenplanung}
\label{sec:Projektphasenplanung}

Nachdem im vorherigen Abschnitt die Organisation des Projektes erläutert wurde,
wird im Folgenden das Projekt in Phasen gegliedert. Die Phasen dienen dabei
dazu das Projekt in einem geordneten Ablauf zu strukturieren und
Arbeitsschritte aufeinander aufbauend zu vergeben. "`Der Aufbau der
Projektphasen ist darauf ausgerichtet, systematisch und aufeinander aufbauend
zu 'lernen', um insgesamt eine optimale Entscheidungsprozedur zu
durchlaufen"'.
% TODO: Verweis auf Quelle walter2006
Jede Projektphase stellt dabei eine in sich geschlossene Einheit dar, die aus
mehreren Arbeitspaketen und Meilensteinen bestehen kann. Arbeitspakete sind
dabei konkrete Arbeitsschritte, die von einer ausgewählten Person in einem
gewählten Zeitraum zu erfüllen sind. Jedes Arbeitspakete arbeitet auf die
Erfüllung des jeweils nächsten Meilensteines hin.

Meilensteine bilden wichtige, elementare "`Stationen"' in einem Projekt ab. Sie
strukturieren das Projekt durch zeitliches Festlegen von Teilzielen.
% TODO: Verweis auf Quelle schels2008
Darüber hinaus kann an den abgeschlossenen Meilensteinen
der Erfüllungsgrad des Projektzieles abgelesen werden.
% TODO: Verweis auf Quelle schels2008
Jede Projektphase endet daher mit einem Meilenstein. Der Projektphasenplan mit
den jeweiligen abschließenden Meilensteinen ist in nachfolgender Abbildung
dargestellt:

% TODO: Abbildung einfügen

\subsubsection{Analysephase}
\label{sec:Analysephase}

Die Analysephase ist Ausgangspunkt der vorliegenden Projektarbeit. Sie setzt
auf der aktuellen Ist-Situation auf und in ihr werden Lösungsmöglichkeiten für
das vorhandene Problem analysiert. Am Ende der Analysephase steht eine
konzeptionelle Problemlösung. Das bedeutet konkret aus der Analyse der
vorliegenden Situation werden verschiedene Lösungsansätze entwickelt und im
Zuge der Analyse und Auswertung dieser Lösungsansätze wird ein Ansatz
ausgewählt der für die Lösung am geeignetsten ist.
% TODO: Verweis auf Quelle
Im vorliegenden Projekt stellt sich die Ist-Situation, wie bereits im Abschnitt
Projektbegründung aufgezeigt, wie folgt dar:

Der neue Campus der Hochschule Osnabrück am Standort Lingen ist seit dem Jahr
201X fertiggestellt und bietet den Studierenden an moderne Art des Studiums.

Diese Modernisierungen sollen dazu genutzt werden, um den Standort Lingen als
Studienstandort zu vergrößern. In diesem Zuge soll mit diesem Projekt eine
Möglichkeit dazu geschaffen werden Studieninteressierte visuell vom
Hochschulstandort Lingen zu überzeugen. Als Mittel um dieses Ziel zu erreichen
wurde, innerhalb der Analyse der Möglichkeiten, eine Darstellungsform mit Hilfe
von 360-Grad Panoramas  als das geeignetste Mittel festgelegt.
% TODO: Verweis auf Modellierung und Betrieb
Dieser Lösungsansatz beinhaltet die Abbildung des ganzen Campus in Form von
360-Grad Bildern, sowie die Anzeige dieser in einer Weboberfläche mit
Navigationsmöglichkeiten im Stile von Google Street View \copyright. Um diese
Webdarstellung zukunftsorientiert wartbar zu gestalten ist zusätzlich die
Erstellung einer Administrationsoberfläche vorgesehen, mit der es bestimmten
Nutzern möglich ist Panoramafotos und zusätzliche Informationen zu pflegen.

Die Ergebnisse der Analyse werden in einem Lastenheft formuliert. Dieses
Lastenheft schildert dabei welches Problem mit welchen Anforderungen umzusetzen
ist.

Die Analysephase wird mit dem Meilenstein "`Problemlösung definieren und
Lastenheft angefertigt"' abgeschlossen.


\subsubsection{Entwurfsphase}
\label{sec:Entwurfsphase}

Ausgangspunkt der Entwurfsphase ist ein angefertigtes Lastenheft. Aufgrund der
darin beschriebenen Anforderungen kann die Planung der Umsetzung (der Entwurf
der Anwendung) begonnen werden. Die Entwurfsphase beinhaltet dabei Modelle und
Entwürfe (Skizzen, Zeichnungen), die konkret beschreiben, wie die Anforderungen
des Lastenheftes umgesetzt werden. Das Ergebnis der Entwurfsphase ist ein
erstelltes Lastenheft, das die konkretisierte Umsetzung des Problems
beschreibt.

Im vorliegenden Projekt wurden systematisch verschiedene Teilgebiete der
entstehenden Lösung betrachtet und mit Modellen individuell konzipiert, um die
Komplexität einer solchen Lösung in kleinere Teilbereiche zu unterteilen. Dabei
wurden hauptsächlich die Bereiche Datenhaltung bzw. Datenbankdesign und
Anwendungslogik voneinander getrennt betrachtet. Dieser getrennten Betrachtung
ging dabei noch der Entwurf eines ersten Oberflächendesigns, ein sogenanntes
Mockup, voraus. Mit diesen Mockup konnte ein erster Eindruck über Umfang und
Komplexität gewonnen werden und zusätzlich Funktionalität und Benutzung
verdeutlicht werden. Eine Abbildung dieses Mockups ist an Anhang X zu sehen.

% TODO: Anhang einfügen

Durch das Mockup konnten Informationen über zu speichernde Daten und
Datenzusammenhänge dargestellt werden. Auf dieser Basis wurde ein Modell der
Datenbank angefertigt. Eine Abbildung dieses Datenbankmodells kann in Anhang X
% TODO: Anhang einfügen
gesehen werden. Die hier dargestellte Abbildung beschreibt ein sogenanntes
Entity-Relationship-Modell (ERM), das in einer einheitlichen Notation
Datensatzattribute und -beziehungen darstellt. Der Prozess des
Datenbankentwurfes kann in der Ausarbeitung Modellierung und Betrieb in Kapitel
X nachgelesen werden.
% TODO: Verweis auf Modellierung und Betrieb

Neben der Erstellung eines konzeptionellen Datenbankentwurfes wurde ein Entwurf
der Anwendung erstellt, die die Lösung des beschriebenen Problems darstellt.
Auf dieser Ebene wurde vor allem die Modellierungsart der
Anwendungsfalldiagramme (engl. use case diagram) genutzt, um typische und
besondere Anwendungsfälle eines Benutzers mit den Aufrufen der entsprechenden
Funktionen aufzuzeigen. Durch diese Technik konnte die notwendigen Funktionen,
die für die Anwendung notwendig sind herausgestellt werden. Ein grober
Überblick über Funktionsumfang, Zeitaufwand, und Komplexität konnte so erstellt
werden. Der Prozess des Anwendungsentwurfs kann in der Ausarbeitung
Modellierung und Betrieb nachgelesen werden.
% TODO: Verweis auf Modellierung und Betrieb
Das konstruierte Anwendungsfalldiagramm ist im Anhang X dargestellt.
% TODO: Anhang einfügen

Sowohl der Datenbankentwurf, als auch der Anwendungsentwurf stellen im Sinne
der Trennung der Bereiche nur ein bestimmten Aspekt der zu erstellenden Lösung
dar. Daher wurde aufbauend auf diese Entwürfe ein Architekturentwurf
angefertigt, der die zu erstellende Lösung umfassend darstellt. Dieser
Architekturentwurf beinhaltet dabei vor allem die Kommunikation zwischen den
Bereichen Datenhaltung und Anwendung und stellt darüber hinaus wichtige
Schnittstellen innerhalb der Anwendung dar. Dieser Architekturentwurf ist in
Anhang X dargestellt.
% TODO: Anhang einfügen

Die Entwurfsphase wurde mit dem Meilenstein "`Pflichtenheft abgeschlossen und
Architekturplan erstellt"' abgeschlossen.


\subsubsection{Implementierungsphase}
\label{sec:Implementierungsphase}

Im Anschluss auf die Entwurfsphase und aufbauend auf einem abgeschlossenen
Pflichtenheft folgt die Implementierungsphase. Diese Phase ist sowohl die
zeitlich größte Phase, als auch die entscheidende, da in dieser Phase alle im
Vorfeld geplanten Schritte und Modelle umgesetzt werden. In der Phase der
Implementierung werden aufbauend auf den vorher angefertigten Entwürfen, die
theoretischen Modelle in ausführbaren Quellcode umgesetzt. In dieser Phase
zeigt sich vor allem die Qualität der angefertigten Modelle, denn am
schnellsten lässt sich Software entwickeln, wenn man das dazugehörige Modell
1:1 in Quellcode übertragen kann. Allerdings ist das, aufgrund der
Schwierigkeit eine Software in seiner Komplexität in einem Modell zu erfassen,
häufig schwer zu erreichen. Bei unerwartetem Abweichen vom Modell muss man
daher, die Konzeption der Software erneut überdenken, um eventuellen logischen
Fehlern vorzubeugen. Der Prozess dieser Implementierung kann in der
Ausarbeitung Modellierung und Betrieb in Kapitel X  nachgelesen werden.
% TODO: Verweis auf Modellierung und Betrieb

Die Implementierungsphase endet mit dem Meilenstein "`Anwendung voll
funktionsfähig fertiggestellt"'.

\subsubsection{Testphase}
\label{sec:Testphase}

Im Anschluss an eine fertiggestellte funktionierende Software kann und muss
diese getestet werden. Idealerweise durch Personen, die nicht im Prozess der
Softwareentwicklung beteiligt waren. Denn diese Personen haben eine
unbeeinflusste Sicht auf die Software und können so zum einen dessen Intuitive
Bedienung besser einschätzen und zum anderen benutzen sie die Software auch
unvoreingenommen, was dazu führen kann, das sie Fehler in der Bedienung oder im
Programm entdecken. Einen solchen Test, der von außenstehenden Personen
durchgeführt wird, nennt man Betatest. Diesem Betatest ist im Idealfall ein
Alphatest der Entwickler vorausgegangen, in dem sie die entwickelten
Anwendungsfälle (use cases) im fertigen Programm simulieren und dabei mögliche
Fehler entdecken und ausbessern wurden.

In der Betatestphase ist es dann besonders wichtig die  aufgetretenen Fehler
und Kritiken zu erfassen und an die Entwickler weiterzugeben. Durch den
Betatest können sogenannte "`Kinderkrankheiten"' früh erkannt und beseitigt
werden. Das steigert die Qualität und erhöht vor allem die
Benutzerzufriedenheit beim verwenden der Software.

Die Testphase ist mit dem Meilenstein "`Betatestergebnisse umgesetzt und
Betatestfehler behoben"' abgeschlossen.

\subsubsection{Einführungsphase}
\label{sec:Einführungsphase}

Nach erfolgreicher Fertigstellung der Software und Korrektur aller
aufgetretenen Fehler kann die Software als Problemlösung in der vorgesehen
Umgebung produktiv eingesetzt werden. Der Abschluss dieser Phase stellt damit
den Abschluss des Projektes dar. Besonders zu beachten ist in dieser Phase,
dass es zu Kompatibilitätsproblemen zwischen der produktiv Umgebung und der
Entwicklungsumgebung kommen kann. Von diesem Problem sind vor allem Elemente
der Anwendung betroffen die von Drittanbietern bezogen wurden, zum Beispiel
eine bestimmte Programmbibliothek die in der produktiv Umgebung in einer
anderen (möglicherweise veralteten) Version vorliegt, als in der
Entwicklungsumgebung. Solche Komplikationen können den Abschluss eines
Projektes entscheidend beeinflussen und herauszögern, daher ist bereits in der
Analysephase das Zielsystem ein wichtiges Element.

Die Einführungsphase endet mit dem letzten Meilenstein "`Anwendung produktiv
eingesetzt"'.

% TODO: Produktivsystem in Analysephase berücksichtigen