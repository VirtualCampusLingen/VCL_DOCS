\subsection{Projektorganisation}
\label{sec:Projektorganisation}

\subsubsection{Möglichkeiten der Projektorganisation}
\label{sec:MoeglichkeitenProjektorganisation}

% Scrum ist eine Methode der agilen Softwareentwicklung. Konkret stellt Scrum eine
% agile Projektmanagementmethode dar. Eine solche Managementmethode behandelt
% nicht die konkrete Struktur und Art der Zusammenarbeit im Team, sondern
% beschränkt sich auf den Ablauf des Projektes. Bei einer Projektorganisation nach
% der Scrum-Methode lassen sich grundsätzlich jeweils drei verschiedene Rollen,
% Zeremonien und Artefakte unterscheiden. Diese sollen im Folgenden näher
% beschrieben werden. Die drei zentralen Rollen der Scrum-Methode sind hierbei\ldots

\subsubsection{Planung der Projektorganisation}
\label{sec:PlanungProjektorganisation}

\subsubsection{Umsetzung der Projektorganisation}
\label{sec:UmsetzungProjektorganisation}

Für eine erfolgreiche Durchführung des Projektes ist es unabdingbar sich vorher
über die Projektorganisation im Klaren zu sein. Bei der Projektorganisation
handelt es sich hierbei darum, wie die Entwicklung eines Projektes vonstatten
gehen soll. Dies bedeutet, dass festgelegt wird, welche Entwicklungsmethodiken
angewandt werden oder welche Hilfsmittel und Zusatztools eingesetzt werden, um
das gewünschte Ziel des Projektes möglichst effizient zu erreichen. Ebenfalls
umfasst dieser Aspekt den Punkt der Kommunikation und Interaktion des
Projektteams und der einzelnen Projektmitglieder untereinander. Hinzu kommt,
dass festgelegt wird, wie die einzelnen Meetings der Projektgruppe vonstatten
gehen und welche Bedingungen erfüllt werden müssen.

Bei dem großen Umfang und der Vielfalt dieses Projektes ist es notwendig
flexibel zu sein und somit auf Probleme oder Änderungen und Wünsche des
Auftraggebers zeitnah reagieren und umsetzen zu können. Um somit die
Entwicklung agil zu gestalten, entschied man sich in diesem Projekt für die
Scrum-Methode. Dies ist eine Methodik der agilen Softwareentwicklung und sorgt
mit sogenannten Scrum-Meetings für regelmäßige Treffen der Projektmitglieder.
Diese Scrum-Meetings werden immer in gleichen Abständen nach sogenannten
Sprints gehalten. Die Sprints können hierbei 1 bis 30 Tage lang sein und
stellen einzelne Iterationsschritte in der Entwicklung der Anwendung dar. Nach
jeweils einem Sprint, entsteht eine weitere lauffähige Anwendung, aber um die
Funktionen des letzten Sprints erweitert. Somit wird gewährleistet, dass immer
eine lauffähige Version zur Verfügung steht und dem Auftraggeber vorgelegt
werden kann. Nach Ablauf eines Sprints und des darauffolgenden
Scrum-Meetings,werden weitere Arbeitspakete und Aufgaben an die einzelnen
Projektmitglieder verteilt, welche in dem folgenden Sprint erledigt werden
müssen. Durch die kurzen Sprints und häufigen Meetings können so Meinungen,
Ideen und auch Probleme bei der Umsetzung unter den Projektmitgliedern
diskutiert und beseitigt werden. Hierbei ist jedes Mitglied gleichberechtigt
involviert. Trotz dessen können bei der Scrum-Methode folgende drei Rollen
unterschieden werden:

\begin{description}
\item[Der Product Owner] stellt einen Vertreter für den Endkunden dar und
vertritt somit dessen Wünsche und Bedürfnisse hinsichtlich der Anwendung. Der
Product Owner trifft somit auch die Entscheidungen bzgl. Kosten und weiterer
gewünschter Änderungen oder Vorgaben. Die Ergebnisse werden ebenfalls von
diesem überprüft.
\item[Der Scrum-Master] dient als Schnittstelle zwischen dem Product Owner und
dem Scrum-Team und fördert die Zusammenarbeit. Ebenfalls sorgt er dafür, dass
das Scrum-Team nach den Regeln der Scrum-Methode arbeiten.
\item[Das Scrum-Team] ist für die Entwicklung und Implementierung der
gewünschten Anwendung zuständig. Es handelt eigenständig und organisiert somit
sich und die Vorgehensweise in großem Maße selbst.
\end{description}

Ebenfalls zu erwähnen ist, dass ein Product-Backlog ein vom Endkunden
definierter Katalog ist, welcher die Anforderungen des Kunden , nach Priorität
und Wichtigkeit sortiert, enthält. Die Anforderungen des Product-Backlog werden
dabei in ein sogenanntes Sprint-Backlog übertragen, welches bei den Sprints als
Vorlage für die zu erfüllenden Aufgaben dient. Nach jedem Sprint werden die
erzielten Ergebnisse, unabhängig von der erreichten Vorgaben, in einem Sprint
Review dokumentiert und an den Product Owner zur Einsicht übergeben.

In folgender Abbildung wird der Scrum-Prozess nochmals visuell dargestellt:
% TODO: Abbildung einfügen

Nach einer kurzen Einführung in die agile Softwareentwicklung kann nun gesagt
werden, dass das Projektteam in diesem Projekt agil entwickelt hat. Dazu wurde
eine Sprintdauer von einer Woche angesetzt. Nach dieser Woche fand immer ein
Scrum-Meeting statt, in der die Erkenntnisse und Ergebnisse der einzelnen
Projektmitglieder reflektiert wurden. Für die einzelnen Scrum-Meetings sind
ebenfalls Gesprächsprotokolle erstellt worden, auf die im Nachhinein
zugegriffen werden kann (siehe Anhang). Bei besonders komplexen Problemen
während der Entwicklung der Anwendung, wurde auf die Methode des Pair
Programming zurückgegriffen. Dies bedeutet, dass mehrere Mitglieder
gleichzeitig und zusammen an einem Rechner an einem Implementierungsproblem
arbeiten. Dadurch konnten komplexe Probleme schneller und effizienter beseitigt
werden, da sich die Teammitglieder während des Arbeitens direkt austauschen
können. Zum Schluss jedes Scrum-Meetings wurden ebenfalls neue Arbeitspakete an
die Gruppenmitglieder verteilt.

Die Verwaltung der Arbeitspakete wurde dabei durch ein zusätzliches Tool
realisiert, welches sich PHProjekt nennt. Dieses Tool ist ein webbasiertes
Ticketsystem, welches es ermöglicht Arbeitspakete einzelnen Gruppenmitgliedern
zuzuweisen und deren einzelne Status anzeigen zu lassen. Ebenfalls ist es hier
möglich Prioritäten und Dauer zu definieren, um die wichtigsten Arbeitspakete
zu kennzeichnen und eine Frist zur Erledigung zu setzen.

Um die einzelnen Programmversionen, welche nach den Sprints entstehen, zu
verwalten, wurde die Versionsverwaltung Github eingesetzt, welche ebenfalls
webbasiert arbeitet. Hiermit ist es möglich Abspaltungen von einer bestehenden
Programmversion zu erzeugen und diese zu bearbeiten und nach Abschluss wieder
der Stammversion hinzuzufügen. So kann gewährleistet werden, dass immer eine
funktionsfähige Version zur Verfügung steht, da isoliert von der Stammversion
gearbeitet und entwickelt wird.