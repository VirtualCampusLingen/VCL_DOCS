\section{Theoretische Grundlagen}
\label{sec:Grundlagen}

\subsection{Stakeholderanalyse}
\label{sec:Stakeholderanalyse}

\subsection{Scrum}
\label{sec:Scrum}

Scrum ist eine Methode der agilen Softwareentwicklung. Konkret stellt Scrum eine
agile Projektmanagementmethode dar. Eine solche Managementmethode behandelt
nicht die konkrete Struktur und Art der Zusammenarbeit im Team, sondern
beschränkt sich auf den Ablauf des Projektes. Bei einer Projektorganisation nach
der Scrum-Methode lassen sich grundsätzlich jeweils drei verschiedene Rollen,
Zeremonien und Artefakte unterscheiden. Diese sollen im Folgenden näher
beschrieben werden. Die drei zentralen Rollen der Scrum-Methode sind hierbei:

\begin{description}
\item[Product Owner]
\item[Team]
\item[Scrum Master]
\end{description}


\subsection{Zielsystem}
\label{sec:Zielsystem}

\subsection{Agile Softwareentwicklung}
\label{sec:AgileSoftwareentwicklung}

\subsection{Projektstrukturplan}
\label{sec:Projektstrukturplan}