\section{Ausgangssituation}
\label{sec:Ausgangssituation}

Der Studienstandort Lingen, der seit 20XX der Hochschule Osnabrück zugeordnet
ist, wurde in den Jahren 2007 bis 2013 komplett ungestaltet. Im Zuge dieser
Umgestaltung wurde das stillgelegte Eisenbahnausbesserungswerk der Stadt zu
einem modernen Campusgebäude umgebaut. Die vier Institute der Fakultät
Management, Kultur und Technik, welche zuvor in ganz Lingen an verschiedenen
Standorten verteilt waren, sind nun alle in diesem Campusgebäude untergebracht.
% TODO: Jahr ergänzen 

Der neue Campus wirkt sich in vielerlei Hinsicht positiv auf den
Studienstandort Lingen und alle Studierenden aus. Seit der Zentralisierung der
Institute am gemeinsamen Campus wächst das Zusammengehörigkeitsgefühl zwischen
den Instituten stetig. Die Studierenden der einzelnen Institute sehen sich nun
nicht mehr nur als Mitglied ihres Instituts, sondern als Teil der Hochschule
Osnabrück. Weiterhin bietet der Campus den Studierenden attraktive
Räumlichkeiten, die mit moderner Technik ausgestattet sind. Auch Aufenthalsräume
für die Studierenden, eine Bibiliothek, Labore und eine Mensa wurden am Standort
eingerichtet. Es wurde somit ein attraktiver Studienstandort mit viel Potenzial
geschaffen.