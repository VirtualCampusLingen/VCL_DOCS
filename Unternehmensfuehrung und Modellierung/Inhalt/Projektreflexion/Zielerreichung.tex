\subsection{Zielerreichung}
\label{sec:Zielerreichung}

Die zu erreicheneden Ziele des Projekts wurden in \verweis{Zielkatalog} aufgestellt und erläutert. Im Folgenden
wird ermittelt, ob und in welchem Ausmaß diese Ziele erreicht werden konnten.
Aus den erreichten Zielen kann darauf folgend der Nutzen der Anwendung für die Auftraggeber verifiziert werden.

Die Verifizierung der Zielerreichung wird dabei in zwei Teilschritten vollzogen. Im ersten Schritt werden alle
Ziele untersucht, deren Erfolg von der Projektgruppe selbst verifiziert werden kann. Im Anschluss daran werden
die Teilziele untersucht, deren Erfolg durch den Fragebogen im Betatest verifiziert wird.

Die folgende Aufstellung listet im ersten Schritt alle Ziele mit dem entsprechenden Grad der Zielerreichung auf,
die von der Projektgruppe selbst verifiziert werden können.

\begin{table}[h]
\centering
\begin{tabular}{ccccl}
\hline
\multicolumn{1}{l}{}                 & Zielkategorie & Zielindikator & Grad der Zielerreichung &              \\ \hline
Abbildung aller Institute                 & Muss-Ziel     & Anzahl Bilder pro Institut > 10   & 100\%   \\ \hline
Aufbau einer Administrationsoberfläche    & Muss-Ziel     & Möglchkeit der Pflege             & 100\%   \\ \hline
Entwicklung einer Anwenderdokumentation   & Muss-Ziel     & Anwenderdokumentation vorhanden   & 100\%   \\ \hline
Einhaltung der Datenschutzrichtlinien     & Muss-Ziel     & Keine Abweichungen                & 100\%   \\ \hline
Einhaltung der IT-Sicherheit              & Muss-Ziel     & Kein Eindringen möglich           & 100\%   \\ \hline
Alleinstellungsmerkmal erzielen           & Soll-Ziel     & Vergleich zu anderen Hochschulen  & 100\%   \\ \hline
Geringe Betriebskosten erzielen           & Soll-Ziel     & Hostinggebühren < 20€             & 100\%   \\ \hline
Geringe Entwicklungskosten erzielen       & Soll-Ziel     & tatsächliche Entwicklungskosten < 100€  & 100\%   \\ \hline
\end{tabular}
\caption{Zielerreichung der selbstverifizierten Ziele}%
\label{tab:Zielerreichung1}%
\end{table}

Aus der dargestellten Auflistung ist zu entnehmen, dass im ersten Schritt alle Ziele erreicht werden konnten.
Dem Auftraggeber kann damit verifizert werden, dass die entwickelte Anwendung ein zukunftsorientertes Projekt mit
Alleinstellungscharakter ist.

In einem zweiten Schritt verifizieren die Antworten der Betatester die Erreichung weiterer Ziele:


% \begin{table}[h]
% \centering
% \begin{tabular}{ccccl}
% \hline
% \multicolumn{1}{l}{}                 & Zielkategorie    & Zielindikator & erreichter Wert & Grad der Zielerreichung &              \\ \hline
% Steigerung der Attraktivität              & Muss-Ziel   & Frage Nr. XX mind 3,0                   & 100\%   \\ \hline
% Aufbau einer Administrationsoberfläche    & Muss-Ziel     & Möglchkeit der Pflege             & 100\%   \\ \hline
% Entwicklung einer Anwenderdokumentation   & Muss-Ziel     & Anwenderdokumentation vorhanden   & 100\%   \\ \hline
% Einhaltung der Datenschutzrichtlinien     & Muss-Ziel     & Keine Abweichungen                & 100\%   \\ \hline
% Einhaltung der IT-Sicherheit              & Muss-Ziel     & Kein Eindringen möglich           & 100\%   \\ \hline
% Alleinstellungsmerkmal erzielen           & Soll-Ziel     & Vergleich zu anderen Hochschulen  & 100\%   \\ \hline
% Geringe Betriebskosten erzielen           & Soll-Ziel     & Hostinggebühren < 20€             & 100\%   \\ \hline
% Geringe Entwicklungskosten erzielen       & Soll-Ziel     & tatsächliche Entwicklungskosten < 100€  & 100\%   \\ \hline
% \end{tabular}
% \caption{Zielerreichung der selbstverifizierten Ziele}%
% \label{tab:Zielerreichung1}%
% \end{table}


