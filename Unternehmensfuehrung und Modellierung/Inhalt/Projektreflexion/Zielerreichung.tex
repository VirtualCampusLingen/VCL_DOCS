\subsection{Zielerreichung}
\label{sec:Zielerreichung}

Die zu erreicheneden Ziele des Projekts wurden in \verweis{Zielkatalog} aufgestellt und erläutert. Im Folgenden
wird ermittelt, ob und in welchem Ausmaß diese Ziele erreicht werden konnten.
Aus den erreichten Zielen kann darauf folgend der Nutzen der Anwendung für die Auftraggeber verifiziert werden.

Die Verifizierung der Zielerreichung wird dabei in zwei Teilschritten vollzogen. Im ersten Schritt werden alle
Ziele untersucht, deren Erfolg von der Projektgruppe selbst verifiziert werden kann. Im Anschluss daran werden
die Teilziele untersucht, deren Erfolg durch den Fragebogen im Betatest verifiziert wird.

Die folgende Aufstellung listet im ersten Schritt alle Ziele mit dem entsprechenden Grad der Zielerreichung auf,
die von der Projektgruppe selbst verifiziert werden können.

\begin{table}[h]
\centering
\begin{tabular}{ccccl}
\hline
\multicolumn{1}{l}{}                 & Zielkategorie  & Zielindikator & Ist-Wert & Grad der Zielerreichung \\ \hline
Abbildung aller Institute                 & Muss-Ziel & Anzahl Bilder pro Institut > 10   & ~20 & 100\%   \\ \hline
Aufbau einer Administrationsoberfläche    & Muss-Ziel & Möglchkeit der Pflege             & gegeben & 100\%   \\ \hline
Entwicklung einer Anwenderdokumentation   & Muss-Ziel & Anwenderdokumentation vorhanden   & erreicht & 100\%   \\ \hline
Einhaltung der Datenschutzrichtlinien     & Muss-Ziel & Keine Abweichungen                & erreicht & 100\%   \\ \hline
Einhaltung der IT-Sicherheit              & Muss-Ziel & Kein Eindringen möglich           & erreicht & 100\%   \\ \hline
Alleinstellungsmerkmal erzielen           & Soll-Ziel & Vergleich zu anderen Hochschulen  & erreicht & 100\%   \\ \hline
Geringe Betriebskosten erzielen           & Soll-Ziel & Hostinggebühren < 20€             & 0€ & 100\%   \\ \hline
Geringe Entwicklungskosten erzielen       & Soll-Ziel & tatsächliche Entwicklungskosten < 100€ & 0€ & 100\%   \\ \hline
\end{tabular}
\caption{Zielerreichung der selbstverifizierten Ziele}%
\label{tab:Zielerreichung1}%
\end{table}

Aus der dargestellten Auflistung ist zu entnehmen, dass im ersten Schritt alle Ziele erreicht werden konnten.
Dem Auftraggeber kann damit verifizert werden, dass die entwickelte Anwendung ein zukunftsorientertes Projekt mit
Alleinstellungscharakter ist.

In einem zweiten Schritt verifizieren die Antworten der Betatester die Erreichung weiterer Ziele:


\begin{table}[h]
\centering
\begin{tabular}{cccccccl}
\hline
\multicolumn{1}{l}{}              & Zielkategorie  & Zielindikator  & erreichter Umfragewert \\ \hline

vermittelte Attraktivität         & Muss-Ziel      & mind 3,0       & 1,78                    \\ \hline
Präsentation der Studiengänge     & Muss-Ziel      & mind 3,0       & 3,00                    \\ \hline
Präsentation der Studienprojekte  & Muss-Ziel      & mind 3,0       & 2,79                    \\ \hline
Art der Informationsbeschaffung   & Muss-Ziel      & mind 3,0       & 2,83                    \\ \hline
Design der Anwendung              & Soll-Ziel      & mind 3,0       & 1,91                    \\ \hline
Auftritt des Campus Lingen        & Soll-Ziel      & mind 3,0       & 1,74                    \\ \hline

Subjektive Bewertung              & Optional       & mind 3,0       & 2,18                    \\ \hline
\end{tabular}
\caption{Auswertung des Betatests}%
\label{tab:Fragebogen}%
\end{table}

Die vorangegange Tabelle zeigt, dass die festgelegten Ziele anhand des durchgeführten 
Betatests verifiziert werden konnten.
Der Zielindikator richtet sich dabei nach dem Schulnotensystem von "`1 = sehr gut"' bis "`6 = sehr schlecht"'.
Der erreichte Umfragewert wird dabei errechnet, indem die jeweilige Schulnote mit 
der Antworthäufigkeit multipliziert und über alle Produkte addiert wird.

Es ist dargestellt, dass alle Ziele erreicht werden. Hierraus ergibt sich, dass das Projekt 
den gewünschten Nutzen erfüllt.
Zudem zeigt sich auch in dieser Darstellung, dass die Anwendnug von den Betatestern
auf positive Resonanz gestoßen ist.