\subsection{Zeitmanagement}
\label{sec:Zeitmanagement}

Aus dem \verweis{TerminplanungUndZeitManagement} ist zu entnehmen, dass 760 Stunden Arbeitsaufwand für das Projekt
eingeplant wurden. Das Gantt-Diagramm in Abbildung XX zeigt, dass tatsächlich
ein Aufwand von 987 Stunden enstanden ist.
% TODO: Abbildung XX fürs Gantt Diagramm richtig einfügen.
Die Differenz dieses Arbeitsaufwandes wird in \tabelle{SollIstVergleichZeit} differenzierter dargestellt.

\tabelleEinfg{Soll-Ist-Vergleich der Zeitplanung}{tab:SollIstVergleichZeit}{ZeitVergleich}


Der Mehraufwand von 227 Stunden ist vorallem durch Abweichungen in der Implementierung des Administrationsbereiches
enstanden. Dieser Teilbereich des Projektes verursachte einen Mehraufwand von
171 Stunden, der vor allem darauf zurückzuführen ist, dass die Implementierungslogik in diesem Bereich oft überarbeitet werden musste.

Darüber hinaus ist zu erkennen, dass in Relation zu dem geplanten Aufwand die Aufbereitung der Fotos die Größte
prozentuale Differenz aufweist. Der Aufwand dieses Arbeitsbereiches wurde zu Beginn des Projektes unterschätzt.
Der Fotobearbeitungsprozess wurde innerhalb der Projektentwicklung ständig verfeinert und stellt am Ende des Projekts
einen komplexen eigenständigen Prozess dar. Die Vertiefung dieser Prozesskomplexität würde an dieser Stelle zu weit
führen, ist aber in \citet{modelierungUndBetrieb2014} im Kapitel 5.1 (Panoramaerstellung) dokumentiert.