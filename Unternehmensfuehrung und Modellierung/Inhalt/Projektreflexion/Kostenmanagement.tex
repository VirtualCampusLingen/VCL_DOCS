\subsection{Kostenmanagement}
\label{sec:Kostenmanagement}

% Was war Kostentechnisch geplant?
% Was ist der tatsächlich ist Zustand
% Woran kann man das festmachen?
% Warum ist das so?

Die Kostenplanung in \verweis{Kostenplanung} zeigt, dass keine Kosten für die Entwicklung des vorliegenden Projektes
geplant waren. Im Verlauf des Projektes konnte wie geplant das gesamte Equipment selbst gestellt oder geliehen werden,
sodass keine realen Kosten im Projekt angefallen sind. Der Kostenplan wurde eingehalten.

Im vorherigen \verweis{Zeitmanagement} wurde aber bereits aufgezeigt, dass der Zeitaufwand nicht der Planung entsprach und
dieser Zeitaufwand beeinflusst das Kostenmanagement. Praktisch fallen zwar keine Personalkosten für das vorliegende Projekt
an, aber sie werden aus zu statistischen Zwecken für den Auftragegeber und zur
Wertbestimmung des Projekt trotzdem berücksichtigt. Im vorherigen Abschnitt wurde ein zeitlicher Mehraufwand von \textbf{227} Stunden festgestellt. Dieser
wird mit einem statistischen Kostensatz von 60€/Stunde berechnet. Daraus ergibt sich folgende dargestellte
Auswirkung auf das Kostenmanagement:

\begin{table}[h]
\centering
\begin{tabular}{ccccl}
\hline
\multicolumn{1}{l}{}              & Plan {[}€{]} & Ist {[}€{]} & Differenz {[}€{]} & Differenz {[}\%{]} \\ \hline
Anschaffungen \& Lizenzen          & 0           & 0          & 0               & 0,00\%          \\ \hline
Personalkosten                    & 45.600      &  59.220    & +13.620          & +29,87\%         \\ \hline
                                  & 45.600      & 59.220     & +13.620          & +29,87\%          \\ \hline
\end{tabular}
\caption{Soll-Ist-Vergleich der Kostenplanung}%
\label{tab:SollIstVergleichKosten}%
\end{table}

Es ergeben sich statistische personelle Mehrkosten von \textbf{13.620€}. Der über die Kosten definierte Projektwert
beläuft sich damit auf \textbf{63.037€}

% TODO: Woher kommen die 2.203€ in der nachfolgenden Tabelle?? Im Fließtext beschreiben
\begin{table}[h]
\centering
\begin{tabular}{ccccl}
\hline
\multicolumn{1}{l}{}            & Kosten {[}€{]}  & \\ \hline
Entwicklungskosten              & 59.220          & \\ \hline
Anschaffungenkosten             & 2.203           & \\ \hline
Projektwert gemessen an Kosten  & 63.037          & \\ \hline
laufende Kosten                 & 0               & \\ \hline
Gesamte laufende Kosten         & 0               & \\ \hline
\end{tabular}
\caption{Kostenübersicht}%
\label{tab:Kostenuebersicht}%
\end{table}