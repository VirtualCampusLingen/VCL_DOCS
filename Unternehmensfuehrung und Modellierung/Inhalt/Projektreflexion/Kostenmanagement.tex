\subsection{Kostenmanagement}
\label{sec:Kostenmanagement}

% Was war Kostentechnisch geplant?
% Was ist der tatsächlich ist Zustand
% Woran kann man das festmachen?
% Warum ist das so?

Die Kostenplanung in \verweis{Kostenplanung} zeigt, dass keine Kosten für die Entwicklung des vorliegenden Projektes
geplant waren. Im Verlauf des Projektes konnte wie geplant das gesamte Equipment selbst gestellt oder geliehen werden,
sodass keine realen Kosten im Projekt angefallen sind. Der Kostenplan wurde eingehalten.

Im vorherigen \verweis{Zeitmanagement} wurde aber bereits aufgezeigt, dass der Zeitaufwand nicht der Planung entsprach und
dieser Zeitaufwand beeinflusst das Kostenmanagement. Praktisch fallen zwar keine Personalkosten für das vorliegende Projekt
an, werden aber zu statistischen Zwecken für den Auftragegeber und zur
Wertbestimmung des Projekt trotzdem berücksichtigt. Im vorherigen Abschnitt wurde ein zeitlicher Mehraufwand von \textbf{227} Stunden festgestellt. Dieser
wird mit einem statistischen Kostensatz von 60€/Stunde berechnet. Daraus ergibt sich folgende dargestellte
Auswirkung auf das Kostenmanagement:

\tabelleEinfg{Soll-Ist-Vergleich der Kostenplanung}{tab:SollIstVergleichKosten}{KostenVergleich}

Es ergeben sich statistische personelle Mehrkosten von \textbf{13.620€}. Die Kosten
für die Weiterführung des Projektes belaufen sich auf \textbf{2.763}. Der abschließende,
über die Kosten definierte, Projektwert beläuft sich damit auf \textbf{61.983€}.

Es müssten damit nur \textbf{4,46\%} des Projektwertes vom Auftraggeber investiert werden,
um das Projekt zu erhalten und die Ziele dieses zu nutzen.