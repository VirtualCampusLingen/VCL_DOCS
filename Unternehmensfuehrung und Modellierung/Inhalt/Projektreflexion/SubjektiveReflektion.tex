\subsection{Subjektive Reflektion}
\label{sec:SubjektiveReflekion}

Die Reflexion des Projektes wird an dieser Stelle mit 
einer subjektiven Reflexion aus Sicht der Projektgruppe abgeschlossen. Ziel ist
es, subjektiv den Verlauf des Projektes mit negativen und positiven Eindrücken
wiederzugeben.

An erster Stelle kann reflektiert werden, dass das geplante Zeitmanagement
des Projektes stark überzogen wurde. Auch die angesetzten Sollstunden von
150 Stunden pro Person summierten sich am Ende zu knapp 250 Stunden pro Person auf.
Grund für diesen Mehraufwand ist zum einen der unterschätzte Aufwand der 
Fotoerstellung und zum anderen eine notwendige Neustrukturierung im 
Administrationsbereich.

Die Aufnahme und Bearbeitung der Fotos hat sich dabei innerhalb des
Projektverlaufs zu einem komplexen Prozess entwickelt, der in
\citet{modelierungUndBetrieb2014} nachgelesen werden kann. Der
positive Effekt dieser Entwicklung ist, dass die
Qualität der Fotos stark gesteigert werden konnte.
Die Neustrukturierung des Administrationsbereiches ist dagegen bedingt durch eine stärkere
Integration der Google Street View \acs{API}. Auch diese Entwicklung konnte am Anfang des Projektes
nicht eingeplant werden, führte aber wiederum zu einer Steigerung der Qualität der Software.

Positiv kann im Rückblick des Projektes an erster Stelle die Kommunikation
innerhalb der Projektgruppe reflektiert werden. Durch eine stark feedbackgetriebene
Entwicklung konnten viele Fehler früh erkannt und beseitigt werden.
Darüber hinaus stimmen alle Projektmitglieder darüber überein, dass mit diesem
Projekt ein sehr gutes Produkt entstanden ist, das auch seitens der Hochschule
sehr positiv aufgenommen wurde. Vor allem ist an dieser Stelle der engagierte Einsatz
vieler Mitarbeiter der Hochschule positiv zu erwähnen. Viele Einblicke wurden der
Projektgruppe durch engagierte Mitarbeiter eröffnet, die in den entstandenen Panoramafotos festgehalten sind.
Besonders der Einsatz der Projektpaten Herrn Stephan Feldker und Herrn Prof. Dr. Westerbusch 
ist der Projektgruppe positiv aufgefallen.
Viele Anfragen und Anliegen wurden von ihnen auf dem kurzen Dienstweg zu Gunsten der Projektgruppe
erledigt. Darüber hinaus hat sich Herr Feldker mit vielen Ideen und hilfreichem Feedback stark in das Projekt
eingebracht.