\subsection{Grobkonzept}
\label{sec:Grobkonzept}

Das Ziel einer möglichst ansprechenden, visuellen Darstellung des
Studienstandorts Lingen und des neuen Campus unter Verwendung moderner
Webtechnologien lässt sich auf verschiedenste Weisen realisieren. 

In einem Ideenfindungsprozess hat die Projektgruppe für die Erreichung der
Projektziele geeignete Umsetzungsmöglichkeiten identifiziert. Mit der
Brainstorming-Methode werden zunächst sämtliche Ideen zur Realisierung der
Projektidee gesammelt. Die Ideensammlung, die auf diese Weise entstand, wird
anschließend gefiltert und sortiert. Die einzelnen Ideen werden zu Clustern, wie
beispielsweise "`Anforderungen aus dem Zielsystem"', "`Technologien"' und
"`Funktionalität des Systems"' zusammengefasst. Anhand dieser strukturierten
Ideensammlung werden dann konkrete Möglichkeiten zur Umsetzung der Projektidee
herausgearbeitet. Nach einer ersten Filterung der Umsetzungsmöglichkeiten hat
sich die Projektgruppe auf die folgenden drei Realiserungsalternativen geeinigt. 

\begin{description}
\item[Fotogalerie] \hfill \\
Innerhalb einer Fotogalerie soll dem Anwender der modernisierte Studienstandort
Lingen näher gebracht werden. Zu den Fotos könnten entsprechende
Informationstexte, Beschreibungen und weiterführende Links für
Studieninteressierte hinterlegt werden.

\item[Virtueller Rundgang im Stile von Google Street View] \hfill \\
Der Anwender soll interaktiv durch Panoramafotos von den Räumlichkeiten des
Campus navigieren können. Auch bei dieser Möglichkeit der Projektrealisierung
könnten dem Nutzer weiterführende Informationen zu den abgebildeten Inhalten
präsentiert werden.

\item[Virtueller Rundgang durch ein 3D-Modell] \hfill \\
Bei dieser Möglichkeit der Umsetzug soll ein 3D-Modell des gesamten
Campusgebäudes erstellt werden. Der Nutzer könnte dann virtuell durch die
modellierten Räumlichkeiten des Campus navigieren und auf diese Weise einen
Eindruck vom Studienstandort erlangen.
\end{description}

Hierbei unterscheiden sich die einzelnen Alternativen in Innovativität,
Attraktivität für den Anwender, Komplexität der Umsetzung und Grad der Zielerreichung voneinander.
Die folgende \tabelle{AlternativenVergleich} stellt die drei Varianten
hinsichtlicher der vier genannten Kritierien gegenüber.

% \begin{table}[h]
% \centering
% 
% \caption{Soll-Ist-Vergleich der Kostenplanung}%
% \label{tab:AlternativenVergleich}%
% \end{table}

\tabelleEinfg{Vergleich der Umsetzungsalternativen}{tab:AlternativenVergleich}{AlternativenVergleich}

Die vorangegangene Tabelle stellt die drei möglichen Lösungsansätze mit den Kriterien 
Innovativität, Attraktivität, Komplexität und dem Grad der Zielerreichung dar. Die 
Fotogalerie hat im Gegensatz zu den anderen Ansätzen eine geringe Innovativität. Dies 
resultiert daraus, dass eine statische Fotogalerie im Vergleich zu virtuellen Rundgängen 
innovationslos ist. Die Fotogalerie erreicht somit nur eine mittlere Attraktivität. Die 
beiden anderen Lösungsansätze besitzen aufgrund der hohen Innovativität ebenfalls eine hohe 
Attraktivität. Die Komplexität der 3D-Modellierung und des virtuellen Rundgangs sind dagegen 
höher als die der Fotogalerie, da der Einarbeitungsaufwand bei diesen beiden Ansätzen höher 
und langwieriger ist. Die Fotogalerie hat im Vergleich zu den anderen Lösungsansätzen nur 
einen geringen Grad der Zielerreichung. Die 3D-Modellierung und der virtuelle Rundgang erreichen 
hingegen einen hohen Grad der Zielerreichung, da die junge Zielgruppe durch innovative und attraktive Ansätze
effektiver angesprochen wird.

% Die Projektgruppe geht davon aus, dass vielen Nutzern innerhalb der Zielgruppe
% diese Darstellungsform durch Google Street View bereits bekannt ist. Um von den
% vorhandenen Erfahrungen der Nutzer profitieren zu können sollen bekannten
% Steuerungselemente von Google Street View adaptiert werden. Diese sind in
% \abbildung{GoogleStreetView} dargestellt.

% \begin{figure}[htb] 
% \centering
% \includegraphics[width=0.7\textwidth]{GoogleStreetView.png}
% \caption[Google Street View]{Google Street View\protect\footnotemark}
% \label{fig:GoogleStreetView}
% \end{figure}
% \footnotetext{Screenshot von \url{https://maps.google.de/}}

% Die Realisierung der Projektziele wäre nach einer ersten Einschätzung der
% Projektmitglieder mit jeder der dei vorgestellten Umsetzungsmöglichkeiten
% vorstellbar. Im weiteren Verlauf der Konzeptionierungsphase soll jedoch die
% Möglichkeit herausgestellt werden, welche der Möglichkeiten für die
% Erreichung der Projektziele am besten geeignet ist.
