\subsection{Projektumfeld}
\label{sec:Projektumfeld}

% TODO: Umsetzung durch Projektteam, jedoch nicht die einzigen Beteiligten
% TODO: Referrenz auf Stakeholderanalyse
% TODO: Projektgruppen nennen und Besonderheiten herausstellen

Nachdem geklärt wurde, auf welche Weise das Projekt realisiert werden soll,
sollen nachfolgend alle Stakeholder, die an dem Projekt "`Virtueller Campus
Lingen"' beteiligt sind und deren Aufgaben im Projekt, herausgestellt werden.
Das Projektteam, welches aus Andreas Makeev, Raphael Otten, Carsten Sandker und
Jannik Fangmann besteht, stellt hierbei nur einen Teil der Stakeholder dar.
Dieser Teil der Stakeholder agiert als Entwickler und ist für die Umsetzung der
Projektidee zuständig. Neben den Entwicklern sind weitere Personen oder
Personengruppen an dem Projekt beteiligt, die die Rollen "`Betreuer"',
"`Aufraggeber"', "`Berater"' und "`Anwender"' in dem Projekt übernehmen.

Wie bereits herausgestellt ist der Projektinhalt die Visualisierung des Campus
Lingen. An diesem Standort der Hochschule Osnabrück ist die Fakultät
Management, Kultur und Technik ansässig. Aus diesem Grund nimmt die Leitung
dieser Fakultät auch die Rolle des Auftraggebers im Projekt ein. Die
Fakultätsleitung besteht hierbei aus dem Dekan der Fakultät, Prof. Dr. Frank
Blümel, und den Leitern der vier Institute aus denen die Fakultät besteht.
Diese sind Prof. Dr. Michael Ryba für das Institut Management und Technik,
Prof. Dr.-Ing. Wolfgang Arens-Fischer für das Institut für Duale Studiengänge,
Prof. Dr. Dagmar Schütte für das Institut für Kommunikationsmanagement und
Prof. Dr. Bernd Ruping für das Institut für Theaterpädagogik.

Die Rolle der Betreuer wird hierbei von den Projektpaten Stefan Feldker und
Prof. Dr. Ralf Westerbusch übernommen. Diese sollen der Projektgruppe bei
Fragen zur Seite stehen und diese durch den Prozess der Projektumsetzung
begleiten. Weiterhin stellen sie die Zwischenstation bei der Kommunikation der
Entwickler mit der Fakultätsleitung dar.

Auch der Datenschutzbeauftragte der Hochschule Osnabrück, Prof. Dr. Alfred
Scheerhorn ist am Projekt beteiligt. Das Projekt soll, wie bereits
herausgestellt, im Stile von Google Street View realisiert werden. Hierbei
sollen die Benutzer durch 360-Grad Fotos des Campus navigieren. Die Umsetzung
des Projektes beinhaltet also auch die Veröffentlichung dieser Fotos im
Internet. Bei den datenschutzrechtlichen Fragen, die sich hieraus ergeben, soll
Herr Scheerhorn dem Projektteam als Berater zur Seite stehen.

Neben den bereits erwähnten Personen und Personengruppen gehören auch die
Benutzer der zu realisierenden Anwendung zu den Stakeholdern des Projektes.
Alle Stakeholder, die an dem Projekt beteiligt sind, sind in Abbildung X
% TODO: Verweis auf Anhang
dargestellt.
