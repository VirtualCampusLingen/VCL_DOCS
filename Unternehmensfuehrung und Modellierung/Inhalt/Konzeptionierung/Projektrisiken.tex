\subsection{Projektrisiken}
\label{sec:Projektrisiken}


Ein Projekt in dieser größenordnung birgt einige Risiken, die innerhalb der
Durchführung eintreffen könnten und das Erreichen der Projektziels gefährden.
Solche Risiken können mittels Maßnahmenplanung im Fall des Eintritts bewältigt
werden. Jedes im Projekt erkannte Risiko wird mit einer
Eintrittswarscheinlichkeit bewertet und ein grober Plan aufgestellt, was im Fall
des Eintritts zu unternehmen ist.

Im durchzuführenden Projekt wurden folgende Risiken erkannt:

\begin{description}
\item[Ausfall der Server]
Sollte der Webserver der Anwendung ausfallen ist diese nicht mehr für die Nutzer
erreichbar und somit nicht vorhanden.
\item[Datenverlust]
Es besteht das Risiko eines Datenverlustes während der Entwicklung.
\item[Ausfall der personellen Ressourcen]
Innerhalb der Entwicklungsphase können Mitglieder des Entwicklungsteams
krankheitsbedingt ausfallen.
\item[Keine rechtzeitige Fertigstellung der Anwendung]
Es besteht das Risiko das die Anwendung nicht bis zum geplanten Übergabetermin
fertiggestellt wird.
\item[Anwendung entspricht nicht Wünschen des Kunden]
Die fertige Anwendung kann nicht den Wünschen des Kunden entsprechen.
\item[Qualität der Anwendung nicht ausreichend]
Die Anwendung entspricht nicht den Qualitätsansprüchen des Kunden.
\item[Administration der Anwendung zu komplex]
Es ist zu kompliziert die fertige Anwendung zu Administrieren.
\item[Anwendung wird gehackt und missbraucht]
Durch eine Sicherheitslücke in der Anwendung erhalten unbefugte zugriff auf die
Anwendung. 
\end{description}
%ToDo: Sind das alle Risiken ? Was ist mit Datenschutz ? Sollen diese hier
% nochmal beschrieben werden?

Die erkannten Projektrisiken sind anhand von zwei Kriterien zu kategorisieren:

\begin{itemize}
\item Warscheinlichkeit des Eintritts
\item Ausmaß des Schadens bei Eintritt 
\end{itemize}

Die Kategorisierung der erfassten Risiken wird in einer sogennanten Risikomatrix
dargestellt. An der visuell erfasst wird, wie hoch die Warscheinlichkeit eines
bestimmten Risikos ist und wie hoch der entstehende Schaden am Projekt
einzuschätzen ist.

\begin{table}[h]
\centering
\begin{tabular}{ccccl}
\hline
\multicolumn{1}{l}{} mögliches Risiko      & Wahrscheinlichkeit & Schadensausmaß
\\ \hline 
Ausfall der Server       	& Gering & Mittel  \\ \hline
Datenverlust             			& Mittel & Hoch  \\ \hline
Ausfall der personellen Ressourcen  & Mittel & Hoch \\ \hline
keine rechtzeitige Fertigstellung des Kunden & Mittel & Hoch \\ \hline
Anwendung entspricht nicht wünschen des Kunden & Gering & Hoch \\ \hline
Qualität der Anwendung nicht ausreichend & Gering & Hoch \\ \hline
Zu hoher Wartungsaufwand & Gering & Mittel \\ \hline
Anwendung wird gehackt und missbraucht & Mittel & Mittel \\ \hline
\end{tabular}
\caption{Risikomatrix}%
\label{tab:Risikomatrix}%
\end{table}

\ldots
in Arbeit 
\ldots


% Beschreibung der Auswirkungen und warum diese Warscheinlichkeit Mittel und das
% Schadensausmaß Hoch ist ??
% Risikoanalyse?!?







