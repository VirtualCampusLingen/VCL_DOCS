\subsection{Projektrisiken}
\label{sec:Projektrisiken}

Ein Projekt in dieser größenordnung birgt einige Risiken, die innerhalb der
Durchführung eintreffen könnten und das Erreichen der Projektziels gefährden.
Solche Risiken können mittels Maßnahmenplanung im Fall des Eintritts bewältigt
werden. Jedes im Projekt erkannte Risiko wird mit einer
Eintrittswarscheinlichkeit bewertet und ein grober Plan aufgestellt, was im Fall
des Eintritts zu unternehmen ist.

Im durchzuführenden Projekt wurden folgende Risiken erkannt:

\begin{description}
\item[Ausfall der Server]
Sollte der Webserver der Anwendung ausfallen ist diese nicht mehr für die Nutzer
erreichbar und somit nicht vorhanden.
\item[Datenverlust]
Es besteht das Risiko eines Datenverlustes während der Entwicklung.
\item[Ausfall der personellen Ressourcen]
Innerhalb der Entwicklungsphase können Mitglieder des Entwicklungsteams
krankheitsbedingt ausfallen.
\item[Keine rechtzeitige Fertigstellung der Anwendung]
Es besteht das Risiko das die Anwendung nicht bis zum geplanten Übergabetermin
fertiggestellt wird.
\item[Anwendung entspricht nicht Wünschen des Auftraggebers]
Die fertige Anwendung kann nicht den Wünschen des Auftraggebers entsprechen.
\item[Qualität der Anwendung nicht ausreichend]
Die Anwendung entspricht nicht den Qualitätsansprüchen des Kunden.
\item[Administration der Anwendung zu komplex]
Es ist zu kompliziert die fertige Anwendung zu administrieren.
\item[Anwendung wird gehackt und missbraucht]
Durch eine Sicherheitslücke in der Anwendung erhalten unbefugte zugriff auf die
Anwendung. 
\end{description}

Die erkannten Projektrisiken sind anhand von zwei Kriterien zu kategorisieren:

\begin{itemize}
\item Warscheinlichkeit des Eintritts
\item Ausmaß des Schadens bei Eintritt 
\end{itemize}

Die Kategorisierung der erfassten Risiken wird in einer sogennanten Risikomatrix
dargestellt. An der visuell erfasst wird, wie hoch die Warscheinlichkeit eines
bestimmten Risikos ist, wie hoch der entstehende Schaden am Projekt
einzuschätzen ist und mit welcher Maßnahem sich dieses Risiko minimieren lässt.

\begin{table}[h]
\centering
\begin{tabular}{ccccl}
\hline
\multicolumn{1}{l}{}              & Wahrscheinlichkeit  & Schadensausmaß & Maßnahme                       \\ \hline 
Ausfall der Server                & Gering              & Mittel         &                                \\ \hline
Datenverlust             			    & Mittel              & Hoch           & regelmäßige Backups            \\ \hline
Ausfall personellen Ressourcen    & Mittel              & Hoch           & zentrale Verwaltung            \\ \hline
Zeitverzug der Fertiggstellung    & Mittel              & Hoch           & Zeitplanung                    \\ \hline
Auftraggeberwünsche nicht erfüllt & Gering              & Hoch           & Feedbackgetriebene Entwicklung \\ \hline
Mangelnde Qualität                & Gering              & Hoch           & Anwendungstest                 \\ \hline
Zu hoher Wartungsaufwand          & Gering              & Mittel         & Usability Best Pracitces       \\ \hline
Anwendungs Missbrauch             & Mittel              & Mittel         & IT-Security Best Practices     \\ \hline
\end{tabular}
\caption{Risikomatrix}%
\label{tab:Risikomatrix}%
\end{table}

Jedes dargestellte Risiko mit Begründung der gewählten Wahrscheinlichkeit und des gewählten
Schadensausmaßes im Detail zu erläutern würde an dieser Stelle zu weit führen
und ist auch nicht Intention dieser Auflistung. Die dargestellten Risiken
sollen dem Auftraggeber und allen Stakeholdern des Projektes lediglich verdeutlicht
werden. Nennenswert ist an dieser Stelle jedoch, dass keine Maßnahme zum Schutz vor
Serverausfall aufgelistet wurde. Es ist zwar möglich sich vor
Ausfall eines Server durch redundante Systeme zu schützen, aber die Möglichkeit
ist in diesem Projekt auf Grund begrenzter Hardwareressourcen nicht gegeben.

Die Projektrisiken sind dem Auftraggeber damit bekannt und die Konzeptionierung der
Anwendung kann abgeschlossen werden.