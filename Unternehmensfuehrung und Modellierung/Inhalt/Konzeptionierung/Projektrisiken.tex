\subsection{Projektrisiken}
\label{sec:Projektrisiken}

Ein Projekt in dieser größenordnung birgt einige Risiken, die
innerhalb der Durchführung eintreffen könnten und das Erreichen der
Projektziele gefährden.Solche Risiken können mittels Maßnahmenplanung im Fall
des Eintritts bewältigt werden. Jedes im Projekt erkannte Risiko wird mit einer
Eintrittswarscheinlichkeit bewertet und ein grober Plan aufgestellt, was im Fall
des Eintritts zu unternehmen ist.

Hierbei können fünf unterschiedliche Risikoarten unterschieden
werden:

\begin{itemize}
  \item Kostenrisiko (nicht ausreichendes Projektbudget)
  \item Terminrisiko (terminliche Planung wird nicht eingehalten)
  \item Qualitätsrisiko (geforderte Qualität kann nicht eingehalten werden)
  \item Auslastungsrisiko (persönliche Ressourcen ausgelastet)
  \item Akzeptanzrisiko (erbrachte Ergebnis wird vom Kunden nicht akzeptiert)
\end{itemize}
%ToDo: Qulle http:// www.pm-handbuch.com/planung/



