\subsection{Projektrisiken}
\label{sec:Projektrisiken}

Ein Projekt in dieser Größenordnung birgt einige Risiken, die innerhalb der
Durchführung eintreffen könnten und das Erreichen der Projektziels gefährden.
Solche Risiken können mittels Maßnahmenplanung im Fall des Eintritts bewältigt
werden. Jedes im Projekt erkannte Risiko wird mit einer
Eintrittswahrscheinlichkeit bewertet und mit einer Gegenmaßnahme versehen.

Im durchzuführenden Projekt wurden folgende Risiken erkannt:

\begin{description}
\item[Ausfall der Server]
Sollte der Webserver der Anwendung ausfallen, ist diese nicht mehr für die Nutzer
erreichbar.
\item[Datenverlust]
Es besteht das Risiko eines Datenverlustes innerhalb des produktiven Einsatzes.
\item[Ausfall der personellen Ressourcen]
Innerhalb der Entwicklungsphase können Mitglieder des Entwicklungsteams
krankheitsbedingt ausfallen.
\item[Keine rechtzeitige Fertigstellung der Anwendung]
Es besteht das Risiko, dass die Anwendung nicht bis zum geplanten Übergabetermin
fertiggestellt wird.
\item[Anwendung entspricht nicht Wünschen des Auftraggebers]
Die fertige Anwendung verfehlt die Wünsche des Auftraggegbers.
\item[Qualität der Anwendung nicht ausreichend]
Die Anwendung entspricht nicht den Qualitätsansprüchen des Kunden.
\item[Administration der Anwendung zu komplex]
Es ist zu kompliziert die fertige Anwendung zu administrieren.
\item[Anwendung wird gehackt und missbraucht]
Durch eine Sicherheitslücke in der Anwendung erhalten Unbefugte Zugriff auf die
Anwendung. 
\end{description}

Die erkannten Projektrisiken sind anhand von zwei Kriterien zu kategorisieren:

\begin{itemize}
\item Wahrscheinlichkeit des Eintritts
\item Ausmaß des Schadens bei Eintritt 
\end{itemize}

Die Kategorisierung der erfassten Risiken wird in einer sogenannten Risikomatrix
dargestellt. In dieser Matrix wird visuell dargestellt, wie hoch die Warscheinlichkeit eines
bestimmten Risikos ist, wie hoch der entstehende Schaden am Projekt
einzuschätzen ist und mit welcher Maßnahem sich dieses Risiko minimieren lässt.

\clearpage
\tabelleEinfg{Risikomatrix}{tab:Risikomatrix}{Risikomatrix}

Jedes dargestellte Risiko, mit Begründung der gewählten Wahrscheinlichkeit und des gewählten
Schadensausmaßes im Detail, zu erläutern würde, an dieser Stelle zu weit führen
und ist auch nicht Intention dieser Auflistung. Die dargestellten Risiken
sollen dem Auftraggeber und allen Stakeholdern des Projektes lediglich verdeutlicht
werden. Nennenswert ist an dieser Stelle jedoch, dass keine Maßnahme zum Schutz vor
Serverausfall aufgelistet wurde. Es ist zwar möglich sich vor
Ausfall eines Server durch redundante Systeme zu schützen, aber diese Möglichkeit
ist in diesem Projekt aufgrund begrenzter Hardwareressourcen nicht gegeben.