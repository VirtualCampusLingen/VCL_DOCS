\subsection{Iterative Erarbeitung des Feinkonzeptes}
\label{sec:ErarbeitungFeinkonzept}

Aufbauend auf dem Grobkonzept wird im folgenden ein Feinkonzept in feedback-getriebenen Iterationen erarbeitet.
Während der Erarbeitung des Konzeptes sollen die Auftraggeber mit eingebunden und stets auf einem aktuellen
Stand gehalten werden. Das Feedback wird dabei in der Konzepterarbeitung berücksichtigt.


\subsubsection*{Erstes Treffen mit den Projektpaten}
\label{sec:Treffen1}

Nach Definition möglicher Lösungsansätze für das bestehende Problem, soll ein Treffen mit den Projektpaten stattfinden, 
welche das Projekt und die Durchführung begleiten und bei Problemen oder Fragen beratend zur Seite stehen. Das 
Gesamtkonzept soll den Projektpaten präsentiert und diskutiert werden. Der Ansatz der 3D-Modellierung wurde von den 
Projektpaten besonders positiv aufgenommen und soll weiter untersucht werden. 
Die 3D-Modellierung wurde ausgewählt, um einer Machbarkeitsstudie unterzogen zu werden, welche nach dem Treffen mit den 
Projektpaten durchgeführt wurde. Anhand der Machbarkeitsstudie können die Komplexität und der Aufwand dieser Lösung im 
weiteren Verlaufe des Projektes genauer eingeschätzt werden.
Im Laufe der Machbarkeitsstudie konnte schnell erkannt werden, dass sich diese Lösung als zu aufwändig und komplex für die 
Realisierung dieses Projektes herausstellt. Dies ist darauf zurückzuführen, dass die Vorkenntnisse der Projektmitglieder 
in der grafischen 3D-Modellierung recht gering sind und eine zu hohe Einarbeitungszeit notwendig wäre. Zusätzlich stellte 
sich der Campus als Objekt zu groß und zu komplex dar. 

Der Ansatz der 3D-Modellierung wurde verworfen, da die erzielbaren Ergebnisse in dem vorgegebenen Zeitraum nicht den 
Ansprüchen genügen würden.


\subsubsection*{Treffen mit dem Studiendekan des IDS}
\label{sec:Treffen2}

Das Treffen mit den Projektpaten hat ergeben, dass der Lösungsansatz mit der 3D-Modellierung zu komplex ist und entfällt.
 Die weiteren Lösungsansätze sollen mit dem Dekan des IDS, Prof. Dr.-Ing. Arens-Fischer, besprochen und ein Feedback 
eingeholt werden. Dieser ist einer der Auftraggeber des Projektes und soll näher mit einbezogen werden. Durch diese 
Vorgehensweise soll gewährleistet werden, dass das erarbeitete Konzept den Ansprüchen der Auftraggeber entspricht.

% TODO: Datum korrekt einfügen
Das Treffen mit dem Dekan des IDS fand am xx.xx.2013 statt.
Im Laufe des Gesprächs ergaben sich wichtige Aspekte, die beachtet werden sollten und als wichtig für die Realisierung 
des Projektes erachtet werden. Hierbei stellte einer der wichtigsten Punkte der Datenschutz dar und sollte bei der 
Durchführung des Projektes berücksichtigt werden. Es sollten keine personenbezogenen Daten abgebildet werden.
Es wurde ebenfalls angesprochen, dass alle vier Institute berücksichtigt und abgebildet werden sollen. Da die Möglichkeit 
der Erweiterung der bestehenden Fotogalerie noch bestand, stellte sich die Frage, ob auf bereits existierende Fotos 
zugegriffen und diese auch genutzt werden können. 
Es wurde auch aus dem Gespräch deutlich, dass die Informationsdarstellung zu den Fakultäten, Instituten, Laboren, 
Studienprojekten sowie Studiengängen ein wichtiger Bestandteil ist und dementsprechend ausführlich und ansprechend 
vermittelt werden sollen.

Das Projekt ist vom Studiendekan des IDS soweit auf Zustimmung gestoßen, dass, wenn nötig, ein Budget für die 
Durchführung beantragt werden kann.

Im Anschluss an das Treffen wurden die erwähnten Aspekte noch einmal in der Projektgruppe besprochen und auf die beiden 
zur Auswahl stehenden Lösungsansätze reflektiert. Als Ergebnis dieser Reflektion stellte sich heraus, dass die Umsetzung 
durch die Fotogalerie ungeeignet ist. Diese Art der Umsetzung entspricht nicht den Ansprüchen der Zielsetzung im Bezug 
auf die Informationsdarstellung.
Aus diesen Gründen wird die Realisierung des Projektes anhand des virtuellen Rundgangs im Stile von Google Street View 
als bestmöglicher Lösungsansatz herausgestellt. Im Anschluss wird ein Prototyp der Anwendung erstellt, welcher darstellen 
soll, wie die Realisierung mit dieser Technologie funktionell und visuell geplant ist. Daraufhin soll wieder ein Feedback 
der Auftraggeber eingeholt werden.

\subsubsection*{Zweites Treffen mit den Projektpaten}
\label{sec:Treffen3}

In den vorherigen Treffen stellte sich anhand der Machbarkeitsstudien für die Fotogalerie und die 3D-Modellierung heraus, 
dass diese nicht den geforderten Ansprüchen entsprechen. Daraufhin wurde ein Prototyp für den Lösungsansatz des Rundgangs 
im Stile von Google Street View erstellt.
Nach Fertigstellung des Prototypen für den virtuellen Rundgang durch den Campus wurde nochmals ein Termin mit den 
Projektpaten vereinbart, um diesen Prototypen zu präsentieren. Dabei war es wichtig, das Konzept mit den Projektpaten 
abzustimmen, um den bisherigen Projektstand transparent zu halten. In dieser Präsentation konnten die 
Projektpaten von dieser Art der
Umsetzung überzeugt werden. Die Umsetzung des virtuellen Campusrundgangs im Stile von Google Street View ist
damit durch die Projektpaten bestätigt. Es wurde jedoch nochmals der verstärkte Datenschutz im Bezug auf das Projekt 
erwähnt. Die Problematik des Datenschutzes soll vermieden werden, indem Datenschutzrichtlinien erstellt wurden, die bei 
der 
Durchführung des Projektes beachtet werden sollen. Die genauen Datenschutzrichtlinien können in Anhang x nachgelesen
werden.
% TODO: Datenschutzrichtlinien in den Anhang
% TODO: Anhang X zu Datenschutzrichtlinien korrekt einfügen

Nachdem die Projektpaten dem Projektkonzept zugestimmt haben, liegt der Projektgruppe ein Prototyp vor, der der 
Fakultätsleitungsrunde als Auftraggeber präsentiert werden kann.
Hierzu wurde ein Termin mit der Fakultätsleitungsrunde zur Präsentation des Konzepts vereinbart.



% Alter Text:
% Die Projektgruppe hat sich einstimmig dazu entschlossen die Projetidee durch die
% Erstellung eines virtuellen Rundgangs im Stile von Google Street View zu
% realisieren. Hierbei sollen die Räumlichkeiten des Campus Lingen in 360 Grad
% Panoramafotos dokumentiert werden. In jedem dieser Fotos kann sich der Nutzer
% frei umsehen. Auf diese Weise erhält der Nutzer einen fotorealistischen Eindruck
% von den dargestellten Räumlichkeiten. Weiterhin ist es dem Nutzer möglich
% selbständig durch die verschiedenen Panoramafotos zu navigieren. Hierzu werden
% die einzelnen Fotos miteinander verknüpft. Mithilfe von in den Fotos
% angezeigten Richtungspfeilen ist der Nutzer in der Lage sich zwischen
% den verknüpften Fotos frei zu bewegen.

\subsubsection*{Fakultätsleitungsrunde}
\label{sec:Treffen4}

Das Konzept, welches von den Projektpaten bestätigt wurde, \colorbox{yellow}{soll} der Fakultätsleitungsrunde präsentiert werden. Dabei \colorbox{yellow}{soll} 
der bereits erstellte Prototyp und dessen Funktionen vorgestellt werden. Die Fakultätsleitungsrunde muss dem Konzept 
zustimmen, damit das Projekt durchgeführt werden kann.
Bei dem Treffen mit der Fakultätsleitungsrunde wurde der Prototyp des virtuellen Rundgangs durch den Campus im Stile von 
Google Street View präsentiert. Hierbei waren alle Dekane der einzelnen Institute vertreten, um eine eigene Einschätzung 
des Projektes zu erlangen.
Als Ergebnis dieses Treffens hat sich ergeben, \colorbox{yellow}{dass} die Fakultätsleitungsrunde dem Konzept zugestimmt hat. Jedoch wurden 
einige Aspekte genannt, die beachtet werden müssen. So wurde der Wunsch geäußert, \colorbox{yellow}{dass} die Möglichkeit bestehen soll, dass 
bei geöffneten Infotexten in der Anwendung Videos und Bilder eingebunden werden können. Weiterführend sollte die 
Nachhaltigkeit des Projektes gewährleistet werden. So soll die Anwendung von anderen Personen administrierbar sein und es 
soll eine Dokumentation für die Anwendung erstellt werden. Ebenfalls kann ein Budget für die Anschaffung für Equipment 
beantragen werden, damit dieses für die Projektweiterführung zur Verfügung steht. Anhand dieser Punkte soll der Aspekt der 
Nachhaltigkeit gesichert werden. 

Außer den zusätzlichen Funktionen wurden ebenfalls Grenzen für das Projekt definiert. So sollen nicht alle einzelnen Räume 
des Campus abgebildet werden, da viele Räume nahezu identisch aufgebaut sind. 
Nach allen angesprochenen Aspekten wurde das Projektkonzept genehmigt und kann durchgeführt werden.
