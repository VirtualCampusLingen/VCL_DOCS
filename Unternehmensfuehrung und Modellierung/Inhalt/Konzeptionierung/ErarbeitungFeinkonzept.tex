\subsection{Iterative Erarbeitung des Feinkonzeptes}
\label{sec:ErarbeitungFeinkonzept}



\subsubsection{Erstes Treffen mit den Projektpaten}
\label{sec:Treffen1}

Nach der Durchführung der Machbarkeitsanalyse konnte die Projektgruppe die
Alternative der 3D-Modellierung ausschließen. Diese wurde von der Projektgruppe
als zu komplex eingestuft. Der Detaillierungsgrad, der bei der Modellierung in
der zur Verfügung stehenden Zeit erreicht werden könnte, würde nicht den
Anforderungen der Projektgruppe entsprechen. So ist es nicht möglich auf diese
Weise eine ansprechende Darstellung des Campus zu realisieren. 

\subsubsection{Treffen mit dem Studiendekan des IDS}
\label{sec:Treffen2}

Die Umsetzung als Fotogalerie wurde ebenfalls von der Projektgruppe 
ausgeschlossen. Da diese sehr schlichte Art der Präsentation nicht dem modernen
Charakter des Campus gerecht werden würde. Weiterhin vertritt die Projektgruppe
die Ansicht, dass diese Präsentationsform aufgrund von fehlender Innovation die
Zielgruppe verfehlen würde und somit keinen erheblichen Mehrwert für die
Hochschule darstellt.

\subsubsection{Zweites Treffen mit den Projektpaten}
\label{sec:Treffen3}

Die Projektgruppe hat sich einstimmig dazu entschlossen die Projetidee durch die
Erstellung eines virtuellen Rundgangs im Stile von Google Street View zu
realisieren. Hierbei sollen die Räumlichkeiten des Campus Lingen in 360 Grad
Panoramafotos dokumentiert werden. In jedem dieser Fotos kann sich der Nutzer
frei umsehen. Auf diese Weise erhält der Nutzer einen fotorealistischen Eindruck
von den dargestellten Räumlichkeiten. Weiterhin ist es dem Nutzer möglich
selbständig durch die verschiedenen Panoramafotos zu navigieren. Hierzu werden
die einzelnen Fotos miteinander verknüpft. Mithilfe von in den Fotos
angezeigten Richtungspfeilen ist der Nutzer in der Lage sich zwischen
den verknüpften Fotos frei zu bewegen.

\subsubsection{Fakultätsleitungsrunde}
\label{sec:Treffen4}