\subsection{Projektzielplanung}
\label{sec:Projektzielplanung}

Mit der Initiierung und Durchführung eines Projekts verfolgt jedes Unternehmen
bestimmte Ziele, welche in komplementärer Beziehung zu den Unternehmenszielen
stehen.
% TODO: Verweis auf Quelle
Hierbei wird der Zweck des Projektes als Zielsetzung ausgedrückt und dem Projekt
vorgegeben. Diese  grobe Zielaussage wird in Unterzielen weiter Konkretisiert.
Diese Teilziele sind für die Zielerreichung des obergeordneten Ziels relevant.
Konkret formuliert werden die Teilziele in einer dritten Ebene in den
sogenannten Unterzielen. In den Unterzielen werden klare Ziele definiert, die
im nachstehenden Zielsystem aufgeführt sind.

% TODO: Grafik des Zielsystems einfügen

In den fünf Teilzielen im Zielsystem spiegelt sich das Hauptziel des Projektes
wieder. Die Unterziele auf der letzten Ebene sind nicht mehr durch andere Ziele
beschreibbar, sondern nur noch als Maßnahmen umsetzbar und auch so formuliert.
Die Ergebnisse dieser Maßnahmen sind messbar und somit wird auch das erreichen
der einzelnen Teilziele messbar. Wodurch auch das erreichen der groben
Zielaussage (Hauptziel) welches sich aus den Teilzielen ergibt bewertbar wird.
Um Unklarheiten und Missverständnisse zu vermeiden wurden die Unterziele
(Maßnahmen) der Teilziele in Muss-, Soll- und Kann-Ziele unterteilt.

\begin{description}
\item[Muss-Ziele]
Diese Ziele sind für das Erreichen des Hauptzieles unabdingbar sie sind
projektentscheidend und können zum Projektabbruch führen.
% TODO: Verweis auf Quelle einfügen
\item[Soll-Ziele]
Stellen eine zusätzliche Funktion für den Auftraggeber dar, ein nicht erreichen
gefährdet nicht das Projekt.
% TODO: Verweis auf Quelle einfügen
\item[Kann-Ziele]
Beschreiben wünschenswerte Ziele, die aber nicht zwingend für das erreichen des
Hauptzieles notwendig sind.
% TODO: Verweis auf Quelle einfügen
\end{description}

Das Hauptziel im vorliegenden Zielsystem ist die "`Konzeptionierung und
Entwicklung einer Anwendung zur Darstellung des Campus Lingen als attraktiven
Studienstandort"'. Um das Hauptziel zu erreichen dienen die Teilziele
"`Verbesserung der Information über Studienangebote"', "`Vermittlung eines
Modernen und ansprechenden Bildes vom Campus"', "`Sicherstellung der Pflege und
Erweiterbarkeit (Fotos \& Informationen)"', "`Verbesserung des
Bekanntheitsgrades"' und "`Verbesserung der Veröffentlichung im Internet"'. Die
Teilziele werden in Unterzielen detailliert beschrieben und so ist es möglich
Arbeitsschritte abzuleiten. Diese Arbeitsschritte werden so formuliert, dass
der Erfolg messbar ist. Während ein Projektteam lediglich die Ziele erreichen
kann, die innerhalb des Projektverlaufs auch gemessen werden können, verfolgt
die Unternehmensführung mit der Durchführung des Projektes langfristige Ziele. 
Um das erreichen des Projektzieles sowohl kurzfristig für das Projektteam, als
auch langfristig für den Studienstandort Lingen messen zu können, wurden für
unterschiedliche Betrachtungszeiträume  bestimmte Zielindikatoren definiert.
% TODO: Zielindikatoren