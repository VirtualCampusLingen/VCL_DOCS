\subsection{Zielkatalog}
\label{sec:Zielkatalog}

Der Zielkatalog der Muss-Ziele besteht aus acht Zielen, von denen vier 
Ziele in einem Betatest\footnotemark\
und vier Ziel von der Projektgruppe selbst verifiziert werden. Das Ziel
gilt dabei als erreicht, wenn der festgelegte Zielindikator erfüllt wird.

\footnotetext{Am Ende des Projektes wird ein Test der Anwendung durchgeführt. Dieser Test 
beinhaltet unter anderem eine Umfrage, die zur Verifikation der oben genannte Ziel dient. 
Eine genauere Betrachtung folgt in \verweis{Projekttest}}

\textbf{Verifikation durch den Betatest:}
\begin{description}
  \item[Steigerung der Attraktivität:] Das Ziel gilt als
  erreicht, wenn innerhalb der Umfrage die Frage "`Wie bewerten Sie die durch
  die Anwendung vermittelte Attraktivität des Campus Lingen?"' von den
  Betatestern im Durchschnitt mit "`eher gut"' oder besser beantwortet wird.

  \item[Verbesserung der Präsentation von Studienprojekten:]
  Das Ziel gilt als erreicht, wenn die Frage "`Wie bewerten Sie die Präsentation
  der Studienprojekte im Vergleich zur bereits bestehenden Hochschulseite?"' von
  den Betatestern im Durchschnitt mit "`eher gut"' oder besser beantwortet wird.

  \item[Verbesserung der Präsentation aller Studiengänge:] Das Ziel gilt als erreicht, wenn
  die Frage "`Wie bewerten Sie die Art der Präsentation der einzelnen
  Studiengänge im Vergleich zur bereits bestehenden Hochschulseite?"' von den
  Betatestern im Durchschnitt mit "`eher gut"' oder besser beantwortet wird.

  \item[Verbesserung der Informationsbeschaffung:]Das Ziel gilt als erreicht, 
  wenn die Frage "`Wie bewerten Sie die
  Art der Informationsbeschaffung im Vergleich zur bereits bestehenden
  Hochschulseite?"' von den Betatestern im Durchschnitt mit "`eher gut"' oder
  besser beantwortet wird.
\end{description}

\textbf{Verifikation durch das Projektteam:}
\begin{description}
  \item[Abbildung aller Institute:]
  Kriterium bei der Bewertung ist hierbei
  die Anzahl der erstellten Fotos je Institut. Das Ziel gilt als erreicht, wenn
  mindestens 10 Bilder pro Institut in der Anwendung vorhanden sind.

  \item[Aufbau einer Administrationsoberfläche:] Das Ziel gilt als erreicht, wenn eine
  Administrationsoberfläche geschaffen wurde. Diese Administarionsoberfläche muss
  die Verwaltung der Daten der Anwendung ermöglichen. Der Datenbestand muss
  hierbei sowohl erweitert als auch gelöscht und angepasst werden können.

  \item[Entwicklung einer Anwendungsdokumentation:] Das Ziel gilt hierbei als erreicht, wenn
  für die zu realisierende Anwendung eine Dokumentation erstellt wurde. In dieser
  müssen alle relevanten Anwendungsfälle dokumentiert sein.

  \item[Erstellung unter Einhaltung der Datenschutzrichtlinien:] 
  Die in Absprache mit der Hochschule Osnabrück entstandenen
  Datenschutzrichtlinien müssen während der Erstellung beachtet werden. Es sind
  keine Abweichungen zu den Datenschutzrichtlinien erlaubt.

  % Todo komplett raus! Auch aus dem Zielkatalog im Anhang
  % \item[Erstellung unter Beachtung der IT-Sicherheit:] Die Zielerreichung wird
  % in einem Penetrationstest im Rahmen der Veranstaltung "`IT-Sicherheit"'
  % festgestellt. In diesem Test werden die Kommilitonen aus dem Studiengang
  % Wirtschaftsinformatik versuchen auf den Administrationsbereich der Anwendung
  % zuzugreifen. Das Ziel gilt als erreicht, wenn innerhalb des
  % Penetrationstests kein kein Zugriff zum Administationsbereich erlangt werden
  % kann.
\end{description}

% TODO: Zielindikatoren beziehen sich auf Funtionen, die erst im Feinkonzept
% erarbeitet werden

Sollten die im Zielkatalog der Muss-Ziele genannten Werte der Zielindikatoren
im Projekt nicht erreicht werden können so sind auch die entsprechenden
Muss-Ziele nicht erfüllt und das Projekt ist gescheitert. Ein erfolgreicher
Abschluss des Projektes bedingt somit auch die Erreichug dieser Werte.

Auch wenn die Erreichung der Kann- und Soll-Ziele nicht kritisch für den
Projekterfolg ist muss es möglich sein diese Ziele zu verifizieren. Aus diesem
Grund wurden auch diesen Zielen Zielindikatoren zugeordnet. Der komplette
Zielkatalog, in dem jedem Projektziel ein Zielindikator zugeordnet ist, ist in
Anhang X dargestellt.
% TODO: Anhang einfügen

% Im nachfolgenden werden examplarisch ein Soll- und ein Kann-Ziel mit dem
% dazugehörigen Zielindikator dargestellt. Es wurden alle Muss-Ziele mit den
% dazugehörigen Indikatoren gezeigt, da diese per Definition bei einem nicht
% erreichen zum Misserfolg des Projektes führen.

% Die Erreichung des Soll-Ziels "`Verbesserung der Usability"', welches dem
% Oberziel "`Nachhaltigkeit sichern"' zugeordnet ist, wird innnerhalb des
% Betatests verifiziert. Das Ziel gilt hierbei als erreicht, wenn die Frage
% "`Konnten Sie die Anwendung intuitiv verwenden?"' von mehr als 50\% der
% Betatester mit "`Ja"' beantwortet wird.

% Ein Kann-Ziel des Oberziels "`Außendartellung des Campus Lingen
% verbessern"' ist "`Förderung eines positiven Images"'. Hierbei ist das festlegen
% eines klaren Zielindikators schwer, da eine positives Image nicht direkt auf die
% Anwendung zurückgeführt werden kann. Es handelt sich bei diesem Unterziel um ein
% qualitatives Ziel. % TODO: Begriff erklären in Fussnote? 
% Dies bedeutet das sich das erreichen des Ziels nicht dirket messen lässt. Es
% müssen Beobachtungskriterien gefunden werden die unter einer Annahme dieses Ziel
% abbilden und bewertbar sind. Die Projektgruppe entscheidet sich herbei für eine
% Frage innerhalb des Betatestfragebogens. Bei ihr wird der Tester nach dem
% potenzial der Anwendung zur Verbesserung des Images vom Campus Lingen gefragt.
% Somit kann von der Projektgruppe die Zielerreichung verifiziert werden.

% Ein detailierter Zielkatalog mit der Auflistung aller Ziele und der
% Klassifizierung nach Muss-, Soll- und Kann-Zielen mit den zugeordneten
% Zielindikatoren zur Bestimmung der Zielerreichung ist im Anhang zu finden.