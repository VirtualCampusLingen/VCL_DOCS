\subsection{Projekttest}
\label{sec:Projekttest}

Der Projekttest dient im vorliegenden Projekt zur Verifikation der beschriebenen
Projektziele und zur Behebung von Fehlern innerhalb der Anwendung. Der
Projekttest gliedert sich dabei in zwei aufeinander folgende Phasen. In der
ersten Phase wird das Projekt vom Entwicklerteam selbst, anhand von erstellten
Anwendungsfällen, getestet. Einen solchen Test bezeichnet man als Alphatest, da
dieser von Personen durchgeführt wird, die an der Entwicklung beteiligt waren.
Der Alphatest dient primär zur Erkennung und Beseitigung von Fehlern in der
Anwendung. Der Abschluss dieses Test verifiziert, dass die Anwendung fehlerfrei
funktioniert. Aufbauend auf dem Alphatest wird ein Betatest der Anwendung von
aussenstehenden Personen durchgeführt. Dieser Test dient primär zur
Verifizierung der Projektziele. Der Alphatest wird an dieser Stelle nicht weiter
vertieft, da dieser für die Betrachtung des vorliegenden Themenbereiches nicht
relevant ist. Die Durchführung und Ergbnisse können in
\citet{modelierungUndBetrieb2014} im Kapitel 6 (Test) nachgelesen werden. Der
Fokus des Projekttests liegt im Folgenden auf dem durchgeführten Betatest.

Zur Durchführung des Betatests wurde im ersten Schritt ein definierter Bereich
des Campus in die erstellte Anwendung eingepflegt. Dieser Bereich zeigt den
Betatestern einen repräsentativen Ausschnitt des Campus.
Im zweiten Schritt wurde ein Onlinefragebogen vorbereitet, der von Betatestern
nach Benutzung der Anwendung ausgefüllt werden soll. Dieser Fragebogen enthält 
sowohl Fragen, die der Verifizierung der Zielerreichung dienen, als auch
Fragen, die dem Entwicklerteam Auskunft über mögliche Schwachstellen des Systems geben.
Da an dieser Stelle nur der Teil der Fragen, der zur Verifikation der Ziele dient,
interessant ist, wird nur ein Auschnitt des gesamten Fragebogens in \tabelle{Fragebogen} dargestellt.

An dem Betatest haben insgesamt 97 Personen teilgenommen. Die Auswertung der Tests ist in nachfolgender Tabelle
festgehalten:

\clearpage
\tabelleEinfg{Auswertung des Betatests}{tab:Fragebogen}{AuswertungBetatest}

Aufbauend auf diesen Ergebnissen erfolgt die Verifikation der Zielerreichung in \verweis{Zielerreichung}.
Der subjektive Eindruck der Betatester, der in obiger Tabelle in der letzten Spalte abzulesen ist,
zeigt dem Entwicklerteam aber bereits eine erste positive Resonanz.