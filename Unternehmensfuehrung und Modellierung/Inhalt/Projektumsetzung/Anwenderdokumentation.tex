\subsection{Anwenderdokumentation}
\label{sec:Anwenderdokumentation}

Zur Gewährleistung der Nachhaltigkeit ist es Ziel des Projektes eine
Dokumentation zu erstellen, die die Administrations- und Pflegefunktionen der
Anwendung erläutern. Der grobe Inhalt dieser Dokumentation wird im Folgenden
kurz vorgestellt, um einen Überblick über die Pflege der Anwendung zu geben. Die
vollständige Version kann in \citet{projektdokumentation2014} nachglesen werden.

Die Dokumentation umfasst die folgenden Teilbereiche:

\begin{itemize}
  \item Hinzufügen und Ändern von Fotos
  \item Hinzufügen und Ändern von Infotexten
  \item Hinzufügen und Ändern von interessanten Orten
  \item Positionieren von Fotos
  \item Zuordnung von Infotexten und Fotos
  \item Festlegen von Nachbarschaftsbeziehungen zwischen Fotos
\end{itemize}

Bei Pflege der Anwendung müssen in erster Linie Fotos, Infotexte und interessante Orte gepflegt werden. Die \textbf{Fotos}
sind dabei das Basiselement der Anwendung. Hier müssen Panoramabilder angefertigt und in kleinere Bildteile zerschnitten
werden. Die Hintergründe zu dieser Vorgehensweise werden in \citet{modelierungUndBetrieb2014} im
Kapitel 5.1 (Panoramaerstellung) beschrieben.
Die \textbf{Infotexte} dienen zum Anzeigen von relevanten Informationen auf einem Foto. In Infotexten können
zum Beispiel Projekte vorgestellt oder Öffnungszeiten von Gebäuden angezeigt werden. Die gesamte Informationsvermittlung
findet im Projekt über Infotexte statt.
Zum Navigieren durch den Campus stehen dem Benutzer zum einen die Navigationspfeile und zum anderen die
\textbf{interessanten Orte} zur Verfügung. Dem Benutzer wird durch Klick auf die
Übersichtskarte eine Liste mit Orten angezeigt, zu denen er springen kann.
Diese Navigationsmöglichkeit hilft Studieninteressierten schnell Orte, wie die
Bibliothek oder das Fachbereichsgebäude zu finden, ohne dass sie diese suchen
müssen. Diese interessanten Orte müssen aber im Administrationsbereich der
Anwendung hinterlegt und gepflegt werden.

Aufbauend auf der Pflege dieser Grundinformationen müssen in der Anwendung noch
weiterführende Zusatzinformationen gepflegt werden. Dazu zählt zum einem das
Positionieren des Fotos auf einer topographischen Karte, die Zuordnung zu
Infotexten und das Festlegen von Nachbarschaftsbeziehungen. Ist ein Foto
positioniert, kann es dem Benutzer im Frontend angezeigt werden. 
Sollen auf diesem Foto zusäzlich ein oder mehrere Infotexte angezeigt werden,
müssen die entsprechende Infotexte an das Foto angehängt werden. Diese Zuordnung
wird im Administrationsbereich gepflegt. Ebenso wird die angesprochene
Navigation zwischen zwei Fotos über die Navigationspfeile im
Administrationsbereich gepflegt. Dazu müssen zu jedem hochgeladenen und
positionierten Foto Nachbarfotos definiert werden. Diese Nachbarfotos werden dem
Benutzer als Navigationspfeile angezeigt.
