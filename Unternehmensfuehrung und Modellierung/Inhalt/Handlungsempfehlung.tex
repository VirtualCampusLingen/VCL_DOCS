\section{Handlungsempfehlung}
\label{sec:Handlungsempfehlung}

Nach Abschluss des Projektes \ac{VCL} ist es für den Auftraggeber wichtig zu wissen,
ob das vorliegende Projekt eingeführt und weitergeführt werden sollte.
Hierzu wurden in der vorliegenden Ausarbeitung die wichtigsten Kennzahlen des Projektes,
einschließlich der anfallenden Kosten und der verifizierten Ziele, vorgestellt.
Diese Kosten sollen an dieser Stelle noch einmal kurz revidiert werden:

Die Weiterführung des Projektes kostet den Auftraggeber \textbf{2.763 €}, wobei
die erstellte Anwendung im Rahmen einer Unternehmensleistung, bedingt
durch Personal- und Materialkosten, ca. \textbf{61.983 €} kosten würde.
Der Auftraggeber hat damit einen Kostenvorteil von \textbf{95,54\%}.

Dieser Kostenvorteil rentiert sich für den Auftraggeber aber nur, wenn
das Projekt auch einen entsprechenden Mehrwehrt liefert.
Dem Auftraggeber wurde dazu nachweisbar in der vorliegenden Ausarbeitung aufgezeigt,
dass das die Ausendarstellung des Campus verbessert, eine junge Zielgruppe
von studieninteressierten direkt angesprochen wird und dabei noch für
Nachhaltigkeit und Innovativität gesorgt ist. Zudem hat keine andere
Hochschule eine solche Präsentation des eigenen Campus. Ein Alleinstellungsmerkmal
kann definitv erreicht und die daraus resultierenden Marketingeffekte effektiv genutzt werden.

Die Entscheidung dieses Projekt unter diesen Bedingungen Einzuführen
bleibt dabei dem Auftraggeber vorbehalten. Eine Empfehlung dafür wird an dieser
Stelle jedoch seitens der Autoren ausgesprochen. Darüber hinaus empfehelen die
Autoren den Erhalt und die Weiterführung dieses Projektes mit den vorgestellten Möglichkeiten.