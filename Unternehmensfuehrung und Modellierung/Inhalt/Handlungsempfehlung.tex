\section{Handlungsempfehlung}
\label{sec:Handlungsempfehlung}

Das durchgeführte Projekt „virtueller Campus Lingen“ konnte erfolgreich abgeschlossen werden. 
Für den Auftraggeber ist es nun wichtig zu wissen, ob dieses Projekt in der Zukunft 
weitergeführt werden sollte. Hierzu ist es notwendig die aufzutretenden Kosten sowie die 
erreichten Ziele zu kennen. Im Bezug auf die Zielerreichung kann anhand dieser Ausarbeitung 
verdeutlicht werden, dass die im Projekt aufgezeigten Ziele erreicht werden können und das 
Projekt somit den gewünschten Nutzen erfüllt.  Den Zielen gegenüber stehen jedoch die auftretenden 
Kosten, welche die Software, das Equipment und die Mitarbeiter zur Administration der Anwendung umfassen. 
Die Kosten betragen dabei 1.149€ für die benötigte Software sowie 1.614€ für das Equipment für 
die Panoramaerstellung. Die Kosten für das Personal können jedoch nicht beziffert werden, 
da keine allgemeine Gehaltseinschätzung möglich ist.
Es ist nicht einfach den Nutzen eines Projektes mit den anfallenden Kosten gleichwertig zu 
vergleichen und eine Entscheidung zu fällen. In diesem Fall kann jedoch eine positive 
Handlungsempfehlung ausgesprochen werden. Dies resultiert daraus, dass das Projekt alle 
gewünschten Ziele erfüllt und aufgrund der Erweiterungsmöglichkeiten ein nachhaltiges und 
zukunftsorientiertes Projekt darstellt. Demgegenüber können die Kosten relativiert werden, 
da bis auf die Personalkosten, die Kosten für die Software und das Equipment einmalig anfallen.