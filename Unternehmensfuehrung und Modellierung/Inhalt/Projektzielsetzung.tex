\section{Projektzielsetzung}
\label{sec:Projektzielsetzung}

Ein wichtiger Bestandteil der Projektplanung ist die Projektzielsetzung. Hierfür
müssen konkrete Ziele formuliert werden. Diese müssen allen Projektbeteiligten
zugänglich sein und richten sich dabei nach einem Soll-Zustand,

\begin{itemize}
  \item der in der Zukunft liegt,
  \item der real sein soll,
  \item dessen Erreichen erwünscht ist,
  \item der bewusst gewählt wird und
  \item der nur durch Handlung erreicht werden kann.
\end{itemize}

Weiter sollen die formulierten Ziele die Zielvorstellung bereinigen,
systematisch strukturieren, auf Vollständigkeit prüfen, ergänzen und letztlich
in einer verbindlichen Form festhalten.  Als Methode wurde hierfür die
"`Zielhierarchie"' gewählt. Hierbei werden Ziele hierarchisch vom Allgemeinen zum
spezielleren definiert.  Diese systematisch erstellte Anordnung der
Projektziele wird als "`Zielsystem"' bezeichnet. Der Vorteil dieser Methode ist
das Ziele auf der hierarchisch letzten Stufe besser analysier werden können.
Die Zielerreichung wird somit messbarer gemacht.

\subsection{Zielsystem des Projektes}
\label{sec:Zielsystem}

Für das Projekt wurde ein Zielsystem welches sich in drei Ebenen gliedert
erstellt. Das bereits in dem Punkt "`Projektidee \& Projektauftrag"' definierte
Ziel "`Konzeptionierung und Entwicklung einer Anwendung zur Darstellung des
Campus Lingen als attraktiven Studienstandort"' stellt das obere Hauptziel das
Projektziel in der obersten Ebene dar. Um dieses Hauptziel zu erreichen wurde
es in sechs Oberziele zerlegt:

\begin{itemize}
  \item Außendarstellung des Campus Lingen verbessern
  \item Kosten gering / Finanzierung sichern
  \item Campus vollständig und realistisch abbilden
  \item Nachhaltigkeit sichern
  \item Zielgruppe junger Studieninteressierter direkt ansprechen
  \item Qualität des Internetauftritts Erhalten
\end{itemize}

In den Oberzielen spiegeln sich Hauptziele des Projektes wieder. Die zweite
Ebene der sechs Oberziele bietet aber noch nicht den gewünschten Grad der
Operationalisierung und die Oberziele werden in mehre Unterziele aufgeteilt.

Die Zeile auf der untersten Ebene sind nicht mehr durch andere Ziele
beschreibbar. Zur Erreichung können diesen Zielen konkrete Maßnahmen zugeordnet
werden oder werden durch diese bereits abgebildet. Zusätzlich sind die
Ergebnisse dieser Ziele messbar und machen die Erreichung des Oberziels und
schließlich auch die Erreichung des Hauptziels bewertbar.
Um eine Priorisierung der Ziele auf der untersten Ebene durchzuführen haben wir
diese in Muss-, Soll-, und Kann-Ziele unterteilt.

Muss-Ziele

Soll-Ziele

Kann-Ziele

Die Darstellung im Zielsystem folgt einer Ampeldarstellung Muss- (Rot), Soll-
(Gelb) und Kann-Ziel (Grün).Die einzelnen Unterziele der Oberziele werden im
folgenden Punkt "`Zielkatalog"' genauer beschrieben und nach Quantitativen- und
Qualitativen-Zielen unterschieden. Es wird weiter auf die Möglichkeit des
Messens der Erreichung der untersten Ziel / Maßnahmen eingegangen, den
sogenannten Zielindikatoren.

\subsection{Zielkatalog}
\label{sec:Zielkatalog}
