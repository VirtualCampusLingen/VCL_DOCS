\section{Einleitung}
% TODO: Auf die gleiche und richtige Zeit im Dokument achten
% TODO: Zitattion im gesamten Dokument anpassen
\label{sec:Einleitung}

Am Studienstandort Lingen der Fachhochschule Osnabrück ist in einer 5 jährigen Planungs- und Bauphase
ein neuer Campus enstanden, der Besucher und den Dekan der Hochschule beeindruckt\footnote{siehe \url{http://www.noz.de/lokales/lingen/artikel/418853/campus-lingen-der-wow-effekt-ist-immer-noch-da}}.
Damit dieser Eindruck des Campus und der daraus resultierende "`Wow-Effekt"', wie es Dekan des Standortes 
Prof. Dr. Frank Blümel nennt\footnote{siehe \url{http://www.noz.de/lokales/lingen/artikel/418853/campus-lingen-der-wow-effekt-ist-immer-noch-da}}, auch Personen außerhalb der Hochschule erreicht, muss der neue Campus
effektiv beworben werden. Vor allem der Zielgruppe der jungen Studieninteressierten soll
der Campus ansprechend dargestellt werden. Die beste Möglichkeit diese Zielgruppe direkt anzusprechen
ist dabei aus Sicht der Autoren das Internet.

Eine Studie des Unternehmens Bitkom zeigt, dass sich über 70\% der befragten Personen in Deutschland
im Internet über Dienstleistung und Waren informieren. Das gilt dabei insbesondere für junge
Erwachsene\footnote{\url{http://www.bitkom.org/de/themen/54842_71833.aspx}}.
Diese Marktkennzahlen zeigen, dass es sich auch für eine Hochschule lohnt die eigenen
Stärken im Internet zu veröffentlichen. An diesem Ansatz setzt das vorliegende Projekt
"`virtueller Campus Lingen"', kurz VCL, an. Ziel dieses Projektes ist es dabei
eine visuelle Darstellung des neuen Campus zu schaffen, die eine junge Zielgruppe direkt anspricht
und im Internet für diese Zielgruppe zugänglich ist.

Um diese Zielerreichung innerhalb dieses Projektes aufzuzeigen wird in vorliegender Ausarbeitung
ein Einblick in die Entwicklung des Projektes aus unternehmerischer Sicht gegeben.
Im Rahmen dieser Betrachtung wird dabei, ausgehend von einem Projektauftrag der Hochschule,
eine Konzeptionierungsphase beschrieben, aus der die genaue Umsetzung des Projektes in Form
eines Lastenheftes hervorgeht. Das hier entwickelte Feinkonzept beschreibt dabei
in welchem Rahmen der neue Campus in Lingen visualisiert werden und darüber hinaus
wie die Darstellung des Campus und damit das vorliegende Projekt kontinuierlich aktuell
gehalten werden kann. Die darin vorgestellte konkretisierte Projektidee wird 
daraufhin umgesetzt und reflektiert.

In dieser Projektreflektion werden geplante und benötigte
Ressourcen in einem Soll-Ist-Vergleich gegenübergestellt und die Erreichung der zuvor
aufgestellten Projektziele verifiziert. Für den Auftraggeber werden an dieser die
angefallenen Kosten den erreichten Projektzielen und somit dem Nutzen des
Projektes gegenübergestellt. Weiterhin werden die subjektiven Eindrücke der
Projektmitglieder in Bezug auf die Kommunikation, das Verhalten in der Gruppe
und die Betreuung der Arbeit durch die Dozenten reflektiert.

Abschließend werden Möglichkeiten der Projekterweiterung in einem Ausblick vorgestellt
und der Hochschule Osnabrück wird als Auftraggeber eine Handlungsemphlung für
die Verwendung des Projektes ausgesprochen.