\section{Einleitung}
\label{sec:Einleitung}
% TODO: Wirtschaftlichkeitsanalyse und oder Nutzwertanalyse im Rahmen der Konzeptentwicklung mit einbeziehen?
% TODO: Untersuchung der Umsetzungmöglichkeiten muss in den Prozess der Konzeptenwicklung mit rein
Für viele Menschen stellt das Internet in der heutigen Zeit das wichtigste
Medium zur Informationsgewinnung dar. Durch die Entwicklung des Internets hin
zum Web 2.0 werden dem Nutzer immer neue Möglichkeiten geboten, auf das
umfassende Angebot an Informationen im Internet zuzugreifen. Der Zugriff auf
dieses Informationsangebot gestaltet sich hierbei zunehmend interaktiver.
Diese Interaktivität wird immer häufiger auch von Unternehmen genutzt.
Unternehmen präsentieren sich nicht mehr nur auf der firmeneigenen
Internetseite, sondern betreiben Marketing in sozialen Netzwerken,
veröffentlichen Neuigkeiten in Blogs und aquirieren neue Mitarbeiter über
Internetplatformen. Der Trend hin zur Nutzung dieser modernen
Kommunikationsmittel zeigt, dass es hiermit möglich ist weltweit eine große
Zielgruppe anzusprechen und auf sich aufmerksam machen zu können.

Im Rahmen des Projektes "`Virtueller Campus Lingen"' sollen diese neuen
Möglichkeiten der Internetpräsenz auch für den Standort Lingen der Hochschule
Osnabück nutzbar gemacht werden. Hierbei soll der neue Campus des
Studienstandorts Lingen einer breiten Masse an Studieninteressierten
präsentiert werden.

Bei der Durchführung eines Projektes von diesem Ausmaß ist es wichtig die
Komplexität handhabbar zu machen und das Projekt möglichst genau zu planen,
steuern und kontrollieren zu können. In dieser Ausarbeitung soll aus diesem
Grund die Vorgehensweise bei der Auftragsfindung, der Konzeptionierung,
Umsetzung und Reflexion des Projektes "`Virtueller Campus Lingen"' aus Sicht der
Unternehmensführung dargestellt werden. Zunächst wird im Abschnitt
"`Projektauftrag"' die Projektidee, sowie das Projektumfeld und die Projektziele
herausgestellt. Hierauf aufbauend soll dann auf die Entwicklung des
Projektkonzeptes im Abschnitt "`Konzeptionierung"' eingegangen werden.
Nachfolgend werden dann die für die Umsetzung des Projektes benötigten
Ressourcen Zeit und Kosten geplant. Daraufhin werden im Abschnitt
"`Projektdurchführung"' die Organisation der Projektgruppe und die einzelnen
Phasen, in denen das Projekt durchgeführt wurde, erläutert. In der
darauffolgenden Projektreflexion werden geplante und benötigte Ressourcen in
einem Soll-Ist-Vergleich gegenübergestellt und die Erreichung der zuvor
aufgestellten Projektziele verifiziert. Weiterhin werden die subjektiven
Eindrücke der Projektmitglieder im Bezug auf die Kommunikation, das Verhalten
in der Gruppe und die Betreuuung der Arbeit durch die Dozenten reflektiert. Nach
der Reflektion wird ein Ausblick auf eine mögliche Weiterführung des Projektes
gegeben. Abschließend wird eine Handlungsempfehlung bezüglich der 
Veröffentlichung und Erweiterung des Projektes ausgesprochen.