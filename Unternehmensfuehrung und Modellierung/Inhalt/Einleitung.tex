\section{Einleitung}
\label{sec:Einleitung}

Für viele Menschen stellt das Internet in der heutigen Zeit das wichtigste
Medium zur Informationsgewinnung dar. Durch die Entwicklung des Internets hin
zum Web 2.0 werden dem Nutzer immer neue Möglichkeiten geboten, auf das
umfassende Angebot an Informationen im Internet zuzugreifen. Der Zugriff auf
dieses Informationsangebot gestaltet sich hierbei zunehmend interaktiver.
Diese Interaktivität wird immer häufiger auch von Unternehmen genutzt.
Unternehmen präsentieren sich nicht mehr nur auf der firmeneigenen
Internetseite, sondern betreiben Marketing in sozialen Netzwerken,
veröffentlichen Neuigkeiten in Blogs und aquirieren neue Mitarbeiter über
Internetplatformen. Der Trend hin zur Nutzung dieser modernen
Kommunikationsmittel zeigt, dass es hiermit möglich ist weltweit eine große
Zielgruppe anzusprechen und auf sich aufmerksam machen zu können.

Im Rahmen des Projektes "`Virtueller Campus Lingen"' sollen diese neuen
Möglichkeiten der Internetpräsenz auch für den Standort Lingen der Hochschule
Osnabück nutzbar gemacht werden. Hierbei soll der neue Campus des
Studienstandorts Lingen einer breiten Masse an Studieninteressierten
präsentiert werden.

In dieser Ausarbeitung soll die Vorgehensweise bei der Umsetzung des Projektes
"`Virtueller Campus Lingen"' aus Sicht der Unternehmensführung dargestellt
werden. Die Ausarbeitung gliedert sich hierbei in die zwei Bereich
"`Projektplanung"' und "`Projektcontrolling"'. Im Bereich der Projektplanung
soll eine Struktur entwickelt werden, die die Grundlage der späteren
Projektumsetzung bildet. Hierbei wird aufbauend auf der Projektbegründung das an
dem Projekt beteiligte Umfeld und das Ziel des Projektes dargestellt. Weiterhin
werden die einzelnen Phasen, in die sich das Projekt gliedert, herausgestellt
und eine Projetstruktur erarbeitet. Nachfolgend werden sowohl der zeitliche
Ablauf des Projektes als auch die Ressoucen und Kosten für die Projektumsetzung
geplant. Die Projektplanung endet mit einer Analyse der Risiken, die während des
Projektes auftreten können. Im zweiten Teil der Ausarbeitung, dem
Projektcontrolling, soll auf die Maßnahmen zur Kontrolle der Projektdurchführung
und zur Sicherung der Zielerreichung eingegangen werden. Im nachfolgenden
Fazit werden die wichtigsten Aspekte der Projektplanung und des
Projektcontrollings noch einmal kurz zusammengefasst. Abschließend erfolgt
dann eine kritische Reflexion der Vorgehensweise im Projekt.