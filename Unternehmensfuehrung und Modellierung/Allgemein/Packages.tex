% Anpassung des Seitenlayouts --------------------------------------------------
%   siehe Seitenstil.tex
% ------------------------------------------------------------------------------
\usepackage[
    automark, % Kapitelangaben in Kopfzeile automatisch erstellen
    headsepline, % Trennlinie unter Kopfzeile
    ilines % Trennlinie linksb�ndig ausrichten
]{scrpage2}

% Anpassung an Landessprache ---------------------------------------------------
\usepackage[ngerman]{babel}

% Umlaute ----------------------------------------------------------------------
%   Umlaute/Sonderzeichen wie äüöß direkt im Quelltext verwenden (CodePage).
%   Erlaubt automatische Trennung von Worten mit Umlauten.
% ------------------------------------------------------------------------------
\usepackage[utf8x]{inputenc}
\usepackage[T1]{fontenc}
\usepackage{textcomp} % Euro-Zeichen etc.

% Schrift ----------------------------------------------------------------------
\usepackage{lmodern} % bessere Fonts
\usepackage{relsize} % Schriftgröße relativ festlegen

% Tabellen ---------------------------------------------------------------------
\PassOptionsToPackage{table}{xcolor}
\usepackage{tabularx}
\usepackage{booktabs}
% für lange Tabellen -----------------------------------------------------------
\usepackage{longtable}
\usepackage{array}
\usepackage{ragged2e}
\usepackage{lscape}
\newcolumntype{w}[1]{>{\raggedleft\hspace{0pt}}p{#1}}

% Grafiken ---------------------------------------------------------------------
% Einbinden von JPG-Grafiken ermöglichen
\usepackage[dvips,final]{graphicx}
% hier liegen die Bilder des Dokuments
\graphicspath{{Bilder/}}

% Befehle aus AMSTeX für mathematische Symbole z.B. \boldsymbol \mathbb --------
\usepackage{amsmath,amsfonts}

% für Index-Ausgabe mit \printindex --------------------------------------------
\usepackage{makeidx}

% Einfache Definition der Zeilenabst�nde und Seitenr�nder etc. -----------------
\usepackage{setspace}
\usepackage{geometry}

% Symbolverzeichnis ------------------------------------------------------------
%   Symbolverzeichnisse bequem erstellen. Beruht auf MakeIndex:
%     makeindex.exe %Name%.nlo -s nomencl.ist -o %Name%.nls
%   erzeugt dann das Verzeichnis. Dieser Befehl kann z.B. im TeXnicCenter
%   als Postprozessor eingetragen werden, damit er nicht ständig manuell
%   ausgeführt werden muss.
%   Die Definitionen sind ausgegliedert in die Datei "Glossar.tex".
% ------------------------------------------------------------------------------
\usepackage[intoc]{nomencl}
\let\abbrev\nomenclature
\renewcommand{\nomname}{Abkürzungsverzeichnis}
\setlength{\nomlabelwidth}{.25\hsize}
\renewcommand{\nomlabel}[1]{#1 \dotfill}
\setlength{\nomitemsep}{-\parsep}

% zum Umfließen von Bildern ----------------------------------------------------
\usepackage{floatflt}
\usepackage{float}

% zum Einbinden von PDFs ------------------------------------------------------
\usepackage{pdfpages}

% zum Einbinden von Programmcode -----------------------------------------------
\usepackage{listings}
% Einstellungen zum listings-Package
\lstset{
    float=hbp,
    basicstyle=\footnotesize,
    columns=flexible,
    tabsize=2,
    frame=single,
    extendedchars=true,
    showspaces=false,
    showstringspaces=false,
    numbers=left,
    numberstyle=\tiny,
    breaklines=true,
    breakautoindent=true,
    captionpos=b,
    }
\usepackage{xcolor} 
\definecolor{hellgelb}{rgb}{1,1,0.9}
\definecolor{colKeys}{rgb}{0,0,1}
\definecolor{colIdentifier}{rgb}{0,0,0}
\definecolor{colComments}{rgb}{0,0.5,0}
\definecolor{colString}{rgb}{1,0,0}
\lstset{
    float=hbp,
    basicstyle=\ttfamily\color{black}\small\smaller,
    identifierstyle=\color{colIdentifier},
    keywordstyle=\color{colKeys},
    stringstyle=\color{colString},
    commentstyle=\color{colComments},
    columns=flexible,
    tabsize=2,
    frame=single,
    extendedchars=true,
    showspaces=false,
    showstringspaces=false,
    numbers=left,
    numberstyle=\tiny,
    breaklines=true,
    backgroundcolor=\color{hellgelb},
    breakautoindent=true
}
% http://blog.yeticode.co.uk/2009/04/c-sharp-style-for-lstinputlisting/
\lstdefinelanguage{cs}
  {morekeywords={abstract,event,new,struct,as,explicit,null,switch
		base,extern,object,this,bool,false,operator,throw,
		break,finally,out,true,byte,fixed,override,try,
		case,float,params,typeof,catch,for,private,uint,
		char,foreach,protected,ulong,checked,goto,public,unchecked,
		class,if,readonly,unsafe,const,implicit,ref,ushort,
		continue,in,return,using,decimal,int,sbyte,virtual,
		default,interface,sealed,volatile,delegate,internal,short,void,
		do,is,sizeof,while,double,lock,stackalloc,
		else,long,static,enum,namespace,string, },
	  sensitive=false,
	  morecomment=[l]{//},
	  morecomment=[s]{/*}{*/},
	  morestring=[b]",
}
\lstdefinelanguage{natural}
  {morekeywords={DEFINE,DATA,LOCAL,END-DEFINE,WRITE,CALLNAT,PARAMETER,USING,%
               IF,NOT,END-IF,ON,*ERROR-NR,ERROR,END-ERROR,ESCAPE,ROUTINE,%
               PERFORM,SUBROUTINE,END-SUBROUTINE,CONST,END-FOR,END,FOR,RESIZE,%
               ARRAY,TO,BY,VALUE,RESET,COMPRESS,INTO,EQ},
	  sensitive=false,
	  morecomment=[l]{/*},
	  morestring=[b]",
	  morestring=[b]',
	  alsodigit={-,*},
}


% URL verlinken, lange URLs umbrechen etc. -------------------------------------
\usepackage{url}

% wichtig für korrekte Zitierweise ---------------------------------------------
\usepackage[square]{natbib}

% PDF-Optionen -----------------------------------------------------------------
\usepackage[
    bookmarks,
    bookmarksnumbered,    
    bookmarksopen=true,
    bookmarksopenlevel=1,
    colorlinks=true,
% diese Farbdefinitionen zeichnen Links im PDF farblich aus
    %linkcolor=red, % einfache interne Verkn�pfungen
    %anchorcolor=black,% Ankertext
    %citecolor=blue, % Verweise auf Literaturverzeichniseintr�ge im Text
    %filecolor=magenta, % Verkn�pfungen, die lokale Dateien �ffnen
    %menucolor=red, % Acrobat-Men�punkte
    %urlcolor=cyan, 
% diese Farbdefinitionen sollten f�r den Druck verwendet werden (alles schwarz)
    linkcolor=black, % einfache interne Verkn�pfungen
    anchorcolor=black, % Ankertext
    citecolor=black, % Verweise auf Literaturverzeichniseintr�ge im Text
    filecolor=black, % Verkn�pfungen, die lokale Dateien �ffnen
    menucolor=black, % Acrobat-Men�punkte
    urlcolor=black, 
%
    backref,
    pdftex,
    plainpages=false, % zur korrekten Erstellung der Bookmarks
    pdfpagelabels=true, % zur korrekten Erstellung der Bookmarks
    hypertexnames=true, % zur korrekten Erstellung der Bookmarks
    linktocpage % Seitenzahlen anstatt Text im Inhaltsverzeichnis verlinken
]{hyperref}
% Befehle, die Umlaute ausgeben, f�hren zu Fehlern, wenn sie hyperref als Optionen �bergeben werden
\hypersetup{
    pdftitle={\titel \untertitel},
    pdfauthor={\autorA},
    pdfcreator={\autorA},
    pdfsubject={\titel \untertitel},
    pdfkeywords={\titel \untertitel},
}

% fortlaufendes Durchnummerieren der Fu�noten ----------------------------------
\usepackage{chngcntr}

% Formatierung von Listen �ndern -----------------------------------------------
\usepackage{paralist}

% bei der Definition eigener Befehle benötigt
\usepackage{ifthen}

% definiert u.a. die Befehle \todo und \listoftodos
\usepackage{todonotes}

% sorgt dafür, dass Leerzeichen hinter parameterlosen Makros nicht als
% Makroendezeichen interpretiert werden
\usepackage{xspace}

% Erstellen von Struktogrammen (Nassi-Shneidermann)
\usepackage{struktex}

% Durchstreichen von Text
\usepackage{cancel}

% Packages zur Gestaltung von Tabellen
\usepackage{color}
\usepackage{colortbl}
\usepackage{tabularx} 
\usepackage{multicol} 
\usepackage{multirow} 
\usepackage{booktabs}

% Farben für Tabellen
\definecolor{dunkelgrau}{rgb}{0.8,0.8,0.8}
\definecolor{hellgrau}{rgb}{0.95,0.95,0.95}
\definecolor{heading}{rgb}{0.64,0.78,0.86}
\definecolor{odd}{rgb}{0.9,0.9,0.9}

\usepackage{subfigure}

\usepackage{acronym}