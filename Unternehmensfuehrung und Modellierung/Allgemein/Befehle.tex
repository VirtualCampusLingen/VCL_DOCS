% Eigene Befehle und typographische Auszeichnungen f�r diese

% einfaches Wechseln der Schrift, z.B.: \changefont{cmss}{sbc}{n}
\newcommand{\changefont}[3]{\fontfamily{#1} \fontseries{#2} \fontshape{#3} \selectfont}

% Abk�rzungen mit korrektem Leerraum 
\newcommand{\ca}{ca.\ }
\newcommand{\Vgl}{Vgl.\ }
\newcommand{\bzw}{bzw.\ }
\newcommand{\etc}{etc.\ }
\newcommand{\inkl}{inkl.\ }
\newcommand{\evtl}{evtl.\ }
\newcommand{\ggfs}{ggfs.\ }
\newcommand{\usw}{usw.\ }

\newcommand{\abbildung}[1]{Abbildung~\ref{fig:#1} (\nameref{fig:#1})}
\newcommand{\tabelle}[1]{Tabelle~\ref{tab:#1} (\nameref{tab:#1})}
\newcommand{\listing}[1]{Listing~\ref{lst:#1} (\nameref{lst:#1})}
\newcommand{\verweis}[1]{Abschnitt~\ref{sec:#1} (\nameref{sec:#1})}
\newcommand{\anhang}[1]{Anhang~\ref{sec:#1} (\nameref{sec:#1})}

\newcommand{\bs}{$\backslash$}

\newcommand{\AO}{\textsc{Alte Oldenburger}\xspace}

% Setzt ein Wort in Anführungszeichen
\newcommand{\gqq}[1]{\glqq{}#1\grqq{}}

% erzeugt ein Listenelement mit fetter Überschrift 
\newcommand{\itemd}[2]{\item{\textbf{#1}}\\{#2}}

% einige Befehle zum Zitieren --------------------------------------------------
\newcommand{\Zitat}[2][\empty]{\ifthenelse{\equal{#1}{\empty}}{\citep{#2}}{\citep[#1]{#2}}}

% zum Ausgeben von Autoren
\newcommand{\Autor}[1]{\textsc{#1}}

% verschiedene Befehle um Wörter semantisch auszuzeichnen ----------------------
\newcommand{\Index}[2][\empty]{\ifthenelse{\equal{#1}{\empty}}{\index{#2}#2}{\index{#1}#2}}
\newcommand{\Fachbegriff}[2][\empty]{\ifthenelse{\equal{#1}{\empty}}{\textit{\Index{#2}}}{\textit{\Index[#1]{#2}}}}
\newcommand{\NeuerBegriff}[2][\empty]{\ifthenelse{\equal{#1}{\empty}}{\textbf{\Index{#2}}}{\textbf{\Index[#1]{#2}}}}

\newcommand{\Ausgabe}[1]{\texttt{#1}}
\newcommand{\Eingabe}[1]{\texttt{#1}}
\newcommand{\Code}[1]{\texttt{#1}}
\newcommand{\Datei}[1]{\texttt{#1}}

\newcommand{\Assembly}[1]{\textsf{#1}}
\newcommand{\Klasse}[1]{\textsf{#1}}
\newcommand{\Methode}[1]{\textsf{#1}}
\newcommand{\Attribut}[1]{\textsf{#1}}

\newcommand{\Datentyp}[1]{\textsf{#1}}
\newcommand{\XMLElement}[1]{\textsf{#1}}
\newcommand{\Webservice}[1]{\textsf{#1}}

\newcommand{\Refactoring}[1]{\Fachbegriff{#1}}
\newcommand{\CodeSmell}[1]{\Fachbegriff{#1}}
\newcommand{\Metrik}[1]{\Fachbegriff{#1}}
\newcommand{\DesignPattern}[1]{\Fachbegriff{#1}}
