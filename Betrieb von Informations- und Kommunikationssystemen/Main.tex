% HS Vorlage
% ------------------------------------------------------------------------------
%   erstellt von Jannik Fangmann, 08.02.2013   
 
\documentclass[
    11pt, % Schriftgröße
    DIV10,
    ngerman, % für Umlaute, Silbentrennung etc.
    a4paper, % Papierformat
    oneside, % einseitiges Dokument
    titlepage, % es wird eine Titelseite verwendet
    parskip=half, % Abstand zwischen Absätzen (halbe Zeile)
    headings=normal, % Größe der Überschriften verkleinern
    listof=totoc, % Verzeichnisse im Inhaltsverzeichnis aufführen
    bibliography=totoc, % Literaturverzeichnis im Inhaltsverzeichnis aufführen
    index=totoc, % Index im Inhaltsverzeichnis aufführen
    captions=tableheading, % Beschriftung von Tabellen unterhalb ausgeben
    final % Status des Dokuments (final/draft)
]{scrartcl}

% Meta-Informationen -----------------------------------------------------------
%   Informationen über das Dokument, wie z.B. Titel, Autor, Matrikelnr. etc
%   werden in der Datei Meta.tex definiert und können danach global
%   verwendet werden.
% ------------------------------------------------------------------------------
% Meta-Informationen -----------------------------------------------------------
%   Definition von globalen Parametern, die im gesamten Dokument verwendet
%   werden können (z.B auf dem Deckblatt etc.).
%
%   ACHTUNG: Wenn die Texte Umlaute oder ein Esszet enthalten, muss der folgende
%            Befehl bereits an dieser Stelle aktiviert werden:
            \usepackage[utf8x]{inputenc}
% ------------------------------------------------------------------------------
\newcommand{\titel}{Virtueller Campus Lingen}
\newcommand{\untertitel}{Unternehmensführung}
\newcommand{\art}{Ausarbeitung}
\newcommand{\autorA}{Jannik Fangmann}
\newcommand{\autorB}{Andreas Makeev}
\newcommand{\autorC}{Raphael Otten}
\newcommand{\autorD}{Carsten Sandker}
\newcommand{\matrikelnrA}{506347}
\newcommand{\matrikelnrB}{517007}
\newcommand{\matrikelnrC}{516975}
\newcommand{\matrikelnrD}{500199}
\newcommand{\studienbereich}{Wirtschaftsinformatik}
\newcommand{\semester}{6}
\newcommand{\gutachterA}{Stephan Feldker (B. Eng.)}
\newcommand{\gutachterB}{Prof. Dr.-Ing. Ralf Westerbusch}
\newcommand{\abgabedatum}{\today}
\newcommand{\jahr}{2014}
\newcommand{\ort}{Lingen}
\newcommand{\logo}{IDS.jpg}
\newcommand{\logoSw}{IDS_SW.jpg}

% Erstellung eines Index aktivieren --------------------------------------------
\makeindex

\input{Allgemein/Packages}

% Kopf- und Fußzeilen, Seitenränder etc. ---------------------------------------
% Zeilenabstand 1,5 Zeilen -----------------------------------------------------
\onehalfspacing

% Überschriften (Chapter höher setzen)
\renewcommand*{\chapterheadstartvskip}{\vspace*{-\topskip}}

% Seitenränder -----------------------------------------------------------------
\setlength{\topskip}{\ht\strutbox} % behebt Warnung von geometry
\geometry{paper=a4paper,left=20mm,right=40mm,top=20mm,bottom=45mm}

% Kopf- und Fußzeilen ----------------------------------------------------------
\pagestyle{scrheadings}
% Kopf- und Fußzeile auch auf Kapitelanfangsseiten
%\renewcommand*{\chapterpagestyle}{scrheadings} 
% Schriftform der Kopfzeile
\renewcommand{\headfont}{\normalfont}

% Kopfzeile
\ihead{\large{\textsc{\titel}}\\ \small{\untertitel} \\[2ex]
\textit{\headmark}}
\chead{}
\ohead{\includegraphics[scale=0.5]{\logo}}
\setlength{\headheight}{21mm} % Höhe der Kopfzeile
% Kopfzeile über den Text hinaus verbreitern
\setheadwidth[0pt]{textwithmarginpar} 
\setheadsepline[text]{0.4pt} % Trennlinie unter Kopfzeile

% Fußzeile
\ifoot{\copyright \ \autorA}
\cfoot{}
\ofoot{\pagemark}

% Größe der Fußzeilenschriftgröße auf 9pt anpassen
\def\footnotesize{\fontsize{9pt}{10pt}\selectfont}

% sonstige typographische Einstellungen ----------------------------------------

% erzeugt ein wenig mehr Platz hinter einem Punkt
\frenchspacing 

% Schusterjungen und Hurenkinder vermeiden
\clubpenalty = 10000
\widowpenalty = 10000 
\displaywidowpenalty = 10000

% Quellcode-Ausgabe formatieren
\lstset{numbers=left, numberstyle=\tiny, numbersep=5pt, breaklines=true}
\lstset{emph={square}, emphstyle=\color{red}, emph={[2]root,base}, emphstyle={[2]\color{blue}}}

% Fu�noten fortlaufend durchnummerieren
%\counterwithout{footnote}{chapter}


% eigene LaTeX-Befehle
% Eigene Befehle und typographische Auszeichnungen f�r diese

% einfaches Wechseln der Schrift, z.B.: \changefont{cmss}{sbc}{n}
\newcommand{\changefont}[3]{\fontfamily{#1} \fontseries{#2} \fontshape{#3} \selectfont}

% Abk�rzungen mit korrektem Leerraum 
\newcommand{\ca}{ca.\ }
\newcommand{\Vgl}{Vgl.\ }
\newcommand{\bzw}{bzw.\ }
\newcommand{\etc}{etc.\ }
\newcommand{\inkl}{inkl.\ }
\newcommand{\evtl}{evtl.\ }
\newcommand{\ggfs}{ggfs.\ }
\newcommand{\usw}{usw.\ }

\newcommand{\abbildung}[1]{Abbildung~\ref{fig:#1} (\nameref{fig:#1})}
\newcommand{\tabelle}[1]{Tabelle~\ref{tab:#1} (\nameref{tab:#1})}
\newcommand{\listing}[1]{Listing~\ref{lst:#1} (\nameref{lst:#1})}
\newcommand{\verweis}[1]{Abschnitt~\ref{sec:#1} (\nameref{sec:#1})}
\newcommand{\anhang}[1]{Anhang~\ref{sec:#1} (\nameref{sec:#1})}

\newcommand{\bs}{$\backslash$}

\newcommand{\AO}{\textsc{Alte Oldenburger}\xspace}

% Setzt ein Wort in Anführungszeichen
\newcommand{\gqq}[1]{\glqq{}#1\grqq{}}

% erzeugt ein Listenelement mit fetter Überschrift 
\newcommand{\itemd}[2]{\item{\textbf{#1}}\\{#2}}

% einige Befehle zum Zitieren --------------------------------------------------
\newcommand{\Zitat}[2][\empty]{\ifthenelse{\equal{#1}{\empty}}{\citep{#2}}{\citep[#1]{#2}}}

% zum Ausgeben von Autoren
\newcommand{\Autor}[1]{\textsc{#1}}

% verschiedene Befehle um Wörter semantisch auszuzeichnen ----------------------
\newcommand{\Index}[2][\empty]{\ifthenelse{\equal{#1}{\empty}}{\index{#2}#2}{\index{#1}#2}}
\newcommand{\Fachbegriff}[2][\empty]{\ifthenelse{\equal{#1}{\empty}}{\textit{\Index{#2}}}{\textit{\Index[#1]{#2}}}}
\newcommand{\NeuerBegriff}[2][\empty]{\ifthenelse{\equal{#1}{\empty}}{\textbf{\Index{#2}}}{\textbf{\Index[#1]{#2}}}}

\newcommand{\Ausgabe}[1]{\texttt{#1}}
\newcommand{\Eingabe}[1]{\texttt{#1}}
\newcommand{\Code}[1]{\texttt{#1}}
\newcommand{\Datei}[1]{\texttt{#1}}

\newcommand{\Assembly}[1]{\textsf{#1}}
\newcommand{\Klasse}[1]{\textsf{#1}}
\newcommand{\Methode}[1]{\textsf{#1}}
\newcommand{\Attribut}[1]{\textsf{#1}}

\newcommand{\Datentyp}[1]{\textsf{#1}}
\newcommand{\XMLElement}[1]{\textsf{#1}}
\newcommand{\Webservice}[1]{\textsf{#1}}

\newcommand{\Refactoring}[1]{\Fachbegriff{#1}}
\newcommand{\CodeSmell}[1]{\Fachbegriff{#1}}
\newcommand{\Metrik}[1]{\Fachbegriff{#1}}
\newcommand{\DesignPattern}[1]{\Fachbegriff{#1}}


% Das eigentliche Dokument -----------------------------------------------------
%   Der eigentliche Inhalt des Dokuments beginnt hier. Die einzelnen Seiten
%   und Kapitel werden in eigene Dateien ausgelagert und hier nur inkludiert.
% ------------------------------------------------------------------------------
\begin{document}

% auch subsubsection nummerieren
\setcounter{secnumdepth}{3}
\setcounter{tocdepth}{3}

% Deckblatt und Abstract ohne Seitenzahl
\ofoot{}
\thispagestyle{plain}
\begin{titlepage}

\begin{center}

\Huge{\textbf{\titel}}\\[1.4ex]
\huge{mit der Themenstellung:\\ {\untertitel}}\\[6ex]

\includegraphics[scale=1.2]{HS_Osna_MKT.jpg}\\[7ex]

\normalsize
\begin{tabular}{w{5.4cm}p{6cm}}\\
vorgelegt von:  & \quad \autorA\\[1.2ex]
Studienbereich: & \quad \studienbereich\\[1.2ex]
Semester: & \quad \semester\\[1.2ex]
Modul: & \quad \modul\\[1.2ex]
  & \quad \themenstellung\\[1.2ex]
Matrikelnummer: & \quad \matrikelnr\\[1.2ex]
Erstgutachter:  & \quad \erstgutachter\\[1.2ex]
Abgabedatum: & \quad \abgabedatum\\[2.4ex]
\end{tabular}

\copyright\ \jahr\\[12ex]

\end{center}

\singlespacing
\small
\noindent Dieses Werk einschließlich seiner Teile ist \textbf{urheberrechtlich
geschützt}. Jede Verwertung außerhalb der engen Grenzen des Urheberrechtgesetzes
ist ohne Zustimmung des Autors unzulässig und strafbar. Das gilt insbesondere
für Vervielfältigungen, Übersetzungen, Mikroverfilmungen sowie die
Einspeicherung und Verarbeitung in elektronischen Systemen.

\end{titlepage}


% Abstract----------------------------------------------------------------------
% \include{Inhalt/Abstract}
% ------------------------------------------------------------------------------
\ofoot{\pagemark}

% Seitennummerierung -----------------------------------------------------------
%   Vor dem Hauptteil werden die Seiten in gro�en r�mischen Ziffern 
%   nummeriert.
% ------------------------------------------------------------------------------

\pagenumbering{Roman}
\pdfbookmark[1]{Inhaltsverzeichnis}{toc} % Inhaltsverzeichnis als PDF-Bookmark
\tableofcontents % Inhaltsverzeichnis

% Abk�rzungsverzeichnis --------------------------------------------------------
% \input{Inhalt/Glossar}

% für korrekte Überschrift in der Kopfzeile
% \clearpage
%\markboth{\nomname}{\nomname} 

% Abbildungsverzeichnis --------------------------------------------------------
% \listoffigures 
% ------------------------------------------------------------------------------

% arabische Seitenzahlen im Hauptteil ------------------------------------------
\clearpage
\pagenumbering{arabic}

% die Inhaltskapitel werden in "Inhalt.tex" inkludiert -------------------------
% Hier können die einzelnen Kapitel inkludiert werden. Sie müssen in den 
% entsprechenden .TEX-Dateien vorliegen. Die Dateinamen können natürlich 
% angepasst werden.

\section{Zielbestimmung}
\label{sec:Zielbestimmung}

Ziel des Projektes Virtueller Campus Lingen ist es den Standort der Hochschule Osnabrück
in Lingen virtuell darzustellen sowie alle vier Institutionen mit ihren Studienangeboten
vorzustellen.

Der virtuelle Rundgang über den Campus Lingen soll Studieninteressierte auf diesen auf-
merksam machen, das Studieren in Lingen aus einer modernen, interaktiven Perspektive
zeigen und bestenfalls vom Studienangebot in Lingen überzeugen. Hierzu soll das gesamte
Gebäude der Hochschule in Lingen an der Kaiserstraße sowie der Standort an der Bac-
cumer Str. fotografiert werden. Diese Fotos werden, im Stile von Googl-Street-View, zu
360-Grad-Panoramen zusammengesetzt, in denen sich der Nutzer frei umschauen kann.

Dem Anwender soll auf diese Weise ein neuartiger Einblick in die Räumlichkeiten und auf
das Umfeld der Hochschule geboten werden. Es soll nicht nur visuell von der Hochschule
überzeugt werden, sondern auch mit Inhalt und Möglichkeiten.

In diesem Rahmen werden Informationen und Wissenswertes dem Nutzer an passenden
Stellen angezeigt. Zum Beispiel kann am Eingangsbereich einer Fakultät eine Infobox
angezeigt werden.

Auf dieser Basis soll, eingebettet in eine ansprechende Web-Oberfläche, eine Web-Anwendung
entstehen, die per Verlinkung auf die aktuelle Webseite der Hochschule eingebunden wer-
den kann.

Neben der öffentlich sichtbaren Ansicht soll es zusätzlich ein Backend-System zur War-
tung und Pflege der Anwendung geben. Hier sollen Texte, Fotos und Informationen von
Mitarbeitern der Hochschule aktualisiert werden können, um die Anwendung auch für die
Zukunft immer aktuell halten zu können.

\section{Produkteinsatz}
\label{sec:Produkteinsatz}

Ziel des Projektes „Virtueller Campus Lingen“ ist es, durch einen virtuellen Rundgang
über den Campus Studieninteressierte auf diesen aufmerksam zu machen und bestenfalls
von ihm und seinen Angeboten zu überzeugen.

Durch moderne Optik sollen Studieninteressierte auf der Suche nach einem geeigneten
Studienort auf den ersten Blick vom Standort Lingen überzeugt werden. Dieses Projekt
soll nicht nur im Internet abrufbar sein, sondern auch als Werbemaßnahme auf Messen
und Ausstellungen Interessenten für den modernen Campus Lingen begeistern.
\section{Produkfunktionen}
\label{sec:Produkfunktionen}

\subsection{Benutzerfunktionen}
\label{sec:Benutzerfunktionen}

\begin{description}
  \item[F0010] Einem beliebigen Internetnutzer muss es möglich sein, ohne Anmeldung mit seinem
  im Internet befindlichen Endgerät mit einer Displaygröße von min. 8 Zoll auf die
  Anwendung zugreifen zu können.
  \item[F0015] Beim Starten der Anwendung soll dem Benutzer ein 360-Grad Panorama vom
  Eingangsbereich des Campus Lingen präsentiert werden. Von dieser Ansicht aus
  besteht die Möglichkeit den Campus zu erkunden (nächstes 360-Grad Foto). Er hat
  aber auch die Möglichkeit sich über die Übersichtkarte einen anderen Standpunkt
  als 360 Grad Foto anzeigen zu lassen.
  \item[F0020] Der Benutzer muss sich in den 360-Grad Fotos der Anwendung frei umsehen können. Außerdem hat er die Möglichkeit in das 360-Grad Foto rein und raus zu zoomen.
  \item[F0030] Der Benutzer muss zwischen den Fotos navigieren können. Hierzu steht ihm ein
  Steuerkreuz im unteren Anwendungsbereich zur Verfügung.
  \item[F0040] Die Anwendung muss eine Minimap enthalten, die sowohl den aktuellen Standpunkt, als auch alle weiteren Einstiegspunkte umfasst.
  \item[F0045] Die Minimap soll in der oberen linken Ecke des 360-Grad-Fotos angezeigt werden.
  Sie stellt einen Ausschnitt der Übersichtskarte an der aktuellen Position des Nutzers
  dar.
  \item[F0050] In den 360-Grad-Fotos müssen relevante Informationen zu den dargestellten Örtlichkeiten angezeigt werden können. Diese werden hinter einem Informationsicon als
  Popup hinterlegt. Sobald der Nutzer diese Icon anklickt erscheint ein Popup, welches die betreffenden Informationen beinhaltet. Außerdem kann das Popup einen
  weiterführenden Link auf die Hochschulwebseite für mehr Informationen enthalten.
\end{description}

\subsection{Administrationsfunktionen}
\label{sec:Administrationsfunktionen}

\begin{description}
  \item[F1010] Der Administrator muss sich per Passwort im Administrationsbereich anmelden können.
  \item[F1020] Der Administrationsbereich bietet ein strukturiertes Menü mit allen Anwendungsfunktionen, die der Administrator verwenden kann.
  \item[F1030] Der Administrator hat die Möglichkeit in einer Medienverwaltung 360-Grad Fotos hochzuladen und zu löschen. Hierzu steht ihm eine Miniaturansicht der 360-Grad Fotos, sowie der Dateinamen der Bilder, die sich auf dem Server befinden, zur Verfügung.
  \item[F1040] Eine definierte Karte muss dem Administrator als Positionierungseinschränkung für neue 360-Grad-Fotos zur Verfügung stehen.
  \item[F1050] Der Administrator muss neue Fotos in die Anwendung einpflegen können. Die Fotos können hierbei an beliebigen Punkten auf einer definierten Karte positioniert werden. Hierzu sieht der Administrator die Übersichtskarten des Campus und hat die Möglichkeit per Mauszeiger auf dem dort angezeigten Wegenetz Fotos frei positionieren oder ändern/löschen zu können.
  \item[F1060] Der Administrator muss veraltete Fotos in der Anwendung austauschen und durch bereits hochgeladene ersetzen können.
  \item[F1070] Der Administrator muss Informationstexte zu den 360-Grad Fotos hinzufügen, abändern und löschen können. Diese Informationstexte können zum Beispiel Projektvorstellungen enthalten. Der Informationstext besteht aus einem Pop-Up mit Einführungstext und weiterführenden Links.
  \item[F1080] Der Administrator muss Texte zu interessanten Orten erstellen können und diese mit bereits hochgeladenen Fots verbinden.
\end{description}
\subsection{Benutzerfunktionen}
\label{sec:Benutzerfunktionen}

\begin{description}
  \item[LF0010] Einem beliebigen Internetnutzer muss es möglich sein, ohne Anmeldung auf die
  Anwendung zugreifen zu können.
  \item[LF0015] Beim Starten der Anwendung soll dem Benutzer ein 360-Grad-Foto vom
  Eingangsbereich des Campus Lingen präsentiert werden.
  \item[LF0020] Der Benutzer muss sich in den 360-Grad-Fotos der Anwendung frei umsehen
  können.
  \item[LF0030] Der Benutzer muss zwischen den Fotos navigieren können.
  \item[LF0040] Die Anwendung muss eine Minimap enthalten, die den aktuellenStandpunkt anzeigt.
  \item[LF0050] Der Benuzer muss auf zu interessanten Orten, zum Beispiel Bibliothek oder Fachbereichsgebäuden schnell und direkt springen können, ohne sich dorthin navigieren zu müssen.
  \item[LF0060] In den 360-Grad-Fotos müssen relevante Informationen zu den dargestellten Ört-
  lichkeiten angezeigt werden können.
\end{description}
\subsection{Administrationsfunktionen}
\label{sec:Administrationsfunktionen}

Der Administrator ist ein Benutzer, der für den geregelten Ablauf des Systems verant-
wortlich ist. Er kann die Eigenschaften des Systems ändern bzw. konfigurieren.

\begin{description}
  \item[LF1010] Der Administrator muss neue Fotos in die Anwendung einpflegen können.
  \item[LF1020] Der Administrator muss veraltete Fotos in der Anwendung austauschen können.
  \item[LF1030] Der Administrator muss hinterlegte Informationen zu einzelnen Fotos, zum Beispiel Name und Beschreibung, verändern können.
  \item[LF1040] Der Administrator muss Informationstexte speichern können.
  \item[LF1050] Der Administrator muss Informationstexte löschen können.
  \item[LF1060] Der Administrator muss den Informationstext abändern können.
  \item[LF1070] Der Administrator muss interessante Orte, zum Beispiel Bibliothek oder Fachbereichsgebäide, mit einem entsprechenden Text erstellen können.
  \item[LF1080] Der Administrator muss interessante Ort löschen können.
  \item[LF1090] Der Administrator muss die hinterlegten Informationen zu interessanten Orten verändern können.
  \item[LF1100] Dem Administrator muss eine topographische Karte als Positionierungseinschränkung für neue
  360-Grad-Fotos zur Verfügung stehen.
  \item[LF1110] Der Administrator muss auf dieser Karte Fotos platzieren und löschen können.
\end{description}
\section{Produktdaten}
\label{sec:Produktdaten}

\begin{description}
  \item[D0010] 360-Grad Fotos
  \item[D0020] Informationstexte zu den 360-Grad Fotos
  \item[D0030] Informationstexte zu interessanten Orten
  \item[D0040] Übersichtskarten des Campus bwz. der einzelnen Gebäude
\end{description}
\section{Produkleistungen}
\label{sec:Produktleistungen}

Es liegen keine besonderen Ansprüche an die Produktleistungen vor.
\section{Qualitätsanforderungen}
\label{sec:Qualitaetsanforderungen}

Für die Anwendung ist sowohl die Portierbarkeit als auch die Kompatibilität maßgebend.
Ebenso ist die intuitive Bedienbarkeit eine weitere Anforderung an die Qualität der An-
wendung. Bei der Umsetzung der Anwendung sollen ebenso Performanzansprüche berück-
sichtigt werden.
\section{Ergänzungen}
\label{sec:Ergaenzungen}

Eine erste Entwurfsskizze ist in \abbildung{Mockup} zu finden.
Weiterhin ist ein Grobentwurf der Infrastruktur in \abbildung{Systemaufbau} dargestellt.

\begin{figure}[htb]
\centering
\includegraphics[width=1.0\textwidth]{Mockup.jpg}
\caption[Mockup Benutzeransicht]{Entwurf der Benutzeroberfläche}
\label{fig:Mockup}
\end{figure}


\begin{figure}[htb]
\centering
\includegraphics[width=1.0\textwidth]{Systemaufbau.png}
\caption[Aufbau des Systems]{Aufbau des Systems}
\label{fig:Systemaufbau}
\end{figure}

% Literaturverzeichnis ---------------------------------------------------------
%   Das Literaturverzeichnis wird aus der BibTeX-Datenbank "Literatur.bib"
%   erstellt.
% ------------------------------------------------------------------------------
\clearpage
\bibliographystyle{natdin}
\bibliography{Literatur}

% Index ------------------------------------------------------------------------
% \printindex
 
\clearpage
\section*{Eidesstattliche Erklärung}
\label{sec:Eidestattliche Erklaerung}

Wir, \autorA, \autorB, \autorC \ und \autorD \ versichern hiermit, dass wir die
\art \ mit dem Thema
\begin{quote}
\textit{\untertitel}
\end{quote}
selbständig verfasst haben und keine anderen als die angegebenen Quellen und
Hilfsmittel benutzt haben, wobei wir alle wörtlichen und sinngemäßen Zitate als
solche gekennzeichnet haben. Die Arbeit wurde bisher keiner anderen
Prüfungsbehörde vorgelegt und auch nicht veröffentlicht.\\[6ex]

\ort, den \today
\\
\\
\rule[-0.2cm]{5cm}{0.5pt}

\textsc{\autorA} 
\\
\\
\rule[-0.2cm]{5cm}{0.5pt}

\textsc{\autorB}
\\
\\
\rule[-0.2cm]{5cm}{0.5pt}

\textsc{\autorC}
\\
\\
\rule[-0.2cm]{5cm}{0.5pt}

\textsc{\autorD}

 % Selbständigkeitserklärung

% Ausarbeitung auf CD ----------------------------------------------------------
% \addchap*{Ausarbeitung auf CD}
\label{cha:Ausarbeitung auf CD}
 
% ------------------------------------------------------------------------------

% Anhang -----------------------------------------------------------------------
%   Die Inhalte des Anhangs werden analog zu den Kapiteln inkludiert.
%   Dies geschieht in der Datei "Anhang.tex".
% ------------------------------------------------------------------------------
\clearpage
\begin{appendix}
    \pagenumbering{roman}
    % Rand der Aufz�hlungen in Tabellen anpassen
    \setdefaultleftmargin{1em}{}{}{}{}{}
    \section{Anhang}
\label{sec:Anhang}

\subsection{Javascript-Datei des Prototypen}
\label{sec:AnhangJavascriptPrototyp}
\lstinputlisting[language=JavaScript,caption={Javascript-Datei des Prototypen},label={lst:Javascript Prototyp}]{Listings/JS_Prototyp.js}
\end{appendix}

\end{document}
