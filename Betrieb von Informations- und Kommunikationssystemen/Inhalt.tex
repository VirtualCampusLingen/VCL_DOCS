% Hier können die einzelnen Kapitel inkludiert werden. Sie müssen in den 
% entsprechenden .TEX-Dateien vorliegen. Die Dateinamen können natürlich 
% angepasst werden.

\section{Einleitung}
\label{sec:Einleitung}

Das Projekt virtueller Campus Lingen, kurz VCL, ist ein Studienprojekt der Hochschule Osnabrück am Standort Lingen. Inahlt dieses Projektes ist eine virtuelle Darstellung des Studienstandortes durch 360-Grad Panoramafotos für Studieninteressierte und Besucher der Hochschulwebseite. Das Projektbild orientiert sich dabei stark an der Google Street View\copyright\ Darstellung von geographischen Orten\footnotemark. Durch eine solche virtuelle Darstellung des Hochschulstandortes Lingen, soll besonders jungen Studieninteressierten die Attraktivität des Campus aufgezeigt werden. Im folgenden Projekt wird daher eine virtuelle Darstellung geschaffen, die Bekanntheit, Image und Attraktivität des Hochschulstandortes Lingen steigern soll.

\footnotetext{siehe \nameref{fig:GoogleStreetViewBeispiel} im Anhang ~\ref{sec:BeispielEinerGoogleStreetViewAnsicht}}

Vorrausgegangen ist der folgenden Betrachtung dabei ein konzeptioneller Management- und Unternehmensführungsansatz zur Planung und Steuerung des Projekts. Der besondere Betrachtungsschwerpunkt wurde dabei auf Kosten, Nutzen und die Zielerreichung des Projektes gesetzt. Diese Inhalte werden im Folgenden nicht weiter betrachtet können aber in der Ausarbeitung "`Unternehmensfürhung und Konzeption ..."'\footnote{\citet{unternehmensfuehrung2014}} nachgelesen werden. Schwerpunkt der folgenden Betrachtung ist dagegen die Modellierung, die Umsetzung und der Betrieb des Projektes VCL.

Zur Einführung der Modellierung wird in folgender Ausarbeitung eine Analyse und Planungphase beschrieben, in der vor allem die Anforderung in Form eines Lastenheftes spezifiziert und die Projektgrenzen abgesteckt werden. Aufbauend auf den erarbeitetenden Erkenntinsse werden Modelle und Entwürfe erstellt, die die technische Umsetzung des Projektes und die wichtigsten Komponenten dieser Umsetzung beschreiben. Vorraussetzung für Planung und Entwurf dieser Modelle ist dabei ein technisches Grundverständis bestimmter eingesetzter Technologien. Diese Technologien werden aus diesem Grund im Vorfeld des Projektes kurz eingeführt und erläutert. Mit Hilfe der erstellten Modelle kann daraufhin das Projekt technisch Umgesetzt werden. Die Umsetzung gliedert sich dabei in die Bereiche der Panoramaerstellung und der Softwareerstellung. Beide Bereiche sind essentielle Elemente des Projektes und sind funktional voneinander abhängig. Nach Abschluss der Umsetzung ist das Projekt funktional fertiggestellt. Eine anschließende Testphase dient, vor der Übergabe an die Hochschule. zur identifizierung und Beseitigung von Fehlern. Nach Übergabe des Projektes wird die vorliegende Ausarbeitung mit einem Ausblick auf mögliche Projektweiterführungen und einem Fazit abgeschlossen.
\section{Architekturgrundlagen}
\label{sec:Architekturgrundlagen}

Das vorliegende Projekt beschreibt die Entwicklung einer Softwarelösung für das im Vorfeld beschriebene Ziel.
Im Zuge dieser Entwicklung wird eine Architektur entwickelt, die auf verschiedenen Technologien aufbaut.
Das nachfolgende Kapitel dient dazu diese Technologien grundlegend zu erläutern, um deren weitere Verwendung
nachvollziehen zu können. Zu diesem Zweck wird die entwickelte Architektur im Schaubild (\abbildung{Architektur} auf Seite \pageref{fig:Architektur}) vorausgestellt.
Die Entwicklung dieser Architektur wird aufbauend auf den folgenden Grundlagen in den nachfolgenden Kapiteln beschrieben.

In diesem Architekturentwurf auf Seite \pageref{fig:Architektur} ist die Verwendung von vier Technologien vermerkt. Diese sind:
\begin{itemize}
  \item HTML
  \item Javascript
  \item PHP
  \item SQL
\end{itemize}

Diese Technologien sind grundlegend für das Verständnis der entwickelten Software.
Aus diesem Grund werden die wichtigsten Grundlagen zu jeder Technologie nachfolgend erläutert.
Der Fokus liegt dabei immer auf den Teilbereichen, die im vorliegenden Projekt eingesetzt werden.
\section{Analyse und Planung}
\label{sec:AnalyseUndPlanung}
% TODO: Seite XX in Fußnoten richtig einfügen
Nach der Fundierung der Architekturgrundlagen wird die enstehende Softwarelösung geplant. Diese Ausarbeitung setzt hierbei
am entwickelten Konzept\footnote{siehe dazu \citet{unternehmensfuehrung2014}, Seite XX} und der daraus resultierenden Problemlösung\footnote{siehe dazu \citet{unternehmensfuehrung2014}, Seite XX} an.
Darauf aufbauend wird zunächst eine Analyse der Ist-Situation vorgenommen. 
Hieraus geht hervor, ob es in der Ist-Situation der Hochschuldarstellung Bestandteile gibt, die für die 
neue Softwarelösung wiederverwendet werden können. Die Planungsphase wird mit der Definition der Anforderungen in 
Form eines Lastenheftes und der Festlegung der Projektgrenzen beendet.
\section{Entwurf}
\label{sec:Entwurf}

\subsection{Technologieauswahl}
\label{sec:Technologieauswahl}

Zu Beginn der Entwurfsphase werden die einleitend vorgestellten und ausgwählten Technologien im Folgenden in einer Analyse untersucht. An dieser Stelle soll herausgestellt werden warum bestimmte Technologien ausgewählt wurden und welche Alternativen vorhanden sind. Betrachtet werden an dieser Stelle wiederrum die vier Technologien der Architektur (HTML, Javascript, PHP und SQL), die zuvor in den Architekturgrundlagen erläutert wurden.

\begin{description}
  \item[HTML] wird als Darstellungssprache für Benutzeransichten verwendet. HTML ist der führende Standard für die Darstellung in Webanwendungen. Die einzige Alternative zu HTML ist das Darstellungsformat XML (Extensible Markup Language). Mit XML können, genau wie mit HTML, strukturiert Textdatein erstellt werden, die einem Benutzer eine Ansicht auf Basis definierter Elemente bietet. Der große Nachteil von XML ist dabei, dass alle Elemente, die man zur Darstellung verwenden möchte, selbst in einer eigenen Datei definiert werden müssen. Dagegen besteht bei HTML der Vorteil, dass alle Projektmitglieder mit dieser Technologie bereits gearbeitet haben und den Einsatz beherrschen. Die Umsetzung des
  Projektes ist mit HTML daher schneller zu realsieren.

  \clearpage
  \item[Javascript] wird als clientseitige Skriptsprache zum Ausführen von Benutzerinteraktonen und zum Nachladen von Webinhalten verwendet. Javascript hat sich über einen Entwicklungszeitraum von 19 Jahren (1995-2014) als Standard in diesen Bereich von Webanwendungen etabliert\footnote{\citet{powers2007}, Seite 2}. Der große Vorteil von Javascript ist die Mächtigkeit, die die Sprache durch 14 Jahre Entwicklung erhalten hat. Auf Javascript basieren heute ganze Frameworks sowohl für Clientanwendungsbereiche als auch für Serveranwendungsbereiche\footnotemark, darüber hinaus gibt es zahlreiche Bibliotheken, die das Entwickeln mit Javascript vereinfachen (z.B.: jQuery). Der Nachteil von Javascript ist, dass es vom jeweiligen Internetbrowser des Anwenders compiliert (übersetzt) wird. Daraus resultiert das Javascript in manchen Browsern anders ausgeführt wird und andere Ergebnisse liefert als in anderen. Das ist aber vor allem ein Problem von älteren Browser, in den neuen Versionen haben sich alle großen Hersteller auf einheitliche Standards geeinigt. Als Alternativen zu Javascript kann das Google Projekt "`Dart"'\footnote{siehe \url{https://www.dartlang.org/}} und das Projekt "`JSX"'\footnote{siehe \url{http://jsx.github.io/}} angeführt werden. Sowohl Dart als auch JSX sind aber keinem Projektmitglied bekannt, die Nutzung dieser Technologie würde damit zusätzlichen Arbeitsaufwand und Einarbeitungszeit bedeuten, die durch Verwendung der bekannten Javascript Technologie effizienter zur Entwicklung des Projektes genutzt werden kann. Zudem wird zumindest für die Verwendung von Dart eine eigene Laufzeitumgebung benötigt, die die Sprache interpretiert und compiliert. Diese ist zum einen noch in der Entwicklungsphase, da das Dart Projekt erst 2011 veröffentlicht wurde und zum anderen noch nicht auf jedem Browser lauffähig\footnote{vergleiche hierzu \url{http://www.golem.de/news/javascript-alternative-googles-dart-1-0-veroeffentlicht-1311-102745.html}}.

  \footnotetext{siehe hierzu zum Beispiel Angular.js (\url{http://angularjs.org/}) oder Node.js (\url{http://nodejs.org/})}

  \item[PHP] ist im Projekt für die Implementierung der Serverseitigen Logik zuständig. In diesem Anwendungsbereich gibt es zahlreiche Alternativen und Möglichkeiten. Die bekanntesten in diesem Bereich sind Java, Python, Ruby, Perl, ASP und 
  Node.js. Alle diese Technologien haben ihre Vorteile und Nachteile im Bezug auf Performanz, Skalierbarkeit, Wartbarkeit, Portierbarkeit usw. Der auschlagebende Grund PHP in diesem Projekt als Servertechnologie zu verwenden ist die schnelle Lernkruve, die durch die einfache Struktur der Sprache gegeben ist und die schnelle Einrichtung eines Webservers mit PHP, sowohl lokal, als auch auf externen Webservern\footnote{\citet{peyton2005}, Seite 14}. Viele fertige Webserverlösungen, zum Beispiel Xampp, bringen darüber hinaus PHP standardmäßig mit und erleichtern damit den Einstieg.

  \item[SQL] ist die Verwendete Abfragesprache für Anfragen an die Datenbank im vorliegenden Projekt. Vorteil dieser Sprache ist die Große Bibliothek an Befehlen und Mächtigkeit der Sprache, besonders durch die Sprachbereiche DCL und DDL. Nachteil ist die relativ komplexe Syntax der Sprache und eine subjektiv wahrgenommene relativ langsame Lernkurve. SQL ist aber auch die einzige Möglichkeit eine Datenbank, wie sie in diesem Projekt vorliegt, anzusprechen. Lediglich wenn man bereit ist das Datenbanksystem gegen eine andere Form der Datenspeicherung zu tauschen bieten sich weitere Alternativen. Zum Beispiel bieten CSV(comma serperated value)-Dateien, eine Möglichkeit Informationen auf Dateibasis zu speichern. Problem der CSV-Dateien ist die schlechte Wartbarkeit und Kontrolle der Informationen, da ein Kommagetrenntes Format bei großen Datenmengen für Menschen nicht gut lesbar ist. Eine weitere Möglichkeit ist die Speicherung von Informationen in XML. Auch diese Speicherung ist auf Dateibasis und auch hier ist die Darstellung großer Datenmengen wenig übersichtlich. Darüber hinaus sind die in relationalen Datenbanken definierten und in SQL abfragbaren Beziehungen, bei diesen Technologien nicht vorhanden und müssten selbst durch Logik und Aufwand abgebildet werden. Zudem müsste bei Verwendung einer alternativen Datenhaltung eine eigene Abfragesprache oder eigene Abfragebefehle definiert werden. SQL stellt vor diesem Hintergrund die beste Wahl dar.
\end{description}
\subsection{Zielplattform}
\label{sec:Zielplattform}

Das zu realisierende Projekt soll dazu dienen, Studieninteressierten die
Attraktivität des Studienstandortes Lingen zu vermitteln. Die zu erstellende
Anwendung soll also jedem Interessenten frei und unbegrenzt zur Verfügung
stehen. Zu diesem Zweck soll die Software über das Internet auf der
Internetseite der Campus Lingen zur Verfügung gestellt werden. Die
Zielplattform der Anwendung stellt daher ein Webserver der Hochschule dar, der
die Technologie PHP unterstützt. Über einen Link wird die Anwendung nach der
Fertigstellung in die bestehende Internetseite der Hochschule eingebettet. Die
Nutzung der Software erfordert somit nur noch einen Internetbrowser, der die
Technologie Javascript unterstützt.

\subsection{Oberflächenentwurf}
\label{sec:Oberflaechenentwuf}

Ein Oberflächenentwurf, auch Mockup genannt, eignet sich besonders in
IT-Projekten für einen ersten Grobentwurf der Benutzeroberfläche des zu
entwickelnden Systems. Ein solcher Oberflächenentwurf beinhaltet die wichtigsten
Benutzerelemente, mit denen die Funktionen des Systems erfüllt werden können.
Das Design und das Layout der Benutzerelemente ist dabei zweitrangig. Ein
Oberflächenentwurf dient in erster Linie dazu, das System für die Entwickler zu
visualisieren. Dies ist besonders von Vorteil, wenn mehrere Personen an der
Entwicklung beteiligt sind. In diesem Fall vermittelt der Oberflächenentwurf ein
einheitliches Bild des zu realisierenden Systems an alle beteiligten
Entwickler. Dieser erste Eindruck vom Endprodukt ist auch für die Auftraggeber
eines Projektes von großem Interesse. Der Oberflächenentwurf dient somit im
zweiten Schritt auch dem Auftraggeber und der Kommunikation zwischen
Auftraggeber und Entwicklerteam. Funktionsvorstellungen und Erweiterung können
mit Hilfe eines Oberflächenentwurfs direkt an einem Modell festgemacht werden.
Auf diese Weise können Fehlentwicklungen von Beginn des Projektes an verhindert
werden.

\subsubsection{Benutzeransicht}
\label{sec:Benutzeransicht}

Ein erster Oberflächenentwurf der Benutzeransicht des vorliegenden Projektes
ist in \abbildung{MockupFrontend} dargestellt.

\begin{figure}[htb]
\centering
\includegraphics[width=1.0\textwidth]{MockupFrontend.jpg}
\caption[Oberflächenentwurf der
Benutzeransicht]{Oberflächenentwurf der
Benutzeransicht\protect\footnotemark}
\label{fig:MockupFrontend}
\end{figure}
\footnotetext{Quelle: Eigene Darstellung}

\clearpage

In diesem Mockup sind folgende vier Elemente abgebildet:

\begin{itemize}
 \item Ein 360-Grad-Foto
 \item Ein Steuerkreuz
 \item Eine Übersichtskarte (Minimap)
 \item Ein Informationsfenster
\end{itemize}

Das \textbf{360-Grad-Foto} ist hierbei zentrales Element des Projektes. Dieses
Foto stellt einen Standpunkt dar, von dem aus sich der Benutzer den Campus
Lingen ansehen kann. Von diesem Standpunkt aus kann sich der Benutzer in alle
Richtungen in dem Panoramafoto frei umsehen.

Mit dem \textbf{Steuerkreuz} am unteren Rand des dargestellten
Oberflächenentwurfs kann der Benutzer zu einem anderen Aufnahmepunkt wechseln.
Er kann auf diese Weise den Campus von einem anderen Standpunkt aus betrachten.
An diesem neuen Standort kann sich der Benutzer wiederum frei umsehen. Die
Pfeile des Steuerkreuzes zeigen dabei zu jedem Panoramafoto, welches von der
aktuellen Position aus erreichbar ist.

Die \textbf{Minimap} ist in einer Ecke des Panoramas platziert und erfüllt
zwei Aufgaben: Zum einen dient sie der Orientierung des Benutzers. Sie zeigt, an
welcher Aufnahmeposition sich der Benutzer aktuell am Campus
befindet\footnote{Im Oberflächentwurf ist diese Position durch einen schwarzen
Punkt gekennzeichnet.}. Hierdurch wird neben dem Einblick in die Räumlichkeiten
des Campus auch ein Bild vom Aufbau des Campus vermittelt. Zum anderen kann die
Minimap, ähnlich wie die Pfeile des Steuerkreuzes, zum Navigieren zu anderen
Standpunkten genutzt werden. Beim Klicken auf die Minimap öffnet sich ein
Informationsfenster, welches interessante Orte der Hochschule anzeigt. Bei der
Auswahl eines der Orte aus der Liste wechselt die Position des Benutzers zu
dem ausgewählten Standpunkt. Auch der Ausschnitt, der auf der Minimap
präsentiert wird, passt sich der neuen Position des Benutzers an.

Das \textbf{Informationsfenster} zeigt interessante Informationen zum aktuellen
Panorama an. Im obigen Oberflächenentwurf sind zwei solcher Informationsfenster
dargestellt\footnote{Die Informationsfenster befinden sich am rechten Rand
sowie oberhalb des Steuerkreuzes}. Diese Darstellungsformen sind als alternativ
zu betrachten. Die Art der Informationsdarstellung ist zum Zeitpunkt des
Oberflächenentwurfs noch nicht eindeutig festgelegt. Inhalt dieser
Informationsfenster können dabei interessante Projekte einzelner Studiengänge,
Öffnungszeiten von Räumlichkeiten oder Wissenwertes aus dem Studienalltag sein.
Die Informationen sollen in Form eines Popup-Fensters dargestellt werden. Sie
sollen also nicht permanent angezeigt werden, sondern erscheinen erst durch
Klicken des Benutzers auf einen Button. Dadurch wird das Blickfeld des Benutzers
nicht durch störende Anzeigen eingeschränkt.

\subsubsection{Administrationsbereich}
\label{sec:Administrationsbereich}

Ein Oberflächenentwurf des Administrationsbereiches ist in
\abbildung{MockupBackend} dargestellt.

\begin{figure}[htb]
\centering
\includegraphics[width=1.0\textwidth]{MockupBackend.png}
\caption[Oberflächenentwurf des
Administrationsbereiches]{Oberflächenentwurf des
Administrationsbereiches\protect\footnotemark}
\label{fig:MockupBackend}
\end{figure}
\footnotetext{Quelle: Eigene Darstellung}

Der Administartionsbereich gliedert sich hierbei in folgende Bereiche:

\begin{itemize}
  \item Infotextverwaltung
  \item Fotoverwaltung
  \item Interessante Ort
  \item Übersichtskarte
\end{itemize}

In der \textbf{Infotextverwaltung} können Informationstexte verfasst und zu
Panoramafotos zugeordnet werden. Ein Informationstext kann dabei auch mehreren
Fotos zugeordnet werden. Dadurch können dem Benutzer alle Informationstexte
angezeigt werden, die sich auf den Standpunkt beziehen, an dem sich dieser
gerade befindet. Alle Informationstexte, die bereits erstellt wurden, werden in
der Infotextverfaltung tabellarisch aufgelistet. Jeder dieser Informationstexte
kann sowohl verändert als auch gelöscht werden.

In der \textbf{Fotoverwaltung} werden analog zu der Infotextverwaltung die
erstellten Panoramafotos hinterlegt und gepflegt. Erstellte Panoramafotos
können in diesem Menüpunkt mit Namen und Beschreibung hochgeladen werden.
Analog zu den Infotexten werden hier die bereits hochgeladenen Fotos
tabellarisch aufgelistet. Auch die Möglichkeit zur Änderung und Löschung der
Fotos ist hier gegeben.

Der Bereich \textbf{Interessante Orte} bietet die
Möglichkeit, die Standorte zu hinterlegen, die dem Benutzer beim Klicken
auf die Minimap angezeigt werden. Zu diesen Standorten kann der Benutzer
dann springen, ohne dorthin navigieren zu müssen. Diese Möglichkeit ist vorallem
für Benutzer der Anwendung von Vorteil, die den Campus nicht kennen und nicht
erst nach bestimmten Orten suchen wollen. Zur Pflege der interessanten Orte muss
im entsprechenden Bereich in der Administrationsoberfläche nur ein
beschreibender Text eingetragen und mit einem hochgeladenen Foto verlinkt werden.
 
Die \textbf{Übersichtskarte} stellt den letzen und komplexesten Bereich des
Administrationsbereiches dar. In der Übersichtskarte werden die hochgeladenen
Panoramafotos auf einer Karte des Hochschulgebäudes platziert. Dazu wird dem
angemeldeten Administrator zunächst eine topographische Karte des Campus
präsentiert. Auf dieser Karte kann der Administrator durch Klicken eine
Position bestimmen, zu der er ein Foto auswählen und speichern kann.
Ein Foto kann dabei nur an einer Position auf der Übersichtskarte platziert
werden. Der Administrator hat die Möglichkeit die Position jedes Fotos beliebig
oft zu ändern. Darüber hinaus werden dem Administrator Steuerelemente angezeigt
mit denen er einen bestimmten Bereich des Campus auswählen kann, in dem er
Fotos positionieren möchte. So kann der Administrator sowohl zwischen
verschiedenen Gebäudekomplexen als auch zwischen den einzelnen Stockwerken
innerhalb eines Gebäudekomplexes wählen. Neben der Möglichkeit ein Foto zu
positionieren hat der Administrator weiterhin die Möglichkeit die Verbindungen
zwischen positionierten Fotos zu pflegen. Aus dem Oberflächenentwurf der
Benutzeroberfläche, welcher im vorherigen Abschnitt vorgestellt wurde, ist
ersichtlich, dass ein Benutzer von einem Panoramafoto aus zu einem anderen
Standpunkt navigieren kann. Diese Navigation beruht auf Verbindungen zwischen
den Panoramafotos. Solche Verbindungen werden in der Übersichtskarte geplegt.
Die Steuerungselemente zu Pflege der Verbindungen zwischen Panoramafotos sind zum
Zeitpunkt der Erstellung des Mockups noch nicht definiert, die Notwendigkeit
dieser Funktion ist aber bedacht.
\subsection{Anwendungsfälle}
\label{sec:Anwendungsfaelle}

% TODO: Anhang der use case diagramme verweisen

Aufbauend auf dem Entwurf der Oberfläche sowohl für den Benutzer, als auch für den Administrator, werden im Folgenden Anwendungsfälle beschrieben, die die Anwendungsmöglichkeiten dieser beiden Benutzergruppen beschreiben. Zur visuellen Modellierung dieser Anwendungsfälle wird die Darstellungsart des Anwendungsfalldiagramms\footnotemark (engl: use case diagram) gewählt. Die Diagramme sind im Anhang dargestellt.

\footnotetext{Ein Anwendungsfalldiagramm ist eine Diagrammart der Unified Modelnig Langauge, kurz UML. In dieser werden sowohl Auslöser einer Funktion als auch Funktionsbeschreibung dargestellt. Eine Funktion kann dabei weitere Funktionen einbinden oder erweitern.}

Die im Folgenden vorgestellten Anwendungsfälle dienen sowohl der Dokumentation der Anwendungs, als auch zur späteren Durchführung der Tests.
\subsection{Datenbankentwurf}
\label{sec:Datenbankentwurf}

Aufbauend auf den vorangegangenen Entwurfsergebnissen wird im Folgenden ein
Entwurf der Datenbank in Form eines Tabellenmodells angefertigt.\footnote{Ein
Tabellenmodell ist ein graphisches Modell zur Darstellung der Struktur einer
Datenbank. In diesem Modell werden Tabellen, deren Attribute (Eigenschaften)
und die Beziehung zu anderen Tabellen dargestellt.}

Bevor das Tabellenmodell erstellt werden kann, müssen in dem vorliegenden
Projekt die Daten identifiziert werden, die persistent in der Datenbank
gespeichert werden. Aus dem vorausgegangenen Oberflächenentwurf wird deutlich,
dass das Projekt aus den zwei Teilen, Benutzeransicht und Administrationsansicht,
besteht. Die Benutzeransicht stellt dabei Informationen dar, die in der
Administrationsansicht gepflegt und hinterlegt wurden. Der
Administrationsbereich ist der datenhaltende Bereich, in dem die Interaktion
mit der Datenbank stattfindet. Aus dem \verweis{Administrationsbereich} ergeben
sich bereits folgende drei Informationsobjekte:

\begin{itemize}
  \item Infotexte
  \item Fotos
  \item Interessante Orte
\end{itemize}

Die Beziehung zwischen Infotexten und Fotos stellt dabei eine sogenannte
n:m Beziehung dar. In dieser Beziehung kann ein Infotext mehreren Fotos
zugeordnet sein und im Gegenzug können ebenso mehrere Infotexte auf demselben
Foto platziert werden. Um eine solche Beziehung in einer relationalen Datenbank
abbilden zu können muss eine neue Tabelle angelegt werden, die diese Beziehung
zwischen Foto und Infotext speichert. Gleiches gilt für die
Nachbarschaftsbeziehung zwischen zwei Fotos. Da ein Foto mehrere Nachbarfotos
haben kann, muss wiederum eine Tabelle erstellt werden, in der die Beziehung
zwischen Foto und Nachbarfoto gespeichert wird. Neben diesen zwei zusätzlichen
Tabellen werden noch drei weitere Tabellen für die Speicherung der Daten der
Übersichtskarte benötigt. In der Übersichtskarte soll grundsätzlich zwischen dem
Studienstandort an der Baccumer Straße und dem Standort an der Kaiserstraße
unterschieden werden. Da beide Standorte unterschiedlich parametresiert werden 
müssen wird hierfür eine eigene Tabelle benötigt. Diese wird mit dem Namen
\textit{Area} bezeichnet. Eine Area beschreibt hierbei einen topologischen
Ausschnitt einer Google Maps Karte. Da eine topologische Karte nur den Umriss
von Gebäuden zeigt, ist es schwer, den genauen Standort eines Panoramas zu
bestimmen. Aus diesem Grund werden Grafiken über die Karte gelegt, die die
innere Struktur (Wände, Gebäudetrackte, etc.) des Campus zeigen. Sowohl die
topologische Karte als auch die übergelegten Grafiken, im folgenden
\textit{Overlays} genannt, benötigen eine eigene Tabelle, da beide jeweils noch
weitere Attribute haben. Ein Overlay ist dabei immer einer Karte zugeordnet.
Eine Karte kann weiterhin mehrere Overlays besitzen. Über die Zuordnung von
mehreren Overlays zu einer Karte werden Stockwerke an einem Standort
dargestellt. Ein Overlay enthält damit unter anderem den Pfad zu einer Grafik,
die ein Stockwerk an einem Standort darstellt. Zusammengefasst ergeben sich
folgende Tabellen:

\begin{itemize}
  \item infotext (Verwaltung der Infotexte)
  \item panorama (Verwaltung der Fotos)
  \item poi (Verwaltung der interessanten Orte)
  \item infotext\_panorama (n:m-Beziehung zwischen Infotexten und Fotos)
  \item neighbour (n:m-Beziehung zwischen zwei Fotos)
  \item area (Kartenbereich)
  \item map (Kartenansicht)
  \item overlay (übergelegte Grafik)
\end{itemize}

Diese Tabellen werden in \abbildung{Tabellenmodell} mit ihren jeweiligen
Attributen und Beziehungen dargestellt.

\begin{figure}[htb]
\centering
\includegraphics[width=1.0\textwidth]{Tabellenmodell.png}
\caption[Tabellenmodell der Anwendung]{Tabellenmodell der Anwendung\protect\footnotemark}
\label{fig:Tabellenmodell}
\end{figure}
\footnotetext{Quelle: Eigene Darstellung}
\subsection{Architektur}
\label{sec:Architektur}
\subsection{Pflichtenheft}
\label{sec:Pflichtenheft}
\subsection{Technologieauswahl}
\label{sec:Technologieauswahl}

Zu Beginn der Entwurfsphase werden die einleitend vorgestellten und ausgwählten Technologien im Folgenden in einer Analyse untersucht. An dieser Stelle soll herausgestellt werden warum bestimmte Technologien ausgewählt wurden und welche Alternativen vorhanden sind. Betrachtet werden an dieser Stelle wiederrum die vier Technologien der Architektur (HTML, Javascript, PHP und SQL), die zuvor in den Architekturgrundlagen erläutert wurden.

\begin{description}
  \item[HTML] wird als Darstellungssprache für Benutzeransichten verwendet. HTML ist der führende Standard für die Darstellung in Webanwendungen. Die einzige Alternative zu HTML ist das Darstellungsformat XML (Extensible Markup Language). Mit XML können, genau wie mit HTML, strukturiert Textdatein erstellt werden, die einem Benutzer eine Ansicht auf Basis definierter Elemente bietet. Der große Nachteil von XML ist dabei, dass alle Elemente, die man zur Darstellung verwenden möchte, selbst in einer eigenen Datei definiert werden müssen. Dagegen besteht bei HTML der Vorteil, dass alle Projektmitglieder mit dieser Technologie bereits gearbeitet haben und den Einsatz beherrschen. Die Umsetzung des
  Projektes ist mit HTML daher schneller zu realsieren.

  \clearpage
  \item[Javascript] wird als clientseitige Skriptsprache zum Ausführen von Benutzerinteraktonen und zum Nachladen von Webinhalten verwendet. Javascript hat sich über einen Entwicklungszeitraum von 19 Jahren (1995-2014) als Standard in diesen Bereich von Webanwendungen etabliert\footnote{\citet{powers2007}, Seite 2}. Der große Vorteil von Javascript ist die Mächtigkeit, die die Sprache durch 14 Jahre Entwicklung erhalten hat. Auf Javascript basieren heute ganze Frameworks sowohl für Clientanwendungsbereiche als auch für Serveranwendungsbereiche\footnotemark, darüber hinaus gibt es zahlreiche Bibliotheken, die das Entwickeln mit Javascript vereinfachen (z.B.: jQuery). Der Nachteil von Javascript ist, dass es vom jeweiligen Internetbrowser des Anwenders compiliert (übersetzt) wird. Daraus resultiert das Javascript in manchen Browsern anders ausgeführt wird und andere Ergebnisse liefert als in anderen. Das ist aber vor allem ein Problem von älteren Browser, in den neuen Versionen haben sich alle großen Hersteller auf einheitliche Standards geeinigt. Als Alternativen zu Javascript kann das Google Projekt "`Dart"'\footnote{siehe \url{https://www.dartlang.org/}} und das Projekt "`JSX"'\footnote{siehe \url{http://jsx.github.io/}} angeführt werden. Sowohl Dart als auch JSX sind aber keinem Projektmitglied bekannt, die Nutzung dieser Technologie würde damit zusätzlichen Arbeitsaufwand und Einarbeitungszeit bedeuten, die durch Verwendung der bekannten Javascript Technologie effizienter zur Entwicklung des Projektes genutzt werden kann. Zudem wird zumindest für die Verwendung von Dart eine eigene Laufzeitumgebung benötigt, die die Sprache interpretiert und compiliert. Diese ist zum einen noch in der Entwicklungsphase, da das Dart Projekt erst 2011 veröffentlicht wurde und zum anderen noch nicht auf jedem Browser lauffähig\footnote{vergleiche hierzu \url{http://www.golem.de/news/javascript-alternative-googles-dart-1-0-veroeffentlicht-1311-102745.html}}.

  \footnotetext{siehe hierzu zum Beispiel Angular.js (\url{http://angularjs.org/}) oder Node.js (\url{http://nodejs.org/})}

  \item[PHP] ist im Projekt für die Implementierung der Serverseitigen Logik zuständig. In diesem Anwendungsbereich gibt es zahlreiche Alternativen und Möglichkeiten. Die bekanntesten in diesem Bereich sind Java, Python, Ruby, Perl, ASP und 
  Node.js. Alle diese Technologien haben ihre Vorteile und Nachteile im Bezug auf Performanz, Skalierbarkeit, Wartbarkeit, Portierbarkeit usw. Der auschlagebende Grund PHP in diesem Projekt als Servertechnologie zu verwenden ist die schnelle Lernkruve, die durch die einfache Struktur der Sprache gegeben ist und die schnelle Einrichtung eines Webservers mit PHP, sowohl lokal, als auch auf externen Webservern\footnote{\citet{peyton2005}, Seite 14}. Viele fertige Webserverlösungen, zum Beispiel Xampp, bringen darüber hinaus PHP standardmäßig mit und erleichtern damit den Einstieg.

  \item[SQL] ist die Verwendete Abfragesprache für Anfragen an die Datenbank im vorliegenden Projekt. Vorteil dieser Sprache ist die Große Bibliothek an Befehlen und Mächtigkeit der Sprache, besonders durch die Sprachbereiche DCL und DDL. Nachteil ist die relativ komplexe Syntax der Sprache und eine subjektiv wahrgenommene relativ langsame Lernkurve. SQL ist aber auch die einzige Möglichkeit eine Datenbank, wie sie in diesem Projekt vorliegt, anzusprechen. Lediglich wenn man bereit ist das Datenbanksystem gegen eine andere Form der Datenspeicherung zu tauschen bieten sich weitere Alternativen. Zum Beispiel bieten CSV(comma serperated value)-Dateien, eine Möglichkeit Informationen auf Dateibasis zu speichern. Problem der CSV-Dateien ist die schlechte Wartbarkeit und Kontrolle der Informationen, da ein Kommagetrenntes Format bei großen Datenmengen für Menschen nicht gut lesbar ist. Eine weitere Möglichkeit ist die Speicherung von Informationen in XML. Auch diese Speicherung ist auf Dateibasis und auch hier ist die Darstellung großer Datenmengen wenig übersichtlich. Darüber hinaus sind die in relationalen Datenbanken definierten und in SQL abfragbaren Beziehungen, bei diesen Technologien nicht vorhanden und müssten selbst durch Logik und Aufwand abgebildet werden. Zudem müsste bei Verwendung einer alternativen Datenhaltung eine eigene Abfragesprache oder eigene Abfragebefehle definiert werden. SQL stellt vor diesem Hintergrund die beste Wahl dar.
\end{description}
\subsection{Zielplattform}
\label{sec:Zielplattform}

Das zu realisierende Projekt soll dazu dienen, Studieninteressierten die
Attraktivität des Studienstandortes Lingen zu vermitteln. Die zu erstellende
Anwendung soll also jedem Interessenten frei und unbegrenzt zur Verfügung
stehen. Zu diesem Zweck soll die Software über das Internet auf der
Internetseite der Campus Lingen zur Verfügung gestellt werden. Die
Zielplattform der Anwendung stellt daher ein Webserver der Hochschule dar, der
die Technologie PHP unterstützt. Über einen Link wird die Anwendung nach der
Fertigstellung in die bestehende Internetseite der Hochschule eingebettet. Die
Nutzung der Software erfordert somit nur noch einen Internetbrowser, der die
Technologie Javascript unterstützt.

\subsection{Oberflächenentwurf}
\label{sec:Oberflaechenentwuf}

Ein Oberflächenentwurf, auch Mockup genannt, eignet sich besonders in
IT-Projekten für einen ersten Grobentwurf der Benutzeroberfläche des zu
entwickelnden Systems. Ein solcher Oberflächenentwurf beinhaltet die wichtigsten
Benutzerelemente, mit denen die Funktionen des Systems erfüllt werden können.
Das Design und das Layout der Benutzerelemente ist dabei zweitrangig. Ein
Oberflächenentwurf dient in erster Linie dazu, das System für die Entwickler zu
visualisieren. Dies ist besonders von Vorteil, wenn mehrere Personen an der
Entwicklung beteiligt sind. In diesem Fall vermittelt der Oberflächenentwurf ein
einheitliches Bild des zu realisierenden Systems an alle beteiligten
Entwickler. Dieser erste Eindruck vom Endprodukt ist auch für die Auftraggeber
eines Projektes von großem Interesse. Der Oberflächenentwurf dient somit im
zweiten Schritt auch dem Auftraggeber und der Kommunikation zwischen
Auftraggeber und Entwicklerteam. Funktionsvorstellungen und Erweiterung können
mit Hilfe eines Oberflächenentwurfs direkt an einem Modell festgemacht werden.
Auf diese Weise können Fehlentwicklungen von Beginn des Projektes an verhindert
werden.

\subsubsection{Benutzeransicht}
\label{sec:Benutzeransicht}

Ein erster Oberflächenentwurf der Benutzeransicht des vorliegenden Projektes
ist in \abbildung{MockupFrontend} dargestellt.

\begin{figure}[htb]
\centering
\includegraphics[width=1.0\textwidth]{MockupFrontend.jpg}
\caption[Oberflächenentwurf der
Benutzeransicht]{Oberflächenentwurf der
Benutzeransicht\protect\footnotemark}
\label{fig:MockupFrontend}
\end{figure}
\footnotetext{Quelle: Eigene Darstellung}

\clearpage

In diesem Mockup sind folgende vier Elemente abgebildet:

\begin{itemize}
 \item Ein 360-Grad-Foto
 \item Ein Steuerkreuz
 \item Eine Übersichtskarte (Minimap)
 \item Ein Informationsfenster
\end{itemize}

Das \textbf{360-Grad-Foto} ist hierbei zentrales Element des Projektes. Dieses
Foto stellt einen Standpunkt dar, von dem aus sich der Benutzer den Campus
Lingen ansehen kann. Von diesem Standpunkt aus kann sich der Benutzer in alle
Richtungen in dem Panoramafoto frei umsehen.

Mit dem \textbf{Steuerkreuz} am unteren Rand des dargestellten
Oberflächenentwurfs kann der Benutzer zu einem anderen Aufnahmepunkt wechseln.
Er kann auf diese Weise den Campus von einem anderen Standpunkt aus betrachten.
An diesem neuen Standort kann sich der Benutzer wiederum frei umsehen. Die
Pfeile des Steuerkreuzes zeigen dabei zu jedem Panoramafoto, welches von der
aktuellen Position aus erreichbar ist.

Die \textbf{Minimap} ist in einer Ecke des Panoramas platziert und erfüllt
zwei Aufgaben: Zum einen dient sie der Orientierung des Benutzers. Sie zeigt, an
welcher Aufnahmeposition sich der Benutzer aktuell am Campus
befindet\footnote{Im Oberflächentwurf ist diese Position durch einen schwarzen
Punkt gekennzeichnet.}. Hierdurch wird neben dem Einblick in die Räumlichkeiten
des Campus auch ein Bild vom Aufbau des Campus vermittelt. Zum anderen kann die
Minimap, ähnlich wie die Pfeile des Steuerkreuzes, zum Navigieren zu anderen
Standpunkten genutzt werden. Beim Klicken auf die Minimap öffnet sich ein
Informationsfenster, welches interessante Orte der Hochschule anzeigt. Bei der
Auswahl eines der Orte aus der Liste wechselt die Position des Benutzers zu
dem ausgewählten Standpunkt. Auch der Ausschnitt, der auf der Minimap
präsentiert wird, passt sich der neuen Position des Benutzers an.

Das \textbf{Informationsfenster} zeigt interessante Informationen zum aktuellen
Panorama an. Im obigen Oberflächenentwurf sind zwei solcher Informationsfenster
dargestellt\footnote{Die Informationsfenster befinden sich am rechten Rand
sowie oberhalb des Steuerkreuzes}. Diese Darstellungsformen sind als alternativ
zu betrachten. Die Art der Informationsdarstellung ist zum Zeitpunkt des
Oberflächenentwurfs noch nicht eindeutig festgelegt. Inhalt dieser
Informationsfenster können dabei interessante Projekte einzelner Studiengänge,
Öffnungszeiten von Räumlichkeiten oder Wissenwertes aus dem Studienalltag sein.
Die Informationen sollen in Form eines Popup-Fensters dargestellt werden. Sie
sollen also nicht permanent angezeigt werden, sondern erscheinen erst durch
Klicken des Benutzers auf einen Button. Dadurch wird das Blickfeld des Benutzers
nicht durch störende Anzeigen eingeschränkt.

\subsubsection{Administrationsbereich}
\label{sec:Administrationsbereich}

Ein Oberflächenentwurf des Administrationsbereiches ist in
\abbildung{MockupBackend} dargestellt.

\begin{figure}[htb]
\centering
\includegraphics[width=1.0\textwidth]{MockupBackend.png}
\caption[Oberflächenentwurf des
Administrationsbereiches]{Oberflächenentwurf des
Administrationsbereiches\protect\footnotemark}
\label{fig:MockupBackend}
\end{figure}
\footnotetext{Quelle: Eigene Darstellung}

Der Administartionsbereich gliedert sich hierbei in folgende Bereiche:

\begin{itemize}
  \item Infotextverwaltung
  \item Fotoverwaltung
  \item Interessante Ort
  \item Übersichtskarte
\end{itemize}

In der \textbf{Infotextverwaltung} können Informationstexte verfasst und zu
Panoramafotos zugeordnet werden. Ein Informationstext kann dabei auch mehreren
Fotos zugeordnet werden. Dadurch können dem Benutzer alle Informationstexte
angezeigt werden, die sich auf den Standpunkt beziehen, an dem sich dieser
gerade befindet. Alle Informationstexte, die bereits erstellt wurden, werden in
der Infotextverfaltung tabellarisch aufgelistet. Jeder dieser Informationstexte
kann sowohl verändert als auch gelöscht werden.

In der \textbf{Fotoverwaltung} werden analog zu der Infotextverwaltung die
erstellten Panoramafotos hinterlegt und gepflegt. Erstellte Panoramafotos
können in diesem Menüpunkt mit Namen und Beschreibung hochgeladen werden.
Analog zu den Infotexten werden hier die bereits hochgeladenen Fotos
tabellarisch aufgelistet. Auch die Möglichkeit zur Änderung und Löschung der
Fotos ist hier gegeben.

Der Bereich \textbf{Interessante Orte} bietet die
Möglichkeit, die Standorte zu hinterlegen, die dem Benutzer beim Klicken
auf die Minimap angezeigt werden. Zu diesen Standorten kann der Benutzer
dann springen, ohne dorthin navigieren zu müssen. Diese Möglichkeit ist vorallem
für Benutzer der Anwendung von Vorteil, die den Campus nicht kennen und nicht
erst nach bestimmten Orten suchen wollen. Zur Pflege der interessanten Orte muss
im entsprechenden Bereich in der Administrationsoberfläche nur ein
beschreibender Text eingetragen und mit einem hochgeladenen Foto verlinkt werden.
 
Die \textbf{Übersichtskarte} stellt den letzen und komplexesten Bereich des
Administrationsbereiches dar. In der Übersichtskarte werden die hochgeladenen
Panoramafotos auf einer Karte des Hochschulgebäudes platziert. Dazu wird dem
angemeldeten Administrator zunächst eine topographische Karte des Campus
präsentiert. Auf dieser Karte kann der Administrator durch Klicken eine
Position bestimmen, zu der er ein Foto auswählen und speichern kann.
Ein Foto kann dabei nur an einer Position auf der Übersichtskarte platziert
werden. Der Administrator hat die Möglichkeit die Position jedes Fotos beliebig
oft zu ändern. Darüber hinaus werden dem Administrator Steuerelemente angezeigt
mit denen er einen bestimmten Bereich des Campus auswählen kann, in dem er
Fotos positionieren möchte. So kann der Administrator sowohl zwischen
verschiedenen Gebäudekomplexen als auch zwischen den einzelnen Stockwerken
innerhalb eines Gebäudekomplexes wählen. Neben der Möglichkeit ein Foto zu
positionieren hat der Administrator weiterhin die Möglichkeit die Verbindungen
zwischen positionierten Fotos zu pflegen. Aus dem Oberflächenentwurf der
Benutzeroberfläche, welcher im vorherigen Abschnitt vorgestellt wurde, ist
ersichtlich, dass ein Benutzer von einem Panoramafoto aus zu einem anderen
Standpunkt navigieren kann. Diese Navigation beruht auf Verbindungen zwischen
den Panoramafotos. Solche Verbindungen werden in der Übersichtskarte geplegt.
Die Steuerungselemente zu Pflege der Verbindungen zwischen Panoramafotos sind zum
Zeitpunkt der Erstellung des Mockups noch nicht definiert, die Notwendigkeit
dieser Funktion ist aber bedacht.
\subsubsection{Benutzeransicht}
\label{sec:Benutzeransicht}

Ein erster Oberflächenentwurf der Benutzeransicht des vorliegenden Projektes
ist in \abbildung{MockupFrontend} dargestellt.

\begin{figure}[htb]
\centering
\includegraphics[width=1.0\textwidth]{MockupFrontend.jpg}
\caption[Oberflächenentwurf der
Benutzeransicht]{Oberflächenentwurf der
Benutzeransicht\protect\footnotemark}
\label{fig:MockupFrontend}
\end{figure}
\footnotetext{Quelle: Eigene Darstellung}

\clearpage

In diesem Mockup sind folgende vier Elemente abgebildet:

\begin{itemize}
 \item Ein 360-Grad-Foto
 \item Ein Steuerkreuz
 \item Eine Übersichtskarte (Minimap)
 \item Ein Informationsfenster
\end{itemize}

Das \textbf{360-Grad-Foto} ist hierbei zentrales Element des Projektes. Dieses
Foto stellt einen Standpunkt dar, von dem aus sich der Benutzer den Campus
Lingen ansehen kann. Von diesem Standpunkt aus kann sich der Benutzer in alle
Richtungen in dem Panoramafoto frei umsehen.

Mit dem \textbf{Steuerkreuz} am unteren Rand des dargestellten
Oberflächenentwurfs kann der Benutzer zu einem anderen Aufnahmepunkt wechseln.
Er kann auf diese Weise den Campus von einem anderen Standpunkt aus betrachten.
An diesem neuen Standort kann sich der Benutzer wiederum frei umsehen. Die
Pfeile des Steuerkreuzes zeigen dabei zu jedem Panoramafoto, welches von der
aktuellen Position aus erreichbar ist.

Die \textbf{Minimap} ist in einer Ecke des Panoramas platziert und erfüllt
zwei Aufgaben: Zum einen dient sie der Orientierung des Benutzers. Sie zeigt, an
welcher Aufnahmeposition sich der Benutzer aktuell am Campus
befindet\footnote{Im Oberflächentwurf ist diese Position durch einen schwarzen
Punkt gekennzeichnet.}. Hierdurch wird neben dem Einblick in die Räumlichkeiten
des Campus auch ein Bild vom Aufbau des Campus vermittelt. Zum anderen kann die
Minimap, ähnlich wie die Pfeile des Steuerkreuzes, zum Navigieren zu anderen
Standpunkten genutzt werden. Beim Klicken auf die Minimap öffnet sich ein
Informationsfenster, welches interessante Orte der Hochschule anzeigt. Bei der
Auswahl eines der Orte aus der Liste wechselt die Position des Benutzers zu
dem ausgewählten Standpunkt. Auch der Ausschnitt, der auf der Minimap
präsentiert wird, passt sich der neuen Position des Benutzers an.

Das \textbf{Informationsfenster} zeigt interessante Informationen zum aktuellen
Panorama an. Im obigen Oberflächenentwurf sind zwei solcher Informationsfenster
dargestellt\footnote{Die Informationsfenster befinden sich am rechten Rand
sowie oberhalb des Steuerkreuzes}. Diese Darstellungsformen sind als alternativ
zu betrachten. Die Art der Informationsdarstellung ist zum Zeitpunkt des
Oberflächenentwurfs noch nicht eindeutig festgelegt. Inhalt dieser
Informationsfenster können dabei interessante Projekte einzelner Studiengänge,
Öffnungszeiten von Räumlichkeiten oder Wissenwertes aus dem Studienalltag sein.
Die Informationen sollen in Form eines Popup-Fensters dargestellt werden. Sie
sollen also nicht permanent angezeigt werden, sondern erscheinen erst durch
Klicken des Benutzers auf einen Button. Dadurch wird das Blickfeld des Benutzers
nicht durch störende Anzeigen eingeschränkt.

\subsubsection{Administrationsbereich}
\label{sec:Administrationsbereich}

Ein Oberflächenentwurf des Administrationsbereiches ist in
\abbildung{MockupBackend} dargestellt.

\begin{figure}[htb]
\centering
\includegraphics[width=1.0\textwidth]{MockupBackend.png}
\caption[Oberflächenentwurf des
Administrationsbereiches]{Oberflächenentwurf des
Administrationsbereiches\protect\footnotemark}
\label{fig:MockupBackend}
\end{figure}
\footnotetext{Quelle: Eigene Darstellung}

Der Administartionsbereich gliedert sich hierbei in folgende Bereiche:

\begin{itemize}
  \item Infotextverwaltung
  \item Fotoverwaltung
  \item Interessante Ort
  \item Übersichtskarte
\end{itemize}

In der \textbf{Infotextverwaltung} können Informationstexte verfasst und zu
Panoramafotos zugeordnet werden. Ein Informationstext kann dabei auch mehreren
Fotos zugeordnet werden. Dadurch können dem Benutzer alle Informationstexte
angezeigt werden, die sich auf den Standpunkt beziehen, an dem sich dieser
gerade befindet. Alle Informationstexte, die bereits erstellt wurden, werden in
der Infotextverfaltung tabellarisch aufgelistet. Jeder dieser Informationstexte
kann sowohl verändert als auch gelöscht werden.

In der \textbf{Fotoverwaltung} werden analog zu der Infotextverwaltung die
erstellten Panoramafotos hinterlegt und gepflegt. Erstellte Panoramafotos
können in diesem Menüpunkt mit Namen und Beschreibung hochgeladen werden.
Analog zu den Infotexten werden hier die bereits hochgeladenen Fotos
tabellarisch aufgelistet. Auch die Möglichkeit zur Änderung und Löschung der
Fotos ist hier gegeben.

Der Bereich \textbf{Interessante Orte} bietet die
Möglichkeit, die Standorte zu hinterlegen, die dem Benutzer beim Klicken
auf die Minimap angezeigt werden. Zu diesen Standorten kann der Benutzer
dann springen, ohne dorthin navigieren zu müssen. Diese Möglichkeit ist vorallem
für Benutzer der Anwendung von Vorteil, die den Campus nicht kennen und nicht
erst nach bestimmten Orten suchen wollen. Zur Pflege der interessanten Orte muss
im entsprechenden Bereich in der Administrationsoberfläche nur ein
beschreibender Text eingetragen und mit einem hochgeladenen Foto verlinkt werden.
 
Die \textbf{Übersichtskarte} stellt den letzen und komplexesten Bereich des
Administrationsbereiches dar. In der Übersichtskarte werden die hochgeladenen
Panoramafotos auf einer Karte des Hochschulgebäudes platziert. Dazu wird dem
angemeldeten Administrator zunächst eine topographische Karte des Campus
präsentiert. Auf dieser Karte kann der Administrator durch Klicken eine
Position bestimmen, zu der er ein Foto auswählen und speichern kann.
Ein Foto kann dabei nur an einer Position auf der Übersichtskarte platziert
werden. Der Administrator hat die Möglichkeit die Position jedes Fotos beliebig
oft zu ändern. Darüber hinaus werden dem Administrator Steuerelemente angezeigt
mit denen er einen bestimmten Bereich des Campus auswählen kann, in dem er
Fotos positionieren möchte. So kann der Administrator sowohl zwischen
verschiedenen Gebäudekomplexen als auch zwischen den einzelnen Stockwerken
innerhalb eines Gebäudekomplexes wählen. Neben der Möglichkeit ein Foto zu
positionieren hat der Administrator weiterhin die Möglichkeit die Verbindungen
zwischen positionierten Fotos zu pflegen. Aus dem Oberflächenentwurf der
Benutzeroberfläche, welcher im vorherigen Abschnitt vorgestellt wurde, ist
ersichtlich, dass ein Benutzer von einem Panoramafoto aus zu einem anderen
Standpunkt navigieren kann. Diese Navigation beruht auf Verbindungen zwischen
den Panoramafotos. Solche Verbindungen werden in der Übersichtskarte geplegt.
Die Steuerungselemente zu Pflege der Verbindungen zwischen Panoramafotos sind zum
Zeitpunkt der Erstellung des Mockups noch nicht definiert, die Notwendigkeit
dieser Funktion ist aber bedacht.
\subsection{Anwendungsfälle}
\label{sec:Anwendungsfaelle}

% TODO: Anhang der use case diagramme verweisen

Aufbauend auf dem Entwurf der Oberfläche sowohl für den Benutzer, als auch für den Administrator, werden im Folgenden Anwendungsfälle beschrieben, die die Anwendungsmöglichkeiten dieser beiden Benutzergruppen beschreiben. Zur visuellen Modellierung dieser Anwendungsfälle wird die Darstellungsart des Anwendungsfalldiagramms\footnotemark (engl: use case diagram) gewählt. Die Diagramme sind im Anhang dargestellt.

\footnotetext{Ein Anwendungsfalldiagramm ist eine Diagrammart der Unified Modelnig Langauge, kurz UML. In dieser werden sowohl Auslöser einer Funktion als auch Funktionsbeschreibung dargestellt. Eine Funktion kann dabei weitere Funktionen einbinden oder erweitern.}

Die im Folgenden vorgestellten Anwendungsfälle dienen sowohl der Dokumentation der Anwendungs, als auch zur späteren Durchführung der Tests.
\subsubsection{Benutzeranwendungen}
\label{sec:Benutzeranwendungen}

Ein Anwender der Software hat verschiedene Möglichkeiten der Interaktion mit der Anwendung. Diese sind im Folgenden mit dem forlaufenden Kennzeichen \textbf{AFB XX}\footnotemark markiert und strukturiert aufgelistet:

\footnotetext{AFBXX = Anwendunsfall eines Benutzers mit Nr. XX}

\begin{description}
  \item[AFB01] Der Benutzer ruft die Webanwendung durch einen Internetbrowser auf.
  \item[AFB02] Der Benutzer dreht sich horizontal und vertikal mit die Achse der Fotoaufnahme, um einen Rundumblick zu erhalten.
  \item[AFB03] Der Benutzer zoomt über die Steuerelemente in die Fotoaufnahme hinein.
  \item[AFB04] Der Benutzer navigiert mit Hilfe der Navigationselemente zu anderen Aufnahmepositionen.
  \item[AFB05] Der Benutzer navigiert mit Hilfe der Minimap zu einer anderen Aufnahmepositionen.
  \item[AFB06] Der Benutzer bewegt sich mit Hilfe der Pfeiltasten seiner Tastatur zu anderen Aufnahmepositionen
  \item[AFB07] Der Benutzer öffnet über ein Steuerungselement das Informationsfenster.
  \item[AFB08] Der Benutzer schließt über ein Steuerungselement das Informationsfenster.
\end{description}

% TODO: Verweise auf Anwenwendungsfalldiagram, Anhang, und Seite korrekt einfügen
Die Anwendungsfälle AFB01 bis AFB08 definieren die Menge an Interaktionsmöglichkeiten. Diese Menge ist inklusive hinterlegten Funktionen im Awendungsfalldiagramm X im Anhang X auf Seite X abgebildet.
\subsubsection{Adminstratoranwendungen}
\label{sec:Adminstratoranwendungen}

Ein Administrator hat in vier Bereichen Möglichkeiten zur
Steuerung und Pflege der Webanwendung. Diese vier Bereiche wurden im
\verweis{Administrationsbereich} vorgestellt. Die Möglichkeiten der Interaktion
in diesen Bereichen sind mit der forlaufenden
Kennzeichnung \textbf{AFAXX}\footnote{AFAXX = Anwendunsfall eines Adminstrator
mit Nr. XX} markiert und strukturiert aufgelistet:

\begin{description}
  \item[AFA01] Der Adminstrator ruft den Administrationsbereich der
  Webanwendung durch einen Internetbrowser auf.
  \item[AFA02] Der Administrator klickt auf den Menüpunkt "`Infotexte"' und ihm
  wird die Infotextverwaltung angezeigt.
  \item[AFA03] Der Administrator erstellt im Menüpunkt "`Infotexte"' einen neuen
  Informationstext.
  \item[AFA04] Der Administrator verändert im Menüpunkt "`Infotexte"' einen
  erstellten Informationstext.
  \item[AFA05] Der Administrator löscht im Menüpunkt "`Infotexte"' einen
  erstellten Informationstext.
  \item[AFA06] Der Administrator klickt auf den Menüpunkt "`Fotos"' und ihm wird
  die Fotoverwaltung angezeigt.
  \item[AFA07] Der Administrator lädt im Menüpunkt "`Fotos"' ein erstelltes
  Panoramafoto mit Beschreibung und Namen hoch.
  \item[AFA08] Der Administrator verändert im Menüpunkt "`Fotos"' den Namen und
  die Beschreibung eines hochgeladenen Fotos.
  \item[AFA09] Der Administrator löscht im Menüpunkt "`Fotos"' ein hochgeladenes
  Foto.
  \item[AFA10] Der Administrator klickt auf den Menüpunkt "`interssante Orte"'
  und ihm wird die Verwaltung der interessanten Orte angezeigt.
  \item[AFA11] Der Administrator erstellt im Menüpunkt "`interssante Orte"'
  einen neuen interessanten Ort.
  \item[AFA12] Der Administrator verändert im Menüpunkt "`interssante Orte"' den
  Namen und die Beschreibung zu einem interessanten Ort.
  \item[AFA13] Der Administrator löscht im Menüpunkt "`interssante Orte"' einen
  interessanten Ort.
  \item[AFA14] Der Administrator klickt auf den Menüpunkt "`Übersichtskarte"'
  und ihm wird die Übersichtskarte angezeigt.
  \item[AFA15] Der Administrator platziert im Menüpunkt "`Übersichtskarte"' ein
  hochgeladenes Foto auf der Übersichtskarte.
  \item[AFA16] Der Administrator verschiebt im Menüpunkt "`Übersichtskarte"' ein
  bereits platziertes Foto an eine andere Position.
  \item[AFA17] Der Administrator navigiert im Menüpunkt "`Übersichtskarte"' über
  die Steuerelemente zu einem anderen Stockwerk.
  \item[AFA18] Der Administrator verbindet im Menüpunkt "`Übersichtskarte"'
  mehrere hochgeladene und positionierte Fotos auf der Übersichtskarte.
  \item[AFA19] Der Administrator verbindet im Menüpunkt "`Übersichtskarte"' ein
  hochgeladenes und positioniertes Foto mit erstellten Informationstexten.
\end{description}

% TODO: Verweise auf Anwenwendungsfalldiagram, Anhang, und Seite korrekt einfügen
Die Anwendungsfälle AFA01 bis AFA19 definieren die Menge an
Interaktionsmöglichkeiten eines Administrators. Diese Menge an
Interaktionsmöglichkeiten ist inklusive der hinterlegten Funktionen in einem
Awendungsfalldiagramm im Anhang X auf Seite X dargestellt.

\subsection{Datenbankentwurf}
\label{sec:Datenbankentwurf}

Aufbauend auf den vorangegangenen Entwurfsergebnissen wird im Folgenden ein
Entwurf der Datenbank in Form eines Tabellenmodells angefertigt.\footnote{Ein
Tabellenmodell ist ein graphisches Modell zur Darstellung der Struktur einer
Datenbank. In diesem Modell werden Tabellen, deren Attribute (Eigenschaften)
und die Beziehung zu anderen Tabellen dargestellt.}

Bevor das Tabellenmodell erstellt werden kann, müssen in dem vorliegenden
Projekt die Daten identifiziert werden, die persistent in der Datenbank
gespeichert werden. Aus dem vorausgegangenen Oberflächenentwurf wird deutlich,
dass das Projekt aus den zwei Teilen, Benutzeransicht und Administrationsansicht,
besteht. Die Benutzeransicht stellt dabei Informationen dar, die in der
Administrationsansicht gepflegt und hinterlegt wurden. Der
Administrationsbereich ist der datenhaltende Bereich, in dem die Interaktion
mit der Datenbank stattfindet. Aus dem \verweis{Administrationsbereich} ergeben
sich bereits folgende drei Informationsobjekte:

\begin{itemize}
  \item Infotexte
  \item Fotos
  \item Interessante Orte
\end{itemize}

Die Beziehung zwischen Infotexten und Fotos stellt dabei eine sogenannte
n:m Beziehung dar. In dieser Beziehung kann ein Infotext mehreren Fotos
zugeordnet sein und im Gegenzug können ebenso mehrere Infotexte auf demselben
Foto platziert werden. Um eine solche Beziehung in einer relationalen Datenbank
abbilden zu können muss eine neue Tabelle angelegt werden, die diese Beziehung
zwischen Foto und Infotext speichert. Gleiches gilt für die
Nachbarschaftsbeziehung zwischen zwei Fotos. Da ein Foto mehrere Nachbarfotos
haben kann, muss wiederum eine Tabelle erstellt werden, in der die Beziehung
zwischen Foto und Nachbarfoto gespeichert wird. Neben diesen zwei zusätzlichen
Tabellen werden noch drei weitere Tabellen für die Speicherung der Daten der
Übersichtskarte benötigt. In der Übersichtskarte soll grundsätzlich zwischen dem
Studienstandort an der Baccumer Straße und dem Standort an der Kaiserstraße
unterschieden werden. Da beide Standorte unterschiedlich parametresiert werden 
müssen wird hierfür eine eigene Tabelle benötigt. Diese wird mit dem Namen
\textit{Area} bezeichnet. Eine Area beschreibt hierbei einen topologischen
Ausschnitt einer Google Maps Karte. Da eine topologische Karte nur den Umriss
von Gebäuden zeigt, ist es schwer, den genauen Standort eines Panoramas zu
bestimmen. Aus diesem Grund werden Grafiken über die Karte gelegt, die die
innere Struktur (Wände, Gebäudetrackte, etc.) des Campus zeigen. Sowohl die
topologische Karte als auch die übergelegten Grafiken, im folgenden
\textit{Overlays} genannt, benötigen eine eigene Tabelle, da beide jeweils noch
weitere Attribute haben. Ein Overlay ist dabei immer einer Karte zugeordnet.
Eine Karte kann weiterhin mehrere Overlays besitzen. Über die Zuordnung von
mehreren Overlays zu einer Karte werden Stockwerke an einem Standort
dargestellt. Ein Overlay enthält damit unter anderem den Pfad zu einer Grafik,
die ein Stockwerk an einem Standort darstellt. Zusammengefasst ergeben sich
folgende Tabellen:

\begin{itemize}
  \item infotext (Verwaltung der Infotexte)
  \item panorama (Verwaltung der Fotos)
  \item poi (Verwaltung der interessanten Orte)
  \item infotext\_panorama (n:m-Beziehung zwischen Infotexten und Fotos)
  \item neighbour (n:m-Beziehung zwischen zwei Fotos)
  \item area (Kartenbereich)
  \item map (Kartenansicht)
  \item overlay (übergelegte Grafik)
\end{itemize}

Diese Tabellen werden in \abbildung{Tabellenmodell} mit ihren jeweiligen
Attributen und Beziehungen dargestellt.

\begin{figure}[htb]
\centering
\includegraphics[width=1.0\textwidth]{Tabellenmodell.png}
\caption[Tabellenmodell der Anwendung]{Tabellenmodell der Anwendung\protect\footnotemark}
\label{fig:Tabellenmodell}
\end{figure}
\footnotetext{Quelle: Eigene Darstellung}
\subsection{Architektur}
\label{sec:Architektur}
\subsection{Pflichtenheft}
\label{sec:Pflichtenheft}
\section{Umsetzung}
\label{sec:Umsetzung}
\subsection{Panoramaerstellung}
\label{sec:Panoramaerstellung}

\subsubsection{Anforderungen an die Panoramafotos}
\label{sec:PanoramaerstellungAnforderungen}

Die Panoramaerstellung stellt, neben der Softwareerstellung einen zentralen
Teilprozess innerhalb der Projektumsetzung dar. Bevor mit der Erstellung der
Panoramas begonnen werden kann müssen zunächst die Anforderungen der Anwendung
an die Panoramafotos festgelegt werden. Hierbei muss definiert werden, welche
Kriterien die Panoramafotos erfüllen müssen, um von der Anwendung korrekt
verarbeitet werden zu können. Relevante Kriterien sind hierbei die
Darstellungsform, die Größe und das Format der Panoramafotos. Nachdem diese
Schnittstelle festgelegt ist kann der Prozess der Panoramaerstellung isoliert
vom Prozess der Softwareerstellung durchgeführt werden. Dies ermöglicht eine
parallele Umsetzung der beiden Teilprozesse.

Wie bereits in \verweis{Architektur} erwähnt soll in der Anwendung die Google
Street View API verwendet werden. Diese Schnittstelle stellt die Funktionalität
zur Anzeige der Panoramafotos zur Verfügung. Somit legt diese Schnittstelle auch
fest, in welcher Form die Panoramafotos zur Verfügung gestellt werden müssen.
Diese Anforderungen können aus der Entwicklerdokumentation zu der API entnommen
werden. Die Schnittstelle erwartet hierbei ein Panoramafoto, welches der
Rektangulatprojektion\footnote{Die Rektangularprojektion stellt eine
horizontale Ansicht von 360 Grad und eine vertikale Ansicht von 180 Grad dar.
Dies bedingt ein Seitenverhälnis von 2:1} entspricht. Durch die API wird diese
Darstellung auf die Fläche einer Kugel projeziert. Der Mittlepunkt dieser Kugel
stellt den Standpunkt des Betrachters dar. Da sich immer nur ein Teil der
Projektion im Blickfeld dieses Betrachters befindet muss auch nur der aktuell
sichtbare Bereich des Panoramafotos dargestellt werden. Aus diesem Grund bietet
die Google Street View API die Möglichkeit, ein in mehrere rechteckige Teile,
sogenannte Kacheln, aufgeteiltes Panoramafoto zu verarbeiten. Auf diese Weise
kann die Performanz der Anwendung erhöht werden, da nicht das komplette
Panoramafoto, sondern nur ein Teil der Kacheln in der Anwendung geladen werden
muss. Damit den einzelnen Kacheln die korrekte Position auf der Planarprojektion
zugewiesen werden kann, muss die Benennung der Kacheldateien dem Namensschema
\texttt{Kachelspalte-Kachelzeile} genügen. Die Anzahl der Kacheln sowie die
Pixelmaße des Panoramafotos können frei gewählt werden. Hierbei ist zu beachten,
dass mit steigender Pixelanzahl die Qualität der 360-Grad-Darstellung steigt,
die Performance der Anwendung jedoch aufgrund der steigenden Datengröße sinkt.
In einer prototypischen Implementierung, auf die im \verweis{Softwareerstellung}
näher eingegangen werden soll, hat die Projektgruppe verschiedene Kombinationen
aus Pixekmaße und Kachelanzahl getestet. Letztendlich hat sich die Projektgruppe
auf die Pixelmaße 4096x2048 für das gesamte Panorama und eine Aufteilung in 32
Kacheln entschieden. Diese Kombination wurde einheitlich als bester Kompromiss
zwischen Qualität und Performanz angesehen.
\subsubsection{Workflow der Panoramaerstellung}
\label{sec:Workflow}

Der Workflow der Panoramaerstellung kann in mehrere Teilprozesse untergliedert
werden. Für die Umsetzung dieser einzelnen Teilschritte wird hierbei spezielles
Hard- und Softwareequipment benötigt. Im Folgenden sollen die einzelnen
Teilschritte der Panoramaerstellung in chronologischer Reihenfolge erläutert
werden. Im Zuge dessen soll auch auf das benötigte Equipment eingegangen
werden.

\paragraph{Aufnahme der Einzelfotos} \hfill \\

Die Panoramaerstellung beginnt mit der Aufnahme mehrerer Einzelfotos. Aus diesen
Einzelfotos wird nachfolgend dann das komplette Panoramafoto zusammengefügt.
Dieses muss, wie zuvor erläutert, der Rektangularprojektion entsprechen. Die
Summe der Einzelfotos muss also eine horizontale Ansicht von 360 Grad und eine
vertikale Ansicht von 180 Grad abbilden. Die Anzahl der Einzelfotos ist somit
von dem Bildwinkel abhängig, der auf einem einzelnen Foto dargestellt werden
kann. Dieser Bildwinkel wird in der Fotografie durch die Brennweite des
Objektives festgelegt. Je geringer die Brennweite eines Objektives ist, desto
größer ist der Bildwinkel, der mit diesem Objektiv eingefangen werden kann und
desto weniger Einzelfotos werden für die Erstellung eines Panoramafotos
benötigt. Durch spezielle Objektivkonstruktionen wie zum Beispiel einem
Fisheye-Obektiv ist es möglich einen besonders großen Bildwinkel aufzunehmen.
Ein solches Objektiv und eine zum dem Objektiv kompatible Kamera wird auch für die Aufnahme der
Panoramafotos im Projekt verwendet. Das Objektiv ermöglicht es in 16 Fotos alle
Ansichten Aufzunehmen, die zur Darstellung einer Rektangularprojektion benötigt
werden. In 15 dieser Einzelfotos ist hierbei die horizontale Rundumansicht
dargestellt. Das sechzehnte Foto bildet den Zenit\footnote{Der Zenit ist der
Punkt über dem Beobachter/der Kamera} im Panoramafoto ab. Der
Nadir\footnote{Der Nadir ist der dem Zent gegenüber gelegene Punkt unter dem
Beobachter/der Kamera} wird in den Panoramafotos nicht abgebildet, da sich hier
das Stativ befindet.

Zur Unterstützung bei der Aufnahme hat sich die Projektgruppe weiterhin
entschieden ein speziell für die Panoramafotografie ausgelegtes Stativ zu
verwenden. Eine Aufnahme ohne dieses Stativ wäre zwar denkbar, würde die
Qualität der Panoramafotos jedoch stark beeinträchtigen. Das Panoramastativ
besteht aus folgenden Komponenten:

\begin{description}
\item[Stativ] Das eigentliche Stativ dient dazu, die Kamera im Raum an einer
festen Position zu fixieren. Hierdurch ist sichergestellt, dass alle
Einzelfotos von der selben Position aus aufgenommen werden.
\item[Nivelliervorrichtung] Die Nivelliervorrichtung dient dazu, den Kopf des
Statives horitontal auszurichten. Auf diese Weise wird ein wellenförmiger
Horizont in dem zusammengefügten Panoramafoto vermieden.
\item[Rotator] Der Rotator ermöglicht es, den Kopf des Stativs um die vertikale
Achse zu drehen. Weiterhin kann eine Gradzahl festgelegt werden, nachdem der
Mechanismus bei einer Drehnung einrastet. Auf diese Weise wird eine gleichmäßige
Drehung bei der Aufnahme der Panoramafotos ermöglicht.
\item[Nodalpunktadapter] Der Nodalpunktadapter ermöglicht es, die Kamera auf
dem Stativ so zu positionieren, dass die Eintrittspupille des Stativs auf Höhe
der vertikalen Drehachse des Stativs liegt. Hierdurch können
Parallaxenfehler\footnote{Der Parallaxenfehler beschreibt eine scheinbare
Verschiebung zweier hintereinandergelegener Gegenstände, wenn sich der
Ausgangspunkt der Betrachtung ändert} bei der Aufnahme vermieden werden
\end{description}

Das komplette Equipment, welches zur Aufnahme der Panoramafotos verwendet wurde,
ist in \abbildung{Equipment} dargestellt.

\begin{figure}[htb]
\centering
\includegraphics[width=0.5\textwidth]{Equipment.png}
\caption[Equipment zur Aufnahme der Einzelfotos]{Equipment zur Aufnahme der Einzelfotos\protect\footnotemark}
\label{fig:Equipment}
\end{figure}
\footnotetext{Eigene Darstellung}

\paragraph{Stitching und Rendering} \hfill \\

Nachdem die Einzelfotos aufgenommen wurden müssen diese zu einem Panoramafoto
zusammengefügt werden. Dieser Prozess wird als Stitching bezeichnet. Das
Stitching erfolgt im Projekt mit der Software Kolor Autopano Giga 3.0. Die
Software ist in der Lage, die Einzelfotos automatisiert zusammenzufügen.
Gegebenenfall muss das hieraus resultierende Panoramafoto nach dem
automatisierten Zusammenfügen noch manuell ausgerichtet werden. Aufgrund
dieser manuellen Ausrichtung ist es nicht möglich, den hier beschriebenen
Prozess vollständig zu automatisieren. In einem letzen Schritt wird das
zusammengefügte Panoramafoto dann in das JPEG-Format konvertiert. Dieser Schritt
wird als Rendering bezeichnet. Das Ergebnis dieses Prozesses ist in
\abbildung{Zusammengefuegt} dargestellt.

\begin{figure}[htb]
\centering
\includegraphics[width=0.8\textwidth]{Zusammengefuegt.png}
\caption[Panoramafoto nach Stitching und Rendering]{Panoramafoto nach Stitching
und Rendering\protect\footnotemark}
\label{fig:Zusammengefuegt}
\end{figure}
\footnotetext{Eigene Darstellung}

\paragraph{Überarbeitung und Optimierung der Panoramafotos} \hfill \\

Da es bei der Aufnahme der Einzelfotos nicht möglich war den Nadir abzubilden,
fehlen diese Informationen folglich auch in dem zusammengefügten
Pa\-no\-ra\-ma\-fo\-to. In \abbildung{Zusammengefuegt} ist der Bereich, in dem
die Bildinformationen fehlen, durch eine schwarze Fläche markiert. Im Zuge
einer Überarbeitung der Panoramafotos soll dieser schwarze Bereich mit einem
Label überdeckt werden, auf dem das Logo des Projektes dargestellt ist. Das
Label muss dabei so verzerrt werden, dass es nach der bereits beschriebenden
Projezierung durch die Google Street View API entzerrt dargestellt ist.

Neben der Überarbeitung der Panoramafotos werden diese weiterhin für die
Darstellung im Internet optimiert. Hierbei wird die Größe des Panoramafotos auf
die Pixelmaße 4096x2048 reduziert und die JPEG-Datei komprimiert abgesspeichert.

Die Bearbeitung der Panoramafotos erfolgt mit der Software Adobe Photoshop CS5.
Durch die Stapelverarbeitungsfunktion des Programms können alle hier
beschriebenen Teilschritte automatisiert durchgeführt werden. Das Ergebnis
dieses Prozesses ist in \abbildung{Ueberarbeitet} dargestellt.

\begin{figure}[htb]
\centering
\includegraphics[width=0.8\textwidth]{Ueberarbeitet.png}
\caption[Panoramafoto nach Überarbeitung und Optimierung]{Panoramafoto nach
Überarbeitung und Optimierung\protect\footnotemark}
\label{fig:Ueberarbeitet}
\end{figure}
\footnotetext{Eigene Darstellung}

\paragraph{Aufteilung des Panoramafotos in Kacheln} \hfill \\

Das Blickfeld des Benutzers in der Anwendung nimmt immer nur einen Teil der
Rundumansicht ein. Somit kann auch immer nur ein Teil des Panoramafotos
dargestellt werden. Es ist daher nicht sinnvoll das komplette Panoramafoto in
der Anwendung zu laden, bevor es dem Benutzer präsentiert wird.
Aus diesem Grund wird das Panoramafoto in einem letzten Bearbeitungsschritt in
mehrere Kacheln aufgeteilt. Ziel dieser Aufteilung ist eine Steigerung der
Performanz und eine Reduzierung der Ladezeiten für die Panoramafotos in der
Anwendung. Es muss immer nur der Teil der Kacheln geladen werden, der dem
Benutzer aktuell präsentiert wird. Wenn sich das Blickfeld des Benutzers ändert,
können einzelne Kacheln schnell nachgeladen werden.

Die Aufteilung der Panoramafotos im Projekt erfolgt, wie auch die Überarbeitung
der Fotos, mit der Software Adobe Photoshop CS5. Auch hier kann die
Stapelverarbeitungsfunktion genutzt werden, um die Aufteilung der Panoramafotos
zu automatisieren.

\subsection{Softwareerstellung}
\label{sec:Softwareerstellung}

Die zu erstellende Software erfüllt die Anforderungen des Lastenheftes, in dem die Modelle und Konzepte der Entwurfsphase implementiert werden. Die Umsetzung eines solchen Projektes mit einem mehrköpfigem Projektteam erfordert neben entsprechenden Softwareentwürfen auch Vorbereitung und den Einsatz spezieller Entwicklungswerkzeuge (Tools), die im nachfolgenden Abschnitt beschrieben sind.
\subsubsection{Vorbereitung und Tools}
\label{sec:VorbereitungUndTools}

Die besondere Herausforderung des vorliegenden Projektes besteht darin, eine
mehrschichtige Software, bestehend aus Benutzer- und Administrationsansicht, in
einem Projektteam zu entwickeln, welches sich in zwei Punkten von der
Projektorganisation eines klassischen\footnotemark\ Softwareprojektes
unterscheidet. Zum einem muss  as Projektteam, bedingt durch unterschiedliche
Arbeitsorte, räumlich getrennt voneinander entwicklen können. Zum anderen
ist das Projektteam während des Entwicklungszeitraums nicht vollständig im
Projekt eingebunden. Es müssen Arbeits-, Urlaubs-, Prüfungs- und
Krankheitszeiten aller Mitglieder des Projektteams als hemmende Zeitfaktoren
berücksichtigt werden. Eine genauere Betrachtung dieser Projektsituation ist an
dieser Stelle nicht notwendig.

\footnotetext{Klassische Softwareprojekte werden hier als Projekte
in einer reinen Projektorganisation verstanden. Zur weiteren Vertiefung siehe
\citet[S.~105]{jenny2001}}

Die beiden genennten Aspekte bergen organisatorische Risiken und können auch zu
Problemen bei der Entwicklung führen. Das Hauptrisiko ist dabei ein Stillstand
des Projektes, welcher dadurch bedingt ist, dass alle Projektmitglieder auf das
Wissen eines Mitglieds angewiesen sind, welches gerade nicht zur Verfügung
steht. Dieses Risiko soll durch den Einsatz von zwei Entwicklungswerkzeugen
minimiert werden.

Zunächst muss dafür gesorgt werden, dass alle Mitglieder immer auf dem
aktuellen Stand des Projektes sind. Wird ein Entwicklungsschritt von einer
einzelnen Person abgeschlossen muss das Ergebnis den anderen zur Verfügung
gestellt werden. Zusätzlich müssen Änderungen am bestehnden Projektstand
kommentiert und dokumentiert werden. Diese Anforderung werden im vorliegenden
Projekt mithilfe einer Projektversionierung erreicht. Ein Versionierungstool
bietet hierbei die Möglichkeit den Quellcode eines Softwareprojektes zentral in
einem sogenannten Repository (zu deutsch "`Depot"') zu halten, um ihn für alle
Mitglieder verfübgar zu machen. Der aktuelle Stand des Projektes kann aus
diesem Repository jederzeit abgefragt werden und Änderungen werden von
Mitgliedern dorthin zurückgeschrieben. Zusätzlich bieten Versionierungstools die
Möglichkeit, die Änderung mit einem Kommentar zu versehen. Die Dokumentation der
konkrteten Änderungen wird vom Tool automatisch durchgeführt. Als
Versionierungssoftware wurde im vorliegenden Projekt "`Github"'\footnotemark\
verwendet, da allen Mitglieder mit diesem Versionierungstool bereits gearbeitet
haben.

\footnotetext{Github ist eine Open Source Projekt, das auf dem
Versionierungstool git aufsetzt. Es stellt den De-facto-Standard für
Webprojekte dar.}

Neben dem Quellcode sollen auch die Anforderungen an die Anwendung und die
aktuellen Aufgaben der Projektmitglieder an zentraler Stelle verwaltet werden.
Bei Ausfall eines Mitgliedes müssen seine aktuellen Aufgaben zentral
dokumentiert sein, um diese auf andere Mitglieder verteilen zu können. Dazu
wurde ein webbasiertes Ticketsystem aufgesetzt, auf das alle Mitglieder zugriff
haben. Hierbei wurde sich für das Ticketsystem "`PHProjekt"' entschieden, da PHP
als Technologie bereits bekannt ist und das Ticketsystem somit schnell
aufgesetzt werden konnte. In diesem Ticketsystem werden alle erstellten
Arbeitspakete des Projektes mit zugeordnetem Mitglied, Bearbeitungszeitraum,
Anforderungsbeschreibung und benötigter Bearbeitungszeit hinterlegt.
\subsubsection{Prototyp}
\label{sec:Prototyp}

Nach Bereitstellung des Ticketsystems und des Versionierungstools kann mit der Implementierung der Software begonnen werden. Für die Auftraggeber des Projektes war es dabei besonders wichtig, dass zunächst ein Prototyp der späteren Software entwickelt wird. Dieser Prototyp sollte nur die 360 Grad Ansicht eines Campus Panoramas und die Möglichkeit zu einem weiteren Panorama zu navigieren enthalten. Der Prototyp sollte genutzt werden um Entscheidungsträger von Beginn an vom Projekt zu überzeugen.

Da der Prototyp nur einen statischen Einblick (kein dynamischer Inhalt aus einer Datenbank) in die späteren Benutzeransicht gewähren sollte, wurde hierfür ein HTML-Dokument geschrieben, das die Google Street View API einbindet und über Javascript dessen Funktionalität implementiert. Das erstellte HTML-Dokument ist in \listing{HTML Prototyp} dargestellt:

\lstinputlisting[language=HTML,caption={HTML Prototyp},label={lst:HTML Prototyp}]{Listings/HTML_Prototyp.html}

Das dargestellte HTML-Dokument bindet in Zeile 5 die angesprochene Google Street View API ein, in der Funktionen zur Darstellung des 360 Grad Panoramas definiert sind. Zusätzlich wird in Zeile 6 eine Javascript-Datei eingebunden, in der die Funktionen der Google Street View API auferufen werden. Im Body-Bereich des HTML-Dokumentes ist dafür ein Element definiert das als Fläche zur Darstellung der Panoramas genutzt wird. Die eingebundene Javascript-Datei aus Zeile 6 ist aus Platzgründen in \listing{Javascript Prototyp} in Anhang ~\ref{sec:AnhangJavascriptPrototyp} dargestellt. Dessen Funktionalität wird an dieser Stelle kurz erläutert.

Sobald der Internetbrowser des Benutzers das HTML-Dokument fertig geladen hat, wird eine Methode aufgerufen, die benötigte Parameter zur Erstellung des Panoramas bereitstellt\footnote{Vergleiche Zeile 3 im Anhang ~\ref{sec:AnhangJavascriptPrototyp}}. Hier werden Zoomstufe, Ausrichtung und weitere Parameter festgelegt. Anschließend wird ein Panorama-Objekt mit Hilfe der Google Street View API erstellt und an das oben beschriebene HTML-Element im Body-Bereich gebunden\footnote{Vergleiche Zeile 16-18 im Anhang ~\ref{sec:AnhangJavascriptPrototyp}}. Über einen weiteren Aufruf einer API-Funktion können an das Panorama sogenannte "`Links"' gehängt werden. Diese Links stellen später für den Benutzer die Pfeile am Boden dar über die zu anderen Panoramas navigiert werden kann. Im Prototyp wird nur der Verweis auf ein anderes Panorama gebraucht, das über die Funktion "`createCustomLinks"'\footnote{vergleiche Zeile 45ff. im Anhang ~\ref{sec:AnhangJavascriptPrototyp}} als Link angehängt wird.

Ein Bildschirmfoto des Prototypen der mit diesen zwei Dateien (HTML- und Javascript-Dokument) erstellt wurde ist in \abbildung{PrototypScreenshot} zu sehen.

%TODO: Screenshot ohne Pixelfehler machen!
\begin{figure}[htb]
\centering
\includegraphics[width=1.0\textwidth]{PrototypScreenshot.png}
\caption[Protoyp Bildschirmfoto]{Bildschirmfoto des Prototypen}
\label{fig:PrototypScreenshot}
\end{figure}
\subsubsection{Administrationsbereich}
\label{sec:Administrationsbereich}

Ein Oberflächenentwurf des Administrationsbereiches ist in
\abbildung{MockupBackend} dargestellt.

\begin{figure}[htb]
\centering
\includegraphics[width=1.0\textwidth]{MockupBackend.png}
\caption[Oberflächenentwurf des
Administrationsbereiches]{Oberflächenentwurf des
Administrationsbereiches\protect\footnotemark}
\label{fig:MockupBackend}
\end{figure}
\footnotetext{Quelle: Eigene Darstellung}

Der Administartionsbereich gliedert sich hierbei in folgende Bereiche:

\begin{itemize}
  \item Infotextverwaltung
  \item Fotoverwaltung
  \item Interessante Ort
  \item Übersichtskarte
\end{itemize}

In der \textbf{Infotextverwaltung} können Informationstexte verfasst und zu
Panoramafotos zugeordnet werden. Ein Informationstext kann dabei auch mehreren
Fotos zugeordnet werden. Dadurch können dem Benutzer alle Informationstexte
angezeigt werden, die sich auf den Standpunkt beziehen, an dem sich dieser
gerade befindet. Alle Informationstexte, die bereits erstellt wurden, werden in
der Infotextverfaltung tabellarisch aufgelistet. Jeder dieser Informationstexte
kann sowohl verändert als auch gelöscht werden.

In der \textbf{Fotoverwaltung} werden analog zu der Infotextverwaltung die
erstellten Panoramafotos hinterlegt und gepflegt. Erstellte Panoramafotos
können in diesem Menüpunkt mit Namen und Beschreibung hochgeladen werden.
Analog zu den Infotexten werden hier die bereits hochgeladenen Fotos
tabellarisch aufgelistet. Auch die Möglichkeit zur Änderung und Löschung der
Fotos ist hier gegeben.

Der Bereich \textbf{Interessante Orte} bietet die
Möglichkeit, die Standorte zu hinterlegen, die dem Benutzer beim Klicken
auf die Minimap angezeigt werden. Zu diesen Standorten kann der Benutzer
dann springen, ohne dorthin navigieren zu müssen. Diese Möglichkeit ist vorallem
für Benutzer der Anwendung von Vorteil, die den Campus nicht kennen und nicht
erst nach bestimmten Orten suchen wollen. Zur Pflege der interessanten Orte muss
im entsprechenden Bereich in der Administrationsoberfläche nur ein
beschreibender Text eingetragen und mit einem hochgeladenen Foto verlinkt werden.
 
Die \textbf{Übersichtskarte} stellt den letzen und komplexesten Bereich des
Administrationsbereiches dar. In der Übersichtskarte werden die hochgeladenen
Panoramafotos auf einer Karte des Hochschulgebäudes platziert. Dazu wird dem
angemeldeten Administrator zunächst eine topographische Karte des Campus
präsentiert. Auf dieser Karte kann der Administrator durch Klicken eine
Position bestimmen, zu der er ein Foto auswählen und speichern kann.
Ein Foto kann dabei nur an einer Position auf der Übersichtskarte platziert
werden. Der Administrator hat die Möglichkeit die Position jedes Fotos beliebig
oft zu ändern. Darüber hinaus werden dem Administrator Steuerelemente angezeigt
mit denen er einen bestimmten Bereich des Campus auswählen kann, in dem er
Fotos positionieren möchte. So kann der Administrator sowohl zwischen
verschiedenen Gebäudekomplexen als auch zwischen den einzelnen Stockwerken
innerhalb eines Gebäudekomplexes wählen. Neben der Möglichkeit ein Foto zu
positionieren hat der Administrator weiterhin die Möglichkeit die Verbindungen
zwischen positionierten Fotos zu pflegen. Aus dem Oberflächenentwurf der
Benutzeroberfläche, welcher im vorherigen Abschnitt vorgestellt wurde, ist
ersichtlich, dass ein Benutzer von einem Panoramafoto aus zu einem anderen
Standpunkt navigieren kann. Diese Navigation beruht auf Verbindungen zwischen
den Panoramafotos. Solche Verbindungen werden in der Übersichtskarte geplegt.
Die Steuerungselemente zu Pflege der Verbindungen zwischen Panoramafotos sind zum
Zeitpunkt der Erstellung des Mockups noch nicht definiert, die Notwendigkeit
dieser Funktion ist aber bedacht.
\subsubsection{Application Programming Interfaces (APIs)}
\label{sec:APIs}

Aufbauend auf dem abgeschlossen Administrationsbereich werden im Folgenden
Application Programming Interfaces, kurz APIs, implementiert. Aufgabe solcher
APIs ist es, Informationen in maschinenleserlicher Form für andere Teilbereiche
des Systems bereitzustellen. Die bereitgestellten Informationen werden von den
aufrufenden Systemen genutzt, um Inhalte dynamisch darzustellen.
In einem Beispiel soll der zuvor vorgestellte Prototyp die Menge der
benachbarten Panoramas über eine solche API beziehen und darauf aufbauend dem
Benutzer die Navigationspfeile entsprechend präsentieren. Die Implementierung
einer API wird nachfolgend an diesem Beispiel erläutert.

Die Realisierung der Schnittstelle vollzieht sich in zwei Schritten.
Zuerst werden die Informationen maschienenleserlich geschrieben und ausgeben. Im
zweiten Schritt werden diese Informationen dann von einem verarbeitenden System
eingelesen und ausgewertet. Das Schreiben von maschinenleserlichen Informationen
hängt stark davon ab, welche Maschine den ausgegeben Text lesen bzw.
interpretieren soll. Im vorliegenden Projekt werden die erstellten APIs
ausschließlich von Javascript-Routinen angefragt, um Inhalte asynchron
nachzuladen. Auf die Bedeutung von asynchronem Nachladen wird später genauer
eingegangen. An dieser Stelle ist lediglich zu beachten, dass die APIs von
Javascript-Routinen angefragt werden. Aus diesem Grund werden die Informationen
der API im JSON-Format dargestellt. JSON steht dabei für "`Javascript Object
Notation"' und ist der de facto Standard für die Kommunikation zwischen
webbasierten Schnittstellen.\footnote{\citet[S.~20]{lubbers2011}} Die
Darstellung im JSON-Format bietet den großen Vorteil, dass innerhalb von
Javascript aus den dargestellten Informationen ein Objekt im Sinne der
objektorientierten Programmierung\footnotemark erstellt werden kann. An dieser
Stelle soll diese Begründung für die Wahl des JSON-Formats ausreichen. Eine
genauere Betrachtung erfolgt im zweiten Schritt der Implementierung der API.
Neben der Klassifikation der API muss noch der darzustellende Inhalt definiert
werden. Für den oben beschriebenen Anwendungsfall müssen hierbei alle Nachbarn
eines gegebenen Panoramas dargestellt werden. Für die Ausrichtung der
Navigationspfeile wird zusätzlich die Himmelsrichtung in Grad jedes Nachbarn
relativ zum Standpunkt des gegeben Panoramas benötigt. Dieser letzte Wert wird
als "`Heading"' bezeichnet und wird bereits bei der Positionierung des
360-Grad-Fotos in der Datenbank gespeichert. Er muss also nur aus der Datenbank
abgefragt werden.

\footnotetext{Die Objektorientierte Programmmierung (OOP) ist
das führende Programmierparadigma für Webanwendungen. Dieses Paradigma
beschreibt eine bestimmte Denkweise für Problemstellungen der Informatik. Für
weitere Einblicke siehe \citet{poetzsch2000}}

Aufbauend auf der vorausgegangenen Beschreibung der API kann diese in PHP
implementiert werden. Dazu wird zunächst das in \verweis{Datenbankentwurf}
beschriebene Tabellenmodell in Bezug auf die darzustellenden Informationen
untersucht. In der \abbildung{Tabellenmodell} ist zu sehen, dass
\textit{heading} ein Attribut der Tabelle \textit{neighbour} ist. Über diese
Tabelle können zu einem gegeben Panorama alle Nachbarn mit entsprechendem
\textit{heading} gefunden werden.
Im Zuge der Implementierung sollen im Folgenden mit Hilfe von PHP über SQL alle
Nachbarn eines gegebenen Panoramas abgefragt werden. Das Ergebnis dieser
Abfrage soll im JSON-Format dargestellt werden. Im \listing{PHP_Nachbar_API}
ist diese Funktionalität implementiert.

\clearpage

\lstinputlisting[language=PHP,caption={PHP Nachbar
API},label={lst:PHP_Nachbar_API}]{Listings/PHP_Nachbar_API.php}

Das gebene Panorama wird im Listing über die aufrufende URL, also einem
HTTP\footnotemark -Parameter, gesetzt. In der URL
\url{http://vcl.example.com/api/api\_test.php?id=1} würde beispielsweise der
Parameter id ("`?id=1"') mit der Panorama-ID 1 übergeben werden. Das
Auslesen dieser Information ist im Listing in Zeile 2 dargestellt.
Unter der Annahme, dass das Panorama mit der ID 1 zwei Nachbarn hat, würde der
Aufruf der API folgendes Ergebnis liefern:

\footnotetext{HTTP steht für Hypertext Transfer Protocol und bezeichnet ein
Protokoll, das den Übertragungsstandard für Webdokumente darstellt. HTTP stellt
damit eine fest protokollierte Struktur auf, in der geregelt ist, wie ein
Dokument über das Internet übertragen wird.}

\begin{figure}[htb]
\centering
\includegraphics[width=0.4\textwidth]{ScreenshotAPIBeispiel.png}
\caption[API Beispiel]{Bildschirmfoto des gegebenen API Beispiels}
\label{fig:ScreenshotAPIBeispiel}
\end{figure}

Die Darstellung dieser Ausgabe wurde mit Hilfe eines Darstellungstools auf
bessere Lesbarkeit optimiert. Im Normalfall würde die Ausgabe in einer Zeile
dargestellt werden. Dies zeigt wiederum, dass das Ausgabeformat nicht auf
menschliche Lesbarkeit ausgelegt ist.

Nachdem das ausgebende System der Beispielschnittstelle implementiert wurde,
soll dieses nachfolgend angefragt und die Antwort des Systems ausgewertet
werden. Die Anfrage an das System erfolgt, wie bereits erwähnt, asynchron
innerhalb einer Javascript-Funktion. Asynchron bedeutet hierbei, dass die
Anfrage unabhängig von dem Aufbau der restlichen Seite ausgeführt wird.
Unabhängig von der aktuell dargestellten Seite wird eine Anfrage ausgeführt,
dessen Ergebnis in die bereits dargestellte Seite integriert wird.

In \verweis{Prototyp} wurde bereits die Funktion "`createCustomLink"' aus dem
Anhang ~\ref{sec:AnhangJavascriptPrototyp}
(\nameref{sec:AnhangJavascriptPrototyp}) vorgestellt. In dieser Funktion
werden die Links festgelegt, die letzenendes die Navigationspfeile in der
Benutzeransicht abbilden. Im \verweis{Prototyp} wurden diese Links statisch
gesetzt. Nachfolgend soll der Prototyp in der Weise abgeändert werden, dass die
Links durch Aufrufen der API dynamisch festgelegt werden. Dazu wird die Funktion
\textit{createCustomLink} zunächst um einen Funktionausruf der Funktion
"`getPanoJson"' erweitert. Diese Funktion ist dafür zuständig, die oben
definierte API mit einer übergebenen ID anzufragen und ein JSON-Objekt an die
aufrufende Methode zurückzuliefern. Die Implementierung dieser Funktion ist in
\listing{Dynamisch_Nachbarn_nachladen} dargestellt.

\lstinputlisting[language=JavaScript,caption={Dynamisch Nachbarn
nachladen},label={lst:Dynamisch_Nachbarn_nachladen}]{Listings/Dynamisch_Nachbarn_nachladen.js}

Die Funktion \textit{getPanoJson} wird in Zeile 5 aufgerufen und fragt daraufhin
über einen sogenannten \textit{XMLHttpRequest} die oben genannte URL an (Zeile
25). Da die Antwort als unformartiertes Textdokument erfolgt und es nicht
möglich ist per HTTP Objekte zu übertragen muss die Antwort zunächst in ein
JSON-Objekt umgewandelt werden. Man spricht dabei von "`parsen"' (Zeile 28). Die
Elemente des zurückgelieferten JSON-Objektes (Zeile 30) können daraufhin von der
aufrufenden Funktion referenziert werden. Über "`pano.neighbours"' (Zeile 7)
erhält man beispielsweise eine Liste aller Nachbarn, die im oben dargestellten
Quellcode durchlaufen und in die \textit{Links}-Liste geschrieben werden (Zeile
8ff.).

Durch die Erweiterung des Prototyps ist dieser in der Lage, die im
Administrationsbereich gepflegten Daten dynamisch abzurufen.

An dieser Stelle des Entwicklungsprozesses sind die Funktionen des
Administrationsbereichs vollständig umgesetzt und die Benutzeransicht greift
über APIs dynamisch auf die hinterlegten Informationen zu. Die Umsetzung des
Projektes ist damit abgeschlossen.
\section{Test}
\label{sec:Test}

Die zu erstellende Software ist mit Abschluss der Umsetzungsphase fertiggestellt und einsatzbereit. Vor Projektabschluss und -übergabe soll die Software auf seine Qualität geprüft werden. Es soll darüber hinaus verifiziert werden, dass die Anforderungen des Lastenheftes umgesetzt wurden. Sowohl Qualität, als auch die Umsetzung der Anforderungen wird im Folgenden in einem zweistufigen Praxistest der Software sichergestellt.

Im ersten Schritt wird die Software einem Alphatest unterzogen. Ein Alphatest liegt immer dann vor, wenn die Tests von Personen durchgeführt werden, die an der Entwicklung eines Produktes beteiligt waren, in diesem Fall das Projektteam. In diesem Test werden vor allem die Produktfunktionen des Lastenheftes (siehe Seite \pageref{sec:Lastenheft}) durchgespielt. In diesem Test sollen erste grobe Fehler in Bedienung, Anzeige und Verhalten gefunden werden und es soll zusätzlich sichergestellt werden, dass alle Anforderungen des Auftraggebers umgesetzt sind. Die Tests der Produktfunktionen werden zusätzlich in den vier führenden Browsern\footnote{Quelle: \url{http://www.chip.de}} durchgeführt, um browserübergreifende Funktionalität sicherzuestellen und dabei jeweils in die Bereiche Bedienung (\textbf{B}), Anzeige (\textbf{A}) und Verhalten(\textbf{V}) unterteilt. Das Ergbeniss des durchgeführten Alphatests ist in nachfolgender Tabelle festgehalten.

\begin{table}[htbp]
  \centering
    \begin{tabular}{l|cccc}
    \toprule
          & Google Chrome & Mozilla Firefox & Internet Explorer & Safari \\
    \midrule
    LF0010 &       &       & X     &  \\
    LF0015 &       &       & X     &  \\
    LF0020 &       &       & X     &  \\
    LF0030 & X     &       &       &  \\
    LF0040 &       &       & X     &  \\
    LF0050 &       & X     &       &  \\
    LF0060 &       & X     &       &  \\
    LF0070 &       & X     &       &  \\
    LF1010 &       & X     &       &  \\
    LF1020 &       & X     &       &  \\
    LF1025 &       & X     &       &  \\
    LF1030 &       & X     &       &  \\
    LF1040 &       & X     &       &  \\
    LF1110 &       & X     &       &  \\
    LF1120 &       & X     &       &  \\
    \bottomrule
    \end{tabular}
  \caption{Ergebnis des Alphatests}
  \label{tab:ErgebnisAlphatest}
\end{table}

Ergbniss des Alphatest ist ....
\subsection{Projektabschluss}
\label{sec:Projektabschluss}

Das Projekt Virtueller Campus Lingen ist mit abschließender Fertigstellung der Anwenderdokumentation
abgeschlossen. Das Projekt muss in dieser Projektphase dem Auftraggeber übergeben werden.
Das zu übergebende Projekt besteht dabei aus den folgenden Teilen:

\begin{itemize}
  \item den vollständigen Quellcode der Software
  \item eine Kopie der Datenbank nach Abschluss des Projektes (Sicherheitskopie)
  \item die Installation der Webanwendung auf einem Webserver
  \item die Internetadresse der Anwendung und alle benötigten Passwörter
  \item eine Einführung in die Verwendung der Software anhand der Dokumentation
\end{itemize}

Auf Basis dieser Informationen und Materialien ist der Auftraggeber in der Lage das Projekt zu pflegen und zu erweitern. Die Wahrung der Nachhaltigkeit ist damit vollständig gewährleistet. Die Installation und Übergabe der Webanwendung ist für Mitte Juli 2014 geplant.
\subsection{Dokumentation}
\label{sec:Dokumentation}

Nach Abschluss und Korrektur des Alpha- und Betatests ist das Projekt funktional abgeschlossen. Zur Förderung von Erweitarbarkeit und Nachhaltigkeit wird eine Dokumentation angefertigt, die es jedem berechtigten Benutzer ermöglicht, die Software zu pflegen und zu erweitern. Die vollstände Dokumentation kann in \citet{dokumentation2014} nachgelesen werden. An dieser Stelle wird zur Transparenz der Benutzung der Awendung der Ablauf zur Erweiterung der Anwendung vorgestellt. Dieser Prozess umfasst ingesamt maximal sechs Teilschritte.

\begin{description}
  \item[1.] Als erstes muss ein neues Panorama aufgenommen werden. Wichtig dabei zu beachten ist, dass die Mitte des Panoramas nach Norden ausgerichtet ist. Es empfiehlt sich darüber hinaus ein Stativ und Panoramakopf zu verwenden, um eine optimale Fotoqualität zu erreichen (siehe \verweis{Panoramaerstellung}).
  \item[2.] Anschließend muss das erstelle Panorama im Administrationsbereich der Software unter dem Menüpunkt "`Fotos"', mit Namen und Beschreibung, hochgeladen werden. An dieser Stelle empfiehlt es sich eine einheitliche Namenskonvention zu verwenden, die Auskunft über den Ort der Aufnahme gibt, also zum Beispiel "`KE\_EG\_001"' (Gebäude KG, Erdgeschoss, Raum 001). Diese Namenskonvention hilft im Folgenden dabei, dieses Foto zu identifizieren.
  \item[3.] Als nächstes kann bei Bedarf ein Informationstext zu diesem Foto erstellt werden. Falls ein interessantes Projekt oder wichtige Informationen hierauf abgebildet sind können diese über einen Informationstext dem Benutzer angezeigt werden.
  \item[4.] Ebenso kann das erstellte und hochgeladene Foto einen interessanten Ort darstellen, für den es sich lohnt eine gesonderte Verlinkung in der Benutzeransicht zu setzen. Beispiele für solche Fälle sind zum Beispiel ein Einstiegsfoto in der Bibliothek, ein zentrales Foto in einem Fachbereich, ein Foto mit Überblick über den gesamten Campus o.ä. Ist dies der Fall kann im Menüpunkt "`Interessante Orte"' ein Eintrag mit Namen und Beschreibung des Ortes hinterlegt und das erstellte Panorama damit verknüpft werden.
  \item[5.] Im Vorletzen Schritt muss das neue Foto auf der Karte des Campus positioniert werden. Dazu navigiert der Administrator zum Menüpunkt "`Übersichtskarte"' und klickt auf die Position der angzeigten Karte, an der das Foto aufgenommen wurde. Es erscheint ein roter Marker und nach einem weiteren Klick auf diesen Marker öffnet sich ein Auswahlmenü. In diesem Menü werden dem Administrator alle hochgeladenen Fotos angezeigt, die er am ausgewählten Punkt positionieren kann. Der Administrator wählt das neu hochgeladene Foto aus (im vorherigen Beispiel "`KE\_EG\_001"') und drückt auf "`Speichern"'.
  \item[6.] Nach dem Positionieren des neuen Fotos erscheint ein grüner Marker an der festgelegten Position. Durch klicken auf diesen Marker hat der Administrator im letzen Schritt die Möglichkeit vorher gespeicherte Informationstexte (siehe punkt 3.) an dieses Foto anzuhängen oder diesen Punkt mit anderen zu verbinden. Durch die Verbindung zu anderen Fotos kann ein Benutzer in der Benutzeransicht über andere Fotos zu dem neu hochgeladenen Foto navigieren. Anders ausgedrückt ist ein hochgeladenes, aber nicht verbundenes Foto für die Endbenutzer nicht sichtbar, da er zu diesem nicht navigieren kann.
\end{description}
\subsection{Übergabe}
\label{sec:Uebergabe}
\section{Ausblick}
\label{sec:Ausblick}

Am Ende des Projektes sollen an dieser Stelle Ideen und Anregungen zur Weiterführung des Projektes gegeben werden. Viele dieser Ideen kamen während des Projektverlaufes auf, ließen sich aber aus zeitlichen Gründen nicht realisieren.

Ein Beispiel dafür ist die Umsetzung einer geführten Tour für Studieninteressierte durch bestimmte Studienbereiche. Auf Messen wäre es dadurch möglich Auschnitte und Informationen zu bestimmten Studienbereichen ohne Benutzerinteraktion zu zeigen. Auch für Studieninteressierte, die sich nicht selbst durch den Campus bewegen, sondern lediglich die wichtigsten Informationen zu Ihrem Studiengang erhalten wollen, wäre dieser Ansatz interessant.

Gleichzeitig wäre, besonders bei den erwähnten Messeauftritten, eine Version der Software von Vorteil, die ohne Internetanbindung verwendet werden kann. Zum Zeitpunkt des Projektabschlusses greift die Software auf einige Programmbibliotheken (vor allem im Javascript Bereich) übers Internet zu. Diese Zugriffe könnten auch offline erfolgen, wenn die Bibliotheken heruntergeladen und lokal eingebunden werden. Die teure Internetanbindung auf Messen, wenn überhaupt eine vorhanden ist, könnte dadurch vermieden werden.

Darüber hinaus gab es auch Ideen den Campus und seine Darstellung durch besondere Aufnahmen, zum Beispiel bei Nacht, noch einzigartiger zu gestalten. Es wäre denkbar Panoramafotos bei Nacht zu erstellen und in die bestehende Anwendung einen "`Nachtmodus"' zu implementieren, der vom Benutzer über einen Schalter ausgewählt werden könnte.
\section{Kritische Reflexion}
\label{sec:KritischeReflexion}

Die entwickelte Softwärelösung ist am Ende des Projektes vollständig funktionsfähig und einsatzbereit.
An dieser Stelle werden die Entwicklungsphasen und -modelle reflektiert und kritisch betrachtet.

Zu Beginn des Projektes wurde das interne Ziel festgelegt, die zu entwickelnde Software möglichst präzise
zu planen, um Verzögerungen in der Entwicklung zu vermeiden. Durch die vorgestellten Modelle der Entwurfsphase
ist die realitätsnahe Modellierung der Software gelungen und zeitliche Verzögerungen konnten so weitesgehend
verhindert werden. Einige zeitliche Verzögerungen durch Anpassungen und Verbesserungsmaßnahmen ließen sich in der
Implementierungsphase des Projektes jedoch nicht vermeiden. Als besonderer Verzögerungsfaktor ist hier die
Google Street View API zu nennen. Die umfassenden Möglichkeiten dieser API wurden erst im Verlauf der
Entwicklung entdeckt und führten dazu, dass einige Planungen der Implementierung im Nachhinein überarbeitet
werden mussten. Konkret konnte die API einige Aufgaben übernehmen, die ursprünglich selbst
entwickelt werden sollten. Die Umstellung der Implementierung verzögerte sich dadurch zwar, aber die
Qualität der Anwendung konnte erheblich gesteigert werden.

Neben der starken Integrationsmöglichkeit der Google Street View API, wurde zu beginn des Projektes auch
der Aufwand der Panoramaerstellung unterschätzt. Das Fotos, wie ursprünglich angenommen, nicht kurzfristig
in Leerlaufphasen des Projektes erstellt werden konnten, verdeutlichte sich zunehmend im Projektverlauf.
Allein der Umfang der benötigten Fotoausrüstung konnte im Vorfeld nicht eingeschätzt werden.
Der Prozess der Fotoaufnahme und -bearbeitung wurde während des Projektes kontinuierlich weiterentwickelt
und verbessert. Zum Abschluss des Projektes konnte durch diese Entwicklung eine sehr hohe Fotoqualität
und ein effizienter Erstellungsprozess erreicht werden.

Insgesamt hat das Projekt deutlich mehr Aufwand in Anspruch genommen als ursprünglich geplant,
aber dieser zeitliche Mehraufwand resultierte in kontinuierlicher Qualitätssteigerung des Projektes, sowohl
im Bereich der Software als auch der erstellten Fotos.
\section{Fazit}
\label{sec:Fazit}

Das Projekt virtueller Campus Lingen konnte durch die aufgezeigte Entwicklung nach einer Entwicklungsphase von 8 Monaten
erfolgreich abgeschlossen werden.
Der strukturierte Entwurf der Anwendung im Vorfeld der Umsetzung konnte genutzt werden, um das Entwicklungstempo
auf einem hohen Niveau zu halten, wobei gleichzeitig eine hohe Qualität der Anwendung gewahrt werden konnte.
Die abschließende Testphase bestätigt, sowohl seitens des Entwicklerteams als auch seitens aussenstehender
Tester, den Erfolg des Projektes.

Am Ende des Projektes steht damit eine qualitativ hochwertige und nachhaltige Softwarelösung zur Verfügung, die die Interessen
des Auftraggebers vollständig erfüllt. Das Projektziel, den Campus Lingen für eine
junge Zielgruppe visuell darzustellen, ist ebenso vollständig erreicht worden.