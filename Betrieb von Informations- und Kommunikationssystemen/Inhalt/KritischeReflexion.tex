\section{Kritische Reflexion}
\label{sec:KritischeReflexion}

Die entwickelte Softwärelösung ist am Ende des Projektes vollständig
funktionsfähig und einsatzbereit. An dieser Stelle werden die Entwicklungsphasen
und -modelle reflektiert und kritisch betrachtet.

Zu Beginn des Projektes wurde innerhalb der Projektgruppe das Ziel festgelegt,
die zu entwickelnde Software möglichst präzise zu planen, um Verzögerungen in
der Entwicklung zu vermeiden. Durch die vorgestellten Modelle der Entwurfsphase
ist die realitätsnahe Modellierung der Software gelungen und zeitliche
Verzögerungen konnten weitesgehend verhindert werden. Einige zeitliche
Verzögerungen durch Anpassungen und Verbesserungsmaßnahmen ließen sich in der
Implementierungsphase des Projektes jedoch nicht vermeiden. Als besonderer
Verzögerungsfaktor ist hier die Google Street View API zu nennen. Die
umfassenden Möglichkeiten dieser API wurden erst im Verlauf der Entwicklung
entdeckt und führten dazu, dass einige Planungen im Bezug auf die
Implementierung im Nachhinein angepasst werden mussten. Konkret stellt die
API einige Funktionen zur Verfügung, die ursprünglich selbst entwickelt werden
sollten. Durch die Umstellung nahm die Implemetierung der Anwendung mehr Zeit in
Anspruch als ursprünglich geplant. Im Gegenzug konnte durch die Umstellung auf
die Funktionen der Google Street View API jedoch die Qualität der Anwendung
erheblich gesteigert werden.

Neben den Integrationsmöglichkeiten der Google Street View API in die Anwendung
wurde auch der Aufwand der Panoramaerstellung zu Beginn des
Projektes unterschätzt. Dass Fotos, wie ursprünglich angenommen, nicht
kurzfristig in Leerlaufphasen des Projektes erstellt werden konnten,
verdeutlichte sich zunehmend im Projektverlauf. Auch der Umfang der benötigten
Fotoausrüstung konnte im Vorfeld nicht korrekt eingeschätzt werden.
Der Prozess der Fotoaufnahme und -bearbeitung wurde während des Projektes
kontinuierlich weiterentwickelt und verbessert. Zum Abschluss des Projektes hin
konnte durch diese Entwicklung eine sehr hohe Fotoqualität in einem
effektiven Erstellungsprozess gewährleistet werden.

Insgesamt hat das Projekt deutlich mehr Aufwand in Anspruch genommen als
ursprünglich geplant. Der zeitliche Mehraufwand resultierte jedoch in einer
kontinuierlicher Qualitätssteigerung innherhalb des Projektes. Diese ließ
sich sowohl in der Anwendung als auch im Bereich der Panoramaerstellung
beobachten.
