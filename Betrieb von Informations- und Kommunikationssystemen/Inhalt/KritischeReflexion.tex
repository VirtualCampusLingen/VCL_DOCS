\section{Kritische Reflexion}
\label{sec:KritischeReflexion}

Die entwickelte Softwärelösung ist am Ende des Projektes vollständig funktionsfähig und einsatzbereit.
An dieser Stelle werden die Entwicklungsphasen und -modelle reflektiert und kritisch betrachtet.

Zu Beginn des Projektes wurde das interne Ziel festgelegt, die zu entwickelnde Software möglichst präzise
zu planen, um Verzögerungen in der Entwicklung zu vermeiden. Durch die vorgestellten Modelle der Entwurfsphase
ist die realitätsnahe Modellierung der Software gelungen und zeitliche Verzögerungen konnten so weitesgehend
verhindert werden. Einige zeitliche Verzögerungen durch Anpassungen und Verbesserungsmaßnahmen ließen sich in der
Implementierungsphase des Projektes jedoch nicht vermeiden. Als besonderer Verzögerungsfaktor ist hier die
Google Street View API zu nennen. Die umfassenden Möglichkeiten dieser API wurden erst im Verlauf der
Entwicklung entdeckt und führten dazu, dass einige Planungen der Implementierung im Nachhinein überarbeitet
werden mussten. Konkret konnte die API einige Aufgaben übernehmen, die ursprünglich selbst
entwickelt werden sollten. Die Umstellung der Implementierung verzögerte sich dadurch zwar, aber die
Qualität der Anwendung konnte erheblich gesteigert werden.

Neben der starken Integrationsmöglichkeit der Google Street View API, wurde zu beginn des Projektes auch
der Aufwand der Panoramaerstellung unterschätzt. Das Fotos, wie ursprünglich angenommen, nicht kurzfristig
in Leerlaufphasen des Projektes erstellt werden konnten, verdeutlichte sich zunehmend im Projektverlauf.
Allein der Umfang der benötigten Fotoausrüstung konnte im Vorfeld nicht eingeschätzt werden.
Der Prozess der Fotoaufnahme und -bearbeitung wurde während des Projektes kontinuierlich weiterentwickelt
und verbessert. Zum Abschluss des Projektes konnte durch diese Entwicklung eine sehr hohe Fotoqualität
und ein effizienter Erstellungsprozess erreicht werden.

Insgesamt hat das Projekt deutlich mehr Aufwand in Anspruch genommen als ursprünglich geplant,
aber dieser zeitliche Mehraufwand resultierte in kontinuierlicher Qualitätssteigerung des Projektes, sowohl
im Bereich der Software als auch der erstellten Fotos.