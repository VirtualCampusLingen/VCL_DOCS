\subsection{Panoramaerstellung}
\label{sec:Panoramaerstellung}

Die Panoramaerstellung stellt, neben der Softwareerstellung einen zentralen
Teilprozess innerhalb der Projektumsetzung dar. Innerhalb dieses Prozesses
werden die 360-Grad Panoramas, welche den eigentlichen Inhalt der Anwendung
darstellen, entworfen. Im Folgenden soll der Prozess der Panoramaerstellung
näher untersucht werden. Hierbei soll zunächst erläutert werden, in welcher
Form die Panoramafotos vorliegen müssen, um von der Anwendung verarbeitet
werden zu können. Weiterhin soll das Equipment vorgestellt werden, welches zur
Umsetzung der Panoramafotos benötigt wird. Anschließend soll dann der Workflow
der Panoramaerstellung dargestellt werden.

\subsubsection{Anforderungen an die Panoramafotos}
\label{sec:PanoramaerstellungAnforderungen}

Bevor mit der Erstellung der Panoramas begonnen werden kann müssen zunächst die
Anforderungen der Anwendung an die Panoramafotos festgelegt werden. Hierbei muss
definiert werden, welche Kriterien die Panoramafotos erfüllen müssen, um von der
Anwendung korrekt verarbeitet werden zu können. Relevante Kriterien sind hierbei
die Darstellungsform, die Größe und das Format der Panoramafotos. Nachdem diese
Schnittstelle festgelegt ist kann der Prozess der Panoramaerstellung isoliert
vom Prozess der Softwareerstellung durchgeführt werden. Dies ermöglicht eine
parallele Umsetzung der beiden Teilprozesse.

Wie bereits in \verweis{Architektur} erwähnt soll in der Anwendung die Google
Street View API verwendet werden. Diese Schnittstelle stellt die Funktionalität
zur Anzeige der Panoramafotos zur Verfügung. Somit legt diese Schnittstelle auch
fest, in welcher Form die Panoramafotos zur Verfügung gestellt werden müssen.
Diese Anforderungen können aus der Entwicklerdokumentation zu der API entnommen
werden. Die Schnittstelle erwartet hierbei ein Panoramafoto, welches der
Rektangulatprojektion\footnote{Die Rektangularprojektion stellt eine
horizontale Ansicht von 360 Grad und eine vertikale Ansicht von 180 Grad dar.
Dies bedingt ein Seitenverhälnis von 2:1} entspricht. Durch die API wird diese
Darstellung auf die Fläche einer Kugel projeziert. Der Mittlepunkt dieser Kugel
stellt den Standpunkt des Betrachters dar. Da sich immer nur ein Teil der
Projektion im Blickfeld dieses Betrachters befindet muss auch nur der aktuell
sichtbare Bereich des Panoramafotos dargestellt werden. Aus diesem Grund bietet
die Google Street View API die Möglichkeit, ein in mehrere rechteckige Teile,
sogenannte Kacheln, aufgeteiltes Panoramafoto zu verarbeiten. Auf diese Weise
kann die Performanz der Anwendung erhöht werden, da nicht das komplette
Panoramafoto, sondern nur ein Teil der Kacheln in der Anwendung geladen werden
muss. Damit den einzelnen Kacheln die korrekte Position auf der Planarprojektion
zugewiesen werden kann, muss die Benennung der Kacheldateien dem Namensschema
\texttt{Kachelspalte-Kachelzeile} genügen. Die Anzahl der Kacheln sowie die
Pixelmaße des Panoramafotos können frei gewählt werden. Hierbei ist zu beachten,
dass mit steigender Pixelanzahl die Qualität der 360-Grad-Darstellung steigt,
die Performance der Anwendung jedoch aufgrund der steigenden Datengröße sinkt.
In einer prototypischen Implementierung, auf die im \verweis{Softwareerstellung}
näher eingegangen werden soll, hat die Projektgruppe verschiedene Kombinationen
aus Pixekmaße und Kachelanzahl getestet. Letztendlich hat sich die Projektgruppe
auf die Pixelmaße 4096x2048 für das gesamte Panorama und eine Aufteilung in 8
mal 4 Kacheln entschieden. Diese Kombination wurde einheitlich als bester
Kompromiss zwischen Qualität und Performanz angesehen.

\subsubsection{Equipment für die Panoramaerstellung}
\label{sec:Equipment}

In dem vorherigen Abschnitt wurde herausgestellt, dass die zu erstellenden
Panoramafotos der Rektangularprojektion genügen müssen, um in der Anwendung
korrekt dargestellt zu werden. Für die Aufname eines Fotos, welches dieser
Darstellungsform genügt, wird spezielles Fotoequipment benötigt\ldots

\subsubsection{Workflow der Panoramaerstellung}
\label{sec:Workflow}

Der Workflow der Panoramaerstellung kann mehrere Teilprozesse untergliedert
werden. Für die Umsetzung dieser einzelnen Teilschritte wird hierbei spezielles
Hard- und Softwareequipment benötigt. Im Folgenden sollen die einzelnen
Teilschritte der Panoramaerstellung in chronologischer Reihenfolge erläutert
werden. Hierbei soll auch auf das benötigte Equipment eingegangen werden.

\begin{description}
\item[Aufnahme der Einzelfotos] Die Panoramaerstellung beginnt mit der Aufnahme
mehrerer Einzelfotos. Aus diesen Einzelfotos wird nachfolgend dann das komplette
Panoramafoto erstellt. Dieses muss, wie zuvor erläutert, der
Rektangularprojektion entsprechen. Die Summe der Einzelfotos muss somit eine
horizontale Ansicht von 360 Grad und eine vertikale Ansicht von 180 Grad
abbilden. Die Anzahl der Einzelfotos ist somit von dem Bildwinkel abhängig, der
auf einem einzelnen Foto dargestellt werden kann. Dieser Bildwinkel wird in der
Fotografie durch die Brennweite des Objektives festgelegt. Je geringer die
Brennweite eines Objektives ist, desto größer ist der Bildwinkel, der mit diesem
Objektiv eingefangen werden kann. Durch spezielle Objektivkonstruktionen wie zum
Beispiel einem Fischaugenobjektiv ist es möglich einen besonders großen
Bildwinkel auzunehmen. 
\end{description}
