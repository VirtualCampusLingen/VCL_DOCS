\subsubsection{Anforderungen an die Panoramafotos}
\label{sec:PanoramaerstellungAnforderungen}

Die Panoramaerstellung stellt einen zentralen Teilprozess innerhalb der
Projektumsetzung dar. Bevor mit der Erstellung der Panoramas begonnen werden
kann, müssen zunächst die Anforderungen der Anwendung an die Panoramafotos
festgelegt werden. Hierbei muss definiert werden, welche Kriterien die
Panoramafotos erfüllen müssen, um von der Anwendung korrekt verarbeitet werden
zu können. Relevante Kriterien sind hierbei die Darstellungsform, die Größe und
das Format der Panoramafotos. Nachdem diese Schnittstelle festgelegt ist kann
der Prozess der Panoramaerstellung isoliert vom Prozess der Softwareerstellung
durchgeführt werden. Dies ermöglicht eine parallele Umsetzung der beiden
Teilprozesse.

Wie bereits in \verweis{Architektur} erwähnt soll in der Anwendung die Google
Street View API verwendet werden. Diese Schnittstelle stellt die Funktionalität
zur Anzeige der Panoramafotos zur Verfügung. Somit legt diese Schnittstelle auch
fest, in welcher Form die Panoramafotos zur Verfügung gestellt werden müssen.
Diese Anforderungen können aus der Entwicklerdokumentation zu der API entnommen
werden. Die Schnittstelle erwartet hierbei ein Panoramafoto, welches der
Rektangulatprojektion\footnote{Die Rektangularprojektion stellt eine
horizontale Ansicht von 360 Grad und eine vertikale Ansicht von 180 Grad dar.
Dies bedingt ein Seitenverhälnis von 2:1} entspricht. Durch die API wird diese
Darstellung auf die Fläche einer Kugel projeziert. Der Mittelpunkt dieser Kugel
stellt den Standpunkt des Betrachters dar. Da sich immer nur ein Teil der
Projektion im Blickfeld dieses Betrachters befindet, muss auch nur der aktuell
sichtbare Bereich des Panoramafotos dargestellt werden. Aus diesem Grund bietet
die Google Street View API die Möglichkeit, ein in mehrere rechteckige Teile,
sogenannte Kacheln, aufgeteiltes Panoramafoto zu verarbeiten. Auf diese Weise
kann die Performanz der Anwendung erhöht werden, da nicht das komplette
Panoramafoto, sondern nur ein Teil der Kacheln in der Anwendung geladen werden
muss. Damit den einzelnen Kacheln die korrekte Position auf der
Rektangulatprojektion zugewiesen werden kann, muss die Benennung der
Kacheldateien dem Namensschema \texttt{Kachelspalte-Kachelzeile} genügen. Die
Anzahl der Kacheln sowie die Pixelmaße des Panoramafotos können frei gewählt
werden. Hierbei ist zu beachten, dass mit steigender Pixelanzahl die Qualität
der 360-Grad-Darstellung steigt, die Performanz der Anwendung jedoch aufgrund
der steigenden Datengröße sinkt. In einer prototypischen Implementierung, auf
die im \verweis{Softwareerstellung} näher eingegangen werden soll, hat die
Projektgruppe verschiedene Kombinationen aus Pixekmaße und Kachelanzahl
getestet. Letztendlich hat sich die Projektgruppe auf die Pixelmaße 4096x2048
für das gesamte Panorama und eine Aufteilung in 32 Kacheln entschieden. Diese
Kombination wurde einheitlich als bester Kompromiss zwischen Qualität und
Performanz angesehen.