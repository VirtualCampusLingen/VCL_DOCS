\subsubsection{Vorbereitung und Tools}
\label{sec:VorbereitungUndTools}

Die besondere Herausforderung des vorliegenden Projektes besteht darin, eine
mehrschichtige Software, bestehend aus Benutzer- und Administrationsansicht, in
einem Projektteam zu entwickeln, welches sich in zwei Punkten von der
Projektorganisation eines klassischen\footnotemark\ Softwareprojektes
unterscheidet. Zum einem muss das Projektteam, bedingt durch unterschiedliche
Arbeitsorte, räumlich getrennt voneinander entwicklen können. Zum anderen
ist das Projektteam während des Entwicklungszeitraums nicht vollständig im
Projekt eingebunden. Es müssen Arbeits-, Urlaubs-, Prüfungs- und
Krankheitszeiten aller Mitglieder des Projektteams als hemmende Zeitfaktoren
berücksichtigt werden. Eine genauere Betrachtung dieser Projektsituation ist an
dieser Stelle nicht notwendig.

\footnotetext{Klassische Softwareprojekte werden hier als Projekte
in einer reinen Projektorganisation verstanden. Zur weiteren Vertiefung siehe
\citet[S.~105]{jenny2001}}

Die beiden genennten Aspekte bergen organisatorische Risiken und können auch zu
Problemen bei der Entwicklung führen. Das Hauptrisiko ist dabei ein Stillstand
des Projektes, welcher dadurch bedingt ist, dass alle Projektmitglieder auf das
Wissen eines Mitglieds angewiesen sind, welches gerade nicht zur Verfügung
steht. Dieses Risiko soll durch den Einsatz von zwei Entwicklungswerkzeugen
minimiert werden.

Zunächst muss dafür gesorgt werden, dass alle Mitglieder immer auf dem
aktuellen Stand des Projektes sind. Wird ein Entwicklungsschritt von einer
einzelnen Person abgeschlossen muss das Ergebnis den anderen zur Verfügung
gestellt werden. Zusätzlich müssen Änderungen am bestehnden Projektstand
kommentiert und dokumentiert werden. Diese Anforderung werden im vorliegenden
Projekt mit Hilfe einer Projektversionierung erreicht. Ein Versionierungstool
bietet hierbei die Möglichkeit den Quellcode eines Softwareprojektes zentral in
einem sogenannten Repository (zu deutsch "`Depot"') zu halten, um ihn für alle
Mitglieder verfübgar zu machen. Der aktuelle Stand des Projektes kann aus
diesem Repository jederzeit abgefragt werden und Änderungen werden von
Mitgliedern dorthin zurückgeschrieben. Zusätzlich bieten Versionierungstools die
Möglichkeit, die Änderung mit einem Kommentar zu versehen. Die Dokumentation der
konkrteten Änderungen wird vom Tool automatisch durchgeführt. Als
Versionierungssoftware wurde im vorliegenden Projekt "`Github"'\footnotemark\
verwendet, da allen Mitglieder mit diesem Versionierungstool bereits gearbeitet
haben.

\footnotetext{Github ist eine Open Source Projekt, das auf dem
Versionierungstool git aufsetzt. Es stellt den De-facto-Standard für
Webprojekte dar.}

Neben dem Quellcode sollen auch die Anforderungen an die Anwendung und die
aktuellen Aufgaben der Projektmitglieder an zentraler Stelle verwaltet werden.
Bei Ausfall eines Mitgliedes müssen seine aktuellen Aufgaben zentral
dokumentiert sein, um diese auf andere Mitglieder verteilen zu können. Dazu
wurde ein webbasiertes Ticketsystem aufgesetzt, auf das alle Mitglieder Zugriff
haben. Hierbei wurde sich für das Ticketsystem "`PHProjekt"' entschieden, da PHP
als Technologie bereits bekannt ist und das Ticketsystem somit schnell
aufgesetzt werden konnte. In diesem Ticketsystem werden alle erstellten
Arbeitspakete des Projektes mit zugeordnetem Mitglied, Bearbeitungszeitraum,
Anforderungsbeschreibung und benötigter Bearbeitungszeit hinterlegt.