\subsubsection{Vorbereitung und Tools}
\label{sec:VorbereitungUndTools}

Die besondere Herausforderung des vorliegenden Projektes besteht darin eine mehrschichtige Software (Benutzeransicht und Administrationsbereich) in einem Projektteam zu entwickeln, das sich in zwei Punkten von der Projektorganisation eines klassischen\footnotemark\ Softwareprojektes unterscheidet. Zum einem muss das Projektteam, bedingt durch unterschiedliche Arbeitsorte, geographisch getrennt von einander entwicklen können. Zum anderen ist das Projektteam während des Entwicklungszeitraums nicht vollständig im Projekt eingebunden. Es müssen Arbeits-, Urlaubs-, Prüfungs- und Krankheitszeiten aller Mitglieder des Projektteams als hemmende Zeitfaktoren berücksichtigt werden. Eine genauere Betrachtung dieser Projektsituation ist an dieser Stelle nicht notwendig.

\footnotetext{Klassische Softwareprojekte werden hier verstanden als Projekte in einer reinen Projektorganisation. Zur weiteren Vertiefung siehe \citet[S.~105]{jenny2001}}

Diese beiden Aspekte bergen organisatorische Risiken und können auch zu Problemen bei der Entwicklung führen.Das
Hauptrisiko ist dabei, dass das Projekt still steht, weil alle Projektmitglieder auf das Wissen oder den Entwicklungsstand eines Mitglieds angewiesen sind, das gerade nicht zur Verfügung steht. Dieses Risiko soll durch den Einsatz von zwei Entwicklungswerkzeugen minimiert werden.

Zunächst muss dafür gesorgt werden, dass alle Mitglieder immer auf dem aktuellen Stand des Projektes sind. Wird ein Entwicklungsschritt von einer einzelnen Person abgeschlossen muss das Ergebnis den anderen zur Verfügung gestellt werden. Zusätzlich müssen Änderungen am bestehnden Projektstand kommentiert und dokumentiert werden. Diese Anforderung werden im vorliegenden Projekt mit Hilfe von Projektversionierung erreicht. Ein Versionierungstool bietet die Möglichkeit den Quellcode eines Softwareprojektes zentral in einem sogenannten Repository (zu deutsch 'Depot') zu halten, um ihn für alle Mitglieder verfübgar zu machen. Der aktuelle Stand des Projektes kann aus diesem Repository jederzeit erfragt werden und Änderungen werden von Mitgliedern dorthin zurückgeschoben. Zustätlich bieten Versionierungstools die Möglichkeit die Änderung mit einem Kommentar zu versehen. Die Dokumentation der Änderungen erledigt das Tool automatisch. Als Versionierungssoftware wurde im vorliegenden Projekt "`Github"'\footnotemark\ verwendet, da alle Mitglieder mit dieser Software bereits gearbeitet haben.

\footnotetext{Github ist eine Open Source Projekt, das auf dem Versionierungstool git aufsetzt. Es stellt den De-facto-Standard für Webprojekte dar.}

Neben dem zentralisierten Quellcode sollen auch die Anforderungen an das Projekt und die aktuellen Aufgaben der einzelnen Mitglieder zentral gehalten werden. Bei Ausfall eines Mitgliedes sollen seine aktuellen Aufgaben zentral dokumentiert sein, um diese auf andere Mitglieder zu verteilen. Dazu wurde ein web-basiertes Ticketsystem aufgesetzt auf das alle Mitglieder zugriff haben. Es wurde sich für das Ticketsystem "`PHProjekt"' entschieden, da PHP als Technologie bereits bekannt ist und das Aufsetzen damit leicht erreicht werden konnte. In diesem Ticketsystem werden alle aufgestellten Arbeitspakete des Projektes mit zugeordnetem Mitglied, Bearbeitungszeitraum, Anforderungsbeschreibung und benötigter Bearbeitungszeit hinterlegt.