\subsection{Datenbankentwurf}
\label{sec:Datenbankentwurf}

Aufbauend auf den vorangegangenen Entwurfsergebnissen wird im Folgenden ein Entwurf der Datenbank angefertigt. Dieser Entwurf benutzt zur grafischen Modellierung das Entity-Relationship-Modell\footnotemark, kurz ERM oder ER-Modell genannt. Auf eine theoretische Fundierung des ER-Modells wird an dieser Stelle bewusst verzichtet. Stattdessen werden die Elemente am nachfolgenden Beispiel erläutert.

\footnotetext{Das Entity-Relationship-Modell ist ein Modell zur Darstellung der inneren Datenbankstruktur einer Datenbank}

Bevor das ERM erstellt werden kann müssen aus dem vorliegenden Projekt die Bestandteile identifiziert werden, die persistent in der Datenbank gespeichert werden müssen.
Aus der vorgangenen Oberflächenentwurf wird deutlich, dass das Projekt Inhaltich aus zwei Teilen besteht, einer Benutzeransicht und einer Administratoransicht. Die Benutzeransicht stellt dabei nur Informationen dar, die in der Administrationsansicht gepflegt und hinterlegt wurden. Der Administrationsbereich ist der Datenhaltende-Bereich, indem die Interaktion mit der Datenbank stattfindet. In dem Kapitel Administrationsbereich werden dabei bereits 3 Informationsobjekte deutlich:

\begin{itemize}
  \item Infotexte
  \item Fotos
  \item Administratordaten
\end{itemize}

--- Rest folgt wenn ERM überarbeitet ---