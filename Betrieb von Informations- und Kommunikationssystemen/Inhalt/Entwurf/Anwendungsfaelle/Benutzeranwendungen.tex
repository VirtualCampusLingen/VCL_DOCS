\subsubsection{Benutzeranwendungen}
\label{sec:Benutzeranwendungen}

Ein Benutzer der Software hat verschiedene Möglichkeiten der Interaktion mit
der Anwendung. Diese sind im Folgenden mit der fortlaufenden
Kennzeichnung \textbf{AFBXX}\footnote{AFBXX = Anwendunsfall eines Benutzers mit
Nr. XX} markiert und strukturiert aufgelistet:

\begin{description}
  \item[AFB01] Der Benutzer ruft die Webanwendung durch einen Internetbrowser
  auf.
  \item[AFB02] Der Benutzer dreht sich horizontal und vertikal um die Achse
  der Fotoaufnahme, um einen Rundumblick zu erhalten.
  \item[AFB03] Der Benutzer zoomt über die Steuerelemente in die Fotoaufnahme
  hinein.
  \item[AFB04] Der Benutzer navigiert mithilfe der Navigationselemente zu
  anderen Aufnahmepositionen.
  \item[AFB05] Der Benutzer navigiert mithilfe der Minimap zu einer anderen
  Aufnahmeposition.
  \item[AFB06] Der Benutzer bewegt sich mithilfe der Pfeiltasten seiner
  Tastatur zu anderen Aufnahmepositionen
  \item[AFB07] Der Benutzer öffnet über ein Steuerungselement ein
  Informationsfenster.
  \item[AFB08] Der Benutzer schließt über ein Steuerungselement ein
  Informationsfenster.
\end{description}

% TODO: Verweise auf Anwenwendungsfalldiagram, Anhang, und Seite korrekt einfügen
Die Anwendungsfälle AFB01 bis AFB08 definieren die Menge an
Interaktionsmöglichkeiten eines Benutzers. Diese Menge an
Interaktionsmöglichkeiten ist inklusive der hinterlegten Funktionen im
Awendungsfalldiagramm im Anhang X auf Seite X abgebildet.
