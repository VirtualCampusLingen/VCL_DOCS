\subsubsection{Benutzeranwendungen}
\label{sec:Benutzeranwendungen}

Ein Anwender der Software hat verschiedene Möglichkeiten der Interaktion mit der Anwendung. Diese sind im Folgenden mit dem forlaufenden Kennzeichen \textbf{AFB XX}\footnotemark markiert und strukturiert aufgelistet:

\footnotetext{AFBXX = Anwendunsfall eines Benutzers mit Nr. XX}

\begin{description}
  \item[AFB01] Der Benutzer ruft die Webanwendung durch einen Internetbrowser auf.
  \item[AFB02] Der Benutzer dreht sich horizontal und vertikal mit die Achse der Fotoaufnahme, um einen Rundumblick zu erhalten.
  \item[AFB03] Der Benutzer zoomt über die Steuerelemente in die Fotoaufnahme hinein.
  \item[AFB04] Der Benutzer navigiert mit Hilfe der Navigationselemente zu anderen Aufnahmepositionen.
  \item[AFB05] Der Benutzer navigiert mit Hilfe der Minimap zu einer anderen Aufnahmepositionen.
  \item[AFB06] Der Benutzer bewegt sich mit Hilfe der Pfeiltasten seiner Tastatur zu anderen Aufnahmepositionen
  \item[AFB07] Der Benutzer öffnet über ein Steuerungselement das Informationsfenster.
  \item[AFB08] Der Benutzer schließt über ein Steuerungselement das Informationsfenster.
\end{description}

% TODO: Verweise auf Anwenwendungsfalldiagram, Anhang, und Seite korrekt einfügen
Die Anwendungsfälle AFB01 bis AFB08 definieren die Menge an Interaktionsmöglichkeiten. Diese Menge ist inklusive hinterlegten Funktionen im Awendungsfalldiagramm X im Anhang X auf Seite X abgebildet.