\subsubsection{Adminstratoranwendungen}
\label{sec:Adminstratoranwendungen}

Ein Administrator hat in vier Bereichen Möglichkeiten zur
Steuerung und Pflege der Webanwendung. Diese vier Bereiche wurden im
\verweis{Administrationsbereich} vorgestellt. Die Möglichkeiten der Interaktion
in diesen Bereichen sind mit der forlaufenden
Kennzeichnung \textbf{AFAXX}\footnote{AFAXX = Anwendunsfall eines Adminstrator
mit Nr. XX} markiert und strukturiert aufgelistet:

\begin{description}
  \item[AFA01] Der Adminstrator ruft den Administrationsbereich der
  Webanwendung durch einen Internetbrowser auf.
  \item[AFA02] Der Administrator klickt auf den Menüpunkt "`Infotexte"' und ihm
  wird die Infotextverwaltung angezeigt.
  \item[AFA03] Der Administrator erstellt im Menüpunkt "`Infotexte"' einen neuen
  Informationstext.
  \item[AFA04] Der Administrator verändert im Menüpunkt "`Infotexte"' einen
  erstellten Informationstext.
  \item[AFA05] Der Administrator löscht im Menüpunkt "`Infotexte"' einen
  erstellten Informationstext.
  \item[AFA06] Der Administrator klickt auf den Menüpunkt "`Fotos"' und ihm wird
  die Fotoverwaltung angezeigt.
  \item[AFA07] Der Administrator lädt im Menüpunkt "`Fotos"' ein erstelltes
  Panoramafoto mit Beschreibung und Namen hoch.
  \item[AFA08] Der Administrator verändert im Menüpunkt "`Fotos"' den Namen und
  die Beschreibung eines hochgeladenen Fotos.
  \item[AFA09] Der Administrator löscht im Menüpunkt "`Fotos"' ein hochgeladenes
  Foto.
  \item[AFA10] Der Administrator klickt auf den Menüpunkt "`interssante Orte"'
  und ihm wird die Verwaltung der interessanten Orte angezeigt.
  \item[AFA11] Der Administrator erstellt im Menüpunkt "`interssante Orte"'
  einen neuen interessanten Ort.
  \item[AFA12] Der Administrator verändert im Menüpunkt "`interssante Orte"' den
  Namen und die Beschreibung zu einem interessanten Ort.
  \item[AFA13] Der Administrator löscht im Menüpunkt "`interssante Orte"' einen
  interessanten Ort.
  \item[AFA14] Der Administrator klickt auf den Menüpunkt "`Übersichtskarte"'
  und ihm wird die Übersichtskarte angezeigt.
  \item[AFA15] Der Administrator platziert im Menüpunkt "`Übersichtskarte"' ein
  hochgeladenes Foto auf der Übersichtskarte.
  \item[AFA16] Der Administrator verschiebt im Menüpunkt "`Übersichtskarte"' ein
  bereits platziertes Foto an eine andere Position.
  \item[AFA17] Der Administrator navigiert im Menüpunkt "`Übersichtskarte"' über
  die Steuerelemente zu einem anderen Stockwerk.
  \item[AFA18] Der Administrator verbindet im Menüpunkt "`Übersichtskarte"'
  mehrere hochgeladene und positionierte Fotos auf der Übersichtskarte.
  \item[AFA19] Der Administrator verbindet im Menüpunkt "`Übersichtskarte"' ein
  hochgeladenes und positioniertes Foto mit erstellten Informationstexten.
\end{description}

% TODO: Verweise auf Anwenwendungsfalldiagram, Anhang, und Seite korrekt einfügen
Die Anwendungsfälle AFA01 bis AFA19 definieren die Menge an
Interaktionsmöglichkeiten eines Administrators. Diese Menge an
Interaktionsmöglichkeiten ist inklusive der hinterlegten Funktionen in einem
Awendungsfalldiagramm im Anhang X auf Seite X dargestellt.
