\subsubsection{Adminstratoranwendungen}
\label{sec:Adminstratoranwendungen}

Ein Administrator der Webanwedung hat in vier Bereichen Möglichkeiten zur Steuerung und Pflege der Webanwendung. Diese vier Bereiche werden im \verweis{Administrationsbereich} vorgestellt. Die Möglichkeiten der Interaktion in diesen Bereichen sind mit den forlaufenden Kennzeichen \textbf{AFA XX}\footnotemark markiert und strukturiert aufgelistet:

\footnotetext{AFAXX = Anwendunsfall eines Adminstrator mit Nr. XX}

\begin{description}
  \item[AFA01] Der Adminstrator ruft den Administrationsbereich der Webanwendung durch einen Internetbrowser auf. Er authentifiziert sich bei der Webanwendung durch seine Email-Adresse und sein Passwort.
  \item[AFA02] Der Administrator klickt auf den Menüpunkt 'Administration' und ihm wird der Bereich der Administrationsdatenpflege angezeigt.
  \item[AFA03] Der Administrator ändert im Menüpunkt 'Administration' seine Email-Adresse.
  \item[AFA04] Der Administrator ändert im Menüpunkt 'Administration' sein Passwort.
  \item[AFA05] Der Administrator klickt auf den Menüpunkt 'Infotexte' und ihm wird der Bereich der Infotexte angezeigt.
  \item[AFA06] Der Administrator erstellt im Menüpunkt 'Infotexte' einen neuen Informationstext.
  \item[AFA07] Der Administrator verändert im Menüpunkt 'Infotexte' einen erstellten Informationstext.
  \item[AFA08] Der Administrator löscht im Menüpunkt 'Infotexte' einen erstellten Informationstext.
  \item[AFA09] Der Administrator klickt auf den Menüpunkt 'Fotos' und ihm wird der Bereich der Fotos angezeigt.
  \item[AFA10] Der Administrator lädt im Menüpunkt 'Fotos' einen erstelltes Panoramafoto mit Beschreibung und Namen hoch.
  \item[AFA11] Der Administrator verändert im Menüpunkt 'Fotos' den Namen und die Beschreibung eines hochgeladenen Fotos.
  \item[AFA12] Der Administrator löscht im Menüpunkt 'Fotos' ein hochgeladenes Foto.
  \item[AFA13] Der Administrator klickt auf den Menüpunkt 'Übersichtskarte' und ihm wird der Bereich der Übersichtskarte angezeigt.
  \item[AFA14] Der Administrator platziert im Menüpunkt 'Übersichtskarte' ein hochgeladenes Foto auf der Übersichtskarte.
  \item[AFA15] Der Administrator verschiebt im Menüpunkt 'Übersichtskarte' ein bereits platziertes Foto an eine andere Position.
  \item[AFA16] Der Administrator navigiert im Menüpunkt 'Übersichtskarte' über die Steuerelemente zu einem anderen Stockwerk.
  \item[AFA17] Der Administrator verbindet im Menüpunkt 'Übersichtskarte' mehrere hochgeladene und positionierte Fotos auf der Übersichtskarte.
\end{description}

% TODO: Verweise auf Anwenwendungsfalldiagram, Anhang, und Seite korrekt einfügen
Die Anwendungsfälle AFA01 bis AFA17 definieren die Menge an Interaktionsmöglichkeiten eines Administrators. Diese Menge ist inklusive hinterlegten Funktionen im Awendungsfalldiagramm X im Anhang X auf Seite X abgebildet.