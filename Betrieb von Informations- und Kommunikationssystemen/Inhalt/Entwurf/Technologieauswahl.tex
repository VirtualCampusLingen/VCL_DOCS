\subsection{Technologieauswahl}
\label{sec:Technologieauswahl}

Zu Beginn der Entwurfsphase werden die zuvor vorgestellten und ausgwählten Technologien im Folgenden in einer Analyse untersucht. An dieser Stelle soll herausgestellt werden warum bestimmte Technologien ausgewählt wurden und welche Alternativen vorhanden sind. Betrachtet werden an dieser Stelle wiederrum die vier Technologien der Architektur (HTML, Javascript, PHP und SQL), die zuvor in den Architekturgrundlagen erläutert wurden.

\begin{description}
  \item[HTML] wird als Darstellungssprache für Benutzeransichten verwendet. HTML ist der führende Standard für die Darstellung in Webanwendungen. Die einzige Alternative zu HTML ist das Darstellungsformat XML (Extensible Markup Language). Mit XML können, genau wie mit HTML, strukturiert Textdatein erstellt werden, die einem Benutzer eine Ansicht auf Basis definierter Elemente bietet. Der große Nachteil von XML ist dabei, dass alle Elemente, die man zur Darstellung verwenden möchte, selbst in einer eigenen Datei definiert werden müssen. Das bedeutet zusätzlichen Zeitaufwand. Dagegen besteht bei HTML der Vorteil, dass alle Projektmitglieder mit dieser Technologie bereits gearbeitet haben und den Einsatz beherrschen.

  \item[Javascript] wird als Clientseitige Skriptsprache zum Ausführen von Benutzerinteraktonen und zum Nachladen von Webinhalten verwendet. Javascript hat sich über einen Entwicklungszeitraum von 19 Jahren (1995-2014) als Standard in diesen Bereich von Webanwendungen etabliert\footnote{\citet{powers2007}}. Der große Vorteil von Javascript ist die Mächtigkeit, die die Sprache durch 14 Jahre Entwicklung erhalten hat. Auf Javascript basieren heute ganze Frameworks sowohl für Clientanwendungsbereiche als auch für Serveranwendungsbereiche\footnotemark, darüber hinaus gibt es zahlreiche Bibliotheken, die das Entwickeln mit Javascript vereinfachen (z.B.: jQuery). Der Nachteil von Javascript ist das es vom jeweiligen Internetbrowser des Anwenders compiliert (übersetzt) wird. Daraus resultiert das Javascript in manchen Browsern anders ausgeführt wird und andere Ergebnisse liefert als in anderen Browser. Das ist aber vor allem ein Problem von älteren Browser, in den neuen Versionen haben sich alle großen Browserhersteller auf einheitliche Standards geeinigt. Als Alternativen zu Javascript kann das Google Projekt "`Dart"'\footnote{siehe \url{https://www.dartlang.org/}} und das Projekt "`JSX"'\footnote{siehe \url{http://jsx.github.io/}} angeführt werden. Sowohl Dart als auch JSX sind aber keinem Projektmitglied bekannt, die Nutzung dieser Technologie würde damit zusätzlichen Arbeitsaufwand und Einarbeitungszeit bedeuten, die durch Verwendung der bekannten Javascript Technologie besser genutzt werden kann. Zudem wird zumindest für die Verwendung von Dart eine eigene Laufzeitumgebung benötigt, die die Sprache interpretiert und compiliert. Diese ist zum einen noch in der Entwicklungsphase, da das Dart Projekt erst 2011 veröffentlicht wurde und zum anderen noch nicht auf jedem Browser lauffähig\footnote{vergleiche hierzu \url{http://www.golem.de/news/javascript-alternative-googles-dart-1-0-veroeffentlicht-1311-102745.html}}.

  \footnotetext{siehe hierzu zum Beispiel Angular.js (\url{http://angularjs.org/}) oder Node.js (\url{http://nodejs.org/})}

  \item[PHP] ist im Projekt für die Implementierung der Serverseitigen Logik zuständig. In diesem Anwendungsbereich gibt es zahlreiche Alternativen und Möglichkeiten. Die bekanntesten in diesem Bereich sind Java, Python, Ruby, Perl, ASP und das in der Fußzeile erwähnte Node.js. Alle diese Technologien haben ihre Vorteile und Nachteile im Bezug auf Performanz, Skalierbarkeit, Wartbarkeit, Portierbarkeit usw. Der auschlagebende Grund PHP in diesem Projekt als Servertechnologie zu verwenden ist die schnelle Lernkruve, die durch die einfache Struktur der Sprache gegeben ist und die schnelle Einrichtung eines Webservers mit PHP, sowohl lokal, als auch auf externen Webservern. Viele fertige Webserverlösungen, zum Beispiel Xampp, bringen darüber hinaus PHP standardmäßig mit und erleichtern damit den Einstieg.

  \item[SQL] ist die Verwendete Abfragesprache für Anfragen an die Datenbank im vorliegenden Projekt. Vorteil dieser Sprache ist die Große Bibliothek an Befehlen und Mächtigkeit der Sprache, besonders durch die Sprachbereiche DCL und DDL. Nachteil ist die relativ komplexe Syntax der Sprache und eine relativ langsame Lernkurve. SQL ist aber auch die einzige Möglichkeit eine Datenbank, wie sie in diesem Projekt vorliegt, anzusprechen. Lediglich wenn man bereit ist das Datenbanksystem gegen eine andere Form der Datenspeicherung zu tauschen bieten sich weitere Alternativen. Zum Beispiel bieten CSV(comma serperated value)-Dateien, eine Möglichkeit Informationen auf Dateibasis zu speichern. Problem der CSV-Dateien ist die schlechte Wartbarkeit und Kontrolle der Informationen, da ein Kommagetrenntes Format bei großen Datenmengen für Menschen nicht gut lesbar ist. Eine weitere Möglichkeit ist die Speicherung von Informationen in XML. Auch diese Speicherung ist auf Dateibasis und auch hier ist die Darstellung großer Datenmengen wenig übersichtlich. Sowohl XML als auch CSV-Dateien sind zudem bei komplexen Datenstruktren wenig performant. Die in relationalen Datenbanken definierten und in SQL abfragbaren Beziehungen, sind bei diesen Technologien nicht vorhanden und müssten selbst durch Logik und aufwand abgebildet werden. Zudem müsste bei Verwendung einer alternativen Datenhaltung eine eigene Abfragesprache oder eigene Abfragebefehle definiert werden. SQL stellt vor diesem Hintergrund die beste Wahl dar.
\end{description}