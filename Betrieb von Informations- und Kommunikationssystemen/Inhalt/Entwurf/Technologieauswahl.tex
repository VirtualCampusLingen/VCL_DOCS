\subsection{Technologieauswahl}
\label{sec:Technologieauswahl}

Zu Beginn der Entwurfsphase werden die einleitend vorgestellten Technologien in
einer Analyse untersucht. An dieser Stelle soll hierbei herausgestellt werden
warum bestimmte Technologien ausgewählt wurden und welche Alternativen vorhanden
sind. Betrachtet werden an dieser Stelle die vier Technologien der Architektur
(HTML, Javascript, PHP und SQL), welche einleitend in den Architekturgrundlagen
erläutert wurden.

\begin{description}
  \item[HTML] wird als Darstellungssprache für Benutzeransichten verwendet.
  HTML ist der führende Standard für die Darstellung in Webanwendungen. Die
  einzige Alternative zu HTML ist das Darstellungsformat XML (Extensible Markup
  Language). Mit XML können, genau wie mit HTML, strukturiert Textdateien
  erstellt werden, die einem Benutzer eine Darstellung auf Basis definierter
  Elemente bietet. Der große Nachteil von XML ist dabei, dass alle Elemente,
  die zur Darstellung verwendet werden sollen, selbst in einer eigenen Datei
  definiert werden müssen. Dagegen bietet HTML den Vorteil, dass alle  
  Projektmitglieder mit dieser Technologie bereits gearbeitet haben und dessen
  Einsatz beherrschen. Die Umsetzung des Projektes ist mit HTML daher
  schneller zu realsieren.
  \item[Javascript] wird als clientseitige Skriptsprache zum Ausführen von
  Benutzerinteraktonen und zum Nachladen von Webinhalten verwendet. Javascript
  hat sich über einen Entwicklungszeitraum von 19 Jahren (1995-2014) als
  Standard in diesem Bereich von Webanwendungen  
  etabliert\footnote{\citet[S.~2]{powers2007}}. Der große Vorteil von Javascript
  ist die Mächtigkeit, die die Sprache durch 14 Jahre Entwicklung erhalten hat.
  Auf Javascript basieren heute ganze Frameworks sowohl für
  Clientanwendungsbereiche als auch für Serveranwendungsbereiche.\footnote{siehe
  hierzu zum Beispiel Angular.js (\url{http://angularjs.org/}) oder Node.js
  (\url{http://nodejs.org/})} Darüber hinaus gibt es zahlreiche Bibliotheken,
  wie beispielsweise jQuery, die das Entwickeln mit Javascript vereinfachen. 
  
  Ein Nachteil der Javascript-Technologie ist, dass der Quellcode vom jeweiligen
  Internetbrowser des Anwenders compiliert wird. Daraus resultiert, dass
  Javascript in einigen Browsern anders ausgeführt wird, und somit andere
  Ergebnisse liefert, als in anderen. Dieses Problem besteht jedoch vor allem
  bei der Verwendung von älteren Browsern. In den neuen Versionen der
  Browser haben sich alle relevanten Hersteller auf einheitliche Standards
  geeinigt. 
  
  Als Alternativen zu Javascript kann das Google Projekt
  "`Dart\footnote{siehe: \url{https://www.dartlang.org/}}"' und das Projekt
  "`JSX\footnote{siehe: \url{http://jsx.github.io/}}"' angeführt werden. Sowohl
  Dart als auch JSX sind aber keinem Projektmitglied bekannt. Die Nutzung
  dieser Technologie würde damit zusätzlichen Arbeitsaufwand und
  Einarbeitungszeit bedeuten, die durch Verwendung der bekannten
  Ja\-va\-script-Technologie zur Entwicklung des Projektes genutzt werden kann.
  Zudem wird zumindest für die Verwendung von Dart eine eigene Laufzeitumgebung
  benötigt, die die Sprache interpretiert und compiliert. Diese ist zum einen
  noch in der Entwicklungsphase, da das Dart Projekt erst 2011 veröffentlicht
  wurde, und zum anderen noch nicht auf jedem Browser
  lauffähig.\footnote{siehe: \url{http://www.golem.de/news/javascript-alternative-googles-dart-1-0-veroeffentlicht-1311-102745.html}}
  \item[PHP] ist im Projekt für die Implementierung der serverseitigen Logik
  zuständig. In diesem Anwendungsbereich gibt es zahlreiche Alternativen und
  Möglichkeiten zur Umsetzung. Die bekanntesten Alternativen in diesem Bereich
  sind Java, Python, Ruby, Perl, ASP und Node.js. All diese Technologien haben
  spezifische Vor- und Nachteile im Bezug auf Performanz, Skalierbarkeit,
  Wartbarkeit und Portierbarkeit. Der ausschlagebende Grund, PHP in diesem
  Projekt zu verwenden, ist zum einen die schnelle Lernkruve, die durch die
  einfache Struktur der Sprache gegeben ist und zum anderen die Möglichkeit der
  schnellen Einrichtung eines Webservers mit
  PHP.\footnote{\citet[S.~14]{peyton2005}}. Darüber hinaus bringen viele
  fertige Webserverlösungen, wie besipielsweise XAMPP, PHP bereits standardmäßig
  mit und erleichtern somit den Einstieg.
  \item[SQL] ist die verwendete Abfragesprache für Anfragen an die Datenbank im
  vorliegenden Projekt. Vorteile dieser Sprache sind zum einen die große
  Bibliothek an Befehlen und zum anderen die Mächtigkeit der Sprache, welche
  besonders durch die Sprachbereiche DCL und DDL gegeben ist.  Nachteile der SQL
  sind hingegen die relativ komplexe Syntax der Sprache und eine subjektiv
  relativ langsam wahrgenommene Lernkurve. SQL stellt jedoch auch die einzige
  Möglichkeit dar, eine Datenbank, wie sie in diesem Projekt vorliegt,
  anzusprechen. Lediglich wenn man bereit ist, das Datenbanksystem gegen eine
  andere Form der Datenspeicherung auszutauschen, bieten sich weitere
  Alternativen an. Zum Beispiel bieten CSV (Comma Serperated Value)-Dateien eine
  Möglichkeit, Informationen auf Dateibasis zu speichern. Ein Problem der
  Datenspeicherung in diesen CSV-Dateien ist die schlechte Wartbarkeit und
  Kontrolle der Informationen, da ein kommagetrenntes Format bei großen
  Datenmengen für Menschen nicht gut lesbar ist. Eine weitere Alternative neben
  CSV ist die Speicherung von Informationen im XML-Format. Auch diese
  Speicherung ist auf Dateibasis und auch hier ist die Darstellung großer
  Datenmengen wenig übersichtlich. Weiterhin bietet sowohl die Speicherung im
  CSV- als auch im XML-Format nicht die Möglichkeit, die Beziehungen zwischen
  den Daten wie in relationalen Datenbanken abzufragen. Bei der Verwendung
  einer alternativen Datenhaltung müssten daher erst eigene Abfragebefehle
  definiert werden. SQL stellt vor diesem Hintergrund die beste Wahl dar.
\end{description}