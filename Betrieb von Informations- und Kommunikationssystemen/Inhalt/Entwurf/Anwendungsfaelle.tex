\subsection{Anwendungsfälle}
\label{sec:Anwendungsfaelle}

% TODO: Anhang der use case diagramme verweisen

Aufbauend auf dem Entwurf der Oberfläche sowohl für den Benutzer, als auch für den Administrator, werden im Folgenden Anwendungsfälle beschrieben, die die Anwendungsmöglichkeiten dieser beiden Benutzergruppen aufzeigen. Zur visuellen Modellierung dieser Anwendungsfälle wird die Darstellungsart des Anwendungsfalldiagramms\footnotemark (engl: use case diagram) gewählt. Die Diagramme sind im Anhang dargestellt.

\footnotetext{Ein Anwendungsfalldiagramm ist eine Diagrammart der Unified Modelnig Langauge, kurz UML. In dieser werden sowohl Auslöser einer Funktion als auch Funktionsbeschreibung dargestellt. Eine Funktion kann dabei weitere Funktionen einbinden oder erweitern.}

Die im Folgenden vorgestellten Anwendungsfälle dienen sowohl der Dokumentation der Anwendungsfunktionen, als auch zur späteren Definition der Testfälle.

\subsubsection{Benutzeranwendungen}
\label{sec:Benutzeranwendungen}

Ein Benutzer der Software hat verschiedene Möglichkeiten der Interaktion mit
der Anwendung. Diese sind im Folgenden mit der fortlaufenden
Kennzeichnung \textbf{AFBXX}\footnote{AFBXX = Anwendunsfall eines Benutzers mit
Nr. XX} markiert und strukturiert aufgelistet:

\begin{description}
  \item[AFB01] Der Benutzer ruft die Webanwendung durch einen Internetbrowser
  auf.
  \item[AFB02] Der Benutzer dreht sich horizontal und vertikal um die Achse
  der Fotoaufnahme, um einen Rundumblick zu erhalten.
  \item[AFB03] Der Benutzer zoomt über die Steuerelemente in die Fotoaufnahme
  hinein.
  \item[AFB04] Der Benutzer navigiert mit Hilfe der Navigationselemente zu
  anderen Aufnahmepositionen.
  \item[AFB05] Der Benutzer navigiert mit Hilfe der Minimap zu einer anderen
  Aufnahmeposition.
  \item[AFB06] Der Benutzer bewegt sich mit Hilfe der Pfeiltasten seiner
  Tastatur zu anderen Aufnahmepositionen
  \item[AFB07] Der Benutzer öffnet über ein Steuerungselement ein
  Informationsfenster.
  \item[AFB08] Der Benutzer schließt über ein Steuerungselement ein
  Informationsfenster.
\end{description}

% TODO: Verweise auf Anwenwendungsfalldiagram, Anhang, und Seite korrekt einfügen
Die Anwendungsfälle AFB01 bis AFB08 definieren die Menge an
Interaktionsmöglichkeiten eines Benutzers. Diese Menge an
Interaktionsmöglichkeiten ist inklusive der hinterlegten Funktionen im
Awendungsfalldiagramm im \anhang{BenutzerAnwendungsfalldiagramme} abgebildet.

\subsubsection{Adminstratoranwendungen}
\label{sec:Adminstratoranwendungen}

Ein Administrator hat in vier Bereichen Möglichkeiten zur
Steuerung und Pflege der Webanwendung. Diese vier Bereiche wurden im
\verweis{Administrationsbereich} vorgestellt. Die Möglichkeiten der Interaktion
in diesen Bereichen sind mit der forlaufenden
Kennzeichnung \textbf{AFAXX}\footnote{AFAXX = Anwendunsfall eines Adminstrator
mit Nr. XX} markiert und strukturiert aufgelistet:

\begin{description}
  \item[AFA01] Der Adminstrator ruft den Administrationsbereich der
  Webanwendung durch einen Internetbrowser auf.
  \item[AFA02] Der Administrator klickt auf den Menüpunkt "`Infotexte"' und ihm
  wird die Infotextverwaltung angezeigt.
  \item[AFA03] Der Administrator erstellt im Menüpunkt "`Infotexte"' einen neuen
  Informationstext.
  \item[AFA04] Der Administrator verändert im Menüpunkt "`Infotexte"' einen
  erstellten Informationstext.
  \item[AFA05] Der Administrator löscht im Menüpunkt "`Infotexte"' einen
  erstellten Informationstext.
  \item[AFA06] Der Administrator klickt auf den Menüpunkt "`Fotos"' und ihm wird
  die Fotoverwaltung angezeigt.
  \item[AFA07] Der Administrator lädt im Menüpunkt "`Fotos"' ein erstelltes
  Panoramafoto mit Beschreibung und Namen hoch.
  \item[AFA08] Der Administrator verändert im Menüpunkt "`Fotos"' den Namen und
  die Beschreibung eines hochgeladenen Fotos.
  \item[AFA09] Der Administrator löscht im Menüpunkt "`Fotos"' ein hochgeladenes
  Foto.
  \item[AFA10] Der Administrator klickt auf den Menüpunkt "`interssante Orte"'
  und ihm wird die Verwaltung der interessanten Orte angezeigt.
  \item[AFA11] Der Administrator erstellt im Menüpunkt "`interssante Orte"'
  einen neuen interessanten Ort.
  \item[AFA12] Der Administrator verändert im Menüpunkt "`interssante Orte"' den
  Namen und die Beschreibung zu einem interessanten Ort.
  \item[AFA13] Der Administrator löscht im Menüpunkt "`interssante Orte"' einen
  interessanten Ort.
  \item[AFA14] Der Administrator klickt auf den Menüpunkt "`Übersichtskarte"'
  und ihm wird die Übersichtskarte angezeigt.
  \item[AFA15] Der Administrator platziert im Menüpunkt "`Übersichtskarte"' ein
  hochgeladenes Foto auf der Übersichtskarte.
  \item[AFA16] Der Administrator verschiebt im Menüpunkt "`Übersichtskarte"' ein
  bereits platziertes Foto an eine andere Position.
  \item[AFA17] Der Administrator navigiert im Menüpunkt "`Übersichtskarte"' über
  die Steuerelemente zu einem anderen Stockwerk.
  \item[AFA18] Der Administrator verbindet im Menüpunkt "`Übersichtskarte"'
  mehrere hochgeladene und positionierte Fotos auf der Übersichtskarte.
  \item[AFA19] Der Administrator verbindet im Menüpunkt "`Übersichtskarte"' ein
  hochgeladenes und positioniertes Foto mit erstellten Informationstexten.
\end{description}

% TODO: Verweise auf Anwenwendungsfalldiagram, Anhang, und Seite korrekt einfügen
Die Anwendungsfälle AFA01 bis AFA19 definieren die Menge an
Interaktionsmöglichkeiten eines Administrators. Diese Menge an
Interaktionsmöglichkeiten ist inklusive der hinterlegten Funktionen in einem
Awendungsfalldiagramm im Anhang X auf Seite X dargestellt.
