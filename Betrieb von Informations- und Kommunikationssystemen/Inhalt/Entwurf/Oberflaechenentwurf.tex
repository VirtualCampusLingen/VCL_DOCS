\subsection{Oberflächenentwurf}
\label{sec:Oberflaechenentwuf}

Ein Oberflächenentwurf, auch Mockup genannt, eignet sich besonders in
IT-Projekten für einen ersten Grobentwurf der Benutzeroberfläche des zu
entwickelnden Systems. Ein solcher Oberflächenentwurf beinhaltet die wichtigsten
Benutzerelemente, mit denen die Funktionen des Systems erfüllt werden können.
Das Design und das Layout der Benutzerelemente ist dabei zweitrangig. Ein
Oberflächenentwurf dient in erster Linie dazu, das System für die Entwickler zu
visualisieren. Dies ist besonders von Vorteil, wenn mehrere Personen an der
Entwicklung beteiligt sind. In diesem Fall vermittelt der Oberflächenentwurf ein
einheitliches Bild des zu realisierenden Systems an alle beteiligten
Entwickler. Dieser erste Eindruck vom Endprodukt ist auch für die Auftraggeber
eines Projektes von großem Interesse. Der Oberflächenentwurf dient somit im
zweiten Schritt auch dem Auftraggeber und der Kommunikation zwischen
Auftraggeber und Entwicklerteam. Funktionsvorstellungen und Erweiterung können
mit Hilfe eines Oberflächenentwurfs direkt an einem Modell festgemacht werden.
Auf diese Weise können Fehlentwicklungen von Beginn des Projektes an verhindert
werden.

\subsubsection{Benutzeransicht}
\label{sec:Benutzeransicht}

Ein erster Oberflächenentwurf der Benutzeransicht des vorliegenden Projektes
ist in \abbildung{MockupFrontend} dargestellt.

\begin{figure}[htb]
\centering
\includegraphics[width=1.0\textwidth]{MockupFrontend.jpg}
\caption[Oberflächenentwurf der
Benutzeransicht]{Oberflächenentwurf der
Benutzeransicht\protect\footnotemark}
\label{fig:MockupFrontend}
\end{figure}
\footnotetext{Quelle: Eigene Darstellung}

\clearpage

In diesem Mockup sind folgende vier Elemente abgebildet:

\begin{itemize}
 \item Ein 360-Grad-Foto
 \item Ein Steuerkreuz
 \item Eine Übersichtskarte (Minimap)
 \item Ein Informationsfenster
\end{itemize}

Das \textbf{360-Grad-Foto} ist hierbei zentrales Element des Projektes. Dieses
Foto stellt einen Standpunkt dar, von dem aus sich der Benutzer den Campus
Lingen ansehen kann. Von diesem Standpunkt aus kann sich der Benutzer in alle
Richtungen in dem Panoramafoto frei umsehen.

Mit dem \textbf{Steuerkreuz} am unteren Rand des dargestellten
Oberflächenentwurfs kann der Benutzer zu einem anderen Aufnahmepunkt wechseln.
Er kann auf diese Weise den Campus von einem anderen Standpunkt aus betrachten.
An diesem neuen Standort kann sich der Benutzer wiederum frei umsehen. Die
Pfeile des Steuerkreuzes zeigen dabei zu jedem Panoramafoto, welches von der
aktuellen Position aus erreichbar ist.

Die \textbf{Minimap} ist in einer Ecke des Panoramas platziert und erfüllt
zwei Aufgaben: Zum einen dient sie der Orientierung des Benutzers. Sie zeigt, an
welcher Aufnahmeposition sich der Benutzer aktuell am Campus
befindet\footnote{Im Oberflächentwurf ist diese Position durch einen schwarzen
Punkt gekennzeichnet.}. Hierdurch wird neben dem Einblick in die Räumlichkeiten
des Campus auch ein Bild vom Aufbau des Campus vermittelt. Zum anderen kann die
Minimap, ähnlich wie die Pfeile des Steuerkreuzes, zum Navigieren zu anderen
Standpunkten genutzt werden. Beim Klicken auf die Minimap öffnet sich ein
Informationsfenster, welches interessante Orte der Hochschule anzeigt. Bei der
Auswahl eines der Orte aus der Liste wechselt die Position des Benutzers zu
dem ausgewählten Standpunkt. Auch der Ausschnitt, der auf der Minimap
präsentiert wird, passt sich der neuen Position des Benutzers an.

Das \textbf{Informationsfenster} zeigt interessante Informationen zum aktuellen
Panorama an. Im obigen Oberflächenentwurf sind zwei solcher Informationsfenster
dargestellt\footnote{Die Informationsfenster befinden sich am rechten Rand
sowie oberhalb des Steuerkreuzes}. Diese Darstellungsformen sind als alternativ
zu betrachten. Die Art der Informationsdarstellung ist zum Zeitpunkt des
Oberflächenentwurfs noch nicht eindeutig festgelegt. Inhalt dieser
Informationsfenster können dabei interessante Projekte einzelner Studiengänge,
Öffnungszeiten von Räumlichkeiten oder Wissenwertes aus dem Studienalltag sein.
Die Informationen sollen in Form eines Popup-Fensters dargestellt werden. Sie
sollen also nicht permanent angezeigt werden, sondern erscheinen erst durch
Klicken des Benutzers auf einen Button. Dadurch wird das Blickfeld des Benutzers
nicht durch störende Anzeigen eingeschränkt.

\subsubsection{Administrationsbereich}
\label{sec:Administrationsbereich}

Ein Oberflächenentwurf des Administrationsbereiches ist in
\abbildung{MockupBackend} dargestellt.

\begin{figure}[htb]
\centering
\includegraphics[width=1.0\textwidth]{MockupBackend.png}
\caption[Oberflächenentwurf des
Administrationsbereiches]{Oberflächenentwurf des
Administrationsbereiches\protect\footnotemark}
\label{fig:MockupBackend}
\end{figure}
\footnotetext{Quelle: Eigene Darstellung}

Der Administartionsbereich gliedert sich hierbei in folgende Bereiche:

\begin{itemize}
  \item Infotextverwaltung
  \item Fotoverwaltung
  \item Interessante Ort
  \item Übersichtskarte
\end{itemize}

In der \textbf{Infotextverwaltung} können Informationstexte verfasst und zu
Panoramafotos zugeordnet werden. Ein Informationstext kann dabei auch mehreren
Fotos zugeordnet werden. Dadurch können dem Benutzer alle Informationstexte
angezeigt werden, die sich auf den Standpunkt beziehen, an dem sich dieser
gerade befindet. Alle Informationstexte, die bereits erstellt wurden, werden in
der Infotextverfaltung tabellarisch aufgelistet. Jeder dieser Informationstexte
kann sowohl verändert als auch gelöscht werden.

In der \textbf{Fotoverwaltung} werden analog zu der Infotextverwaltung die
erstellten Panoramafotos hinterlegt und gepflegt. Erstellte Panoramafotos
können in diesem Menüpunkt mit Namen und Beschreibung hochgeladen werden.
Analog zu den Infotexten werden hier die bereits hochgeladenen Fotos
tabellarisch aufgelistet. Auch die Möglichkeit zur Änderung und Löschung der
Fotos ist hier gegeben.

Der Bereich \textbf{Interessante Orte} bietet die
Möglichkeit, die Standorte zu hinterlegen, die dem Benutzer beim Klicken
auf die Minimap angezeigt werden. Zu diesen Standorten kann der Benutzer
dann springen, ohne dorthin navigieren zu müssen. Diese Möglichkeit ist vorallem
für Benutzer der Anwendung von Vorteil, die den Campus nicht kennen und nicht
erst nach bestimmten Orten suchen wollen. Zur Pflege der interessanten Orte muss
im entsprechenden Bereich in der Administrationsoberfläche nur ein
beschreibender Text eingetragen und mit einem hochgeladenen Foto verlinkt werden.
 
Die \textbf{Übersichtskarte} stellt den letzen und komplexesten Bereich des
Administrationsbereiches dar. In der Übersichtskarte werden die hochgeladenen
Panoramafotos auf einer Karte des Hochschulgebäudes platziert. Dazu wird dem
angemeldeten Administrator zunächst eine topographische Karte des Campus
präsentiert. Auf dieser Karte kann der Administrator durch Klicken eine
Position bestimmen, zu der er ein Foto auswählen und speichern kann.
Ein Foto kann dabei nur an einer Position auf der Übersichtskarte platziert
werden. Der Administrator hat die Möglichkeit die Position jedes Fotos beliebig
oft zu ändern. Darüber hinaus werden dem Administrator Steuerelemente angezeigt
mit denen er einen bestimmten Bereich des Campus auswählen kann, in dem er
Fotos positionieren möchte. So kann der Administrator sowohl zwischen
verschiedenen Gebäudekomplexen als auch zwischen den einzelnen Stockwerken
innerhalb eines Gebäudekomplexes wählen. Neben der Möglichkeit ein Foto zu
positionieren hat der Administrator weiterhin die Möglichkeit die Verbindungen
zwischen positionierten Fotos zu pflegen. Aus dem Oberflächenentwurf der
Benutzeroberfläche, welcher im vorherigen Abschnitt vorgestellt wurde, ist
ersichtlich, dass ein Benutzer von einem Panoramafoto aus zu einem anderen
Standpunkt navigieren kann. Diese Navigation beruht auf Verbindungen zwischen
den Panoramafotos. Solche Verbindungen werden in der Übersichtskarte geplegt.
Die Steuerungselemente zu Pflege der Verbindungen zwischen Panoramafotos sind zum
Zeitpunkt der Erstellung des Mockups noch nicht definiert, die Notwendigkeit
dieser Funktion ist aber bedacht.