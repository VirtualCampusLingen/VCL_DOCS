\subsection{Oberflächenentwurf}
\label{sec:Oberflaechenentwuf}

Ein Oberflächenentwurf, auch Mockup genannt, eignet sich besonders bei IT-Projekten für einen ersten Grobentwurf des zu entwickelnden Systems. Ein Oberflächenentwurf beinhaltet die wichtigsten Benutzungselemente, mit denen die Funktion des Systems erfüllt wird. Das Design oder Layout ist dabei zweitrangig. Ein solcher Entwurf dient in erster Linie dazu, das System für die Entwickler zu visualisieren. Das ist besonders von Vorteil, wenn mehrere Personen an der Entwicklung beteiligt sind, denn dann bekommen alle Beteiligten das gleiche Bild des Systems und ein erster Eindruck vom Endprodukt wird vermittelt. Dieser Ersteindruck vom Endprodukt ist auch für die Auftraggeber eines Projektes von großem Interesse. Der Oberflächenentwurf dient damit im zweiten Schritt auch dem Auftraggeber und der Kommunikation zwischen Auftraggeber und Entwicklerteam. Funktionsvorstellungen und Erweiterung können mit Hilfe eines Oberflächenentwurfs direkt an einem Modell festgemacht werden. Es kann zudem von Beginn des Projektes an verhindert werden, dass das Projekt in eine andere Richtung verläuft, als die Interessen des Auftraggebers.

