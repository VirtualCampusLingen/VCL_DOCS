\subsection{Lastenheft}
\label{sec:Lastenheft}

% TODO: Verweis auf Unternehmensführung richtig machen
% TODO: Lastenheft hier und im Original überarbeiten!!!!
Nach der Analyse der Ist-Situation gilt es im Folgenden das konkrete Ziel und die Funktionen des Projektes zu definieren. 
Zu diesem Zweck wurde aufbauend auf der Konzeptentwicklung (siehe Kapitel XX Ausarbeitung Unternehmensführung) eine 
Lastenheft angefertigt (Verweis auf Lastenheft Literatur).

Im Lastenheft ist das folgende Ziel für das vorliegende Projekt definiert:

"`"'Ziel des Projektes Virtueller Campus Lingen ist es den Standort der Hochschule Osnabrück
in Lingen virtuell darzustellen sowie alle vier Institutionen mit ihren Studienangeboten
vorzustellen."`"' (Quelle: Lastenheft)
Die Darstellung des Campus soll dabei im Stile von Goole Street View\copyright in 360-Grad Panoramafotos erfolgen, in 
denen sich ein Nutzer der Software frei umsehen kann. Darüber hinaus soll ein Backendsystem zur Pflege und Wartung der 
Software entwickelt werden (Quelle: Lastenheft).

Zur Erfüllung der Zielbestimmung des Lastenheftes sind darin folgende Produktfunktionen definiert:

\textbf{Benutzerfunktionen}:

\begin{description}
  \item[LF0010] Einem beliebigen Internetnutzer muss es möglich sein, ohne Anmeldung auf die Anwendung zugreifen zu können.
  \item[LF0015] Beim Starten der Anwendung soll dem Benutzer ein 360-Grad Panorama vom Eingangsbereich des Campus Lingen 
  präsentiert werden.
  \item[LF0020] Der Benutzer muss sich in den 360-Grad Fotos der Anwendung frei umsehen können.
  \item[LF0030] Der Benutzer muss zwischen den Fotos navigieren können.
  \item[LF0040] Die Anwendung muss eine Übersichtkarte enthalten, die sowohl den aktuellen Standpunkt, als auch alle 
  weiteren Einstiegspunkte umfasst.
  \item[LF0050] Die Übersichtskarte stellt alle Gebäude des Campus Lingen in Vogelperspektive dar. Innerhalb der 
  Übersichtskarte sind alle Einstiegspunkte in die 360-Grad-Ansicht enthalten.
  \item[LF0060] Die Minimap soll in der oberen linken Ecke des 360-Grad-Fotos angezeigt werden.Sie stellt einen Ausschnitt 
  der Übersichtskarte dar.
  \item[LF0070] In den 360-Grad-Fotos müssen relevante Informationen zu den dargestellten Örtlichkeiten angezeigt werden können.
\end{description}

\textbf{Backendfunktionen}:

\begin{description}
  \item[LF1010] Die Anwendung muss durch den Administrator offline geschaltet werden können.
  \item[LF1020] Der Administrator muss neue Fotos in die Anwendung einpflegen können. Die Fotos können hierbei an beliebigen Punkten auf einem Wegenetz positioniert werden.
  \item[LF1025] Ein Wegenetz muss dem Administrator als Positionierungseinschränkung für neue 306-Grad-Fotos zur Verfügung stehen.
  \item[LF1030] Der Administrator muss veraltete Fotos in der Anwendung austauschen können.
  \item[LF1040] Der Administrator muss Informationstexte zu den 360-Grad-Fotos hinzufügen, abändern und löschen können. Diese Informationstexte können zum Beispiel Projektvorstellungen enthalten. Der Informationstext besteht aus einem Pop-Up mit Einführungstext und weiterführenden Links.
  \item[LF1110] Der Administrator muss sich mit einem Passwort authentifizieren können.
  \item[LF1120] Der Administrator muss das Passwort abändern können.
\end{description}

Diese Funktionen sind zur Erfüllung der Zielsetzung zu implementieren. Neben diesen Funktionen gehen aus dem Lastenheft 
auch folgende Produktdaten hervor:

\begin{description}
  \item[LD0010] 360-Grad Fotos
  \item[LD0020] Informationstexte zu den 360-Grad-Fotos
  \item[LD0030] Benutzername des Administrators
  \item[LD0040] Passwort des Administrators
  \item[LD0050] Wegenetz
\end{description}

Diese sind vor dem Hintergrund der späteren Datenbankplanung zu berücksichtigen.