\subsection{Lastenheft}
\label{sec:Lastenheft}

Nach der Analyse der Ist-Situation gilt es im Folgenden das konkrete Ziel und
die Funktionen des Projektes zu definieren. Zu diesem Zwecke wurde aufbauend auf
der Konzeptentwicklung des Projektes ein Lastenheft
angefertigt\footnote{\citet{lastenheft2013}}. Die Konzeptentwicklung ist
dabei Teil der Ausarbeitung "`Unternehmensführung"' und wird in dieser
detailliert betrachtet.

Im Lastenheft ist das folgende Ziel für das vorliegende Projekt definiert:

\begin{quote}
"`Ziel des Projektes Virtueller Campus Lingen ist es den Standort der Hochschule Osnabrück
in Lingen virtuell darzustellen sowie alle vier Institutionen mit ihren Studienangeboten
vorzustellen.\footnote{\citet[S.~1]{lastenheft2013}}"'
\end{quote}

Die Darstellung des Campus soll dabei im Stile von Goole Street View, in
360-Grad-Fotos erfolgen. In diesen Fotos soll sich der Nutzer der zu
erstellenden Anwendung frei umsehen können. Darüberhinaus soll ein
Administrationssystem zur Pflege und Wartung der Anwendung entwickelt
werden.\footnote{\citet[S.~1]{lastenheft2013}}

Zur Erfüllung der Zielbestimmung sind im Lastenheft folgende Produktfunktionen
definiert:

\textbf{Benutzerfunktionen}:

\begin{description}
  \item[LF0010] Einem beliebigen Internetnutzer muss es möglich sein, ohne
  Anmeldung auf die Anwendung zugreifen zu können.
  \item[LF0015] Beim Starten der Anwendung soll dem Benutzer ein 360-Grad-Foto
  vom Eingangsbereich des Campus Lingen präsentiert werden.
  \item[LF0020] Der Benutzer muss sich in den 360-Grad-Fotos der Anwendung frei
  umsehen können.
  \item[LF0030] Der Benutzer muss zwischen den Fotos navigieren können.
  \item[LF0040] Die Anwendung muss eine Minimap enthalten, die den
  aktuellen Standpunkt anzeigt.
  \item[LF0050] Der Benuzer muss direkt zu interessanten Orten, wie der
  Bibliothek oder dem Fachbereichsgebäuden, direkt springen können, ohne
  über mehrere Fotos dorthin navigieren zu müssen.
  \item[LF0060] In den 360-Grad-Fotos müssen relevante Informationen zu den
  dargestellten Ört\-lich\-kei\-ten angezeigt werden können.
\end{description}

\textbf{Administrationsfunktionen}:

\begin{description}
  \item[LF1010] Der Administrator muss neue Fotos in die Anwendung einpflegen
  können.
  \item[LF1020] Der Administrator muss veraltete Fotos in der Anwendung
  austauschen können.
  \item[LF1030] Der Administrator muss hinterlegte Informationen zu einzelnen
  Fotos, zum Beispiel Name und Beschreibung, verändern können.
  \item[LF1040] Der Administrator muss Informationstexte speichern können.
  \item[LF1050] Der Administrator muss Informationstexte löschen können.
  \item[LF1060] Der Administrator muss den Informationstext abändern können.
  \item[LF1070] Der Administrator muss interessante Orte, wie die Bibliothek
  oder die Fachbereichsgebäude, mit einem beschreibenden Text erstellen können.
  \item[LF1080] Der Administrator muss interessante Orte löschen können.
  \item[LF1090] Der Administrator muss die hinterlegten Informationen zu
  interessanten Orten verändern können.
  \item[LF1100] Dem Administrator muss eine topographische Karte als
  Positionierungseinschränkung für neue 360-Grad-Fotos zur Verfügung stehen.
  \item[LF1110] Der Administrator muss auf dieser Karte Fotos platzieren und
  löschen können.
\end{description}

Diese Funktionen sind zur Erfüllung der Zielsetzung zu implementieren. Neben
diesen Funktionen gehen aus dem Lastenheft weiterhin die folgenden Produktdaten
hervor:

\textbf{Produktdaten}:

\begin{description}
  \item[LD0010] 360-Grad Fotos
  \item[LD0020] Informationstexte zu den 360-Grad-Fotos
  \item[LD0030] Beschreibungstexte zu interessanten Orten
  \item[LD0030] Karte/Architekturskizze des Campus zum Positionieren der
  360-Grad-Fotos
\end{description}

Diese Produktdaten sind vor dem Hintergrund der späteren Datenbankplanung zu
berücksichtigen.