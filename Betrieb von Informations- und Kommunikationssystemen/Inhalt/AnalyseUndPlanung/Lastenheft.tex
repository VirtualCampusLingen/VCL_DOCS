\subsection{Lastenheft}
\label{sec:Lastenheft}

Nach der Analyse der Ist-Situation gilt es im Folgenden das konkrete Ziel und die Funktionen des Projektes zu definieren. 
Zu diesem Zweck wurde aufbauend auf der Konzeptentwicklung des Projektes ein Lastenheft angefertigt\footnote{siehe \citet{lastenheft2013}}.
Diese Konzeptentwicklung ist dabei Teil der Ausarbeitung "`Virtueller Campus Lingen - Unternehmensfürhung und Konzeptionierung"'.

Im Lastenheft ist das folgende Ziel für das vorliegende Projekt definiert:

\begin{quote}
"`Ziel des Projektes Virtueller Campus Lingen ist es den Standort der Hochschule Osnabrück
in Lingen virtuell darzustellen sowie alle vier Institutionen mit ihren Studienangeboten
vorzustellen."' \footnote{\citet{lastenheft2013}, Seite 1}
\end{quote}

Die Darstellung des Campus soll dabei im Stile von Goole Street View in 360-Grad-Fotos erfolgen, in 
denen sich ein Nutzer der Software frei umsehen kann. Darüber hinaus soll ein Backendsystem zur Pflege und Wartung der 
Software entwickelt werden\footnote{\citet{lastenheft2013}, Seite 1}.

Zur Erfüllung der Zielbestimmung sind im Lastenhefe folgende Produktfunktionen definiert:

\textbf{Benutzerfunktionen}:

\begin{description}
  \item[LF0010] Einem beliebigen Internetnutzer muss es möglich sein, ohne Anmeldung auf die
  Anwendung zugreifen zu können.
  \item[LF0015] Beim Starten der Anwendung soll dem Benutzer ein 360-Grad-Foto vom
  Eingangsbereich des Campus Lingen präsentiert werden.
  \item[LF0020] Der Benutzer muss sich in den 360-Grad-Fotos der Anwendung frei umsehen
  können.
  \item[LF0030] Der Benutzer muss zwischen den Fotos navigieren können.
  \item[LF0040] Die Anwendung muss eine Minimap enthalten, die den aktuellenStandpunkt anzeigt.
  \item[LF0050] Der Benuzer muss auf zu interessanten Orten, zum Beispiel Bibliothek oder Fachbereichsgebäuden schnell und direkt springen können, ohne sich dorthin navigieren zu müssen.
  \item[LF0060] In den 360-Grad-Fotos müssen relevante Informationen zu den dargestellten Ört-
  lichkeiten angezeigt werden können.
\end{description}

\textbf{Administrationsfunktionen}:

\begin{description}
  \item[LF1010] Der Administrator muss neue Fotos in die Anwendung einpflegen können.
  \item[LF1020] Der Administrator muss veraltete Fotos in der Anwendung austauschen können.
  \item[LF1030] Der Administrator muss hinterlegte Informationen zu einzelnen Fotos, zum Beispiel Name und Beschreibung, verändern können.
  \item[LF1040] Der Administrator muss Informationstexte speichern können.
  \item[LF1050] Der Administrator muss Informationstexte löschen können.
  \item[LF1060] Der Administrator muss den Informationstext abändern können.
  \item[LF1070] Der Administrator muss interessante Orte, zum Beispiel Bibliothek oder Fachbereichsgebäide, mit einem entsprechenden Text erstellen können.
  \item[LF1080] Der Administrator muss interessante Ort löschen können.
  \item[LF1090] Der Administrator muss die hinterlegten Informationen zu interessanten Orten verändern können.
  \item[LF1100] Dem Administrator muss eine topographische Karte als Positionierungseinschränkung für neue
  360-Grad-Fotos zur Verfügung stehen.
  \item[LF1110] Der Administrator muss auf dieser Karte Fotos platzieren und löschen können.
\end{description}

Diese Funktionen sind zur Erfüllung der Zielsetzung zu implementieren. Neben diesen Funktionen gehen aus dem Lastenheft 
auch folgende Produktdaten hervor:

\textbf{Produktdaten}:

\begin{description}
  \item[LD0010] 360-Grad Fotos
  \item[LD0020] Informationstexte zu den 360-Grad-Fotos
  \item[LD0030] Beschreibungstexte zu interessanten Orten
  \item[LD0030] Karte/Skizze des Campus zum Positionieren der 360-Grad-Fotos
\end{description}

Diese Produktdaten sind vor dem Hintergrund der späteren Datenbankplanung zu berücksichtigen.