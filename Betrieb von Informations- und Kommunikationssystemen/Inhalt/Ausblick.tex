\section{Ausblick}
\label{sec:Ausblick}

Am Ende des Projektes sollen an dieser Stelle Ideen und Anregungen zur Weiterführung des Projektes gegeben werden. Viele dieser Ideen kamen während des Projektverlaufes auf, ließen sich aber aus zeitlichen Gründen nicht realisieren.

Ein Beispiel dafür ist zum Beispiel die Umsetzung einer geführten Tour für Studieninteressierte durch bestimmte Studienbereiche. Auf Messen oder ähnlichem wäre es dadurch möglich Auschnitte und Informationen zu bestimmten Studienbereichen ohne Benutzerinteraktion zu zeigen. Auch für Studieninteressierte, die sich nicht selbst durch den Campus bewegen, sondern lediglich die wichtigsten Informationen zu Ihrem Studiengang erhalten wollen, wäre dieser Ansatz interessant.

Gleichzeitig wäre besonders bei den erwähnten Messeauftritten eine Version der Software von Vorteil, die ohne Internetanbindung verwendet werden kann. Zum Zeitpunkt des Projektabschlusses greift die Software auf einige Programmbibliotheken (vor allem im Javascript Bereich) übers Internet zu. Diese Zugriffe könnten auch offline erfolgen, wenn man die Bilbiotheken runterlädt und lokal einbindet. Die teure Internetanbindung auf Messen, wenn überhaupt eine vorhanden ist, könnte dadurch vermieden werden.

Darüber hinaus gab es auch Ideen den Campus und seine Darstellung durch besondere Aufnahmen, zum Beispiel bei Nacht, noch einzigartiger zu gestalten. Es wäre denkbar Panoramafotos bei Nacht zu erstellen und in die bestehende Anwendung einen "`Nachtmodus"' zu implementieren, der vom Benutzer über einen Schalter ausgewählt werden könnte. Der Kreativität sind an dieser Stelle keine Grenzen gesetzt.