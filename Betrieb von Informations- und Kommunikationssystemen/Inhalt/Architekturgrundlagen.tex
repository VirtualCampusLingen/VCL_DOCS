\section{Architekturgrundlagen}
\label{sec:Architekturgrundlagen}

Das vorliegende Projekt beschreibt die Entwicklung einer Softwarelösung für das
Ziel der "`Konzeptionierung und Entwicklung einer Anwendung zur Darstellung
des Campus Lingen als attraktiven
Studienstandort"'\footnote{\citet[S.~5]{unternehmensfuehrung2014}}.
Im Zuge dieser Entwicklung wird eine Architektur erarbeitet, die auf
verschiedenen Technologien aufbaut. Das nachfolgende Kapitel dient dazu diese
Technologien grundlegend zu erläutern, um deren Verwendung im Projekt
nachvollziehen zu können. Zu diesem Zwecke wird an dieser Stelle auf
die Architektur vorgegriffen, die im späteren Verlauf dieser Ausarbeitung
entwickelt wird. Die Architektur des vorliegenden Projekts ist auf Seite
\pageref{fig:Architektur} in \abbildung{Architektur} zu sehen.

In diesem Architekturentwurf ist die Verwendung von vier Technologien
dargestellt.
Diese sind:

\begin{itemize}
  \item HTML
  \item Javascript
  \item PHP
  \item SQL
\end{itemize}

Diese Technologien sind essentiell für das Verständnis der zu entwickelten
Software. Aus diesem Grund werden die wichtigsten Grundlagen zu jeder
Technologie nachfolgend erläutert. Der Fokus liegt dabei immer auf den
Teilbereichen der Technologie, die im vorliegenden Projekt eingesetzt werden.

\subsection{HTML}
\label{sec:Html}

Die Hypertext Markup Language, abgekürzt HTML, stellt eine deklarative Sprache zur Beschreibung und Gestaltung von Internetseiten dar. Die erste Spezifikation von HTML erschien 1989, wobei sich die Spezifikation im Laufe der Zeit verändert und erweitert haben\footnotetext{\citet{taglinger2003}}.

Mithilfe von HTML können Dokumente erstellt werden, welche von Internetbrowsern gelesen, interpretiert und dem Inhalt entsprechend dargestellt werden. Dargestellt werden können Text, Bilder, Tabellen, Verlinkungen, Videos usw.

Der grundlegende Aufbau von HTML erfolgt hierbei durch sogenannte Tags, wobei die meisten Befehle einen Start- und einen Endtag besitzen.
Das Tag <html> dient zur Öffnung eines HTML-Bereichs, welcher die weiteren Bereiche und Inhalte für die Darstellung einer Internetseite besitzt. Dieses Tag und das gesamte Dokument werden zum Schluss mit dem Tag </html> geschlossen. Das Tag <head> enthält Kopfdaten, z.B. den Titel einer Internetseite, welcher in der Kopfzeile des Internetbrowser angezeigt wird. Im <body>-Bereich des HTML-Dokuments, welcher ebenfalls mit einem Tag geschlossen werden muss, wird der eigentliche Inhalt der Internetseite angegeben. Dieser wird dann im Browserfenster entsprechend den Angaben und Inhalten dargestellt.
Die Struktur eines stellt sich somit folgendermaßen dar:

%TODO: Listing erstellen

% <html>
% <head>
% <title>Titel der Internetseite </title>
% </head>
% <body>    Inhalt der Internetseite
% </body>
% </html>

Im vorherigen Beispiel würde somit eine Internetseite im Browser angezeigt werden, welche den Titel "`Titel der Internetseite"' in der Kopfzeile anzeigen würde. Als Inhalt würde nur der Text "`Inhalt der Internetseite"' angezeigt werden, ohne jegliche zusätzliche Formatierungen.

Die neueste Spezifikation von HTML, HTML5, bietet einige nützliche Erneuerungen, so können nun unter anderem direkt im HTML-Dokument Audio oder Videodateien eingebunden werden, ohne das zusätzliche Plugins, wie z.B. Adobe Flash benötigt werden, um diese abspielen zu können.
\subsection{Javascript}
\label{sec:Javascript}
\subsection{PHP}
\label{sec:Php}

PHP (Hypertext Preprocessor) ist eine serverseitige Skriptsprache, die mit dem Ziel entwickelt wurde Webseiten dynamisch
zu gestalten.
Mit dieser Intention wurde auch im vorliegenden Projekt PHP als Serversprache verwendet. Die nachfolgend beschriebenen
Grundlagen vermitteln die wichtigsten Aspekte der Sprache PHP, die nötig sind, um die Umsetzung des Projektes
nachvollziehen zu können.
Im Gegensatz zu der vorher beschrieben Skriptsprache Javascript ist PHP eine serverseitige Skriptsprache. Dadurch sind
zwei grundlegende Unterschiede zu Javascript bedingt:

\textbf{1}. PHP wird, wie die Beschreibung Serverskriptsprache bereits suggeriert, nicht auf dem Client (dem Browser des Nutzers),
sondern auf dem Webserver der Anwendung ausgeführt. Bei einer Nutzeranfrage an den Webserver wird also erst eine PHP-Routine serverseitig ausgeführt und anschließend wird das angefragte Dokument, inklusive
Javascriptdateien, an den Client (Browser) ausgeliefert und dort wird dann Javascript ausgeführt.

\textbf{2}. PHP hat weitestgehend vollen Ressourcenzugriff auf dem ausführenden System. Durch PHP-Befehle ist es zum Beispiel möglich Dateien auf dem ausführenden System (Webserver) anzulegen und zu löschen.
Diese beiden Unterschiede sind wichtig, um das Einsatzfeld von PHP zu verstehen.

PHP wird im vorliegenden Projekt für die folgenden Aufgaben eingesetzt:
\begin{itemize}
  \item Kommunikation mit der Datenbank
  \item Schreiben von dynamischen HTML
  \item Kopieren von Benutzerdateien auf das ausführende System.
\end{itemize}

Diese Aufgaben werden im späteren Verlauf ausführlicher beleuchtet, an dieser Stelle soll die Listung dieser Aufgaben
ausreichen, um ein Verständnis für die Einsatzfelder von PHP zu bekommen.
Um diese Aufgaben zu erfüllen sind in PHP, wie in jeder anderen Programmiersprache auch, Befehle mit einer vorgebenen
Syntax definiert. Diese Befehle werden im Quellcode in einem besonders gekennzeichneten Bereich geschrieben. Dieser
Bereich wird durch die Zeichenfolge '<?php' eingeleitet und mit der Zeichenfolge '?>' beendet.
Abschließend soll die folgende Abbildung die Anwendung von PHP verdeutlichen:

\lstinputlisting[language=PHP,caption={PHP Grundlagen},label={lst:PHP Grundlagen}]{Listings/PHP_Grundlagen.php}

In diesem Beispiel wird die Anfrage eines Benutzers ausgewertet. Die eigentliche Anfrage ist dabei nebensächlich und aus diesem Grund nicht dargestellt (auskommentiert in Zeile 2). Die Auswertung der Anfrage soll an dieser Stelle zeigen auf welchem Weg mit Hilfe von PHP dynamisches HTML geschrieben werden kann. In dem dargestellten Beispiel wird die Benutzeranfrage ausgewertet und je nachdem, was diese Anfrage zurückliefert wird ein HTML p-Tag mit anderem Inhalt geschrieben. Im Browser des Benutzers würde dadurch ausgegebn, ob seine Anfrage positiv oder negativ ausgewertet wurde.
\subsection{Datenbanken und SQL}
\label{sec:DatenbankenUndSql}

Eine Datenbank ist das Rückgrat jedes datenverarbeitenden Informationssystems. Dementsprechend häufig sind Datenbanken in 
verschiedenen Kontexten anzutreffen (Web-, ERP-, Mobile-Systeme, etc.). Die nachfolgende theoretische Fundierung dient zum 
Verständnis des Einsatzes der Datenbank im vorliegenden Projekt.

"`Eine Datenbank ist eine Sammlung von Daten, die einen Ausschnitt der realen Welt beschreibt."'\footnote{\citet{elmasri2009}}
Diese
Sammlung von Daten kann dabei sowohl in Papierform, als auch elektronisch und computergestützt aufbewahrt werden. Vor dem 
Hintergrund des vorliegenden Projektes gilt die weitere Betrachtung aber ausschließlich den computergestützten Datenbanken.
In diesem Teilbereich gibt es wiederum verschiedene Datenbanktypen, die sich in der Verwaltung und Haltung der Daten 
unterscheiden. Die wichtigsten Datenbanktypen sind im Folgenden aus Gründen der Abgrenzung aufgelistet:

\begin{itemize}
  \item hierarchische Datenbanken
  \item objektorientierte Datenbanken
  \item relationale Datenbanken
  \item NoSql Datenbanken
\end{itemize}

Der Fokus dieser Spezialisierung liegt im Folgenden wiederum auf der eingesetzten Technologie des Projektes und damit auf 
relationalen Datenbanken. 

Eine relationale Datenbank beruht auf dem relationalen Datenbankmodell von Edgar F. Codd aus dem Jahre 1970.\footnote{siehe dazu \citet{codd1970}} Grundlage dieses Modells ist der Begriff der Relation, der im Wesentlichen
eine mathematische Beschreibung einer Tabelle darstellt. Eine relationale Datenbank ist damit eine Sammlung von Tabellen.

Auf den Relationen (Tabellen) einer solchen Datenbank sind bestimmte Operationen definiert, die im Kontext der Mathematik 
der relationalen Algebra und im Kontext von Datenbanken dem Begriff SQL entsprechen.

SQL steht für Structured Query Language und stammt, genau wie das relationale Datenbankmodell, aus dem Jahre 1970. Es 
wurde ebenfalls von Edgar F. Codd mitentwickelt.\footnote{\citet{kline2005}} Über einen Entwicklungszeitraum von mehreren Jahrzehnten 
wurde SQL zu einer komplexen Sprache, die umfangreiche Möglichkeiten bietet. Dieser Gesamtkomplex wird hier stark verkürzt
dargestellt und beschränkt sich nur auf die wesentlichen Aspekte, die dem Verständnis des Projektes dienen.

SQL dient im Wesentlichen zum Definieren, Kontrollieren und Manipulieren von Datenstrukturen. Um diese 
Bereiche zu trennen sind in SQL drei Sprachbereiche definiert.

\begin{description}
  \item[Data Definition Language (DDL)] dient zum Erstellen, Löschen und Ändern von Tabellen.
  \item[Data Control Language (DCL)] dient zum Vergeben und Entziehen von Berechtigungen.
  \item[Data Manipulation Language (DML)] dient zum Lesen, Schreiben, Löschen und Ändern von Datenstrukturen.
\end{description}

Die Sprachbereiche DDL und DCL haben dabei in der Projektentwicklung eher einmaligen Charakter, wohingegen DML 
kontinuierlich verwendet wird, um dynamisch Informationen zu schreiben oder zu lesen. Die DML tritt dabei als 
Schnittstelle zwischen Anwendungsprogramm und Datenbank auf (siehe Abbildung: \nameref{fig:Architektur}). Über die DML sind beispielsweise 
Befehle definiert, die es erlauben einen Datensatz einer Tabelle zu lesen oder zu verändern. Diese Befehle werden im 
Anwendungsprogramm aufgerufen und erzeugen damit dynamischen Inhalt für den Anwender.