\section{Einleitung}
\label{sec:Einleitung}

Das Projekt virtueller Campus Lingen, kurz VCL, ist ein Studienprojekt der
Hochschule Osnabrück am Standort Lingen. Inhalt dieses Projektes ist eine
virtuelle Darstellung des Studienstandortes durch 360-Grad-Fotos für
Studieninteressierte und Besucher der Hochschulwebseite. Das Projektbild
orientiert sich dabei stark an der Google Street View Darstellung von
geographischen Orten\footnotemark. Durch eine solche virtuelle Darstellung des
Hochschulstandortes Lingen, soll besonders jungen Studieninteressierten die
Attraktivität des Campus aufgezeigt werden. Im folgenden Projekt wird daher
eine virtuelle Darstellung geschaffen, die Bekanntheitsgrad, Image und
Attraktivität des Hochschulstandortes Lingen steigern soll.

\footnotetext{siehe \nameref{fig:GoogleStreetViewBeispiel} im Anhang ~\ref{sec:BeispielEinerGoogleStreetViewAnsicht}}

Vorausgegangen ist der folgenden Betrachtung dabei ein konzeptioneller
Management- und Unternehmensführungsansatz zur Planung und Steuerung des
Projekts. Der besondere Betrachtungsschwerpunkt wurde dabei auf Kosten, Nutzen
und die Zielerreichung des Projektes gelegt. Diese Inhalte werden im Folgenden
nicht weiter betrachtet, können jedoch in der Ausarbeitung
"`Unternehmensführung"'\footnote{\citet{unternehmensfuehrung2014}} nachgelesen
werden. Schwerpunkt der folgenden Betrachtung ist dagegen die Modellierung, die
Umsetzung und der Betrieb des virtuellen Campus Lingen.

Zur Einführung der Modellierung wird in folgender Ausarbeitung eine Analyse und
Planungsphase beschrieben, in der vor allem die Anforderung in Form eines
Lastenheftes spezifiziert und die Projektgrenzen abgesteckt werden. Aufbauend
auf den erarbeiteten Erkenntnisse werden Modelle und Entwürfe erstellt, die
die technische Umsetzung des Projektes und die wichtigsten Komponenten dieser
Umsetzung beschreiben. Voraussetzung für Planung und Entwurf dieser Modelle
ist dabei ein technisches Grundverständnis bestimmter eingesetzter Technologien.
Diese Technologien werden im Vorfeld dieser Ausarbeitung kurz eingeführt und
erläutert. Mit Hilfe der erstellten Modelle kann daraufhin das Projekt
technisch umgesetzt werden. Die Umsetzung gliedert sich dabei in die Bereiche
der Panoramaerstellung und der Softwareerstellung. Beide Bereiche sind
essentielle Elemente des Projektes.
Nach Abschluss der Umsetzung ist das Projekt funktional fertiggestellt und kann
der Hochschule übergeben werden. Vor dieser Übergabe werden abschließende Tests
zur Identifizierung und Beseitigung von Fehlern durchgeführt.
Nach der Beschreibung von Übergabe und Test wird die vorliegende Ausarbeitung
mit einem Ausblick auf mögliche Projektweiterführungen und einem Fazit
abgeschlossen.