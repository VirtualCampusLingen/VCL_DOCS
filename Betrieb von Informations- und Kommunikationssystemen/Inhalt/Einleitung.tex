\section{Einleitung}
\label{sec:Einleitung}

Das Projekt virtueller Campus Lingen, kurz VCL, ist ein Studienprojekt der Hochschule Osnabrück am Standort Lingen. Inhalt dieses Projektes ist eine virtuelle Darstellung des Studienstandortes durch 360-Grad-Fotos für Studieninteressierte und Besucher der Hochschulwebseite. Das Projektbild orientiert sich dabei stark an der Google Street View Darstellung von geographischen Orten\footnotemark. Durch eine solche virtuelle Darstellung des Hochschulstandortes Lingen, soll besonders jungen Studieninteressierten die Attraktivität des Campus aufgezeigt werden. Im folgenden Projekt wird daher eine virtuelle Darstellung geschaffen, die Bekanntheit, Image und Attraktivität des Hochschulstandortes Lingen steigern soll.

\footnotetext{siehe \nameref{fig:GoogleStreetViewBeispiel} im Anhang ~\ref{sec:BeispielEinerGoogleStreetViewAnsicht}}

Vorausgegangen ist der folgenden Betrachtung dabei ein konzeptioneller Management- und Unternehmensführungsansatz zur Planung und Steuerung des Projekts. Der besondere Betrachtungsschwerpunkt wurde dabei auf Kosten, Nutzen und die Zielerreichung des Projektes gesetzt. Diese Inhalte werden im Folgenden nicht weiter betrachtet, können aber in der Ausarbeitung "`Unternehmensführung und Konzeption ..."'\footnote{\citet{unternehmensfuehrung2014}} nachgelesen werden. Schwerpunkt der folgenden Betrachtung ist dagegen die Modellierung, die Umsetzung und der Betrieb des Projektes VCL.

Zur Einführung der Modellierung wird in folgender Ausarbeitung eine Analyse und Planungphase beschrieben, in der vor allem die Anforderung in Form eines Lastenheftes spezifiziert und die Projektgrenzen abgesteckt werden. Aufbauend auf den erarbeitetenden Erkenntinsse werden Modelle und Entwürfe erstellt, die die technische Umsetzung des Projektes und die wichtigsten Komponenten dieser Umsetzung beschreiben. Vorraussetzung für Planung und Entwurf dieser Modelle ist dabei ein technisches Grundverständis bestimmter eingesetzter Technologien. Diese Technologien werden aus diesem Grund im Vorfeld dieser Ausarbeitung kurz eingeführt und erläutert. Mit Hilfe der erstellten Modelle kann daraufhin das Projekt technisch umgesetzt werden. Die Umsetzung gliedert sich dabei in die Bereiche der Panoramaerstellung und der Softwareerstellung. Beide Bereiche sind essentielle Elemente des Projektes und sind funktional voneinander abhängig. Nach Abschluss der Umsetzung ist das Projekt funktional fertiggestellt und kann der Hochschule übergeben werden. Vor dieser Übergabe werden abschließende Tests zur identifizierung und Beseitigung von Fehlern durchgeführt.
Daraufhin wird die vorliegende Ausarbeitung mit einem Ausblick auf mögliche Projektweiterführungen und einem Fazit abgeschlossen. 