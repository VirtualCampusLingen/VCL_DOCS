\section{Analyse und Planung}
\label{sec:AnalyseUndPlanung}

Nach der Fundierung der Architekturgrundlagen wird die enstehende Softwarelösung geplant. Diese Ausarbeitung setzt hierbei
am entwickelten Konzept und der daraus resultierenden Problemlösung\footnote{siehe dazu \citet{unternehmensfuehrung2014}, Kapitel Konzeptionierung} an.
Darauf aufbauend wird zunächst eine Analyse der Ist-Situation vorgenommen. 
Hieraus geht hervor, ob es in der Ist-Situation der Hochschuldarstellung Bestandteile gibt, die für die 
neue Softwarelösung wiederverwendet werden können. Die Planungsphase wird mit der Definition der Anforderungen in 
Form eines Lastenheftes und der Festlegung der Projektgrenzen beendet.