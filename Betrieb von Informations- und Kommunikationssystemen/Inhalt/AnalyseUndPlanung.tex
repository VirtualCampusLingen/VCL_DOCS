\section{Analyse und Planung}
\label{sec:AnalyseUndPlanung}

% TODO: Verweis auf andere Ausarbeitung einbinden. 2x
Nach der Fundierung der Architekturgrundlagen wird die enstehende Softwarelösung geplant. Es wird angesetzt am 
entwickelten Konzept (siehe Ausarbeitung Unternehmensführung) des Projektes und an der daraus resultierten Problemlösung 
(siehe Ausarbeitung Unternehmensführung). Darauf aufbauend wird zunächst eine Analyse der Ist-Situation vorgenommen. 
Hieraus geht hervor, ob es in der Ist-Situation der Hochschuldarstellung Elemente (anderes Wort finden) gibt, die für die 
neue Softwarelösung wiederverwendet werden können. Die Planungsphase wird mit der Definition der Anforderungen in 
Form eines Lastenheftes und der Festlegung der Projektgrenzen beendet.