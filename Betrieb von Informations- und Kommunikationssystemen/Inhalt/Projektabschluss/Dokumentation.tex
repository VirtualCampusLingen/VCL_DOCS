\subsection{Dokumentation}
\label{sec:Dokumentation}

Nach Abschluss und Korrektur des Alpha- und Betatests ist das Projekt funktional abgeschlossen. Zur Förderung von Erweitarbarkeit und Nachhaltigkeit wird eine Dokumentation angefertigt, die es jedem berechtigten Benutzer ermöglicht, die Software zu pflegen und zu erweitern. Die vollstände Dokumentation kann in \citet{dokumentation2014} nachgelesen werden. An dieser Stelle wird zur Transparenz der Benutzung der Awendung der Ablauf zur Erweiterung der Anwendung vorgestellt. Dieser Prozess umfasst ingesamt maximal sechs Teilschritte.

\begin{description}
  \item[1.] Als erstes muss ein neues Panorama aufgenommen werden. Wichtig dabei zu beachten ist, dass die Mitte des Panoramas nach Norden ausgerichtet ist. Es empfiehlt sich darüber hinaus ein Stativ und Panoramakopf zu verwenden, um eine optimale Fotoqualität zu erreichen (siehe \verweis{Panoramaerstellung}).
  \item[2.] Anschließend muss das erstelle Panorama im Administrationsbereich der Software unter dem Menüpunkt "`Fotos"', mit Namen und Beschreibung, hochgeladen werden. An dieser Stelle empfiehlt es sich eine einheitliche Namenskonvention zu verwenden, die Auskunft über den Ort der Aufnahme gibt, also zum Beispiel "`KE\_EG\_001"' (Gebäude KG, Erdgeschoss, Raum 001). Diese Namenskonvention hilft im Folgenden dabei, dieses Foto zu identifizieren.
  \item[3.] Als nächstes kann bei Bedarf ein Informationstext zu diesem Foto erstellt werden. Falls ein interessantes Projekt oder wichtige Informationen hierauf abgebildet sind können diese über einen Informationstext dem Benutzer angezeigt werden.
  \item[4.] Ebenso kann das erstellte und hochgeladene Foto einen interessanten Ort darstellen, für den es sich lohnt eine gesonderte Verlinkung in der Benutzeransicht zu setzen. Beispiele für solche Fälle sind zum Beispiel ein Einstiegsfoto in der Bibliothek, ein zentrales Foto in einem Fachbereich, ein Foto mit Überblick über den gesamten Campus o.ä. Ist dies der Fall kann im Menüpunkt "`Interessante Orte"' ein Eintrag mit Namen und Beschreibung des Ortes hinterlegt und das erstellte Panorama damit verknüpft werden.
  \item[5.] Im Vorletzen Schritt muss das neue Foto auf der Karte des Campus positioniert werden. Dazu navigiert der Administrator zum Menüpunkt "`Übersichtskarte"' und klickt auf die Position der angzeigten Karte, an der das Foto aufgenommen wurde. Es erscheint ein roter Marker und nach einem weiteren Klick auf diesen Marker öffnet sich ein Auswahlmenü. In diesem Menü werden dem Administrator alle hochgeladenen Fotos angezeigt, die er am ausgewählten Punkt positionieren kann. Der Administrator wählt das neu hochgeladene Foto aus (im vorherigen Beispiel "`KE\_EG\_001"') und drückt auf "`Speichern"'.
  \item[6.] Nach dem Positionieren des neuen Fotos erscheint ein grüner Marker an der festgelegten Position. Durch klicken auf diesen Marker hat der Administrator im letzen Schritt die Möglichkeit vorher gespeicherte Informationstexte (siehe punkt 3.) an dieses Foto anzuhängen oder diesen Punkt mit anderen zu verbinden. Durch die Verbindung zu anderen Fotos kann ein Benutzer in der Benutzeransicht über andere Fotos zu dem neu hochgeladenen Foto navigieren. Anders ausgedrückt ist ein hochgeladenes, aber nicht verbundenes Foto für die Endbenutzer nicht sichtbar, da er zu diesem nicht navigieren kann.
\end{description}