\subsection{Dokumentation}
\label{sec:Dokumentation}

Nach Abschluss und Korrektur des Alpha- und Betatests kann das Projekt als
funktional abgeschlossen bezeichnet werden. Zur Förderung von Erweiterbarkeit
und Nachhaltigkeit, wird eine Dokumentation angefertigt, die es jedem
berechtigten Administrator ermöglicht die Anwendung zu pflegen und zu erweitern.
Die vollstände Dokumentation kann in \citet{dokumentation2014} eingesehen
werden. An einem Anwendungsbeispiel soll im Folgenden die Benutzung der
Anwendung verdeutlicht werden. Als Beispiel wird dafür die Erweiterung der
Anwendung um ein Foto vorgestellt. Dieser Prozess umfasst ingesamt sechs
Teilschritte.

\begin{enumerate}
  \item Als erstes muss ein neues Panorama aufgenommen werden. Hierbei
  sollten die in \verweis{Panoramaerstellung} beschriebenen Hinweise beachtet
  werden. Es empfiehlt sich ein Stativ und einen Panoramakopf für die Aufnahme
  zu verwenden, um eine optimale Fotoqualität zu erreichen.
  \item Anschließend muss das erstelle Panorama im Administrationsbereich
  der Software unter dem Menüpunkt "`Fotos"' mit Namen und Beschreibung
  hochgeladen werden. An dieser Stelle empfiehlt es sich eine einheitliche
  Namenskonvention zu verwenden, die Auskunft über den Ort der Aufnahme gibt.
  Eine möglich Namenskonvention wäre "`KE\_EG\_001"' (Gebäude KG, Erdgeschoss,
  Raum 001). Diese Namenskonvention hilft im Folgenden dabei, das Foto zu
  identifizieren.
  \item Als nächstes kann bei Bedarf ein Informationstext zu dem Foto
  erstellt werden. Falls ein interessantes Projekt oder eine wichtige
  Information auf dem Foto abgebildet ist kann diese dem Benutzer über einen
  solchen Informationstext angezeigt werden.
  \item Ebenso kann das erstellte und hochgeladene Foto einen interessanten
  Ort darstellen, für den es sich lohnt eine gesonderte Verlinkung in der
  Benutzeransicht zu erstellen. Dies könnte zum Beispiel bei einem Foto von dem
  Eingang zur Bibiliothek oder zum Campusgebäude sinnvoll sein. In diesem Fall
  kann im Menüpunkt "`Interessante Orte"' ein Eintrag mit Namen und
  Beschreibung des Ortes hinterlegt und das erstellte Panorama mit diesem
  verknüpft werden.
  \item Im vorletzen Schritt muss das neue Foto auf der Karte des Campus
  positioniert werden. Dazu navigiert der Administrator zum Menüpunkt
  "`Übersichtskarte"' und klickt auf die Position in der angzeigten Karte, an
  der das Foto aufgenommen wurde. Es erscheint ein roter Marker an der
  ausgewählten Position. Nach einem weiteren Klick auf diesen Marker öffnet sich
  ein Auswahlmenü. In diesem Menü werden dem Administrator alle hochgeladenen
  Fotos angezeigt, die an der ausgewählten Position speichern kann. Der
  Administrator wählt das neu hochgeladene Foto aus und klickt auf
  "`Speichern"'.
  \item Nach dem Positionieren des neuen Fotos erscheint ein grüner Marker
  an der festgelegten Position. Durch Klicken auf diesen Marker hat der
  Administrator im letzen Schritt die Möglichkeit vorher gespeicherte
  Informationstexte (siehe Punkt 3.) an dieses Foto anzuhängen oder diese
  Position mit anderen Fotos zu verbinden. Durch die Verbindung zu anderen Fotos
  kann ein Benutzer in der Benutzeransicht über die verbundenen Fotos zu dem neu
  hochgeladenen Foto navigieren. Ein hochgeladenes, aber nicht verbundenes  
  Foto ist für dem Endbenutzer zunächst nicht sichtbar, da er zu diesem nicht
  navigieren kann.
\end{enumerate}