\subsection{PHP}
\label{sec:Php}

PHP (Hypertext Preprocessor) ist eine serverseitige Skriptsprache, die mit den Ziel entwickelt wurde Webseiten dynamisch
zu gestalten.
Mit dieser Intention wurde auch im vorliegenden Projekt PHP als Serversprache verwendet. Die nachfolgend beschriebenen
Grundlagen vermitteln die wichtigsten Aspekte der Sprache PHP, die nötig sind, um die Umsetzung des Projektes
nachvollziehen zu können.
Im Gegensatz zu der vorher beschrieben Skriptsprache Javascript ist PHP eine serverseitige Skriptsprache. Dadurch sind
zwei grundlegende Unterschiede zu Javascript bedingt:

\textbf{1}. PHP wird, wie die Beschreibung Serverskriptsprache bereits suggeriert, nicht auf dem Client (dem Browser des Nutzers),
sondern auf dem Webserver der Anwendung ausgeführt. Bei einer Nutzeranfrage an den Webserver wird also erst eine PHP-Routine serverseitig ausgeführt und anschließend wird das angefragte Dokument, inklusive
Javascriptdateien, an den Client (Browser) ausgeliefert und dort wird dann Javascript ausgeführt.

\textbf{2}. PHP hat weitestgehend vollen Ressourcenzugriff auf dem ausführenden System. Durch PHP-Befehle ist es zum Beispiel möglich Dateien auf dem ausführenden System (Webserver) anzulegen und zu löschen.
Diese beiden Unterschiede sind wichtig, um das Einsatzfeld von PHP zu verstehen.

PHP wird im vorliegenden Projekt für die folgenden Aufgaben eingesetzt:
\begin{itemize}
  \item Kommunikation mit der Datenbank
  \item Schreiben von dynamischen HTML
  \item Kopieren von Benutzerdateien auf das ausführende System.
\end{itemize}

Diese Aufgaben werden im späteren Verlauf ausführlicher beleuchtet, an dieser Stelle soll die Listung dieser Aufgaben
ausreichen, um ein Verständnis für die Einsatzfelder von PHP zu bekommen.
Um diese Aufgaben zu erfüllen sind in PHP, wie in jeder anderen Programmiersprache auch, Befehle mit einer vorgebenen
Syntax definiert. Diese Befehle werden im Quellcode in einem besonders gekennzeichneten Bereich geschrieben. Dieser
Bereich wird durch die Zeichenfolge '<?php' eingeleitet und mit der Zeichenfolge '?>' beendet.
Abschließend soll die folgende Abbildung die Anwendung von PHP verdeutlichen:

% TODO: Quellcode Listung einfügen
-- PHP Quellcode Beispiel --

-- kurze Erläuterung zum Beispiel --
% TODO: In diesem Beispiel wird das und das und das gemacht