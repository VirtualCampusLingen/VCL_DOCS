\subsection{PHP}
\label{sec:Php}

PHP (Hypertext Preprocessor) ist eine serverseitige Skriptsprache, die mit dem
Ziel entwickelt wurde, Webseiten dynamisch zu gestalten. Mit dieser Intention
wurde PHP auch im vorliegenden Projekt verwendet. Die nachfolgend beschriebenen
Grundlagen vermitteln die wichtigsten Aspekte der Sprache PHP, die nötig sind,
um die Umsetzung des Projektes nachvollziehen zu können. Im Gegensatz zu der
vorher beschrieben Skriptsprache Javascript ist PHP eine serverseitige
Skriptsprache. Dadurch sind zwei grundlegende Unterschiede zu Javascript
gegeben:

\begin{enumerate}
  \item PHP wird nicht auf dem Client (dem Browser des Nutzers), sondern auf dem
  Webserver ausgeführt. Bei einer Nutzeranfrage an den Webserver wird also
  erst serverseitig eine PHP-Routine ausgeführt und anschließend das vom Nutzer
  angefragte Dokument, inklusive Javascriptdateien, an den Client (Browser)
  ausgeliefert. Dort wird der übertragene Javascript-Quellcode dann ausgeführt.
  \item PHP hat weitestgehend vollen Ressourcenzugriff auf dem ausführenden
  System. Durch PHP-Befehle ist es zum Beispiel möglich, Dateien auf dem
  ausführenden System (Webserver) anzulegen und zu löschen. Diese beiden
  Unterschiede sind wichtig, um das Einsatzfeld von PHP zu verstehen.
\end{enumerate}

PHP wird im vorliegenden Projekt für die folgenden Aufgaben eingesetzt:

\begin{itemize}
  \item Kommunikation mit der Datenbank
  \item Schreiben von dynamischen HTML
  \item Kopieren von Benutzerdateien auf das ausführende System.
\end{itemize}

Auf eine konkrete Beschreibung der Aufgaben soll an dieser Stelle verzichtet
werden, da diese noch im weiteren Verlauf der Ausarbeitung erfolgt. Es wird hier
lediglich kurz aufgezeigt, auf welche Weise die oben stehenden Aufgaben von PHP
umgesetzt werden.

Um die Aufgaben zu erfüllen sind in PHP, wie in jeder anderen Programmiersprache
auch, Befehle mit einer vorgebenen Syntax definiert. Diese Befehle werden im
Quellcode in einem besonders gekennzeichneten Bereich geschrieben. Dieser
Bereich wird durch die Zeichenfolge '<?php' eingeleitet und mit der
Zeichenfolge '?>' beendet. Das \listing{PHP Grundlagen} soll diese Anwendung
von PHP verdeutlichen:

\lstinputlisting[language=PHP,caption={PHP Grundlagen},label={lst:PHP Grundlagen}]{Listings/PHP_Grundlagen.php}

In dem Beispiel wird die Anfrage eines Benutzers ausgewertet. Die
eigentliche Anfrage ist hierbei nebensächlich und ist aus diesem Grund im
Listing nicht dargestellt (siehe Kommentar in Zeile 2). Die Auswertung der
Anfrage soll an dieser Stelle zeigen auf welchem Weg mit Hilfe von PHP
dynamisches HTML geschrieben werden kann. In dem dargestellten Beispiel wird die
Anfrage des Benutzers ausgewertet und ein HTML p-Tag geschrieben, dessen Inhalt
von dem Ergebnis der Abfrage abhängig ist. Dem Benutzer wird durch die Ausgabe
des p-Tags im Browser angezeigt, ob seine Anfrage positiv oder negativ
ausgewertet wurde.