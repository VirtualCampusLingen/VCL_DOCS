\subsection{Datenbanken und SQL}
\label{sec:DatenbankenUndSql}

Eine Datenbank ist das Rückgrat jedes datenverarbeitenden Informationssystems.
Dementsprechend häufig sind Datenbanken in verschiedensten Kontexten, wie
beispielsweise Web-Systemen, ERP-Systemen oder mobilen Systemen anzutreffen.
Die nachfolgende theoretische Fundierung dient zum Verständnis des Einsatzes der
Datenbank im vorliegenden Projekt.

"`Eine Datenbank ist eine Sammlung von Daten, die einen Ausschnitt der realen
Welt beschreibt."'\footnote{\citet[S.~18]{elmasri2009}} Diese Sammlung von Daten
kann dabei sowohl in Papierform, als auch in elektronischer und
computergestützter Form aufbewahrt werden. Vor dem Hintergrund des vorliegenden
Projektes gilt die weitere Betrachtung jedoch ausschließlich den
computergestützten Datenbanken. In diesem Teilbereich gibt es wiederum
verschiedene Datenbanktypen, die sich jeweils in der Verwaltung und Haltung der
Daten voneinander unterscheiden. Die wichtigsten Datenbanktypen sind hierbei:

\begin{itemize}
  \item hierarchische Datenbanken
  \item objektorientierte Datenbanken
  \item relationale Datenbanken
  \item NoSql Datenbanken
\end{itemize}

Der Fokus liegt im Folgenden auf den relationalen Datenbanken, da diese im
Projekt eingesetzt werden.

Eine relationale Datenbank beruht auf dem relationalen Datenbankmodell von
Edgar F. Codd aus dem Jahre 1970.\footnote{siehe dazu \citet{codd1970}}
Grundlage dieses Modells ist der Begriff der Relation, der im Wesentlichen eine
mathematische Beschreibung einer Tabelle darstellt. Eine relationale Datenbank
lässt sich somit als eine Sammlung von Tabellen beschreiben.

Auf den Relationen (Tabellen) einer solchen Datenbank sind bestimmte
Operationen definiert, die im Kontext der Mathematik der relationalen Algebra
und im Kontext von Datenbanken der Structured Query Language entsprechen.

Die Structured Query Language (SQL) stammt, genau wie das relationale
Datenbankmodell, aus dem Jahre 1970. Über einen Entwicklungszeitraum von
mehreren Jahrzehnten wurde SQL zu einer komplexen Sprache, welche umfangreiche
Möglichkeiten bietet. Dieser Gesamtkomplex wird hier stark verkürzt dargestellt
und beschränkt sich nur auf die wesentlichen Aspekte, die dem Verständnis des
Projektes dienen.

SQL dient im Wesentlichen zum Definieren, Kontrollieren und Manipulieren von
Datenstrukturen. Um diese Bereiche voneinander zu trennen sind in der SQL die
folgenden drei Sprachbereiche definiert:

\begin{description}
  \item[Data Definition Language (DDL)] dient zum Erstellen, Löschen und Ändern
  von Relationen.
  \item[Data Control Language (DCL)] dient zum Vergeben und Entziehen von
  Berechtigungen.
  \item[Data Manipulation Language (DML)] dient zum Lesen, Schreiben, Löschen
  und Ändern von Datenstrukturen.
\end{description}

Die Sprachbereiche DDL und DCL haben dabei in der Projektentwicklung eher
einmaligen Charakter, wohingegen DML kontinuierlich verwendet wird, um
dynamisch Informationen zu schreiben oder zu lesen. Die DML tritt als
Schnittstelle zwischen Anwendungsprogramm und Datenbank auf (siehe
\abbildung{Architektur}). Über die DML sind beispielsweise Befehle definiert,
die es erlauben einen Datensatz einer Tabelle zu lesen oder zu verändern. Diese
Befehle werden im Anwendungsprogramm aufgerufen und erzeugen hierdurch
dynamischen Inhalt für den Anwender.