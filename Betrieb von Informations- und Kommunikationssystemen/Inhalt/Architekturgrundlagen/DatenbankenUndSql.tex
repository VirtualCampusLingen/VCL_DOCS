\subsection{Datenbanken und SQl}
\label{sec:DatenbankenUndSql}

Eine Datenbank ist das Rückrad jedes datenverarbeitenden Informationssystems. Dementsprechend häufig sind Datenbanken in 
verschiedenen Kontexten anzutreffen (Web-, ERP-, Mobile-Systeme, etc.). Die nachfolgende theoretische Fundierung dient zum 
Verständnis des Einsatzes der Datenbank im vorliegenden Projekt.

"`Eine Datenbank ist eine Sammlung von Daten, die einen Ausschnitt der realen Welt beschreibt."' (Quelle: elmasri2009) 
Diese
Sammlung von Daten kann dabei sowohl in Papierform, als auch elektronisch und computergestützt aufbewahrt werden. Vor dem 
Hintergrund des vorliegenden Projektes gilt die weitere Betrachtung aber ausschließlich den computergestützten Datenbanken.
In diesem Teilbereich gibt es wiederrum verschiedene Datenbanktypen, die sich in der Verwaltung und Haltung der Daten 
unterscheiden. Die wichtigsten Datenbanktypen sind im Folgenden aus Gründen der Abgrenzung aufgelistet:

\begin{itemize}
  \item hierarchische Datenbanken
  \item objektorientierte Datenbanken
  \item relationale Datenbanken
  \item NoSql Datenbanken
\end{itemize}

Der Fokus dieser Spezialisierung liegt im Folgenden wiederrum auf der eingesetzten Technologie des Projektes und damit auf 
relationalen Datenbanken. 

Eine relationale Datenbank beruht auf dem relationalen Datenbankmodell von Edgar F. Codd aus dem Jahre 1970 (Verweis: 
codd1970). Grundlage dieses Modells ist der Begriff der Relation, der im wesentlichen (schreibt man wesentliche hier groß?)
eine mathematische Beschreibung einer Tabelle darstellt. Eine relationale Datenbank ist damit eine Sammlung von Tabellen.

Auf den Relationen (Tabellen) einer solchen Datenbank sind bestimmte Operationen definiert, die im Kontext der Mathematik 
der relationalen Algebra und im Kontext von Datenbanken dem Begriff SQL entsprechen.

SQL steht für Structured Query Language und stammt, genau wie das relationale Datenbankmodell, aus dem Jahre 1970. Es 
wurde ebenfalls von Edgar F. Codd mitentwickelt (Quelle: kline2005). Über einen Entwicklungszeitraum von mehreren Jahrzehnten 
wurde SQL zu einer komplexen Sprache, die umfangreiche Möglichkeiten bietet. Dieser Gesamtkomplex wird hier stark verkürzt
dargestellt und beschränkt sich nur auf die wesentlichen Aspekten, die dem Verständnis des Projektes dienen.

SQL dient im wesentlichen (wesentlich groß?) zum Definieren, Kontrollieren und Manipulieren von Datenstrukturen. Um diese 
Bereiche zu trennen sind in SQL drei Sprachbereiche definiert.

\begin{description}
  \item[Data Definition Language (DDL)] dient zum Erstellung, Löschen und Ändern von Tabellen.
  \item[Data Control Language (DCL)] dient zum Vergeben und Entziehen von Berechtigungen.
  \item[Data Manipulation Language (DML)] dient zum lesen, schreiben, löschen und ändern von Datenstrukturen.
\end{description}

Die Sprachbereiche DDL und DCL haben dabei in der Projektentwicklung eher einmaligen Charakter, wohingegen DML 
kontinuierlich verwendet wird um dynamischen Informationen zu schreiben oder zu lesen. Die DML tritt dabei als 
Schnittstelle zwischen Anwendungsprogramm und Datenbank auf (siehe Architekturentwurf). Über die DML sind beispielsweise 
Befehle definiert, die es erlauben einen Datensatz einer Tabelle zu lesen oder zu verändern. Diese Befehle werden im 
Anwendungsprogramm aufgerufen und erzeugen damit dynamischen Inhalt für den Anwender.