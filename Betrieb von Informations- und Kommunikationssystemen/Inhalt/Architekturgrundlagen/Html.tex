\subsection{HTML}
\label{sec:Html}

Die Hypertext Markup Language, abgekürzt HTML, stellt eine deklarative Sprache
zur Beschreibung und Gestaltung von Internetseiten dar. Die erste Spezifikation
von HTML erschien 1989, wobei sich die Spezifikation im Laufe der Zeit
verändert und erweitert haben\footnote{\citet[S.~27ff]{taglinger2003}}.

Mit Hilfe von HTML können Dokumente erstellt werden, welche von Internetbrowsern
gelesen, interpretiert und dem Inhalt entsprechend dargestellt werden.
Dargestellt werden können Text, Bilder, Tabellen, Verlinkungen, Videos usw.

Der grundlegende Aufbau von HTML erfolgt durch sogenannte Tags, wobei
die meisten Befehle einen Start- und einen Endtag besitzen.
Das Tag <html> dient zur Öffnung eines HTML-Bereichs, welcher die weiteren
Bereiche und Inhalte für die Darstellung einer Internetseite beinhaltet. Der
HTML-Bereich und das gesamte Dokument wird mit dem Tag </html> geschlossen. Das
Tag <head> enthält Kopfdaten, z.B. den Titel einer Internetseite, welcher in
der Kopfzeile des Internetbrowser angezeigt wird. Im <body>-Bereich des
HTML-Dokuments wird der eigentliche Inhalt der Internetseite angegeben. Dieser
wird dann im Browserfenster entsprechend den Angaben und Inhalten dargestellt.

Die grundlegende Struktur eines HTML-Dokumentes stellt sich somit folgendermaßen
dar:

\lstinputlisting[language=HTML,caption={Architekturgrundlagen
HTML},label={lst:HTML Grundlagen}]{Listings/HTML_Grundlagen.html}

Würde das Beispiel in einem Browser geladen werden, so würde eine
Internetseite mit der Kopfzeile "`Titel der Internetseite"' darstellt werden.
Als Inhalt würde nur der Text "`Inhalt der Internetseite"', ohne jegliche
Formatierung, angezeigt werden.

Die neueste Spezifikation von HTML, HTML5, bietet einige nützliche Neuerungen.
So können nun beispielsweise Audio oder Videodateien direkt im
HTML-Dokument eingebunden werden, ohne dass für das Abspielen zusätzliche
Plugins benötigt werden.
