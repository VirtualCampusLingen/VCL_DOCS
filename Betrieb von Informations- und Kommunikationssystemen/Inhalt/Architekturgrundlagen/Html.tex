\subsection{HTML}
\label{sec:Html}

Die Hypertext Markup Language, abgekürzt HTML, stellt eine deklarative Sprache zur Beschreibung und Gestaltung von Internetseiten dar. Die erste Spezifikation von HTML erschien 1989, wobei sich die Spezifikation im Laufe der Zeit verändert und erweitert haben\footnote{\citet{taglinger2003}}.

Mithilfe von HTML können Dokumente erstellt werden, welche von Internetbrowsern gelesen, interpretiert und dem Inhalt entsprechend dargestellt werden. Dargestellt werden können Text, Bilder, Tabellen, Verlinkungen, Videos usw.

Der grundlegende Aufbau von HTML erfolgt hierbei durch sogenannte Tags, wobei die meisten Befehle einen Start- und einen Endtag besitzen.
Das Tag <html> dient zur Öffnung eines HTML-Bereichs, welches die weiteren Bereiche und Inhalte für die Darstellung einer Internetseite besitzt. Der HTML-Bereich und das gesamte Dokument wird mit dem Tag </html> geschlossen. Das Tag <head> enthält Kopfdaten, z.B. den Titel einer Internetseite, welcher in der Kopfzeile des Internetbrowser angezeigt wird. Im <body>-Bereich des HTML-Dokuments wird der eigentliche Inhalt der Internetseite angegeben. Dieser wird dann im Browserfenster entsprechend den Angaben und Inhalten dargestellt.

Die Struktur eines HTML-Dokumentes stellt sich somit folgendermaßen dar:

\lstinputlisting[language=HTML,caption={Architekturgrundlagen HTML},label={lst:HTML Grundlagen}]{Listings/HTML_Grundlagen.html}

Im vorherigen Beispiel würde somit eine Internetseite im Browser angezeigt werden, welche den Titel "`Titel der Internetseite"' in der Kopfzeile anzeigen würde. Als Inhalt würde nur der Text "`Inhalt der Internetseite"' angezeigt werden, ohne jegliche zusätzliche Formatierungen.

Die neueste Spezifikation von HTML, HTML5, bietet einige nützliche Neuerungen, so können nun unter anderem direkt im HTML-Dokument Audio oder Videodateien eingebunden werden, ohne das zusätzliche Plugins, wie z.B. Adobe Flash benötigt werden, um diese abspielen zu können.