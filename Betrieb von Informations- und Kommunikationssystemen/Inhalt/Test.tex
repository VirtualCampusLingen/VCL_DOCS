\section{Test}
\label{sec:Test}

Die zu erstellende Software ist mit Abschluss der Umsetzungsphase fertiggestellt und einsatzbereit. Vor Projektabschluss und -übergabe soll die Software auf seine Qualität geprüft werden. Es soll darüber hinaus verifiziert werden, dass die Anforderungen des Lastenheftes umgesetzt wurden. Sowohl Qualität, als auch die Umsetzung der Anforderungen wird im Folgenden in einem zweistufigen Praxistest der Software sichergestellt.

Im ersten Schritt wird die Software einem Alphatest unterzogen. Ein Alphatest liegt immer dann vor, wenn die Tests von Personen durchgeführt werden, die an der Entwicklung eines Produktes beteiligt waren, in diesem Fall das Projektteam. In diesem Test werden vor allem die Produktfunktionen des Lastenheftes (siehe Seite \pageref{sec:Lastenheft}) durchgespielt. In diesem Test sollen erste grobe Fehler in Bedienung, Anzeige und Verhalten gefunden werden und es soll zusätzlich sichergestellt werden, dass alle Anforderungen des Auftraggebers umgesetzt sind. Die Tests der Produktfunktionen werden zusätzlich in den vier führenden Browsern\footnote{Quelle: \url{http://www.chip.de}} durchgeführt, um browserübergreifende Funktionalität sicherzuestellen und dabei jeweils in die Bereiche Bedienung (\textbf{B}), Anzeige (\textbf{A}) und Verhalten(\textbf{V}) unterteilt. Das Ergbeniss des durchgeführten Alphatests ist in nachfolgender Tabelle festgehalten.

\begin{table}[htbp]
  \centering
    \begin{tabular}{l|cccc}
    \toprule
          & Google Chrome & Mozilla Firefox & Internet Explorer & Safari \\
    \midrule
    LF0010 &       &       & X     &  \\
    LF0015 &       &       & X     &  \\
    LF0020 &       &       & X     &  \\
    LF0030 & X     &       &       &  \\
    LF0040 &       &       & X     &  \\
    LF0050 &       & X     &       &  \\
    LF0060 &       & X     &       &  \\
    LF0070 &       & X     &       &  \\
    LF1010 &       & X     &       &  \\
    LF1020 &       & X     &       &  \\
    LF1025 &       & X     &       &  \\
    LF1030 &       & X     &       &  \\
    LF1040 &       & X     &       &  \\
    LF1110 &       & X     &       &  \\
    LF1120 &       & X     &       &  \\
    \bottomrule
    \end{tabular}
  \caption{Ergebnis des Alphatests}
  \label{tab:ErgebnisAlphatest}
\end{table}

Ergbniss des Alphatest ist ....