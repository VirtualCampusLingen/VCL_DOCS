% Meta-Informationen -----------------------------------------------------------
%   Definition von globalen Parametern, die im gesamten Dokument verwendet
%   werden können (z.B auf dem Deckblatt etc.).
%
%   ACHTUNG: Wenn die Texte Umlaute oder ein Esszet enthalten, muss der folgende
%            Befehl bereits an dieser Stelle aktiviert werden:
            \usepackage[utf8x]{inputenc}
% ------------------------------------------------------------------------------
\newcommand{\titel}{Virtueller Campus Lingen}
\newcommand{\untertitel}{Teil Modellierung und Betrieb}
\newcommand{\art}{Prüfungsleistung\ }
\newcommand{\autorA}{Jannik Fangmann}
\newcommand{\autorB}{Andreas Makeev}
\newcommand{\autorC}{Raphael Otten}
\newcommand{\autorD}{Carsten Sandker}
\newcommand{\matrikelnrA}{506347}
\newcommand{\matrikelnrB}{517007}
\newcommand{\matrikelnrC}{516975}
\newcommand{\matrikelnrD}{500199}
\newcommand{\studienbereich}{Wirtschaftsinformatik}
\newcommand{\semester}{6}
\newcommand{\gutachterA}{Stephan Feldker (B. Eng.)}
\newcommand{\gutachterB}{Prof. Dr.-Ing. Ralf Westerbusch }
\newcommand{\abgabedatum}{\today}
\newcommand{\jahr}{2014}
\newcommand{\ort}{Lingen}
\newcommand{\logo}{HS_Osna_MKT.jpg}