\section{Zielbestimmung}
\label{sec:Zielbestimmung}

Ziel des Projektes Virtueller Campus Lingen ist es den Standort der Hochschule Osnabrück
in Lingen virtuell darzustellen sowie alle vier Institutionen mit ihren Studienangeboten
vorzustellen.

Der virtuelle Rundgang über den Campus Lingen soll Studieninteressierte auf diesen auf-
merksam machen, das Studieren in Lingen aus einer modernen, interaktiven Perspektive
zeigen und bestenfalls vom Studienangebot in Lingen überzeugen. Hierzu soll das gesamte
Gebäude der Hochschule in Lingen an der Kaiserstraße sowie der Standort an der Bac-
cumer Str. fotografiert werden. Diese Fotos werden, im Stile von Googl-Street-View, zu
360-Grad-Panoramen zusammengesetzt, in denen sich der Nutzer frei umschauen kann.

Dem Anwender soll auf diese Weise ein neuartiger Einblick in die Räumlichkeiten und auf
das Umfeld der Hochschule geboten werden. Es soll nicht nur visuell von der Hochschule
überzeugt werden, sondern auch mit Inhalt und Möglichkeiten.

In diesem Rahmen werden Informationen und Wissenswertes dem Nutzer an passenden
Stellen angezeigt. Zum Beispiel kann am Eingangsbereich einer Fakultät eine Infobox
angezeigt werden.

Auf dieser Basis soll, eingebettet in eine ansprechende Web-Oberfläche, eine Web-Anwendung
entstehen, die per Verlinkung auf die aktuelle Webseite der Hochschule eingebunden wer-
den kann.

Neben der öffentlich sichtbaren Ansicht soll es zusätzlich ein Backend-System zur War-
tung und Pflege der Anwendung geben. Hier sollen Texte, Fotos und Informationen von
Mitarbeitern der Hochschule aktualisiert werden können, um die Anwendung auch für die
Zukunft immer aktuell halten zu können.
