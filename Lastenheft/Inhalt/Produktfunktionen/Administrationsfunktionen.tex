\subsection{Administrationsfunktionen}
\label{sec:Administrationsfunktionen}

Der Administrator ist ein Benutzer, der für den geregelten Ablauf des Systems verant-
wortlich ist. Er kann die Eigenschaften des Systems ändern bzw. konfigurieren.

\begin{description}
  \item[LF1010] Der Administrator muss neue Fotos in die Anwendung einpflegen können.
  \item[LF1020] Der Administrator muss veraltete Fotos in der Anwendung austauschen können.
  \item[LF1030] Der Administrator muss hinterlegte Informationen zu einzelnen Fotos, zum Beispiel Name und Beschreibung, verändern können.
  \item[LF1040] Der Administrator muss Informationstexte speichern können.
  \item[LF1050] Der Administrator muss Informationstexte löschen können.
  \item[LF1060] Der Administrator muss den Informationstext abändern können.
  \item[LF1070] Der Administrator muss interessante Orte, zum Beispiel Bibliothek oder Fachbereichsgebäide, mit einem entsprechenden Text erstellen können.
  \item[LF1080] Der Administrator muss interessante Ort löschen können.
  \item[LF1090] Der Administrator muss die hinterlegten Informationen zu interessanten Orten verändern können.
  \item[LF1100] Dem Administrator muss eine topographische Karte als Positionierungseinschränkung für neue
  360-Grad-Fotos zur Verfügung stehen.
  \item[LF1110] Der Administrator muss auf dieser Karte Fotos platzieren und löschen können.
\end{description}