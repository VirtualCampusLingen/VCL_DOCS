\section{Globale Testszenarien und Testfälle}
\label{sec:GlobaleTestszenarienundTestfaelle}


\subsection{Benutzerfunktionen}
\label{sec:Benutzerfunktionen}

\begin{description}
  \item[T0010] Der Benutzer startet einen Testzugriff ohne Anmeldung mit einem im Internet befindlichen Endgerät mit einer Displaygröße von min. 8 Zoll auf die Anwendung.
  \item[T0015] Der Benutzer startet die Anwendung und lässt sich einen weiteren Standpunkt anhand der Auswahl auf der Übersichtskarte anzeigen.
  \item[T0020] Der Benutzer sieht sich in den 360-Grad Fotos der Anwendung frei um und zoomt in das Foto rein bzw. raus.
  \item[T0030] Der Benutzer navigiert zwischen den Fotos mit Hilfe des Steuerkreuzes im unteren Anwendungsbereich.
  \item[T0040] Der Benutzer vergrößert die Minimap und springt durch Auswahl auf der Übersichtskarte an einen weiteren Einstiegspunkt.
  \item[T0045] Der Benutzer startet die Anwendung und prüft, ob sich die Minimap in der linken
oberen Ecke des 360-Grad Fotos befindet.
  \item[T0050] Der Benutzer springt durch Auswahl in der Übersichtskarte einmal an alle Ein-
stiegspunkte.
  \item[T0060] Der Benutzer klickt ein Informationsicon an und lässt sich die zugehörigen Informationen anzeigen. Enthalten die Informationen einen weiterführenden Link, so wird dieser ebenfalls aufgerufen.
\end{description}

\subsection{Administrationsfunktionen}
\label{sec:Administrationsfunktionen}

\begin{description}
  \item[T1010] Der Administrator benutzt die vorhandenen Anwendungsfunktionen, welche im Administrationsbereich angezeigt werden.
  \item[T1020] Der Administrator lädt in der Medienverwaltung neue 360-Grad Fotos hoch und löscht diese daraufhin.
  \item[T1030] Der Administrator lädt ein neues 360-Grad Foto hoch und positioniert es an einem Punkt auf der vorgegebenen Karte
  \item[T1040] Der Administrator lösche das ein hochgeladenes Foto. Es wird daraufhin nicht mehr auf der Übersichtskarte angezeigt.
  \item[T1050] Der Administrator tauscht ein veraltetes Foto durch ein neu hochgeladenes oder bereits vorhandenes Foto aus.
  \item[T1050] Der Administrator fügt neue Informationstexte hinzu, ändert diese ab und löscht veraltete Informationen.
  \item[T1060] Der Administrator ertsellt einen neuen Text zu einem interessanten Ort.
  \item[T1070] Der Administrator verbindet ein Interessanten Ort und einen Informationstext mit einem Foto auf der Übersichtskarte.
\end{description}
