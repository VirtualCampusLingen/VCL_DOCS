\section{Produkfunktionen}
\label{sec:Produkfunktionen}

\subsection{Benutzerfunktionen}
\label{sec:Benutzerfunktionen}

\begin{description}
  \item[F0010] Einem beliebigen Internetnutzer muss es möglich sein, ohne Anmeldung mit seinem
  im Internet befindlichen Endgerät mit einer Displaygröße von min. 8 Zoll auf die
  Anwendung zugreifen zu können.
  \item[F0015] Beim Starten der Anwendung soll dem Benutzer ein 360-Grad Panorama vom
  Eingangsbereich des Campus Lingen präsentiert werden. Von dieser Ansicht aus
  besteht die Möglichkeit den Campus zu erkunden (nächstes 360-Grad Foto). Er hat
  aber auch die Möglichkeit sich über die Übersichtkarte einen anderen Standpunkt
  als 360 Grad Foto anzeigen zu lassen.
  \item[F0020] Der Benutzer muss sich in den 360-Grad Fotos der Anwendung frei umsehen können. Außerdem hat er die Möglichkeit in das 360-Grad Foto rein und raus zu zoomen.
  \item[F0030] Der Benutzer muss zwischen den Fotos navigieren können. Hierzu steht ihm ein
  Steuerkreuz im unteren Anwendungsbereich zur Verfügung.
  \item[F0040] Die Anwendung muss eine Minimap enthalten, die sowohl den aktuellen Standpunkt, als auch alle weiteren Einstiegspunkte umfasst.
  \item[F0045] Die Minimap soll in der oberen linken Ecke des 360-Grad-Fotos angezeigt werden.
  Sie stellt einen Ausschnitt der Übersichtskarte an der aktuellen Position des Nutzers
  dar.
  \item[F0050] In den 360-Grad-Fotos müssen relevante Informationen zu den dargestellten Örtlichkeiten angezeigt werden können. Diese werden hinter einem Informationsicon als
  Popup hinterlegt. Sobald der Nutzer diese Icon anklickt erscheint ein Popup, welches die betreffenden Informationen beinhaltet. Außerdem kann das Popup einen
  weiterführenden Link auf die Hochschulwebseite für mehr Informationen enthalten.
\end{description}

\subsection{Administrationsfunktionen}
\label{sec:Administrationsfunktionen}

\begin{description}
  \item[F1010] Der Administrator muss sich per Passwort im Administrationsbereich anmelden können.
  \item[F1020] Der Administrationsbereich bietet ein strukturiertes Menü mit allen Anwendungsfunktionen, die der Administrator verwenden kann.
  \item[F1030] Der Administrator hat die Möglichkeit in einer Medienverwaltung 360-Grad Fotos hochzuladen und zu löschen. Hierzu steht ihm eine Miniaturansicht der 360-Grad Fotos, sowie der Dateinamen der Bilder, die sich auf dem Server befinden, zur Verfügung.
  \item[F1040] Eine definierte Karte muss dem Administrator als Positionierungseinschränkung für neue 360-Grad-Fotos zur Verfügung stehen.
  \item[F1050] Der Administrator muss neue Fotos in die Anwendung einpflegen können. Die Fotos können hierbei an beliebigen Punkten auf einer definierten Karte positioniert werden. Hierzu sieht der Administrator die Übersichtskarten des Campus und hat die Möglichkeit per Mauszeiger auf dem dort angezeigten Wegenetz Fotos frei positionieren oder ändern/löschen zu können.
  \item[F1060] Der Administrator muss veraltete Fotos in der Anwendung austauschen und durch bereits hochgeladene ersetzen können.
  \item[F1070] Der Administrator muss Informationstexte zu den 360-Grad Fotos hinzufügen, abändern und löschen können. Diese Informationstexte können zum Beispiel Projektvorstellungen enthalten. Der Informationstext besteht aus einem Pop-Up mit Einführungstext und weiterführenden Links.
  \item[F1080] Der Administrator muss Texte zu interessanten Orten erstellen können und diese mit bereits hochgeladenen Fots verbinden.
\end{description}